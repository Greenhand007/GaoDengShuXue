% -*- coding: utf-8 -*-
% !TEX program = xelatex

\documentclass[14pt,notheorems,leqno,xcolor={rgb}]{beamer} % ignorenonframetext

% -*- coding: utf-8 -*-
% ----------------------------------------------------------------------------
% Author:  Jianrui Lyu <tolvjr@163.com>
% Website: https://github.com/lvjr/theme
% License: Creative Commons Attribution-ShareAlike 4.0 International License
% ----------------------------------------------------------------------------

\ProvidesPackage{beamerthemeriemann}[2018/06/05 v0.6 Beamer Theme Riemann]

\makeatletter

% compatible with old versions of beamer
\providecommand{\beamer@endinputifotherversion}[1]{}

\RequirePackage{tikz,etoolbox,adjustbox}
\usetikzlibrary{shapes.multipart}

\mode<presentation>

\setbeamersize{text margin left=8mm,text margin right=8mm}

%% ----------------- background canvas and background ----------------

\newif\ifbackgroundmarkleft
\newif\ifbackgroundmarkright

\newcommand{\insertbackgroundmark}{
  \ifbackgroundmarkleft
    \foreach \x in {1,2,3,4,5} \draw[very thick,markcolor] (0,\x*\paperheight/6) -- +(1.2mm,0);
  \fi
  \ifbackgroundmarkright
    \foreach \x in {1,2,3,4,5} \draw[very thick,markcolor] (\paperwidth,\x*\paperheight/6) -- +(-1.2mm,0);
  \fi
}

\defbeamertemplate{background}{line}{%
  \begin{tikzpicture}
    \useasboundingbox (0,0) rectangle (\paperwidth,\paperheight);
    \draw[xstep=\paperwidth,ystep=1mm,color=tcolor] (0,0) grid (\paperwidth,\paperheight);
    \insertbackgroundmark
  \end{tikzpicture}%
}

\defbeamertemplate{background}{linear}{%
  \begin{tikzpicture}
    \useasboundingbox (0,0) rectangle (\paperwidth,\paperheight);
    \draw[pattern=horizontal lines, pattern color=tcolor]
      (0,0) rectangle (\paperwidth,\paperheight);
    \insertbackgroundmark
  \end{tikzpicture}%
}

\defbeamertemplate{background}{lattice}[1][1mm]{%
  \begin{tikzpicture}
    \useasboundingbox (0,0) rectangle (\paperwidth,\paperheight);
    \draw[step=#1,color=tcolor,semithick] (0,0) grid (\paperwidth,\paperheight);
    \insertbackgroundmark
  \end{tikzpicture}%
}

\defbeamertemplate{background}{empty}{
  \begin{tikzpicture}
    \useasboundingbox (0,0) rectangle (\paperwidth,\paperheight);
    \insertbackgroundmark
  \end{tikzpicture}%
}

%% -------------------------- title page -----------------------------

% add \occasion command
\newcommand{\occasion}[1]{\def\insertoccasion{#1}}
\occasion{}

\defbeamertemplate{title page}{banner}{%
  \nointerlineskip
  \begin{adjustbox}{width=\paperwidth,center}%
    \usebeamertemplate{title page content}%
  \end{adjustbox}%
}

% need "text badly ragged" option for correct space skips
% see http://tex.stackexchange.com/a/132748/8956
\defbeamertemplate{title page content}{hexagon}{%
  \begin{tikzpicture}
  \useasboundingbox (0,0) rectangle (\paperwidth,\paperheight);
  \path[draw=dcolor,fill=fcolor,opacity=0.8]
      (0,0) rectangle (\paperwidth,\paperheight);
  \node[text width=0.86\paperwidth,text badly ragged,inner ysep=1.5cm] (main) at (0.5\paperwidth,0.55\paperheight) {%
    \begin{minipage}[c]{0.86\paperwidth}
      \centering
      \usebeamerfont{title}\usebeamercolor[fg]{title}\inserttitle
      \ifx\insertsubtitle\@empty\else
        \\[5pt]\usebeamerfont{subtitle}\usebeamercolor[fg]{subtitle}
        \insertsubtitle
      \fi
    \end{minipage}
  };
  \node[rectangle,inner sep=0pt,minimum size=3mm,fill=dcolor,right] (a) at (0,0.55\paperheight) {};
  \node[rectangle,inner sep=0pt,minimum size=3mm,fill=dcolor,left] (b) at (\paperwidth,0.55\paperheight) {};
  \ifx\insertoccasion\@empty
      \draw[thick,dcolor] (a.north east) -- (main.north west)
                   -- (main.north east) -- (b.north west);
  \else
      \node[text badly ragged] (occasion) at (main.north west -| 0.5\paperwidth,\paperheight) {
          \usebeamerfont{occasion}\usebeamercolor[fg]{occasion}\insertoccasion
      };
      \draw[thick,dcolor] (a.north east) -- (main.north west) -- (occasion.west)
                          (b.north west) -- (main.north east) -- (occasion.east);
  \fi
  \node[text badly ragged] (date) at (main.south west -| 0.5\paperwidth,0) {
      \usebeamerfont{date}\usebeamercolor[fg]{date}\insertdate
  };
  \draw[thick,dcolor] (a.south east) -- (main.south west) -- (date.west)
                      (b.south west) -- (main.south east) -- (date.east);
  \node[below=4mm,text width=0.9\paperwidth,inner xsep=0.05\paperwidth,
        text badly ragged,fill=white] at (date.south) {%
      \begin{minipage}[c]{0.9\paperwidth}
          \centering
          \textcolor{brown75}{$\blacksquare$}\hspace{0.2em}%
          \usebeamerfont{institute}\usebeamercolor[fg]{institute}\insertinstitute
          \hspace{0.4em}\textcolor{brown75}{$\blacksquare$}\hspace{0.2em}%
          \usebeamerfont{author}\usebeamercolor[fg]{author}\insertauthor
      \end{minipage}
  };
  \end{tikzpicture}
}

\defbeamertemplate{title page content}{rectangle}{%
  \begin{tikzpicture}
  \useasboundingbox (0,0) rectangle (\paperwidth,\paperheight);
  \path[draw=dcolor,fill=fcolor,opacity=0.8]
      (0,0.25\paperheight) rectangle (\paperwidth,0.85\paperheight);
  \path[draw=dcolor,very thick]
    %%(0.0075\paperwidth,0.26\paperheight) rectangle (0.9925\paperwidth,0.84\paperheight);
      (0.0375\paperwidth,0.26\paperheight) -- (0.9625\paperwidth,0.26\paperheight)
         -- ++(0,0.02\paperheight) -- ++(0.03\paperwidth,0)
         -- ++(0,-0.02\paperheight) -- ++(-0.015\paperwidth,0)
         -- ++(0,0.04\paperheight) -- ++(0.015\paperwidth,0)
      -- (0.9925\paperwidth,0.8\paperheight)
         -- ++(-0.015\paperwidth,0) -- ++(0,0.04\paperheight)
         -- ++(0.015\paperwidth,0) -- ++(0,-0.02\paperheight)
         -- ++(-0.03\paperwidth,0) -- ++(0,0.02\paperheight)
      -- (0.0375\paperwidth,0.84\paperheight)
         -- ++(0,-0.02\paperheight) -- ++(-0.03\paperwidth,0)
         -- ++(0,0.02\paperheight) -- ++(0.015\paperwidth,0)
         -- ++(0,-0.04\paperheight) -- ++(-0.015\paperwidth,0)
      -- (0.0075\paperwidth,0.3\paperheight)
         -- ++(0.015\paperwidth,0) -- ++(0,-0.04\paperheight)
         -- ++(-0.015\paperwidth,0) -- ++(0,0.02\paperheight)
         -- ++(0.03\paperwidth,0) -- ++(0,-0.02\paperheight)
      -- cycle;
  \node[text width=0.9\paperwidth,text badly ragged] at (0.5\paperwidth,0.55\paperheight) {%
    \begin{minipage}[c][0.58\paperheight]{0.9\paperwidth}
      \centering
      \usebeamerfont{title}\usebeamercolor[fg]{title}\inserttitle
      \ifx\insertsubtitle\@empty\else
        \\[5pt]\usebeamerfont{subtitle}\usebeamercolor[fg]{subtitle}
        \insertsubtitle
      \fi
    \end{minipage}
  };
  \ifx\insertoccasion\@empty\else
    \node[text badly ragged,below,draw=dcolor,fill=white] at (0.5\paperwidth,0.84\paperheight) {%
      \usebeamerfont{occasion}\usebeamercolor[fg]{occasion}\insertoccasion
    };
  \fi
  \node[text width=0.9\paperwidth,text badly ragged,below] at (0.5\paperwidth,0.25\paperheight) {%
    \begin{minipage}[t][0.25\paperheight]{0.9\paperwidth}
      \centering
      {\color{brown75}$\blacksquare$}
      \usebeamerfont{institute}\usebeamercolor[fg]{institute}\insertinstitute
      \hfill
      {\color{brown75}$\blacksquare$}
      \usebeamerfont{author}\usebeamercolor[fg]{author}\insertauthor
      \hfill
      {\color{brown75}$\blacksquare$}
      \usebeamerfont{date}\usebeamercolor[fg]{date}%
      \the\year-\ifnum\month<10 0\fi\the\month-\ifnum\day<10 0\fi\the\day
    \end{minipage}
  };
  \end{tikzpicture}
}

%% ----------------------- section and subsection --------------------

\newcounter{my@pgf@picture@count}

\def\sectionintocskip{0.5pt plus 0.1fill}
\patchcmd{\beamer@sectionintoc}{\vskip1.5em}{\vskip\sectionintocskip}{}{}

\AtBeginSection[]{%
  \begin{frame}%[plain]
    \sectionpage
  \end{frame}%
}

\defbeamertemplate{section name}{simple}{\insertsectionnumber.}

\defbeamertemplate{section name}{chinese}[1][节]{第\CJKnumber{\insertsectionnumber}#1}

\defbeamertemplate{section page}{single}{%
  \centerline{%
    \usebeamerfont{section name}%
    \usebeamercolor[fg]{section name}%
    \usebeamertemplate{section name}%
    \hspace{0.8em}%
    \usebeamerfont{section title}%
    \usebeamercolor[fg]{section title}%
    \insertsection
  }%
}

\defbeamertemplate{section name in toc}{simple}{%
  Section \ifnum\the\beamer@tempcount<10 0\fi\inserttocsectionnumber
}

\defbeamertemplate{section name in toc}{chinese}[1][节]{%
  第\CJKnumber{\inserttocsectionnumber}#1%
}

\newcounter{my@section@from}
\newcounter{my@section@to}

\defbeamertemplate{show sections in toc}{total}{%
  \setcounter{my@section@from}{1}%
  \setcounter{my@section@to}{50}%
}

% show at most five sections
\defbeamertemplate{show sections in toc}{partial}{%
  \setcounter{my@section@from}{\value{section}}%
  \addtocounter{my@section@from}{-2}%
  \setcounter{my@section@to}{\value{section}}%
  \addtocounter{my@section@to}{2}%
  \ifnum\my@totalsectionnumber>0%
    \ifnum\value{my@section@to}>\my@totalsectionnumber
      \setcounter{my@section@to}{\my@totalsectionnumber}%
      \setcounter{my@section@from}{\value{my@section@to}}%
      \addtocounter{my@section@from}{-4}%
    \fi
  \fi
  \ifnum\value{my@section@from}<1\setcounter{my@section@from}{1}%
    \setcounter{my@section@to}{\value{my@section@from}}%
    \addtocounter{my@section@to}{4}%
  \fi
}

% reset pgfid to get correct result with \tikzmark in second run
\defbeamertemplate{section page}{fill}{%
  \usebeamertemplate{show sections in toc}%
  \setcounter{my@pgf@picture@count}{\the\pgf@picture@serial@count}%
  \tableofcontents[sectionstyle=show/shaded,subsectionstyle=hide,
                   sections={\arabic{my@section@from}-\arabic{my@section@to}}]%
  \global\pgf@picture@serial@count=\value{my@pgf@picture@count}%
  \unskip
}

\defbeamertemplate{section in toc}{fill}{%
  \noindent
  \begin{tikzpicture}
  \node[rectangle split, rectangle split horizontal, rectangle split parts=2,
        rectangle split part fill={sectcolor,bg}, draw=darkgray,
        inner xsep=0pt, inner ysep=5.5pt]
       {
         \nodepart[text width=0.255\textwidth,align=center]{text}
         \usebeamertemplate{section name in toc}
         \nodepart[text width=0.74\textwidth]{second}%
         \hspace{7pt}\inserttocsection
       };
  \end{tikzpicture}%
  \par
}

\AtBeginSubsection{%
  \begin{frame}%[plain]
    \setlength{\parskip}{0pt}%
    \offinterlineskip
    \subsectionpage
  \end{frame}%
}

\defbeamertemplate{subsection name}{simple}{%
  \insertsectionnumber.\insertsubsectionnumber
}

\defbeamertemplate{subsection page}{single}{%
  \centerline{%
    \usebeamerfont{subsection name}%
    \usebeamercolor[fg]{subsection name}%
    \usebeamertemplate{subsection name}%
    \hspace{0.8em}%
    \usebeamerfont{subsection title}%
    \usebeamercolor[fg]{subsection title}%
    \insertsubsection
  }%
}

% reset pgfid to get correct result with \tikzmark in second run
\defbeamertemplate{subsection page}{fill}{%
  \setcounter{my@pgf@picture@count}{\the\pgf@picture@serial@count}%
  \tableofcontents[sectionstyle=show/hide,subsectionstyle=show/shaded/hide]%
  \global\pgf@picture@serial@count=\value{my@pgf@picture@count}%
  \unskip
}

\defbeamertemplate{subsection in toc}{fill}{%
  \noindent
  \begin{tikzpicture}
    \node[rectangle split, rectangle split horizontal, rectangle split parts=2,
          rectangle split part fill={white,bg}, draw=darkgray,
          inner xsep=0pt, inner ysep=5.5pt]
         {
           \nodepart[text width=0.255\textwidth,align=right]{text}
           \inserttocsectionnumber.\inserttocsubsectionnumber\kern7pt%
           \nodepart[text width=0.74\textwidth]{second}%
           \hspace{7pt}\inserttocsubsection
         };
  \end{tikzpicture}%
  \par
}

% chinese sections and subsections
\defbeamertemplate{section and subsection}{chinese}[1][节]{%
  \setbeamertemplate{section name in toc}[chinese][#1]%
  \setbeamertemplate{section name}[chinese][#1]%
}

%% ---------------------- headline and footline ----------------------

\defbeamertemplate{footline left}{author}{%
  \insertshortauthor
}

\defbeamertemplate{footline center}{title}{%
  \insertshorttitle
}

\defbeamertemplate{footline right}{number}{%
  \Acrobatmenu{GoToPage}{\insertframenumber{}/\inserttotalframenumber}%
}
\defbeamertemplate{footline right}{normal}{%
  \hyperlinkframeendprev{$\vartriangle$}
  \Acrobatmenu{GoToPage}{\insertframenumber{}/\inserttotalframenumber}
  \hyperlinkframestartnext{$\triangledown$}%
}

\defbeamertemplate{footline}{simple}{%
  \hbox{%
  \begin{beamercolorbox}[wd=.2\paperwidth,ht=2.25ex,dp=1ex,left]{footline}%
    \usebeamerfont{footline}\kern\beamer@leftmargin
    \usebeamertemplate{footline left}%
  \end{beamercolorbox}%
  \begin{beamercolorbox}[wd=.6\paperwidth,ht=2.25ex,dp=1ex,center]{footline}%
    \usebeamerfont{footline}\usebeamertemplate{footline center}%
  \end{beamercolorbox}%
  \begin{beamercolorbox}[wd=.2\paperwidth,ht=2.25ex,dp=1ex,right]{footline}%
    \usebeamerfont{footline}\usebeamertemplate{footline right}%
    \kern\beamer@rightmargin
  \end{beamercolorbox}%
  }%
}

\defbeamertemplate{footline}{sectioning}{%
  % default height is 0.4pt, which is ignored by adobe reader, so we increase it by 0.2pt
  {\usebeamercolor[fg]{separator line}\hrule height 0.6pt}%
  \hbox{%
  \begin{beamercolorbox}[wd=.8\paperwidth,ht=2.25ex,dp=1ex,left]{footline}%
    \usebeamerfont{footline}\kern\beamer@leftmargin\insertshorttitle
    \ifx\insertsection\@empty\else\qquad$\vartriangleright$\qquad\insertsection\fi
    \ifx\insertsubsection\@empty\else\qquad$\vartriangleright$\qquad\insertsubsection\fi
  \end{beamercolorbox}%
  \begin{beamercolorbox}[wd=.2\paperwidth,ht=2.25ex,dp=1ex,right]{footline}%
     \usebeamerfont{footline}\usebeamertemplate{footline right}%
     \kern\beamer@rightmargin
  \end{beamercolorbox}%
  }%
}

% customize mini frames template to get a section navigation bar

\defbeamertemplate{navigation box}{current}{%
  \colorbox{accent2}{%
    \rule[-1ex]{0pt}{3.25ex}\color{white}\kern1.4pt\my@navibox\kern1.4pt%
  }%
}

\defbeamertemplate{navigation box}{other}{%
  %\colorbox{white}{%
    \rule[-1ex]{0pt}{3.25ex}\color{black}\kern1.4pt\my@navibox\kern1.4pt%
  %}%
}

\newcommand{\my@navibox@subsection}{$\blacksquare$}
\newcommand{\my@navibox@frame}{$\square$}
\let\my@navibox=\my@navibox@frame

% optional navigation box for some special frame
\newcommand{\my@navibox@frame@opt}{$\boxplus$}
\newcommand{\my@change@navibox}{\let\my@navibox=\my@navibox@frame@opt}
\newcommand{\changenavibox}{%
  \addtocontents{nav}{\protect\headcommand{\protect\my@change@navibox}}%
}

\newcommand{\my@sectionentry@show}[5]{%
  \ifnum\c@section=#1%
    \setbeamertemplate{navigation box}[current]%
  \else
    \setbeamertemplate{navigation box}[other]%
  \fi
  \begingroup
    \def\my@navibox{#1}%
    \hyperlink{Navigation#3}{\usebeamertemplate{navigation box}}%
  \endgroup
}

\newif\ifmy@hidesection

\newcommand{\my@sectionentry@hide}[5]{\my@hidesectiontrue}

\pretocmd{\beamer@setuplinks}{\renewcommand{\beamer@subsectionentry}[5]{}}{}{}
\apptocmd{\beamer@setuplinks}{\global\let\beamer@subsectionentry\mybeamer@subsectionentry}{}{}

\newcommand{\mybeamer@subsectionentry}[5]{\global\let\my@navibox=\my@navibox@subsection}

\newcommand{\my@slideentry@empty}[6]{}

\newcommand{\my@slideentry@section}[6]{%
  \ifmy@hidesection
    \my@hidesectionfalse
  \else
    \ifnum\c@section=#1%
      \setbeamertemplate{navigation box}[other]%
      \ifnum\c@subsection=#2\ifnum\c@subsectionslide=#3%
         \setbeamertemplate{navigation box}[current]%
      \fi\fi
      \beamer@link(#4){\usebeamertemplate{navigation box}}%
    \fi
  \fi
  \global\let\my@navibox=\my@navibox@frame
}

\AtEndDocument{%
   \immediate\write\@auxout{%
     \noexpand\gdef\noexpand\my@totalsectionnumber{\the\c@section}%
   }%
}

\def\my@totalsectionnumber{0}

\defbeamertemplate{footline}{navigation}{%
  % default height is 0.4pt, which is ignored by adobe reader, so we increase it by 0.2pt
  {\usebeamercolor[fg]{separator line}\hrule height 0.6pt}%
  \begin{beamercolorbox}[wd=\paperwidth,ht=2.25ex,dp=1ex]{footline}%
    \usebeamerfont{footline}%
    \kern\beamer@leftmargin
    \setlength{\fboxsep}{0pt}%
    \ifnum\my@totalsectionnumber=0%
      \insertshorttitle
    \else
      \let\sectionentry=\my@sectionentry@show
      \let\slideentry=\my@slideentry@empty
      \dohead
    \fi
    \hfill
    \let\sectionentry=\my@sectionentry@hide
    \let\slideentry=\my@slideentry@section
    \dohead
    \kern\beamer@rightmargin
  \end{beamercolorbox}%
}

%% ------------------------- frame title -----------------------------

\defbeamertemplate{frametitle}{simple}[1][]
{%
  \nointerlineskip
  \begin{beamercolorbox}[wd=\paperwidth,sep=0pt,leftskip=\beamer@leftmargin,%
                         rightskip=\beamer@rightmargin,#1]{frametitle}
    \usebeamerfont{frametitle}%
    \rule[-3.6mm]{0pt}{12mm}\insertframetitle\rule[-3.6mm]{0pt}{12mm}\par
  \end{beamercolorbox}
}

%% ------------------- block and theorem -----------------------------

\defbeamertemplate{theorem begin}{simple}
{%
  \upshape%\bfseries\inserttheoremheadfont
  {\usebeamercolor[fg]{theoremname}%
  \inserttheoremname\inserttheoremnumber
  \ifx\inserttheoremaddition\@empty\else
    \ \usebeamercolor[fg]{local structure}(\inserttheoremaddition)%
  \fi%
  %\inserttheorempunctuation
  }%
  \quad\normalfont
}
\defbeamertemplate{theorem end}{simple}{\par}

\defbeamertemplate{proof begin}{simple}
{%
  %\bfseries
  \let\@addpunct=\@gobble
  {\usebeamercolor[fg]{proofname}\insertproofname}%
  \quad\normalfont
}
\defbeamertemplate{proof end}{simple}{\par}

%% ---------------------- enumerate and itemize ----------------------

\expandafter\patchcmd\csname beamer@@tmpop@enumerate item@square\endcsname
         {height1.85ex depth.4ex}{height1.85ex depth.3ex}{}{}
\expandafter\patchcmd\csname beamer@@tmpop@enumerate subitem@square\endcsname
         {height1.85ex depth.4ex}{height1.85ex depth.3ex}{}{}
\expandafter\patchcmd\csname beamer@@tmpop@enumerate subsubitem@square\endcsname
         {height1.85ex depth.4ex}{height1.85ex depth.3ex}{}{}

%% ------------------------ select templates -------------------------

\setbeamertemplate{background canvas}[default]
\setbeamertemplate{background}[line]
\setbeamertemplate{footline}[navigation]
\setbeamertemplate{footline left}[author]
\setbeamertemplate{footline center}[title]
\setbeamertemplate{footline right}[number]
\setbeamertemplate{title page}[banner]
\setbeamertemplate{title page content}[hexagon]
\setbeamertemplate{section page}[fill]
\setbeamertemplate{show sections in toc}[partial]
\setbeamertemplate{section name}[simple]
\setbeamertemplate{section name in toc}[simple]
\setbeamertemplate{section in toc}[fill]
\setbeamertemplate{section in toc shaded}[default][50]
\setbeamertemplate{subsection page}[fill]
\setbeamertemplate{subsection name}[simple]
\setbeamertemplate{subsection in toc}[fill]
\setbeamertemplate{subsection in toc shaded}[default][50]
\setbeamertemplate{theorem begin}[default]
\setbeamertemplate{theorem end}[default]
\setbeamertemplate{proof begin}[default]
\setbeamertemplate{proof end}[default]
\setbeamertemplate{frametitle}[simple]
\setbeamertemplate{navigation symbols}{}
\setbeamertemplate{itemize items}[square]
\setbeamertemplate{enumerate items}[square]

%% --------------------------- font theme ----------------------------

\setbeamerfont{title}{size=\LARGE}
\setbeamerfont{subtitle}{size=\large}
\setbeamerfont{author}{size=\normalsize}
\setbeamerfont{institute}{size=\normalsize}
\setbeamerfont{date}{size=\normalsize}
\setbeamerfont{occasion}{size=\normalsize}
\setbeamerfont{section in toc}{size=\large}
\setbeamerfont{subsection in toc}{size=\large}
\setbeamerfont{frametitle}{size=\large}
\setbeamerfont{block title}{size=\normalsize}
\setbeamerfont{item projected}{size=\footnotesize}
\setbeamerfont{subitem projected}{size=\scriptsize}
\setbeamerfont{subsubitem projected}{size=\tiny}

\usefonttheme{professionalfonts}
%\usepackage{arev}

%% ---------------------------- color theme --------------------------

% always use rgb colors in pdf files
\substitutecolormodel{hsb}{rgb}

\definecolor{red99}{Hsb}{0,0.9,0.9}
\definecolor{brown74}{Hsb}{30,0.7,0.4}
\definecolor{brown75}{Hsb}{30,0.7,0.5}
\definecolor{yellow86}{Hsb}{60,0.8,0.6}
\definecolor{yellow99}{Hsb}{60,0.9,0.9}
\definecolor{cyan95}{Hsb}{180,0.9,0.5}
\definecolor{blue67}{Hsb}{240,0.6,0.7}
\definecolor{blue74}{Hsb}{240,0.7,0.4}
\definecolor{blue77}{Hsb}{240,0.7,0.7}
\definecolor{blue99}{Hsb}{240,0.9,0.9}
\definecolor{magenta88}{Hsb}{300,0.8,0.8}

\colorlet{text1}{black}
\colorlet{back1}{white}
\colorlet{accent1}{blue99}
\colorlet{accent2}{cyan95}
\colorlet{accent3}{red99}
\colorlet{accent4}{yellow86}
\colorlet{accent5}{magenta88}
\colorlet{filler1}{accent1!40!back1}
\colorlet{filler2}{accent2!40!back1}
\colorlet{filler3}{accent3!40!back1}
\colorlet{filler4}{accent4!40!back1}
\colorlet{filler5}{accent5!40!back1}
\colorlet{gray1}{black!20}
\colorlet{gray2}{black!35}
\colorlet{gray3}{black!50}
\colorlet{gray4}{black!65}
\colorlet{gray5}{black!80}
\colorlet{tcolor}{text1!10!back1}
\colorlet{dcolor}{white}
\colorlet{fcolor}{blue77}
\colorlet{markcolor}{gray}
\colorlet{sectcolor}{brown74}

\setbeamercolor{normal text}{bg=white,fg=black}
\setbeamercolor{structure}{fg=blue99}
\setbeamercolor{local structure}{fg=cyan95}
\setbeamercolor{footline}{bg=,fg=black}
\setbeamercolor{title}{fg=yellow99}
\setbeamercolor{subtitle}{fg=white}
\setbeamercolor{author}{fg=black}
\setbeamercolor{institute}{fg=black}
\setbeamercolor{date}{fg=white}
\setbeamercolor{occasion}{fg=white}
\setbeamercolor{section name}{fg=brown75}
\setbeamercolor{section in toc}{fg=yellow99,bg=blue67}
\setbeamercolor{section in toc shaded}{fg=white,bg=blue74}
\setbeamercolor{subsection name}{parent=section name}
\setbeamercolor{subsection in toc}{use={structure,normal text},fg=structure.fg!90!normal text.bg}
\setbeamercolor{subsection in toc shaded}{parent=normal text}
\setbeamercolor{frametitle}{parent=structure}
\setbeamercolor{separator line}{fg=accent2}
\setbeamercolor{theoremname}{parent=subsection in toc}
\setbeamercolor{proofname}{parent=subsection in toc}
\setbeamercolor{block title}{fg=accent1,bg=gray}
\setbeamercolor{block body}{bg=lightgray}
\setbeamercolor{block title example}{fg=accent2,bg=gray}
\setbeamercolor{block body example}{bg=lightgray}
\setbeamercolor{block title alerted}{fg=accent3,bg=gray}
\setbeamercolor{block body alerted}{bg=lightgray}

%% ----------------------- handout mode ------------------------------

\mode<handout>{
  \setbeamertemplate{background canvas}{}
  \setbeamertemplate{background}[empty]
  \setbeamertemplate{footline}[sectioning]
  \setbeamertemplate{section page}[single]
  \setbeamertemplate{subsection page}[single]
  \setbeamerfont{subsection in toc}{size=\large}
  \colorlet{dcolor}{darkgray}
  \colorlet{fcolor}{white}
  \colorlet{sectcolor}{white}
  \setbeamercolor{normal text}{fg=black, bg=white}
  \setbeamercolor{title}{fg=blue}
  \setbeamercolor{subtitle}{fg=gray}
  \setbeamercolor{occasion}{fg=black}
  \setbeamercolor{date}{fg=black}
  \setbeamercolor{section in toc}{fg=blue!90!gray,bg=}
  \setbeamercolor{section in toc shaded}{fg=lightgray,bg=}
  \setbeamercolor{subsection in toc}{fg=blue!80!gray}
  \setbeamercolor{subsection in toc shaded}{fg=lightgray}
  \setbeamercolor{frametitle}{fg=blue!70!gray,bg=}
  \setbeamercolor{theoremname}{fg=blue!60!gray}
  \setbeamercolor{proofname}{fg=blue!60!gray}
  \setbeamercolor{footline}{bg=white,fg=black}
}

\mode
<all>

\makeatother

% -*- coding: utf-8 -*-

% ----------------------------------------------
% 中文显示相关代码
% ----------------------------------------------

% 以前要放在 usetheme 后面,否则报错;但是现在没问题了
\PassOptionsToPackage{CJKnumber}{xeCJK}
\usepackage[UTF8,noindent]{ctex}
%\usepackage[UTF8,indent]{ctexcap}

% 开明式标点:句末点号用全角,其他用半角。
%\punctstyle{kaiming}

% 在旧版本 xecjk 中用 CJKnumber 选项会自动载入 CJKnumb 包
% 但在新版本 xecjk 中 CJKnumber 选项已经被废弃,需要在后面自行载入它
\usepackage{CJKnumb}

%\CTEXoptions[today=big] % 数字年份前会有多余空白,中文年份前是正常的

\makeatletter
\ifxetex
  \setCJKsansfont{SimHei} % fix for ctex 2.0
  \renewcommand\CJKfamilydefault{\CJKsfdefault}%
\else
  \@ifpackagelater{ctex}{2014/03/01}{}{\AtBeginDocument{\heiti}} %无效?
\fi
\makeatother

%% 在旧版本 ctex 中,\today 命令生成的中文日期前面有多余空格
\makeatletter
\@ifpackagelater{ctex}{2014/03/01}{}{%
  \renewcommand{\today}{\number\year 年 \number\month 月 \number\day 日}
}
\makeatother

%% 在 xeCJK 中,默认将一些字符排除在 CJK 类别之外,需要时可以加入进来
%% 可以在 “附件->系统工具->字符映射表”中查看某字体包含哪些字符
% https://en.wikipedia.org/wiki/Number_Forms
% Ⅰ、Ⅱ、Ⅲ、Ⅳ、Ⅴ、Ⅵ、Ⅶ、Ⅷ、Ⅸ、Ⅹ、Ⅺ、Ⅻ
\xeCJKsetcharclass{"2150}{"218F}{1} % 斜线分数,全角罗马数字等
% https://en.wikipedia.org/wiki/Enclosed_Alphanumerics
\xeCJKsetcharclass{"2460}{"24FF}{1} % 带圈数字字母,括号数字字母,带点数字等

\ifxetex
% 在标点后,xeCJK 会自动添加空格,但不会去掉换行空格
%\catcode`,=\active  \def,{\textup{,} \ignorespaces}
%\catcode`;=\active  \def;{\textup{;} \ignorespaces}
%\catcode`:=\active  \def:{\textup{:} \ignorespaces}
%\catcode`。=\active  \def。{\textup{.} \ignorespaces}
%\catcode`.=\active  \def.{\textup{.} \ignorespaces}
\catcode`。=\active   \def。{.}
% 在公式中使用中文逗号和分号
%\let\douhao, \def\zhdouhao{\text{,\hskip-0.5em}}
%\let\fenhao; \def\zhfenhao{\text{;\hskip-0.5em}}
%\begingroup
%\catcode`\,=\active \protected\gdef,{\text{,\hskip-0.5em}}
%\catcode`\;=\active \protected\gdef;{\text{;\hskip-0.5em}}
% 似乎 beamer 的 \onslide<1,3> 不受影响
% 但是如果 tikz 图形中包含逗号,可能无法编译
%\catcode`\,=\active
%\protected\gdef,{\ifmmode\expandafter\zhdouhao\else\expandafter\douhao\fi}
%\catcode`\;=\active
%\protected\gdef;{\ifmmode\expandafter\zhfenhao\else\expandafter\fenhao\fi}
%\endgroup
%\AtBeginDocument{\catcode`\,=\active \catcode`\;=\active}
% 这样写反而无效
%\def\zhpunct{\catcode`\,=\active \catcode`\;=\active}
%\AtBeginDocument{%
%  \everymath\expandafter{\the\everymath\zhpunct}%
%  \everydisplay\expandafter{\the\everydisplay\zhpunct}%
%}
% 改为使用 kerkis 字体的逗号
\DeclareSymbolFont{myletters}{OML}{mak}{m}{it}
\SetSymbolFont{myletters}{bold}{OML}{mak}{b}{it}
\AtBeginDocument{%
  \DeclareMathSymbol{,}{\mathpunct}{myletters}{"3B}%
}
\fi

% 汉字下面加点表示强调
\usepackage{CJKfntef}

% ----------------------------------------------
% 字体选用相关代码
% ----------------------------------------------

% 虽然 arevtext 字体的宽度较大,但考虑到文档的美观还是同时使用 arevtext 和 arevmath
% 若在 ctex 包之前载入它,其设定的 arevtext 字体会在载入 ctex 时被重置为 lm 字体
% 因此我们在 ctex 宏包之后才载入它,但此时字体编码被改为 T1,需要修正 \nobreakspace
\usepackage{arev}
\DeclareTextCommandDefault{\nobreakspace}{\leavevmode\nobreak\ }

% 即使只需要 arevmath,也不能直接用 \usepackage{arevmath},
% 因为旧版本 fontspec 有问题,这样会导致它错误地修改数学字体

% 旧版本的 XeTeX 无法使用 arev sans 等 T1 编码字体的单独重音字符
% 因此我们恢复使用组合重音字符,见 t1enc.def, fntguide.pdf 和 encguide.pdf
\ifxetex\ifdim\the\XeTeXversion\XeTeXrevision pt<0.9999pt
  \DeclareTextAccent{\'}{T1}{1}
\fi\fi
% 在 T1enc.def 文件中定义了 T1 编码中的重音字符
% 先用 \DeclareTextAccent{\'}{T1}{1} 表示在 T1 编码中 \'e 等于 \accent"01 e
% 再用 \DeclareTextComposite{\'}{T1}{e}{233} 表示在 T1 编码中 \'e 等于 \char"E9

% ----------------------------------------------
% 版式定制相关代码
% ----------------------------------------------

\usepackage{hyperref}
\hypersetup{
  %pdfpagemode={FullScreen},
  bookmarksnumbered=true,
  unicode=true
}

%% 保证在新旧 ctex 宏包下编译得到相同的结果
\renewcommand{\baselinestretch}{1.3} % ctex 2.4.1 开始为 1,之前为 1.3

%% LaTeX 中 默认 \parskip=0pt plus 1pt,而 Beamer 中默认 \parskip=0pt

%% \parskip 用 plus 1fil 没有效果,用 plus 1fill 则节标题错位
\setlength{\parskip}{5pt plus 1pt minus 1pt}       % 段间距为 5pt + 1pt - 1pt
%\setlength{\baselineskip}{19pt plus 1pt minus 1pt} % 行间距为 5pt + 1pt - 1pt
\setlength{\lineskiplimit}{4pt}                    % 行间距小于 4pt 时重新设置
\setlength{\lineskip}{4pt}                         % 行间距太小时自动改为 4pt

\AtBeginDocument{
  \setlength{\baselineskip}{19pt plus 1pt minus 1pt} % 似乎不能放在导言区中
  \setlength{\abovedisplayskip}{4pt minus 2pt}
  \setlength{\belowdisplayskip}{4pt minus 2pt}
  \setlength{\abovedisplayshortskip}{2pt}
  \setlength{\belowdisplayshortskip}{2pt}
}

% 默认是 \raggedright,改为两边对齐
\usepackage{ragged2e}
\justifying
\let\oldraggedright\raggedright
\let\raggedright\justifying

% ----------------------------------------------
% 文本环境相关代码
% ----------------------------------------------

\setlength{\fboxsep}{0.02\textwidth}\setlength{\fboxrule}{0.002\textwidth}

\usepackage{adjustbox}
\newcommand{\mylinebox}[1]{\adjustbox{width=\linewidth}{#1}}

\usepackage{comment}
\usepackage{multicol}

% 带圈的数字
%\newcommand{\digitcircled}[1]{\textcircled{\raisebox{.8pt}{\small #1}}}
\newcommand{\digitcircled}[1]{%
  \tikz[baseline=(char.base)]{%
     \node[shape=circle,draw,inner sep=0.01em,line width=0.07em] (char) {\small #1};
  }%
}

\usepackage{pifont}
% 因为 xypic 将 \1 定义为 frm[o]{--},这里改为在文档内部定义
%\def\1{\ding{51}} % 勾
%\def\0{\ding{55}} % 叉

% 若在 enumerate 中使用自定义模板,则各项前的间距由第七项决定
% 对于我们使用的 arev 数学字体来说各个数字是等宽的,所以没问题
% 参考 https://tex.stackexchange.com/q/377959/8956
% 以及 https://chat.stackexchange.com/transcript/message/38541073#38541073
\newenvironment{enumskip}[1][]{%
  \setbeamertemplate{enumerate mini template}[default]%
  \if\relax\detokenize{#1}\relax % empty
    \begin{enumerate}[\quad(1)]
  \else
    \begin{enumerate}[#1][\quad(1)]
  \fi
}{\end{enumerate}}
\newenvironment{enumzero}[1][]{%
  \setbeamertemplate{enumerate mini template}[default]%
  \if\relax\detokenize{#1}\relax % empty
    \begin{enumerate}[(1)\,]
  \else
    \begin{enumerate}[#1][(1)\,]
  \fi
}{\end{enumerate}}
%
\newenvironment{enumlite}[1][]{%
  \setbeamertemplate{enumerate mini template}[default]%
  \setbeamercolor{enumerate item}{fg=,bg=}%
  \if\relax\detokenize{#1}\relax % empty
    \begin{enumerate}[(1)\,]%
  \else
    \begin{enumerate}[#1][(1)\,]%
  \fi
}{\end{enumerate}}
%
\newcounter{mylistcnt}
%
\newenvironment{enumhalf}{%
  \par\setcounter{mylistcnt}{0}%
  \def\item##1~{%
    \leavevmode\ifhmode\unskip\fi\linebreak[2]%
    \makebox[.5002\textwidth][l]{\stepcounter{mylistcnt}(\arabic{mylistcnt}) \,##1\ignorespaces}%
  }%
  \ignorespaces%
}{\par}
%
\newenvironment{choiceline}[1][]{%
  \par\vskip-0.5em\relax
  \setbeamertemplate{enumerate mini template}[default]%
  \setbeamercolor{enumerate item}{fg=,bg=}%
  \if\relax\detokenize{#1}\relax % empty
    \begin{enumerate}[(A)\,]
  \else
    \begin{enumerate}[#1][(A)\,]
  \fi
}{\end{enumerate}}
%
\newenvironment{choicehalf}{%
  \par\setcounter{mylistcnt}{0}%
  \def\item##1~{%
    \leavevmode\ifhmode\unskip\fi\linebreak[2]%
    \makebox[.5001\textwidth][l]{\stepcounter{mylistcnt}(\Alph{mylistcnt}) \,##1\ignorespaces}%
  }%
  \ignorespaces%
}{\par}
\newenvironment{choicequar}{%
  \par\setcounter{mylistcnt}{0}%
  \def\item##1~{%
    \leavevmode\ifhmode\unskip\fi\linebreak[0]%
    \makebox[.2501\textwidth][l]{\stepcounter{mylistcnt}(\Alph{mylistcnt}) \,##1\ignorespaces}%
  }%
  \ignorespaces%
}{\par}

% ----------------------------------------------
% 定理环境相关代码
% ----------------------------------------------

\makeatletter
\patchcmd{\@thm}{ \csname}{\kern0.18em\relax\csname}{}{}
\makeatother

\newcommand{\mynewtheorem}[2]{%
  \newtheorem{#1}{#2}[section]%
  \expandafter\renewcommand\csname the#1\endcsname{\arabic{#1}}%
}

\mynewtheorem{theorem}{定理}
\newtheorem*{theorem*}{定理}

\mynewtheorem{algorithm}{算法}
\newtheorem*{algorithm*}{算法}

\mynewtheorem{conjecture}{猜想}
\newtheorem*{conjecture*}{猜想}

\mynewtheorem{corollary}{推论}
\newtheorem*{corollary*}{推论}

\mynewtheorem{definition}{定义}
\newtheorem*{definition*}{定义}

\mynewtheorem{example}{例}
\newtheorem*{example*}{例子}

\mynewtheorem{exercise}{练习}
\newtheorem*{exercise*}{练习}

\mynewtheorem{fact}{事实}
\newtheorem*{fact*}{事实}

\mynewtheorem{guess}{猜测}
\newtheorem*{guess*}{猜测}

\mynewtheorem{lemma}{引理}
\newtheorem*{lemma*}{引理}

\mynewtheorem{method}{解法}
\newtheorem*{method*}{解法}

\mynewtheorem{origin}{引例}
\newtheorem*{origin*}{引例}

\mynewtheorem{problem}{问题}
\newtheorem*{problem*}{问题}

\mynewtheorem{property}{性质}
\newtheorem*{property*}{性质}

\mynewtheorem{proposition}{命题}
\newtheorem*{proposition*}{命题}

\mynewtheorem{puzzle}{题}
\newtheorem*{puzzle*}{题目}

\mynewtheorem{remark}{注记}
\newtheorem*{remark*}{注记}

\mynewtheorem{review}{复习}
\newtheorem*{review*}{复习}

\mynewtheorem{result}{结论}
\newtheorem*{result*}{结论}

\newtheorem*{analysis}{分析}
\newtheorem*{answer}{答案}
\newtheorem*{choice}{选择}
\newtheorem*{hint}{提示}
\newtheorem*{solution}{解答}
\newtheorem*{thinking}{思考}

\newcommand{\mynewtheoremx}[2]{%
  \newtheorem{#1}{#2}%
  \expandafter\renewcommand\csname the#1\endcsname{\arabic{#1}}%
}
\mynewtheoremx{bonus}{选做}
\newtheorem*{bonus*}{选做}

\renewcommand{\proofname}{证明}
\renewcommand{\qedsymbol}{}
\renewcommand{\tablename}{表格}

% ----------------------------------------------
% 数学环境相关代码
% ----------------------------------------------

% 选学内容的角标星号
\newcommand{\optstar}{\texorpdfstring{\kern0pt$^\ast$}{}}

\usepackage{mathtools} % \mathclap 命令
\usepackage{extarrows}

% 切换 amsmath 的公式编号位置
% 不使用 leqno 选项而在这里才修改,会导致编号与公式重叠
% 因此在 \documentclass 里都加上了 leqno 选项
\makeatletter
\newcommand{\leqnomath}{\tagsleft@true}
\newcommand{\reqnomath}{\tagsleft@false}
\makeatother
%\leqnomath

% 定义带圈数字的 tag 格式,需要 mathtools 包
\newtagform{circ}[\color{accent2}\digitcircled]{}{}
\newtagform{skip}[\quad\color{accent2}\digitcircled]{}{}

\newcounter{savedequation}

\newenvironment{aligncirc}{%
  \setcounter{savedequation}{\value{equation}}%
  \setcounter{equation}{0}%
  \usetagform{circ}%
  \align
}{
  \endalign
  \setcounter{equation}{\value{savedequation}}%
}
\newenvironment{alignskip}{%
  \setcounter{savedequation}{\value{equation}}%
  \setcounter{equation}{0}%
  \usetagform{skip}%
  \align
}{
  \endalign
  \setcounter{equation}{\value{savedequation}}%
}
\newenvironment{alignlite}{%
  \setcounter{savedequation}{\value{equation}}%
  \setcounter{equation}{0}%
  \align
}{
  \endalign
  \setcounter{equation}{\value{savedequation}}%
}

% cases 环境开始时 \def\arraystretch{1.2}
% 在中文文档中还有 \linespread{1.3},这样公式的左花括号就太大了
% 这里利用 etoolbox 包将 \linespread 临时改回 1
\AtBeginEnvironment{cases}{\linespread{1}\selectfont}
% 奇怪的是在最新的 miktex 中无此问题,
% 而且即使这样修改,在新旧 miktex 中用 arev 字体时花括号大小还是有差别
% 而用默认的 lm 字体时花括号却没有差别

% 用于给带括号的方程组编号
\usepackage{cases}

\newcommand{\R}{\mathbb{R}}
\newcommand{\Rn}{\mathbb{R}^n}

% 大型的积分号
\usepackage{relsize}
\newcommand{\Int}{\mathop{\mathlarger{\int}}}

% \oiint 命令
% \usepackage[integrals]{wasysym}

% http://tex.stackexchange.com/q/84302
\DeclareMathOperator{\arccot}{arccot}

% https://tex.stackexchange.com/a/178948/8956
% 保证 \d x 和 \d(2x) 和 \d^2 x 的间距都合适
\let\oldd=\d
\renewcommand{\d}{\mathop{}\!\mathrm{d}}
\newcommand{\dx}{\d x}
\newcommand{\dy}{\d y}
\def\dz{\d z} % 不确定命令是否已经定义
\newcommand{\du}{\d u}
\newcommand{\dv}{\d v}
\newcommand{\dr}{\d r}
\newcommand{\ds}{\d s}
\newcommand{\dt}{\d t}
\newcommand{\dS}{\d S}

\newcommand{\e}{\mathrm{e}}
\newcommand{\limit}{\lim\limits}

% 分数线长一点的分数,\wfrac[2pt]{x}{y} 表示左右加 2pt
% 参考 http://tex.stackexchange.com/a/21580/8956
\DeclareRobustCommand{\wfrac}[3][2pt]{%
  {\begingroup\hspace{#1}#2\hspace{#1}\endgroup\over\hspace{#1}#3\hspace{#1}}}

% 划去部分公式
% 横着划线,参考 http://tex.stackexchange.com/a/20613/8956
\newcommand{\hcancel}[2][accent3]{%
  \setbox0=\hbox{$#2$}%
  \rlap{\raisebox{.3\ht0}{\textcolor{#1}{\rule{\wd0}{1pt}}}}#2%
}
% 斜着划线,参考 https://tex.stackexchange.com/a/15958
\newcommand{\dcancel}[2][accent3]{%
    \tikz[baseline=(tocancel.base),ultra thick]{
        \node[inner sep=0pt,outer sep=0pt] (tocancel) {$#2$};
        \draw[#1] (tocancel.south west) -- (tocancel.north east);
    }%
}%

% 竖直居中的 \dotfill
\newcommand\cdotfill{\leavevmode\xleaders\hbox to 0.5em{\hss\footnotesize$\cdot$\hss}\hfill\kern0pt\relax}

% 保持居中的 \not 命令
\usepackage{centernot}

% 使用 stix font 中的 white arrows
\ifxetex
    \IfFileExists{STIX-Regular.otf}{%
        \newfontfamily{\mystix}{STIX} % stix v1.1
    }{%
        \newfontfamily{\mystix}{STIXGeneral} % stix v1.0
    }
    \DeclareRobustCommand\leftwhitearrow{%
      \mathrel{\text{\normalfont\mystix\symbol{"21E6}}}%
    }
    \DeclareRobustCommand\upwhitearrow{%
      \mathrel{\text{\normalfont\mystix\symbol{"21E7}}}%
    }
    \DeclareRobustCommand\rightwhitearrow{%
      \mathrel{\text{\normalfont\mystix\symbol{"21E8}}}%
    }
    \DeclareRobustCommand\downwhitearrow{%
      \mathrel{\text{\normalfont\mystix\symbol{"21E9}}}%
    }
\else
    \let \leftwhitearrow = \Leftarrow
    \let \rightwhitearrow = \Rightarrow
    \let \upwhitearrow = \Uparrow
    \let \downwhitearrow = \Downarrow
\fi

% ----------------------------------------------
% 绘图动画相关代码
% ----------------------------------------------

% pgf/tikz 的所有选项都称为 key,它们按照 unix 路径的方式组织,
% 例如:/tikz/external/force remake={boolean}
% 这些 key 可以用 \pgfkeys 定义,用 \tikzset 设置
% \tikzset 实际上等同于 \pgfkeys{/tikz/.cd,#1}.
% Using Graphic Options: P120 in manual 2.10 (\tikzset)
% Key Management: P481 in manual 2.10 (\pgfkeys)

\usepackage{tikz}
\usepackage{pgfplots}
%\usepackage{calc}

% 文档标注,通常需要编译两次就可以得到正确结果
% 但如果主题的 section page 用 tikz 绘图,将需要编译三次
% 这是因为 tikzmark 依赖 aux 文件的 pgfid 编号
% 第一次编译时缺少 toc 文件,将缺少若干个 tikz 图片
% 第二次编译时图片个数正确了,但是 aux 文件的 pgfid 仍然是错误的
% 这个问题在主题文件中已经修正了
\newcommand\tikzmark[1]{%
  \tikz[overlay,remember picture] \node[coordinate] (#1) {};%
}

% pgf 包含的  xxcolor 包存在问题,导致与 xeCJKfntef 包冲突
% 见 https://github.com/CTeX-org/ctex-kit/issues/323
% 注意此冲突在 ctex 2.9 中不存在,仅在最新的 miktex 2.9 中出现
\makeatletter
\g@addto@macro\XC@mcolor{\edef\current@color{\current@color}}
\makeatother

\usetikzlibrary{matrix} % 用于在 node 四周加括号
\usetikzlibrary{decorations}
\usetikzlibrary{decorations.markings} % 用于在箭头上作标记
\usetikzlibrary{intersections} % 用于计算路径的交点
\usetikzlibrary{positioning} % 可以更方便地定位
\usetikzlibrary{shapes.geometric} % 钻石形状节点

\usetikzlibrary{calc}
\usetikzlibrary{snakes}

% Externalizing Graphics: P343 and P651 in manual 2.10
\usetikzlibrary{external}
% 编译时需加上 --shell-escape(texlive)或 -enable-write18(miktex)选项
%\tikzexternalize[prefix=figure/] %\tikzexternalize[shell escape=-enable-write18]

% 默认 tikz 图片会用 pdflatex 编译,可以自己改为 xelatex
%\tikzset{external/system call={%
%  xelatex \tikzexternalcheckshellescape -halt-on-error -interaction=batchmode -jobname "\image" "\texsource"}}

% 强制重新生成图片, pgf 3.0 中会自动比较文件的 md5
%\tikzset{external/force remake} %\tikzset{external/remake next}

%\tikzset{draw=black,color=black}
%\mode<beamer>{\tikzset{every path/.style={color=white!90!black}}}

\usetikzlibrary{patterns}

% hack pgf prior to version 3.0 for pgf patterns in xetex
% code taken from pgfsys-dvipdfmx.def and pgfsys-xetex.def in pgf 3.0
\makeatletter
\def\myhackpgf{
  % fix typo in pgfsys-common-pdf-via-dvi.def in pgf 2.10
  \pgfutil@insertatbegineverypage{%
     \ifpgf@sys@pdf@any@resources%
        \special{pdf:put @resources
           << \ifpgf@sys@pdf@patterns@exists /Pattern @pgfpatterns \fi >>}%
     \fi%
  }
  % required to give colors on pattern objects.
  \pgfutil@addpdfresource@colorspaces{ /pgfprgb [/Pattern /DeviceRGB] }
  % hook for xdvipdfmx
  \def\pgfsys@dvipdfmx@patternobj##1{%
	 \pgfutil@insertatbegincurrentpagefrombox{##1}%
  }%
  % dvipdfmx provides a new special `pdf:stream' for a stream object
  \def\pgfsys@dvipdfmx@stream##1##2##3{%
     \special{pdf:stream ##1 (##2) << ##3 >>}%
  }%
  % declare patterns and set patterns
  \def\pgfsys@declarepattern##1##2##3##4##5##6##7##8##9{%
     \pgf@xa=##2\relax \pgf@ya=##3\relax%
     \pgf@xb=##4\relax \pgf@yb=##5\relax%
     \pgf@xc=##6\relax \pgf@yc=##7\relax%
     \pgf@sys@bp@correct\pgf@xa \pgf@sys@bp@correct\pgf@ya%
     \pgf@sys@bp@correct\pgf@xb \pgf@sys@bp@correct\pgf@yb%
     \pgf@sys@bp@correct\pgf@xc \pgf@sys@bp@correct\pgf@yc%
     \pgfsys@dvipdfmx@patternobj{%
        \pgfsys@dvipdfmx@stream{@pgfpatternobject##1}{##8}{%
           /Type /Pattern
           /PatternType 1
           /PaintType \ifnum##9=0 2 \else 1 \fi
           /TilingType 1
           /BBox [\pgf@sys@tonumber\pgf@xa\space\pgf@sys@tonumber\pgf@ya\space
                  \pgf@sys@tonumber\pgf@xb\space\pgf@sys@tonumber\pgf@yb]
           /XStep \pgf@sys@tonumber\pgf@xc\space
           /YStep \pgf@sys@tonumber\pgf@yc\space
           /Resources << >> %<<
        }%
     }%
     \pgfutil@addpdfresource@patterns{/pgfpat##1\space @pgfpatternobject##1}%
  }
  \def\pgfsys@setpatternuncolored##1##2##3##4{%
     \pgfsysprotocol@literal{/pgfprgb cs ##2 ##3 ##4 /pgfpat##1\space scn}%
  }
  \def\pgfsys@setpatterncolored##1{%
     \pgfsysprotocol@literal{/Pattern cs /pgfpat##1\space scn}%
  }
}
\@ifpackagelater{pgf}{2013/12/18}{}{\ifxetex\expandafter\myhackpgf\fi}%
\makeatother

% ----------------------------------------------
% 表格制作相关代码
% ----------------------------------------------

\newcommand{\narrowsep}[1][2pt]{\setlength{\arraycolsep}{#1}}
\newcommand{\narrowtab}[1][3pt]{\setlength{\tabcolsep}{#1}}

% diagbox 依赖 pict2e,但 miktex 中旧版本 pict2e 打包错误,使得引擎判别错误
% 从而导致在编译时出现大量警告,以及导致底栏右下角按钮链接错位
\ifxetex\PassOptionsToPackage{xetex}{pict2e}\fi
\usepackage{diagbox}

\usepackage{multirow} % 跨行表格

\usepackage{array} % 可以用 \extrarowheight
% 双倍宽度的横线和竖线,\arrayrulewidth 默认为 0.4pt
\setlength{\doublerulesep}{0pt}
\newcommand{\dhline}{\noalign{\global\arrayrulewidth0.8pt}\hline\noalign{\global\arrayrulewidth0.4pt}}
\newcolumntype{?}{!{\vrule width 0.8pt}} % 即使 \doublerulesep 为 0pt,|| 也不能得到双倍宽度
% 最好还是用 tabu,更简单

\usepackage{tabularx}

%\usepackage{arydshln} % 在分块矩阵中加虚线
%\setlength{\dashlinedash}{2pt} % 默认4pt
%\setlength{\dashlinegap}{2pt} % 默认4pt

% tabu 与 arydshln 会冲突,可以不使用 arydshln,
% 而用 tabu 定义虚线 \newcolumntype{:}{|[on 2pt off 2pt]}
% 参考 http://bbs.ctex.org/forum.php?mod=viewthread&tid=63944#pid405057
\usepackage{tabu}
\newcolumntype{:}{|[on 2pt off 2pt]}
\newcommand{\hdashline}{\tabucline[0.4pt on 2pt off 2pt]{-}} % 兼容 arydshln 的命令
\setlength{\tabulinesep}{4pt} % 拉开大型公式与表格横线的距离

%\usepackage{colortbl} % 否则 \taburowcolors 命令无效

% ----------------------------------------------
% 绝对定位相关代码
% ----------------------------------------------

\usepackage[absolute,overlay]{textpos}

% 将整个页面分为 32 乘 24 个边长为 4mm 的小正方形
\setlength{\TPHorizModule}{4mm}
\setlength{\TPVertModule}{4mm}

\setlength{\TPboxrulesize}{0.6pt}
\newlength{\tpmargin}
\setlength{\tpmargin}{2mm}

\newenvironment{bblock}[1][black]{%
  \begingroup
  \TPshowboxestrue\TPMargin{\tpmargin}%
  \textblockrulecolor{#1}\textblockcolour{}%
  \begin{textblock}%
}{%
  \end{textblock}%
  \endgroup
}
\newenvironment{cblock}[2][black]{%
  \begingroup
  \TPshowboxestrue\TPMargin{\tpmargin}%
  \textblockrulecolor{#1}\textblockcolour{#2}%
  \begin{textblock}%
}{%
  \end{textblock}%
  \endgroup
}
\newenvironment{cblocka}{\begin{cblock}{filler1}}{\end{cblock}}
\newenvironment{cblockb}{\begin{cblock}{filler2}}{\end{cblock}}
\newenvironment{cblockc}{\begin{cblock}{filler3}}{\end{cblock}}
\newenvironment{cblockd}{\begin{cblock}{filler4}}{\end{cblock}}
\newenvironment{cblocke}{\begin{cblock}{filler5}}{\end{cblock}}

% ----------------------------------------------
% 模版定制相关代码
% ----------------------------------------------

\usepackage{bookmark}

\newcommand{\mybookmark}[1]{%
  \bookmark[page=\thepage,level=3]{#1}%
  \changenavibox
}

%\setbeamercovered{transparent=5}

\setbeamersize{text margin left=4mm,text margin right=4mm}
\mode<beamer>{\setbeamertemplate{background}[linear]}
\setbeamertemplate{footline}[sectioning]
\setbeamertemplate{footline right}[normal]
\setbeamertemplate{theorem begin}[simple]
\setbeamertemplate{theorem end}[simple]
\setbeamertemplate{proof begin}[simple]
\setbeamertemplate{proof end}[simple]

% 段间距在 block 中也许无效 http://tex.stackexchange.com/q/6111/8956
%\addtobeamertemplate{block begin}{}{\setlength{\parskip}{6pt plus 2pt minus 2pt}}

%\mode<beamer>{\tikzset{every path/.style={color=black}}}

% 在 amsfonts.sty 中已经废弃 \bold 命令,改用 \mathbf 命令
\def\lead#1{\textcolor{accent1}{#1}}
\def\bold#1{\textcolor{accent2}{#1}}
\def\warn#1{\textcolor{accent3}{#1}}
\def\clead{\color{accent1}}
\def\cbold{\color{accent2}}
\def\cwarn{\color{accent3}}

\mode<handout>{
  \colorlet{filler1}{filler1!40!white}
  \colorlet{filler2}{filler2!40!white}
  \colorlet{filler3}{filler3!40!white}
  \colorlet{filler4}{filler4!40!white}
  \colorlet{filler5}{filler5!40!white}
  \colorlet{gray1}{gray1!60!white}
  \colorlet{gray2}{gray2!60!white}
  \colorlet{gray3}{gray3!60!white}
  \colorlet{gray4}{gray4!60!white}
  \colorlet{gray5}{gray5!60!white}
}

% 兼容性命令,在 beamer 中应该避免使用它们,而改用上面几个命令
\let\textbf=\bold \def\pmb{\usebeamercolor[fg]{local structure}}
\let\emph=\warn   \def\bm{\usebeamercolor[fg]{alerted text}}

\newcommand{\vpause}{\pause\vskip 0pt plus 0.5fill\relax}
\newcommand{\ppause}{\par\pause}

\newcommand{\mybackground}{\setbeamertemplate{background}[lattice][4mm]}
% 几个 \varxxx 命令是 arevmath 包提供的
% $\heartsuit\varheart\diamondsuit\vardiamond$
% $\varspade\spadesuit\varclub\clubsuit$
% rframe 为例题解答,sframe 为练习解答,可以选择不包含它们
\newenvironment{rframe}{\mybackground\begin{frame}}{\end{frame}}
\newenvironment{sframe}{%
  \mybackground
  \colorlet{markcolor}{accent4}%
  \backgroundmarklefttrue\backgroundmarkrighttrue
  \begin{frame}
}{\end{frame}}
\ifdefined\slide
  \setbeamertemplate{footline}[navigation]
  \renewenvironment{rframe}{\begin{frame}<beamer:0>}{\end{frame}}%
  \renewenvironment{sframe}{\begin{frame}<beamer:0>}{\end{frame}}%
\fi
\ifdefined\print
  \renewenvironment{sframe}{\begin{frame}<handout:0>}{\end{frame}}%
\fi
% 用于标示只针对内招或外招的内容:iframe 为内招,oframe 为外招
\newenvironment{iframe}{\backgroundmarklefttrue\begin{frame}}{\end{frame}}
\newenvironment{oframe}{\backgroundmarkrighttrue\begin{frame}}{\end{frame}}
\newenvironment{jframe}{\mybackground\backgroundmarklefttrue\begin{frame}}{\end{frame}}
\newenvironment{pframe}{\mybackground\backgroundmarkrighttrue\begin{frame}}{\end{frame}}
\def\myimode{i}
\def\myomode{o}
\ifx\slide\myimode
  \renewenvironment{oframe}{\begin{frame}<presentation:0>}{\end{frame}}%
  \renewenvironment{pframe}{\begin{frame}<presentation:0>}{\end{frame}}%
  \renewenvironment{jframe}{\begin{frame}<presentation:0>}{\end{frame}}%
\fi
\ifx\slide\myomode
  \renewenvironment{iframe}{\begin{frame}<presentation:0>}{\end{frame}}%
  \renewenvironment{jframe}{\begin{frame}<presentation:0>}{\end{frame}}%
  \renewenvironment{pframe}{\begin{frame}<presentation:0>}{\end{frame}}%
\fi
\ifx\print\myimode
  \renewenvironment{oframe}{\begin{frame}<presentation:0>}{\end{frame}}%
  \renewenvironment{pframe}{\begin{frame}<presentation:0>}{\end{frame}}%
\fi
\ifx\print\myomode
  \renewenvironment{iframe}{\begin{frame}<presentation:0>}{\end{frame}}%
  \renewenvironment{jframe}{\begin{frame}<presentation:0>}{\end{frame}}%
\fi

% 利用 tikzmark 作边注
\newcommand{\imark}[1][gray]{%
  \begin{tikzpicture}[overlay,remember picture]
    \node[coordinate] (A) {};
    \fill[color=#1] (current page.west |- A) rectangle +(1.2mm,0.6em);
  \end{tikzpicture}%
}
\newcommand{\omark}[1][gray]{%
  \begin{tikzpicture}[overlay,remember picture]
    \node[coordinate] (A) {};
    \fill[color=#1] (A -| current page.east) rectangle +(-1.2mm,0.6em);
  \end{tikzpicture}%
}
\newcommand{\smark}{%
  \imark[accent4]\omark[accent4]%
}
\newcommand{\itext}[1]{%
  \ifx\slide\myomode\else
    \ifx\print\myomode\else
      #1%
    \fi
  \fi
}
\newcommand{\otext}[1]{%
  \ifx\slide\myimode\else
    \ifx\print\myimode\else
      #1%
    \fi
  \fi
}
\newcommand{\stext}[1]{%
  \ifdefined\slide\else
    \ifdefined\print\else
      #1%
    \fi
  \fi
}

% 选择题的答案
\newcommand{\select}[1]{\qquad\stext{\llap{\makebox[2em]{\color{accent4}#1}}}}

%% 内外招同编号的定理,例子或练习等,需要将编号减一
\newcommand{\minusone}[1]{%
  \ifdefined\slide\else
    \ifdefined\print\else
      \addtocounter{#1}{-1}%
    \fi
  \fi
}

%\mode<beamer>{
%\def\mytoctemplate{
%  \setbeamerfont{section in toc}{size=\normalsize}
%  \setbeamerfont{subsection in toc}{size=\small}
%  \setbeamertemplate{section in toc shaded}[default][100]
%  \setbeamertemplate{subsection in toc}[subsections numbered]
%  \setbeamertemplate{subsection in toc shaded}[default][100]
%  \setbeamercolor{section in toc}{fg=structure.fg}
%  \setbeamercolor{section in toc shaded}{fg=structure.fg!50!black}
%  \setbeamercolor{subsection in toc}{fg=structure.fg}
%  \setbeamercolor{subsection in toc shaded}{fg=normal text.fg}
%  \begin{multicols}{2}
%  \tableofcontents[sectionstyle=show/shaded,subsectionstyle=show/shaded]
%  \end{multicols}
%}
%\AtBeginSection[]{\begin{frame}\frametitle{目录结构}\mytoctemplate\end{frame}}
%\AtBeginSubsection[]{\begin{frame}\frametitle{目录结构}\mytoctemplate\end{frame}}
%}

\mode<presentation>

\setbeamertemplate{section and subsection}[chinese]
\usebeamertemplate{section and subsection}

\mode
<all>

% -*- coding: utf-8 -*-

% ----------------------------------------------
% 高等数学中的定义和改动
% ----------------------------------------------

\newif\ifligong % 理工类或经济类
\ligongtrue

% Repeating Things: P504 in manual 2.10
\newcommand{\drawline}[4][]{%
  \foreach \v [remember=\v as \u,count=\i] in {#4} {
    \ifnum \i > 1
      \ifodd \i \draw[#1,#3] \u -- \v; \else \draw[#1,#2] \u -- \v; \fi
    \fi
  }
}
\newcommand{\drawplot}[5][]{%
  \foreach \v [remember=\v as \u,count=\i] in {#4} {
    \ifnum \i > 1
      \ifodd \i \draw[#1,#3] plot[domain=\u:\v] #5; \else \draw[#1,#2] plot[domain=\u:\v] #5; \fi
    \fi
  }
}

% http://tex.stackexchange.com/q/84302
\DeclareMathOperator{\Prj}{Prj}
\DeclareMathOperator{\grad}{grad}

\newcommand{\va}{\vec{a\vphantom{b}}}
\newcommand{\vb}{\vec{b}}
\newcommand{\vc}{\vec{c\vphantom{b}}}
\newcommand{\vd}{\vec{d}}
\newcommand{\ve}{\vec{e}}
\newcommand{\vi}{\vec{i}}
\newcommand{\vj}{\vec{j}}
\newcommand{\vk}{\vec{k}}
\newcommand{\vn}{\vec{n}}
\newcommand{\vs}{\vec{s}}
\newcommand{\vv}{\vec{v}}

\let\ov=\overrightarrow

% xcolor 支持 hsb 色彩模型,但 pgf 不支持,因此需要指定输出的色彩模型为 rgb
% 在 article 中可以用 \usepackage[rgb]{xcolor} \usepackage{tikz} 解决此问题
% 在 beamer 中可以用 \documentclass[xcolor={rgb}]{beamer} 解决此问题

%\definecolor{bcolor0}{Hsb}{0,0.6,0.9}   % red 红色
%\definecolor{bcolor1}{Hsb}{60,0.6,0.9}  % yellow 黄色
%\definecolor{bcolor2}{Hsb}{120,0.6,0.9} % green 绿色
%\definecolor{bcolor3}{Hsb}{180,0.6,0.9} % cyan 青色
%\definecolor{bcolor4}{Hsb}{240,0.6,0.9} % blue 蓝色
%\definecolor{bcolor5}{Hsb}{300,0.6,0.9} % magenta 洋红色

\colorlet{bcolor0}{accent3}
\colorlet{bcolor1}{accent1}
\colorlet{bcolor2}{accent2}
\colorlet{bcolor3}{accent4}
\colorlet{bcolor5}{accent5}


\begin{document}

\occasion{高等数学课程}
\title{第一章·函数与极限}
\author{\href{https://lvjr.bitbucket.io}{吕荐瑞}}
\institute{暨南大学数学系}

\begin{frame}[plain]
\titlepage
\end{frame}

\section{数集与函数}

\subsection{数集与区间}

\begin{frame}
\frametitle{数集}
人类对数的认识是逐步发展的:
\begin{itemize}[<+->]
  \item 自然数集$\mathbb{N}$
  \item 整数集$\mathbb{Z}$
  \item 有理数集$\mathbb{Q}$
  \item \bold{实数集$\mathbb{R}$}\onslide<6->{\space\lead{$\longleftarrow$\space 微积分的研究对象}}
  \item 复数集$\mathbb{C}$
\end{itemize}
\vpause
\begin{thinking}
$\bold{\mathbb{R}}$ 与$\mathbb{Z}$、$\mathbb{C}$、$\mathbb{Q}$分别有何区别?
\end{thinking}
\end{frame}

\begin{sframe}
\frametitle{区间的定义}
\begin{remark*}
我们希望下册定义的开区域(闭区域)概念是这里的开区间(闭区间)概念在
$\mathbb{R}^n$中的推广,因此
\begin{enumzero}
  \item \hspace{-0.4em}开区间分为有限开区间和无限开区间,排除空集
  \item \hspace{-0.4em}闭区间分为有限闭区间和无限闭区间,排除独点集
  \item \hspace{-0.4em}规定整个实数集$\mathbb{R}$既是开区间又是闭区间
\end{enumzero}
\end{remark*}
\end{sframe}

\begin{frame}
\frametitle{有限区间}
\vspace{-1em}
数轴上长度大于零的一段称为\bold{区间}\pause :
$\left\{\begin{array}{@{}l@{}}\text{有限区间}\\\text{无限区间}\end{array}\right.$
\ppause\vspace{0.3em}\hrule\vspace{0.2em}
\bold{有限区间}有四种($a<b$,$a$和$b$称为区间的\bold{端点}):
\begin{align*}
  (a,b)&=\{x \mid a<x<b\}          &\text{有限\bold{开区间}}\\[-0.2em]
  [a,b]&=\{x \mid a \le x \le b\}  &\text{有限\bold{闭区间}}\\[-0.2em]
  (a,b]&=\{x \mid a<x \le b\}      &\stext{\smark\text{左开右闭区间}}\\[-0.2em]
  [a,b)&=\{x \mid a \le x<b\}      &\stext{\smark\text{左闭右开区间}}
\end{align*}
\vspace{-1.5em}\pause
\begin{example*}
用区间表示下列数集:
\begin{enumhalf}
  \item $\{x \mid 1<x<3\}$ ~
  \item $\{x \mid -5 \le x < 0\}$ ~
\end{enumhalf}
\end{example*}
\end{frame}

\begin{frame}
\frametitle{无限区间}
\bold{无限区间}有五种(其中$a$或$b$称为区间的\bold{端点}):
\begin{align*}
  &(-\infty, b)=\{x \mid x<b\}     &\text{无限开区间} \\[-0.2em]
  &(a,+\infty)=\{x \mid x>a\}      &\text{无限开区间} \\[-0.2em]
  &(-\infty, b]=\{x \mid x \le b\} &\text{无限闭区间} \\[-0.2em]
  &[a,+\infty)=\{x \mid x \ge a\}  &\text{无限闭区间} \\[-0.2em]
  &(-\infty, +\infty)=\mathbb{R}   &\stext{\smark\text{既是开区间又是闭区间}}
\end{align*}
\vspace{-1.5em}\pause
\begin{example*}
用区间表示下列数集:
\begin{enumhalf}
  \item $\{x \mid x<3\}$ ~
  \item $\{x \mid x \ge 2\}$ ~
\end{enumhalf}
\end{example*}
\end{frame}

%\begin{frame}
%\frametitle{绝对值}
%一个实数的\bold{绝对值}定义为$|x|=\left\{\begin{array}{ll}x,&x\ge0\\-x,&x<0\end{array}\right.$。
%\ppause
%去绝对值号的方法(设$a,b>0$):
%\begin{itemize}[<+->]
%  \item $|x|<b$ 等价于 $-b<x<b$
%  \item $|x|>a$ 等价于 $x<-a$或$x>a$
%  \item $a<|x|<b$ 等价于 $-b<x<-a$或$a<x<b$
%\end{itemize}
%\onslide<+->
%绝对值的\bold{三角不等式}:
%\begin{multicols}{2}
%\begin{itemize}[<+->]
%  \item $|x+y|\le|x|+|y|$
%  \item $|x-y|\ge|x|-|y|$
%\end{itemize}
%\end{multicols}
%\end{frame}

\begin{frame}
\frametitle{区间}
\begin{example}
用区间表示下列数集:
\begin{enumlite}
  \item $\left\{x \,\big|\, |x-2|<1\right\}$
  \item $\left\{x \,\big|\, |x+3|\ge5\right\}$
  \item $\left\{x \,\big|\, 1\le|x+1|<4\right\}$
\end{enumlite}
\end{example}
\end{frame}

\begin{oframe}
\frametitle{区间}
\begin{exercise}
用区间表示下列数集:
\begin{enumlite}
  \item $\left\{x \,\big|\, |x+2|\le3\right\}$
  \item $\left\{x \,\big|\, |x-4|>7\right\}$
  \item $\left\{x \,\big|\, 2<|x+3|\le5\right\}$
\end{enumlite}
\end{exercise}
\vpause
\begin{solution}
\begin{enumlite}
  \item $[-5,1]$
  \item $(-\infty,-3)\cup(11,+\infty)$
  \item $[-8,-5)\cup(-1,2]$
\end{enumlite} 
\end{solution}
\end{oframe}

%\begin{iframe}
%\frametitle{区间}
%\begin{example}
%用区间表示下列数集:
%\begin{enumerate}
%  \item $\left\{x \,\big|\, x^2-x-2<0\right\}$
%  \item $\left\{x \,\big|\, x^2+x-6\ge0\right\}$
%\end{enumerate}
%\end{example}
%\pause
%\begin{exercise}
%用区间表示下列数集:
%\begin{enumerate}
%  \item $\left\{x \,\big|\, x^2+2x-3\le0\right\}$
%  \item $\left\{x \,\big|\, x^2-3x+2>0\right\}$
%\end{enumerate}
%\end{exercise}
%\pause
%\begin{example}
%用区间表示数集$\left\{x \,\big|\, |x^2-3x-2|<2\right\}$。
%\end{example}
%\end{iframe}

\begin{frame}
\frametitle{邻域}
\begin{itemize}
  \item $a$的\bold{邻域} $U(a,\delta)$:$$\left\{x \,\big|\, |x-a|<\delta\right\}=(a-\delta,a+\delta)$$
  \item $a$的\bold{去心邻域} $\mathring{U}(a,\delta)$:$$\left\{x \,\big|\, 0<|x-a|<\delta\right\}=(a-\delta,a)\cup(a,a+\delta)$$
\end{itemize}
\pause
\begin{itemize}
  \item $a$的\bold{左邻域}:$(a-\delta,a)$
  \item $a$的\bold{右邻域}:$(a,a+\delta)$
\end{itemize}
\pause
其中$a$称为邻域的\bold{中心},$\delta$称为邻域的\bold{半径}。
\end{frame}

\subsection{函数的概念}

\begin{frame}
\begin{definition}
设非空数集$D\subset\mathbb{R}$,如果存在一个对应规则$f$,使得对每个$x\in D$,都有一个确定的实数$y$ 与之对应,
则称$f$为定义在$D$上的一个\bold{函数},记为$f:D\longrightarrow\mathbb{R}$,简记为$y=f(x)$。
\end{definition}%
\pause
\begin{itemize}
  \item $x$称为\bold{自变量};
  \item $y$称为\bold{因变量};
  \item $D$称为\bold{定义域};
  \item $Z=\left\{y \,\big|\, y=f(x), x\in D\right\}$称为\bold{值域}。
\end{itemize}
\end{frame}

\begin{frame}
\begin{remark*}
两个函数相同,当且仅当两者的定义域和对应规则都相同。
\end{remark*}
\pause
\begin{example}
研究$y=x$和$y=\dfrac{x^2}x$是不是相同的函数。
\end{example}
\pause
\begin{example}
研究$y=x$和$y=\sqrt{x^2}$是不是相同的函数。
\end{example}
\end{frame}

%\begin{frame}
%\frametitle{函数的记号}
%\begin{example}
%已知$f(x)=x(x+1)$,求$f(x-1)$。
%\end{example}
%\pause
%\begin{example}
%已知$f(x-1)=x^2+1$,求$f(x)$。
%\end{example}
%\end{frame}
%
%\begin{frame}
%\frametitle{函数的记号}
%\begin{exercise}
%已知$f(3x-1)=9x^2+6x-2$,求$f(x)$。
%\end{exercise}
%\pause
%\begin{solution}
%$f(x)=x^2+4x+1$。
%\end{solution}
%\vpause
%\begin{exercise}
%已知$f(\frac{x-1}{x+1})=x+2$,求$f(x)$。
%\end{exercise}
%\pause
%\begin{solution}
%$f(x)=\frac{x-3}{x-1}$。
%\end{solution}
%\end{frame}

\begin{frame}
\frametitle{自然定义域}
对未指明定义域的函数,通常根据函数表达式确定它的\bold{自然定义域}。
例如
\begin{enumerate}[<+->]
\item $y=\sqrt{x}$的定义域为$D=[0,+\infty)$,
\item $y=\log_a{x}$的定义域为$D=(0,+\infty)$,
\item $y=\frac1{x}$ 的定义域为$D=(-\infty,0)\cup(0,+\infty)$。
\end{enumerate}
\onslide<+->
求函数的自然定义域时有三个基本要求:
\begin{enumerate}[<+->]
  \item 根号里面要求大于等于零;
  \item 对数里面要求大于零;
  \item 分母要求不能等于零。
\end{enumerate}
\end{frame}

\begin{oframe}
\frametitle{自然定义域}
\begin{example}
用区间表示下列函数的定义域:
\begin{enumlite}
  \item $y = \sqrt{x-2}$
  \item $y= \ln(2x+6)+\sqrt{5-x}$
  \item $y = \dfrac{1}{x-4}$
\end{enumlite}
\end{example}
\end{oframe}

\begin{oframe}
\frametitle{自然定义域}
\begin{exercise}
用区间表示下列函数的定义域:
\begin{enumlite}
  \item $y=\ln(3-x)$
  \item $y=\sqrt{4-x^2}+\dfrac{x+1}{x-1}$
\end{enumlite}
\end{exercise}
\vpause
\begin{solution}
\begin{enumlite}
  \item $(-\infty,3)$
  \item $[-2,1)\cup(1,2]$
\end{enumlite}
\end{solution}
\end{oframe}

\begin{iframe}
\frametitle{自然定义域}
\begin{example}
用区间表示下列函数的定义域:
\begin{enumlite}
  \item $y= \ln(2x+6)+\sqrt{5-x}$
  \item $y= \frac{1}{\ln(x-5)}$
\end{enumlite}
\end{example}
\pause
\begin{exercise}
用区间表示下列函数的定义域:
\begin{enumlite}
  \item $y= \frac1{4-x^2}+\sqrt{x+3}$
  \item $y= \ln(\frac{x+1}{x-1})$
\end{enumlite}
\end{exercise}
\end{iframe}

\subsection{函数的性质}

\begin{frame}
\frametitle{函数的奇偶性}
\begin{definition}
给定函数$y=f(x)$,
\begin{enumerate}
  \item 如果$\forall x\in D$,总有$f(-x)=f(x)$,则称$f(x)$为\bold{偶函数}。
  \item 如果$\forall x\in D$,总有$f(-x)=-f(x)$,则称$f(x)$为\bold{奇函数}。
\end{enumerate}
\end{definition}
\vpause
\begin{example*}
$x$, $x^3$, $\dfrac1x$, $\dfrac1{x^3}$, $\sin x$, $\tan x$ 为奇函数。
\end{example*}
\pause
\begin{example*}
$x^2$, $x^4$, $\dfrac1{x^2}$, $\dfrac1{x^4}$, $\cos x$ 为偶函数。
\end{example*}
\end{frame}

\begin{frame}
\frametitle{函数的奇偶性}
\begin{example}
判断下列函数的奇偶性:
\begin{enumlite}
  \item $f(x)=x^4-2x^2+1$ %为偶函数。
  \item $f(x)=x^3+x$ %为奇函数。
  \item $f(x)=x^2+x+1$ %为非奇非偶函数。
\end{enumlite}
\end{example}
\pause
\begin{exercise}
判断下列函数的奇偶性:
\begin{enumlite}
  \item $f(x)=\frac{1-x^2}{\cos x}$ \onslide<3->{\cdotfill 偶函数}
  \item $f(x)=\frac{e^x-1}{e^x+1}$ \onslide<3->{\cdotfill 奇函数}
\end{enumlite}
\end{exercise}
\end{frame}

\begin{frame}
\frametitle{函数的周期性}
\begin{definition}
对于函数$y=f(x)$,如果存在正常数$T$使得$f(x+T)=f(x)$恒成立,则称此函数为\bold{周期函数};
满足这个等式的最小正数$T$,称为此函数的\bold{周期}。
\end{definition}
\vpause
\begin{example*}
$y=\sin x$ 和 $y=\cos x$ 以 $2\pi$ 为周期。
\end{example*}
\pause
\begin{example*}
$y=\tan x$ 和 $y=\cot x$ 以 $\pi$ 为周期。
\end{example*}
\end{frame}

\begin{frame}
\frametitle{函数的单调性}
\begin{definition}
设函数$y=f(x)$在区间$I$上有定义,对于区间$I$上的任意两点$x_1$和$x_2$,
\begin{enumerate}
  \item 若当$x_1<x_2$时有$f(x_1)<f(x_2)$,则称$f(x)$在区间$I$上是\bold{单调增加的};
  \item 若当$x_1<x_2$时有$f(x_1)>f(x_2)$,则称$f(x)$在区间$I$上是\bold{单调减少的};
\end{enumerate}
\end{definition}
\end{frame}

\begin{frame}
\frametitle{函数的单调性}
\begin{example*}
$y=x$在$(-\infty,+\infty)$上是单调增加的。
\end{example*}
\vpause
\begin{example*}
$y=\ln x$在$(0,+\infty)$上是单调增加的。
\end{example*}
\vpause
\begin{example*}
$y=1/x$在$(-\infty,0)$和$(0,+\infty)$上单调减少。
\end{example*}
\vpause
\begin{example*}
$y=x^2$在$(-\infty,0]$上单调减少,在$[0,+\infty)$上单调增加。
\end{example*}
\end{frame}

\begin{frame}
\frametitle{函数的有界性}
\begin{definition}
设函数$y=f(x)$在数集$I$上有定义,如果存在一个正数$M$,对于所有$x\in I$,恒有$|f(x)|\le M$,
则称函数$f(x)$在数集$I$上\bold{有界}。若这样的$M$不存在,则称$f(x)$在$I$上\bold{无界}。\ppause
如果函数在其定义域上有界,则称它为\bold{有界函数};否则称它为\bold{无界函数}。
\end{definition}
\vpause
\begin{example*}
$y=\sin x$,$y=\cos x$ 是有界函数。
\end{example*}
\pause
\begin{example*}
$y=x^2$,$y=\tan x$,$y=x\cos x$ 是无界函数。
\end{example*}
\end{frame}

\begin{frame}
\frametitle{函数的有界性}
\begin{example}
证明下列函数为有界函数:
\begin{enumlite}
  \item $y=\sin x - 2\cos x$;
  \item $y=\frac1{3+x^2}$。
\end{enumlite}
\end{example}
\pause
\begin{exercise}
证明下列函数为有界函数:
\begin{enumlite}
  \item $y=\sin x -\frac{1}{2+x^2}$\pause ;
  \item $y=\frac{x^2}{1+x^2}$。
\end{enumlite}
\end{exercise}
\pause
\begin{solution}
\begin{multicols}{2}
\begin{enumlite}
  \item $|y|\le\frac32$;
  \item $|y|\le1$。
\end{enumlite}
\end{multicols}
\end{solution}
\end{frame}

\begin{iframe}
\frametitle{函数的有界性}
类似地,我们可以定义函数有上界和有下界的概念。
\begin{itemize}
  \item 对于所有$x$,恒有 $|f(x)|\leq M$ \dotfill\bold{有界}
  \item 对于所有$x$,恒有 $f(x)\le M_1$ \dotfill\bold{有上界}
  \item 对于所有$x$,恒有 $f(x)\ge M_2$ \dotfill\bold{有下界}
\end{itemize}
\vpause
\begin{theorem*}
$f(x)$有界 $\Longleftrightarrow$ $f(x)$有上界而且有下界。
\end{theorem*}
\end{iframe}

\subsection{函数的构建}

\begin{frame}
\frametitle{反函数}
\begin{definition}
设$y=f(x)$的定义域为$D$,值域为$Z$。如果对每个$y\in Z$,有唯一的$x\in D$满足$y=f(x)$,
则可以得到定义在 $Z$ 上的函数 $x=f^{-1}(y)$,称为$y=f(x)$的\bold{反函数}。
\end{definition}
\vpause
\begin{example}
求函数$y=3x-1$的反函数。
\end{example}
\pause
\begin{example}
求函数$y=2\ln x+1$的反函数。
\end{example}
\end{frame}

\begin{oframe}
\frametitle{反函数}
\begin{exercise}
求下列函数的反函数:
\begin{multicols}{2}
\begin{enumlite}
  \item $y=\frac{x-1}{x+2}$;
  \item $y=\frac{e^x-2}{e^x}$。
\end{enumlite}
\end{multicols}
\end{exercise}
\vpause
\begin{solution}
\begin{multicols}{2}
\begin{enumlite}
  \item $y=\frac{-2x-1}{x-1}$;
  \item $y=\ln\big(\frac{2}{1-x}\big)$。
\end{enumlite}
\end{multicols}
\end{solution}
\end{oframe}

\begin{oframe}
\frametitle{复合函数}
\begin{definition*}
将$u=g(x)$代入$y=f(u)$,得到的新函数$y=f[g(x)]$,称为这两个函数的\bold{复合函数}。\pause
复合函数的定义域是那些使得它有意义的$x$所组成的集合。
\end{definition*}
\vpause
\begin{example*}
两个函数$y=\sqrt{u}$ 和 $u=1-x^2$ 的复合函数是$y=\sqrt{1-x^2}$。
\end{example*}
\vpause
\begin{example*}
三个函数$y=\sin u$、$u=v^2-1$ 和 $v=e^x$的复合函数是$y=\sin(e^{2x}-1)$。
\end{example*}
\end{oframe}

\begin{iframe}
\frametitle{复合函数}
\begin{definition}
设$y=f(u)$的定义域是$D(f)$,$u=g(x)$的值域是$Z(g)$,$D(f)\cap Z(G)$非空,
则称$y=f[g(x)]$为$y=f(u)$和$u=g(x)$的\bold{复合函数}。
\end{definition}
\vpause
\begin{example}
两个函数$y=\sqrt{u}$ 和 $u=1-x^2$ 的复合函数是$y=\sqrt{1-x^2}$。
\end{example}
\vpause
\begin{example}
三个函数$y=\sin u$、$u=v^2-1$ 和 $v=e^x$的复合函数是$y=\sin(e^{2x}-1)$。
\end{example}
\end{iframe}

\begin{frame}
\frametitle{三角函数}
\begin{enumerate}[<+->]
  \item 正弦函数$y=\sin x$
  \item 余弦函数$y=\cos x$
  \item 正切函数$y=\tan x$
  \item 余切函数$y=\cot x$
  \item 正割函数$y=\sec x = \dfrac1{\cos x}$
  \begin{itemize}
    \item $\sec^2x=\tan^2x+1$
  \end{itemize}
  \item 余割函数$y=\csc x = \dfrac1{\sin x}$
  \begin{itemize}
    \item $\csc^2x=\cot^2x+1$
  \end{itemize}
\end{enumerate}
\end{frame}

\begin{frame}
\frametitle{反三角函数}
\begin{enumerate}[<+->]
  \item 反正弦函数$y=\arcsin x$ \dotfill $x=\sin y$
  \begin{itemize}
    \item $x\in[-1,1],\quad y\in[-\frac{\pi}2,\frac{\pi}2]$
  \end{itemize}
  \item 反余弦函数$y=\arccos x$ \dotfill $x=\cos y$
  \begin{itemize}
    \item $x\in[-1,1],\quad y\in[0,\pi]$
  \end{itemize}
  \item 反正切函数$y=\arctan x$ \dotfill $x=\tan y$
  \begin{itemize}
    \item $x\in(-\infty,+\infty),\quad y\in(-\frac{\pi}2,\frac{\pi}2)$
  \end{itemize}
  \item 反余切函数$y=\arccot x$ \dotfill $x=\cot y$
  \begin{itemize}
    \item $x\in(-\infty,+\infty),\quad y\in(0,\pi)$
  \end{itemize}
\end{enumerate}%
\onslide<+->{\begin{example*}
$\arccos(\frac12)=\frac\pi3$,$\arctan(\frac{\sqrt3}3)=\frac\pi6$。
\end{example*}}
\end{frame}

\begin{frame}{初等函数}
下面这六种函数,统称为\bold{基本初等函数}:
\begin{enumerate}[<+->]
  \item 常值函数$y=c$;
  \item 幂函数$y=x^\mu$;
  \item 指数函数$y=a^x$;
  \item 对数函数$y=\log_ax$;
  \item 三角函数$y=\sin x$,$y=\cos x$,等;
  \item 反三角函数$y=\arcsin x$,$y=\arccos x$,等。
\end{enumerate}
%\vpause
%由六种基本初等函数经过有限次四则运算所得到的函数,称为\bold{简单函数}。
\onslide<+->
由六种基本初等函数经过有限次四则运算和函数复合所得到的函数,称为\bold{初等函数}。
\end{frame}

\begin{oframe}
\frametitle{初等函数的分解}
\begin{example}
将下列初等函数分解为简单函数的复合:
\begin{enumlite}
  \item $y=e^{2x^2-1}$;
  \item $y=\sqrt{2+\cos^2 x}$;
  %\item $y=\frac{e^{-x}+1}{e^{-x}-1}$。
\end{enumlite}
\end{example}
\vpause
\begin{exercise}
将下列初等函数分解为简单函数的复合:
\begin{enumlite}
  \item $y=(1+\ln x)^5$;
  \item $y=\sin^2(3x+1)$。
  %\item $y=\frac1{e^{-x^2}}$;
\end{enumlite}
\end{exercise}
\pause
\begin{solution}
\begin{enumlite}
  \item $y=u^5$,$u=1+\ln x$;
  \item $y=u^2$,$u=\sin v$,$v=3x+1$。
\end{enumlite}
\end{solution}
\end{oframe}

%\begin{frame}
%\frametitle{经济学常用函数}
%\begin{itemize}[<+->]
%  \item 总成本函数$C(Q)$,平均成本函数$\overline{C}(Q)$
%  \item 总收益函数$R(Q)$,平均收益函数$\overline{R}(Q)$
%  \item 总利润函数$L(Q)$,平均利润函数$\overline{L}(Q)$
%\end{itemize}
%\end{frame}

\section{数列的极限}

\subsection{数列极限的定义}

\begin{frame}
\begin{definition}
\frametitle{数列的定义}
一列按照顺序排列的数 $x_1, x_2, x_3, \cdots, x_n, \cdots$ 称为\bold{数列},记为 $\{x_n\}$.
第$n$项$x_n$的表达式称为数列的\bold{通项}或一般项。
\end{definition}
\vpause
\begin{problem*}
随着$n$的增大,$x_n$也跟着变化。当$n$趋于无穷大时,$x_n$是否会\bold{无限接近}一个确定的数?
\end{problem*}
\end{frame}

\begin{frame}
\frametitle{数列的例子}
\begin{enumerate}[<+->]
  \item \makebox[7em][l]{$x_n=3$} $3,\ 3,\ 3,\ 3,\ \cdots
                             \onslide<9->{\hfill\longrightarrow3}$
  \item \makebox[7em][l]{$x_n=\frac1n$} $1,\ \frac12,\ \frac13,\ \frac14,\ \cdots
                             \onslide<10->{\hfill\longrightarrow0}$
  \item \makebox[7em][l]{$x_n=\frac{1}{2^n}$} $\frac12,\ \frac14,\ \frac18,\ \frac1{16},\ \cdots
                             \onslide<11->{\hfill\longrightarrow0}$
  \item \makebox[7em][l]{$x_n=\frac{n}{n+1}$} $\frac12,\ \frac23,\ \frac34,\ \frac45,\ \cdots
                             \onslide<12->{\hfill\longrightarrow1}$
  \item \makebox[7em][l]{$x_n=\frac{(-1)^n}{n}$}  $-1,\ \frac12,\ -\frac13,\ \frac14,\ \cdots
                             \onslide<13->{\hfill\longrightarrow0}$
  \item \makebox[7em][l]{$x_n=2^n$} $2,\ 4,\ 8,\ 16,\ \cdots
                             \onslide<14->{\hfill\text{\warn{\ding{53}}}}$
  \item \makebox[7em][l]{$x_n=(-1)^n$} $-1,\ 1,\ -1,\ 1,\ \cdots
                             \onslide<15->{\hfill\text{\warn{\ding{53}}}}$
%  \item \makebox[7em][l]{$x_n=\frac{n+(-1)^n}{n}$} $0,\ \frac32,\ \frac23,\ \frac54,\ \frac45,\ \cdots
%                            \onslide<16->{\hfill\longrightarrow1}$
\end{enumerate}
\end{frame}

\begin{iframe}
\frametitle{数列的趋势}
\begin{example*}
如下数列不容易凭借观察得出其变化趋势:
\begin{enumerate}
  \item $x_n = \sqrt[n]{n}$
  \item $x_n = \left(1+\frac1n\right)^n$
\end{enumerate}
\end{example*}
\end{iframe}

\begin{oframe}
\frametitle{数列的极限}
\begin{definition*}
如果当$n$趋于无穷大时,$x_n$无限接近一个确定的常数$A$,我们称数列$\{x_n\}$的\bold{极限}等于$A$,
或者称数列$\{x_n\}$ \bold{收敛}于$A$,记为
\[ \lim_{n\to\infty}x_n=A. \]
否则,称数列$\{x_n\}$的极限不存在,或者称数列\bold{发散}。
\end{definition*}
%\vpause
%\begin{remark*}
%此定义是不严格的,严格的定义可以见下一页。
%\end{remark*}
\end{oframe}

\begin{iframe}
\frametitle{数列的极限}
\begin{definition}
设$\{x_n\}$为一个数列,如果存在常数$A$,对任何$\epsilon>0$,总存在$N>0$,使得当$n>N$时,总有
\[ |x_n-A|<\epsilon \]
则称数列$\{x_n\}$的\bold{极限}等于$A$,或者称数列$\{x_n\}$ \bold{收敛}于$A$,记为
\[ \lim_{n\to\infty}x_n=A. \]
\end{definition}
如果这样的常数$A$不存在,则称数列$\{x_n\}$ \bold{发散}.
\end{iframe}

\begin{frame}
\frametitle{数列极限的基本公式}
\noindent\fbox{\parbox{0.956\textwidth}{%
\begin{enumerate}
\item $\limit_{n\to\infty}C=C$
\item $\limit_{n\to\infty}\dfrac1{n^k}=0$,($k>0$)
\item $\limit_{n\to\infty}\dfrac{(-1)^n}{n^k}=0$,($k>0$)
\item $\limit_{n\to\infty}\dfrac{1}{a^n}=0$,($|a|>1$)
\end{enumerate}
}}
\end{frame}

\begin{iframe}
\frametitle{数列极限}
\begin{example*}
设$x_n=C$,证明$\limit_{n\to\infty}x_n= C$。
\end{example*}
\pause
\begin{proof}
$\forall\epsilon>0$,取$N=1$,则当$n>N$时就有
\[ |x_n-C|=\left|C-C\right|=0<\epsilon.\]
\end{proof}
\pause
\begin{example*}
证明$\limit_{n\to\infty}\dfrac{1}{n}= 0$。
\end{example*}
\pause
\begin{proof}
$\forall\epsilon>0$,取$N=\dfrac1\epsilon$,则当$n>N$时就有
\[ |x_n-0|=\left|\dfrac{1}{n}-0\right|=\frac1n<\epsilon.\]
\end{proof}
\end{iframe}

\begin{iframe}
\frametitle{数列极限}
\begin{example*}
证明$\limit_{n\to\infty}\frac{(-1)^n}{n}= 0$。
\end{example*}
\pause
\begin{proof}
$\forall\epsilon>0$,取$N=\frac1\epsilon$,则当$n>N$时就有
\[ |x_n-0|=\left|\tfrac{(-1)^n}{n}-0\right|=\tfrac1n<\epsilon.\]
\end{proof}
\pause
\begin{example*}
证明$\limit_{n\to\infty}\frac1{2^n}= 0$。
\end{example*}
\pause
\begin{proof}
$\forall\epsilon>0$,取$N=\frac1\epsilon$,则当$n>N$时就有
\[ |x_n-0|=\left|\tfrac1{2^n}-0\right|=\tfrac1{2^n}<\tfrac1n<\epsilon.\]
其中不等式$2^n>n$可由数学归纳法得到.
\end{proof}
\end{iframe}

\begin{frame}
\frametitle{发散数列}
发散的数列至少有这两种可能:
\begin{enumerate}
  \item 无界型的:比如 $x_n=2^n$;
  \item 摆动型的:比如 $x_n=(-1)^n$。
\end{enumerate}
\end{frame}

\subsection{数列极限的运算}

\begin{frame}
\frametitle{数列极限的四则运算}
\begin{theorem}
如果$\limit_{n\to\infty}x_n=A$,$\limit_{n\to\infty}y_n=B$,那么
\begin{enumerate}
  \item $\limit_{n\to\infty}(x_n \pm y_n) = \limit_{n\to\infty}x_n \pm \limit_{n\to\infty}y_n = A \pm B$;
  \item $\limit_{n\to\infty}(x_n \cdot y_n) = \limit_{n\to\infty}x_n \cdot \limit_{n\to\infty}y_n = A \cdot B$;
  \item $\limit_{n\to\infty}\dfrac{x_n}{y_n} = \dfrac{\limit_{n\to\infty}x_n}{\limit_{n\to\infty}y_n} = \dfrac{A}{B}$ 
      (要求分母不为零).
\end{enumerate}
\end{theorem}
\vpause
\begin{corollary*}
$\limit_{n\to\infty}(c\cdot x_n) = c \limit_{n\to\infty}x_n$。
\end{corollary*}
\end{frame}

\begin{jframe}
\frametitle{四则运算法则}
\small
\begin{proof}
(1) $\forall\epsilon>0$, 取$\epsilon_1=\epsilon/2$,则由$\limit_{n\to\infty}x_n=A$得到,
$\exists N_1>0$ 使得当$n>N_1$时总有
\[ |x_n-A|<\epsilon_1; \]
再取$\epsilon_2=\epsilon/2$,则由$\limit_{n\to\infty}y_n=B$得到,
$\exists N_1>0$ 使得当$n>N_1$时总有
\[|y_n-B|<\epsilon_2. \]
令 $N=\max\{N_1,N_2\}$,则当 $n>N$ 时总有
\begin{align*}
|(x_n+y_n)-(A+B)|&=|(x_n-A)+(y_n-B)|\\
 &\le|x_n-A|+|y_n-B| <\epsilon_1+\epsilon_2=\epsilon.
\end{align*}
%\par
%(2) 参考后面$x\to\infty$的情形,略。\par
%(3) 参考后面$x\to x_0$的情形,略。
\end{proof}
\end{jframe}

\begin{frame}
%\frametitle{数列极限的计算}
\begin{example}
求数列极限$\limit_{n\to\infty}\bigg(1-\dfrac1{n^2}\bigg)$.
\end{example}
\pause
\begin{solution}
原式$=\limit_{n\to\infty}1-\limit_{n\to\infty}\dfrac1{n^2}=1-0=1$。
\end{solution}
\vpause
\begin{example}
 求数列极限$\limit_{n\to\infty}\dfrac{n}{n+1}$.
\end{example}
\pause
\begin{solution}
原式\unskip$\begin{aligned}[t]
&=\limit_{n\to\infty}\dfrac{1}{1+\frac1n}
         =\dfrac{\limit_{n\to\infty}1}{\limit_{n\to\infty}1+\limit_{n\to\infty}\frac1n}\\
&=\dfrac{1}{1+0}=1.\end{aligned}$
\end{solution}
\end{frame}

\begin{frame}
\frametitle{数列极限的计算}
\begin{example}
求数列极限$\limit_{n\to\infty}\dfrac{3n^2+1}{n^2+4n}$.
\end{example}
\pause
\begin{solution}
原式\unskip$\begin{aligned}[t]
&=\limit_{n\to\infty}\dfrac{3+\dfrac1{n^2}}{1+4\cdot\dfrac1n}
 =\dfrac{\limit_{n\to\infty}3+\limit_{n\to\infty}\dfrac1{n^2}}{\limit_{n\to\infty}1+4\limit_{n\to\infty}\dfrac1n}\\
&=\dfrac{3+0}{1+4\cdot0}=3.
\end{aligned}$
\end{solution}
\end{frame}

\begin{frame}
\frametitle{数列极限的计算}
\begin{example}
求数列极限$\limit_{n\to\infty}\dfrac{n+4}{n^2+1}$.
\end{example}
\pause
\begin{solution}
原式\unskip$\begin{aligned}[t]
&=\limit_{n\to\infty}\dfrac{\dfrac1n+4\cdot\dfrac1{n^2}}{1+\dfrac1{n^2}}
 =\dfrac{\limit_{n\to\infty}\dfrac1n+4\limit_{n\to\infty}\dfrac1{n^2}}{\limit_{n\to\infty}1+\limit_{n\to\infty}\dfrac1{n^2}}\\
&=\dfrac{0+4\times0}{1+0}=0.
\end{aligned}$
\end{solution}
\end{frame}

\begin{frame}
\frametitle{数列极限的计算}
\begin{example}
求数列极限$\limit_{n\to\infty}\dfrac{n+(-1)^n}{n+1}$.
\end{example}
\pause
\begin{solution}
原式\unskip$\begin{aligned}[t]
&=\limit_{n\to\infty}\dfrac{1+\frac{(-1)^n}{n}}{1+\frac1n}
 =\dfrac{\limit_{n\to\infty}1+\limit_{n\to\infty}\frac{(-1)^n}{n}}{\limit_{n\to\infty}1+\limit_{n\to\infty}\frac1n}\\
&=\frac{1+0}{1+0}=1.
\end{aligned}$
\end{solution}
\end{frame}

\begin{frame}
\frametitle{数列极限的计算}
\begin{example}
求数列极限$\limit_{n\to\infty}\dfrac{2\times 3^n}{3^n+1}$.
\end{example}
\pause
\begin{solution}
原式$\begin{aligned}[t]
&=\limit_{n\to\infty}\dfrac{2}{1+\dfrac1{3^n}}
 =\dfrac{\limit_{n\to\infty}2}{\limit_{n\to\infty}1+\limit_{n\to\infty}\dfrac1{3^n}}\\
&=\frac{2}{1+0}=2.
\end{aligned}$
\end{solution}
\end{frame}

\begin{oframe}
\frametitle{数列极限的计算}
\begin{exercise}
求数列极限
\begin{enumlite}
  \item $\limit_{n\to\infty}\dfrac{3n-1}{n+2}$
  \item $\limit_{n\to\infty}\dfrac{n^2+2n}{5n^2+1}$
  \item $\limit_{n\to\infty}\dfrac{3^n+1}{6^n+1}$
\end{enumlite}
\end{exercise}
\end{oframe}

\begin{iframe}
\frametitle{数列极限的计算}
\begin{exercise}
求数列极限
\begin{enumlite}
  \item $\limit_{n\to\infty}\dfrac{2n+3n^2}{1+n^3}$
  \item $\limit_{n\to\infty}\dfrac{3n+(-1)^n}{n+(-1)^n}$
  \item $\limit_{n\to\infty}\dfrac{3^n+1}{6^n+1}$
\end{enumlite}
\end{exercise}
\end{iframe}

\subsection{数列极限的性质}

\begin{iframe}
\frametitle{数列极限的性质}
\begin{property}[唯一性]
若$\{x_n\}$收敛,则其极限是唯一的。
\end{property}
\end{iframe}

\begin{iframe}
\frametitle{数列极限的性质}
\begin{property}[有界性]
设$\{x_n\}$收敛,则存在$M>0$使得$|x_n|\le M$。
\end{property}
\vpause
\begin{proof}
设$\limit_{n\to\infty}x_n=A$。取$\epsilon=1$,则存在$N>0$,
使得当$n>N$时有$|x_n-A|<\epsilon=1$。此时
\[ |x_n|=|(x_n-A)+A|\le|x_n-A|+|A|< 1+|A| \]
取$M=\max\{x_1,x_2,\cdots,x_{[N]},1+|A|\}$,则对任何$n$都有$|x_n|\le M$。
\end{proof}
\vpause
\begin{example*}
$x_n=1-5/n$收敛于$1$,此时有$|x_n|\le4$。
\end{example*}
\end{iframe}

\begin{iframe}
\frametitle{数列极限的性质}
\begin{property}[保号性]
设数列收敛于$A>0$(或$A<0$),则存在$N>0$,使得当$n>N$时有$x_n>0$(或$x_n<0$)。
\end{property}
\vpause
\begin{proof}
取$\epsilon=A/2$,则存在$N>0$,使得当$n>N$时有$|x_n-A|<\epsilon=A/2$。
此时$x_n>A/2>0$。
\end{proof}
\vpause
\begin{example*}
$x_n=1-5/n$收敛于$1>0$,此时当$n>5$时,有$x_n>0$。
\end{example*}
\end{iframe}

\begin{frame}
\frametitle{数列极限的性质}
\begin{theorem*}[保号性]
设数列$x_n\ge0$(或$x_n\le0$),且$\limit_{n\to\infty}x_n=A$,则有$A\ge0$(或$A\le0$)。
\end{theorem*}
\vpause
\begin{corollary*}
如果$x_n\ge y_n$,而且$\limit_{n\to\infty}x_n=A$,$\limit_{n\to\infty}y_n=B$,则有$A\ge B$。
\end{corollary*}
\end{frame}

\begin{frame}
\frametitle{复习与提高}
\begin{choice}
已知数列$\{x_n\}$的通项为$x_n=(-1)^n\frac{n}{n+1}$,则该数列\dotfill(\select{C})
\begin{choicehalf}
  \item 收敛且有界 ~
  \item 收敛且无界 ~
  \item 发散且有界 ~
  \item 发散且无界 ~
\end{choicehalf}
\end{choice}
\end{frame}

\section{函数的极限I}

\subsection{函数极限的定义}

\begin{frame}
\frametitle{函数极限的例子}
\begin{enumerate}[<+->]
  \item $y=\wfrac1x$
  \begin{itemize}
    \item $x\to+\infty$ 时 $y\to0$
    \item $x\to-\infty$ 时 $y\to0$
  \end{itemize}
  \item $y=\wfrac1{x^2}$
  \begin{itemize}
    \item $x\to+\infty$ 时 $y\to0$
    \item $x\to-\infty$ 时 $y\to0$
  \end{itemize}
  \item $y=\left(\frac12\right)^x$
  \begin{itemize}
    \item $x\to+\infty$ 时 $y\to0$
  \end{itemize}
  \item $y=2^x$
  \begin{itemize}
    \item $x\to-\infty$ 时 $y\to0$
  \end{itemize}
\end{enumerate}
\end{frame}

\begin{oframe}
\frametitle{函数的极限($x\to\infty$)}
\begin{definition*}
如果当$x$趋于无穷时,$f(x)$无限接近一个确定的常数$A$,
则称当$x\to\infty$时$f(x)$以$A$为\bold{极限},记为
\[ \lim_{x\to\infty}f(x)=A. \]
%否则,称当$x\to\infty$时函数$f(x)$的极限不存在。
\end{definition*}
\vpause
\begin{remark*}
$x\to\infty$有两种方向,即$x\to-\infty$和$x\to+\infty$。类似地可以定义
$\limit_{x\to-\infty}f(x)=A$和$\limit_{x\to+\infty}f(x)=A$。
\end{remark*}
\vpause
\begin{property*}
$\limit_{x\to\infty}f(x)=A$ 当且仅当 $\limit_{x\to-\infty}f(x)=A$ 且 $\limit_{x\to+\infty}f(x)=A$。
\end{property*}
\end{oframe}

\begin{iframe}
\frametitle{函数的极限($x\to\infty$)}
\begin{definition}
设$f(x)$在$|x|$足够大时有定义,如果存在常数$A$,对任何$\epsilon>0$,总存在$N>0$,
使得当$|x|>N$时,总有$|f(x)-A|<\epsilon$,
则称当$x\to\infty$时$f(x)$以$A$为\bold{极限},记为
\[ \lim_{x\to\infty}f(x)=A. \]
%如果这样的$A$不存在,则称当$x\to\infty$时 $f(x)$ 的极限不存在.
\end{definition}
\vpause
\begin{remark*}
$x\to\infty$有两种方向,即$x\to-\infty$和$x\to+\infty$。类似地可以定义
$\limit_{x\to-\infty}f(x)=A$和$\limit_{x\to+\infty}f(x)=A$。
\end{remark*}
\vpause
\begin{property*}
$\limit_{x\to\infty}f(x)=A$ 当且仅当 $\limit_{x\to-\infty}f(x)=A$ 且 $\limit_{x\to+\infty}f(x)=A$。
\end{property*}
\end{iframe}

\begin{jframe}
\frametitle{函数极限的例子}
\begin{example*}
证明$\limit_{x\to+\infty}\dfrac1{2^x}= 0$。
\end{example*}
\begin{proof}
$\forall\epsilon>0$,由数列极限$\limit_{n\to\infty}\dfrac1{2^n}= 0$知道,
存在$N_1>0$使得当$n>N_1$时有$\dfrac1{2^n}<\epsilon$.
\par
取$N=N_1+1$,则当$x>N$时有$[x]>N_1$,从而
$$\left|\frac1{2^x}-0\right|=\frac1{2^x}\le\frac1{2^{[x]}}<\frac1{2^{N_1}}<\epsilon.$$
\end{proof}
\end{jframe}

\begin{frame}
\frametitle{函数极限的基本公式I}
\noindent\fbox{\parbox{0.956\textwidth}{%
\begin{align}
&\lim_{x\to\infty}C= C\\
&\lim_{x\to\infty}\frac1{x^k}= 0 \quad (k\text{为正整数})\\
&\lim_{x\to+\infty}\frac1{a^x}= 0 \quad (a>1)\\
&\lim_{x\to-\infty}b^x= 0 \quad (b>1)
\end{align}%
}}
\end{frame}

\subsection{函数极限的运算}

\begin{frame}
\frametitle{四则运算法则}
\begin{theorem}
如果$\limit_{x\to\infty}f(x)=A$,$\limit_{x\to\infty}g(x)=B$,那么
\begin{enumerate}
  \item $\limit_{x\to\infty}(f(x) \pm g(x)) = \limit_{x\to\infty}f(x) \pm \limit_{x\to\infty}g(x) = A \pm B$
  \item $\limit_{x\to\infty}(f(x) \cdot g(x)) = \limit_{x\to\infty}f(x) \cdot \limit_{x\to\infty}g(x) = A \cdot B$
  \item $\limit_{x\to\infty}\dfrac{f(x)}{g(x)} = \dfrac{\limit_{x\to\infty}f(x)}{\limit_{x\to\infty}g(x)} = \dfrac{A}{B}$ 
       (要求分母不为零)
\end{enumerate}
\end{theorem}
\vpause
\begin{corollary*}
$\limit_{x\to\infty}\big(c\cdot f(x)\big) = c \limit_{x\to\infty}f(x)$。
\end{corollary*}
\end{frame}

\begin{jframe}[plain]
\frametitle{四则运算法则 \optstar}
\footnotesize
\begin{proof}
%(1) 参考数列极限的情形,略。\par
(2) $\limit_{x\to\infty}f(x)=A$,由局部有界性知道,存在 $M>0$ 和 $N_0>0$ 使得当$|x|>N_0$时有
$ |f(x)|<M$.\par
且由$\limit_{x\to\infty}f(x)=A$ 知道,$\forall\epsilon_1>0$, $\exists N_1>0$ 使得当$|x|>N_1$时有
\[ |f(x)-A|<\epsilon_1. \]
再由$\limit_{x\to\infty}g(x)=B$ 知道,$\forall\epsilon_2>0$,$\exists N_2>0$ 使得当$|x|>N_2$时有
\[ |g(x)-B|<\epsilon_2. \]
$\forall\epsilon>0$,令 $\epsilon_1=\frac{\epsilon}{2|B|+1}$,$\epsilon_2=\frac{\epsilon}{2M}$,
$N=\max\{N_0,N_1,N_2\}$,则当 $|x|>N$ 时总有
\begin{align*}
|f(x)g(x)-AB|&=\big|f(x)[g(x)-B]+B[f(x)-A]\big| \\
             &\le |f(x)|\cdot|g(x)-B|+|B|\cdot|f(x)-A|<M\epsilon_2+|B|\epsilon_1\\
             &=M\cdot\frac{\epsilon}{2M}+|B|\cdot\frac{\epsilon}{2|B|+1}<\epsilon/2+\epsilon/2=\epsilon.
\end{align*}%\par
%(3) 参考后面$x\to x_0$的情形,略。
\end{proof}
\end{jframe}

\begin{frame}
\frametitle{函数的极限($x\to\infty$)}
\begin{example}
求函数极限$\limit_{x\to\infty}\dfrac{2x-1}{3x+4}$。
\end{example}
\vpause
\begin{solution}
原式\unskip$\begin{aligned}[t]
&=\limit_{x\to\infty}\dfrac{2-\dfrac1x}{3+4\cdot\dfrac1x}
 =\dfrac{\limit_{x\to\infty}2-\limit_{x\to\infty}\dfrac1x}{\limit_{x\to\infty}3+4\limit_{x\to\infty}\dfrac1x}\\
&=\frac{2-0}{3+4\times0}=\frac32
\end{aligned}$
\end{solution}
\end{frame}

\begin{frame}
\frametitle{函数的极限($x\to\infty$)}
\begin{example}
求函数极限$\limit_{x\to\infty}\dfrac{2x+1}{3x^2+2}$。
\end{example}
\vpause
\begin{solution}
原式\unskip$\begin{aligned}[t]
&=\limit_{x\to\infty}\dfrac{\dfrac2x+\dfrac1{x^2}}{3+2\cdot\dfrac1{x^2}}
 =\dfrac{\limit_{x\to\infty}\dfrac2x+\limit_{x\to\infty}\dfrac1{x^2}}{\limit_{x\to\infty}3+2\limit_{x\to\infty}\dfrac1{x^2}}\\
&=\frac{0+0}{3+2\times0}=0
\end{aligned}$
\end{solution}
\end{frame}

\begin{frame}
\frametitle{函数的极限($x\to\infty$)}
\begin{example}
求函数极限$\limit_{x\to\infty}\dfrac{3x^2+1}{x^2+5x}$。
\end{example}
\vpause
\begin{solution}
原式\unskip$\begin{aligned}[t]
&=\limit_{x\to\infty}\dfrac{3+\dfrac1{x^2}}{1+5\cdot\dfrac1{x}}
 =\dfrac{\limit_{x\to\infty}3+\limit_{x\to\infty}\dfrac1{x^2}}{\limit_{x\to\infty}1+5\limit_{x\to\infty}\dfrac1{x}}\\
&=\frac{3+0}{1+5\times0}=3
\end{aligned}$
\end{solution}
\end{frame}

\begin{oframe}
\frametitle{函数的极限($x\to\infty$)}
\begin{exercise}
求下列函数极限:
\begin{enumlite}
  \item $\limit_{x\to\infty}\dfrac{x^2+5}{2x^2+1}$;
  \item $\limit_{x\to\infty}\dfrac{1000x+1}{x^2+1}$。
\end{enumlite}
\end{exercise}
\end{oframe}

\begin{iframe}
\frametitle{函数的极限($x\to\infty$)}
\begin{exercise}
求下列函数极限:
\begin{enumlite}
  \item $\limit_{x\to\infty}\dfrac{1+2x^2}{x+2x^2+3x^3}$;\pause
  \item $\limit_{x\to+\infty}\dfrac{3^x+4^x}{2^x+5^x}$。
\end{enumlite}
\end{exercise}
\end{iframe}

\subsection{函数极限的性质}

\setcounter{property}{0}

\begin{iframe}
\frametitle{函数极限的性质}
\begin{property}[唯一性]
若$\limit_{x\to\infty}f(x)$存在,则极限唯一。
\end{property}
\end{iframe}

\begin{iframe}
\frametitle{函数极限的性质}
\begin{property}[局部有界性]
如果$\limit_{x\to\infty}f(x)=A$,则存在$N>0$和$M>0$,使得当$|x|>N$时有$|f(x)|\le M$。
\end{property}
\vpause
\begin{example*}
设$f(x)=1-5/x$,则$\limit_{x\to\infty}f(x)=1$,\pause 此时当$|x|>5$时有$|f(x)|\le2$。
\end{example*}
\end{iframe}

\begin{iframe}
\frametitle{函数极限的性质}
\begin{property}[局部保号性]
如果$\limit_{x\to\infty}f(x)=A$,且$A>0$(\bold{或$A<0$}),则存在$N>0$,
使得当$|x|>N$时有$f(x)>\frac A2>0$(\bold{或$f(x)<\frac A2<0$})。
\end{property}
\vpause
\begin{example*}
设$f(x)=1-5/x$,则$\limit_{x\to\infty}f(x)=1>0$,\pause 此时当$|x|>5/2$时,有$f(x)>1/2>0$。
\end{example*}
\end{iframe}

\begin{frame}
\frametitle{函数极限的性质}
\begin{theorem*}[保号性]
设$f(x)\ge0$(\bold{或$f(x)\le0$}),且$\limit_{x\to\infty}f(x)=A$,则$A\ge0$(\bold{或$A\le0$})。
\end{theorem*}
\vpause
\begin{corollary*}
如果函数$g(x)  \ge h(x)$,而且$\limit_{x\to\infty}g(x)=A$,$\limit_{x\to\infty}h(x)=B$,则有$A\ge B$。
\end{corollary*}
\end{frame}

\begin{frame}
\frametitle{前面两节复习题}
\begin{review}
求下列极限:
\begin{enumlite}
  \item $\limit_{n\to\infty}\dfrac{(-1)^n-2n^2}{n^2+(-1)^n}$;
  \item $\limit_{x\to\infty}\dfrac{2x^2+3}{x^2+2x+1}$;\pause
  \item $\limit_{x\to+\infty}\dfrac{2^x+4}{4^x+1}$。
\end{enumlite}
\end{review}
\end{frame}

\section{函数的极限II}

\subsection{函数极限的定义}

\begin{oframe}
\frametitle{函数极限的例子}
\begin{enumerate}[<+->]
  \item $y=c$
  \begin{itemize}
    \item 当 $x\to 1$ 时,$y\to c$
  \end{itemize}
  \item $y=x$
  \begin{itemize}
    \item 当 $x\to 2$ 时,$y\to 2$
  \end{itemize}
  \item $y=2x+1$
  \begin{itemize}
    \item 当 $x\to 3$ 时,$y\to 7$
  \end{itemize}
  \item $y=\sqrt{x}$
  \begin{itemize}
    \item 当 $x\to 4$ 时,$y\to 2$
  \end{itemize}
\end{enumerate}
\end{oframe}

\begin{iframe}
\frametitle{函数极限的例子}
\begin{enumerate}[<+->]
  \item $y=c$
  \begin{itemize}
    \item 当 $x\to x_0$ 时,$y\to c$
  \end{itemize}
  \item $y=x$
  \begin{itemize}
    \item 当 $x\to x_0$ 时,$y\to x_0$
  \end{itemize}
  \item $y=2x+1$
  \begin{itemize}
    \item 当 $x\to x_0$ 时,$y\to 2x_0+1$
  \end{itemize}
  \item $y=\sqrt{x}$
  \begin{itemize}
    \item 当 $x\to x_0$ 时,$y\to\sqrt{x_0}$
  \end{itemize}
\end{enumerate}
\end{iframe}

\begin{oframe}
\frametitle{函数的极限($x\to x_0$)}
\begin{definition*}
设$f(x)$在$x_0$附近有定义,如果当$x$从\emph{左右}两边趋于$x_0$时,
$f(x)$都无限接近一个确定的常数$A$,则称当$x\to x_0$ 时$f(x)$以$A$为\bold{极限},记为
\[ \lim_{x\to x_0}f(x)=A. \]
%否则,称当$x\to x_0$时函数$f(x)$的极限不存在。
\end{definition*}
%\vpause
%\begin{remark*}
%此定义是不严格的,严格的定义可以见下一页。
%\end{remark*}
\end{oframe}

\begin{iframe}
\frametitle{函数的极限($x\to x_0$)}
\begin{definition}
设$f(x)$在$x_0$的某个去心邻域内有定义,如果存在常数$A$,对于任何$\epsilon>0$,总存在$\delta>0$,使得当$0<|x-x_0|<\delta$时,总有
\[ |f(x)-A|<\epsilon \]
则称当$x\to x_0$时$f(x)$以$A$为\bold{极限},记为
\[ \lim_{x\to x_0}f(x)=A. \]
%如果这样的常数$A$不存在,就称当$x\to x_0$时函数 $f(x)$ 的极限不存在.
\end{definition}
\end{iframe}

\begin{iframe}
\frametitle{函数极限的例子}
\begin{example*}
证明$\limit_{x\to x_0}x=x_0$。
\end{example*}
\vpause
\begin{proof}
$\forall\epsilon>0$,取$\delta=\epsilon$,则当$0<|x-x_0|<\delta$时,就有
\begin{align*}
|f(x)-A|=|x-x_0|<\epsilon
\end{align*}
所以$\limit_{x\to x_0}x=x_0$。
\end{proof}
\end{iframe}

\begin{jframe}
\frametitle{函数极限的例子}
\begin{example*}
证明$\limit_{x\to x_0}\sqrt{x}= \sqrt{x_0}$($x_0>0$)。
\end{example*}
\vpause
\begin{proof}
$\forall\epsilon>0$,取$\delta=\min\{x_0,\sqrt{x_0}\,\epsilon\}$,则当$0<|x-x_0|<\delta$时,有
\begin{align*}
\left|\sqrt{x}-\sqrt{x_0}\right|&=\left|\frac{x-x_0}{\sqrt{x}+\sqrt{x_0}}\right|\\
&\le\frac{|x-x_0|}{\sqrt{x_0}}<\frac{\sqrt{x_0}\,\epsilon}{\sqrt{x_0}}=\epsilon
\end{align*}
\end{proof}
\end{jframe}

\begin{sframe}
\frametitle{详细解释}
\tikzset{
  box/.style =
  { rectangle, rounded corners=5pt,
    minimum width=40pt, minimum height=20pt, inner sep=5pt,
    draw=accent2, %fill=lightgray
  },
  conn/.style = { -stealth,double=white,double distance=1pt }
}
\begin{tikzpicture}[thick]
  \node[box] (a0) at (4.5,6) {$\delta=\min\{x_0,\sqrt{x_0}\,\epsilon\}$};
  \node[box] (a1) at (0,3) {$\delta\le x_0$};
  \node[box] (b0) at  (4.5,3) {$0<|x-x_0|<\delta$};
  \node[box] (a2) at (9.5,3) {$\delta\le\sqrt{x_0}\,\epsilon$};
  \node[box] (c1) at (2,0) {$\mathring{U}(x_0,\delta)\in D$ (定义域)};
  \node[box] (c2) at (8,0) {$|x-x_0|<\sqrt{x_0}\,\epsilon$};
  \draw[conn] (a0) -- (a1);
  \draw[conn] (a0) -- (a2);
  \draw[conn] (a1) .. controls (0,2.2) and (2,2) .. (2,1.5) -- (c1);
  \draw[conn] (b0) .. controls (3,2.2) and (2,2) .. (2,1.5) -- (c1);
  \draw[conn] (a2) .. controls (9.5,2.2) and (8,2) .. (8,1.5) -- (c2);
  \draw[conn] (b0.south) .. controls (4.5,2.2) and (8,2) .. (8,1.5) -- (c2);
\end{tikzpicture}
\end{sframe}

\begin{frame}
\frametitle{函数极限的基本公式II}
\noindent\fbox{\parbox{0.956\textwidth}{%
\begin{theorem*}[初等函数的连续性]
如果初等函数$f(x)$在$x_0$的某个邻域有定义,则有
\[ \lim_{x\to x_0}f(x)=f(x_0). \]
\end{theorem*}%
}}
\end{frame}

\begin{frame}
\begin{example*}
对于六种基本初等函数,我们有这些极限:
\begin{enumerate}[<+->]
  \item $\limit_{x\to x_0}c=c$;
  \item $\limit_{x\to 2}x^3=2^3=8$;
  \item $\limit_{x\to 3}e^x=e^3$;
  \item $\limit_{x\to 9}\log_3x=\log_3 9=2$;
  \item $\limit_{x\to \frac{\pi}6}\sin x=\sin\frac{\pi}6=\frac12$;
  \item $\limit_{x\to 1}\arctan x=\arctan1=\frac{\pi}4$。
\end{enumerate}
\end{example*}
\end{frame}

\begin{frame}
\frametitle{函数的极限($x\to x_0$)}
\begin{remark*}
$\limit_{x\to x_0}f(x)$与$f(x_0)$未必总是相等。\pause
\end{remark*}
\pause
\begin{example}
设$f(x)=\left\{\begin{array}{ll}x+1, & x\neq 1; \\ 3, & x=1.\end{array}\right.$
则$\limit_{x\to 1}f(x)=2$。
\end{example}
\pause
\begin{remark*}
即使$f(x)$在$x_0$处无定义,极限$\limit_{x\to x_0}f(x)$仍可能存在。
\end{remark*}
\pause
\begin{example}
函数极限$\limit_{x\to 1}\dfrac{x^2-1}{x-1}=2$。
\end{example}
\pause
\begin{example}
函数极限$\limit_{x\to 1}\dfrac{x+1}{x-1}$不存在。
\end{example}
\end{frame}

\subsection{函数极限的运算}

\begin{frame}[shrink=6]
\frametitle{四则运算法则}
\begin{theorem}
如果$\limit_{x\to x_0}f(x)=A$且$\limit_{x\to x_0}g(x)=B$,那么
\begin{enumerate}
  \item $\limit_{x\to x_0}(f(x) \pm g(x)) = \limit_{x\to x_0}f(x) \pm \limit_{x\to x_0}g(x) = A \pm B$
  \item $\limit_{x\to x_0}(f(x) \cdot g(x))=\limit_{x\to x_0}f(x) \cdot \limit_{x\to x_0}g(x) = A \cdot B$
  \item $\limit_{x\to x_0}\dfrac{f(x)}{g(x)}=\dfrac{\limit_{x\to x_0}f(x)}{\limit_{x\to x_0}g(x)}= \dfrac{A}{B}$
       (要求分母不为零)
\end{enumerate}
\end{theorem}
\vpause
\begin{corollary*}
$\limit_{x\to x_0}\big(c\cdot f(x)\big) = c \limit_{x\to x_0}f(x)$。
\end{corollary*}
\end{frame}

\begin{jframe}[plain]
\frametitle{四则运算法则 \optstar}
\small
\begin{proof}
%(1) 参考数列极限的情形,略。\par
%(2) 参考$x\to\infty$的情形,略。\par
我们只证明(3)。 事实上,利用(2)我们只需要证明
$$\limit_{x\to x_0}\dfrac{1}{g(x)} = \dfrac{1}{B}.\eqno{(*)}$$
%因为此时由(2)自然有
%$$\limit_{x\to x_0}\dfrac{f(x)}{g(x)}=\limit_{x\to x_0}\left(f(x)\cdot\dfrac{1}{g(x)}\right)=A\cdot\dfrac{1}{B}=\frac AB.$$\par
现在我们来证明$(*)$。由$\limit_{x\to x_0}g(x)=B$,可以知道,$\forall\epsilon_1>0$,$\exists\delta_1>0$,使得当$0<|x-x_0|<\delta_1$时有
$|g(x)-B|<\epsilon_1.$
\par
再由局部保号性,$\exists\delta_2>0$,使得当 $0<|x-x_0|<\delta_2$ 时有
$|g(x)|>|B|/2.$
\par
因此,$\forall\epsilon>0$,取 $\delta=\min\{\delta_1,\delta_2\}$ 和 $\epsilon_1=|B|^2\epsilon/2$,则当$0<|x-x_0|<\delta$时有
$$\left|\frac1{g(x)}-\frac1B\right|=\left|\frac{g(x)-B}{Bg(x)}\right|<\frac{\epsilon_1}{|B|\cdot|B|/2}=\epsilon.$$
%这就证明了结论。
\end{proof}
\end{jframe}

\begin{frame}
\frametitle{函数的极限($x\to x_0$)}
\begin{example}
求函数极限$\limit_{x\to1}(3x^2-2x+1)$。
\end{example}
\vpause
\begin{solution}
$\begin{aligned}[t]
\text{原式}&=3\limit_{x\to1}x^2-2\limit_{x\to1}x+\limit_{x\to1}1\\
           &=3\times1^2-2\times1+1=2
\end{aligned}$
\end{solution}
\end{frame}

\begin{frame}
\frametitle{函数的极限($x\to x_0$)}
\begin{example}
求函数极限$\limit_{x\to3}\dfrac{x^2-2x-3}{x^2-9}$。
\end{example}
\vpause
\begin{solution}
$\begin{aligned}[t]
\text{原式}&=\limit_{x\to3}\dfrac{(x+1)(x-3)}{(x+3)(x-3)}=\limit_{x\to3}\dfrac{x+1}{x+3}\\
&=\limit_{x\to3}\dfrac{3+1}{3+3}=\frac23
\end{aligned}$
\end{solution}
\end{frame}

\begin{frame}
\frametitle{函数的极限($x\to x_0$)}
\begin{example}
求函数极限$\limit_{x\to4}\dfrac{\sqrt{x}-2}{x-4}$。
\end{example}
\vpause
\begin{solution}
原式\unskip$\begin{aligned}[t]
&=\limit_{x\to4}\dfrac{(\sqrt{x}-2)(\sqrt{x}+2)}{(x-4)(\sqrt{x}+2)}\\
&=\limit_{x\to4}\dfrac{x-4}{(x-4)(\sqrt{x}+2)}=\limit_{x\to4}\dfrac{1}{\sqrt{x}+2}\\
&=\limit_{x\to4}\dfrac{1}{\sqrt4+2}=\frac14
\end{aligned}$
\end{solution}
\end{frame}

\begin{frame}
\frametitle{函数的极限($x\to x_0$)}
\begin{exercise}
求下列函数极限:
\begin{enumlite}
  \item $\limit_{x\to0}(x+2\ln(1+x)+e^x+2)$;
  \item $\limit_{x\to1}\dfrac{2x-2}{\sqrt{x}-1}$;
  \item $\limit_{x\to3}\dfrac{x^2-x-6}{x^2-2x-3}$。
\end{enumlite}
\end{exercise}
\end{frame}

\begin{frame}
\frametitle{函数的极限($x\to x_0$)}
\begin{example}
求函数极限$\limit_{x\to1}\left(\dfrac1{1-x}-\dfrac2{1-x^2}\right)$。
\end{example}
\vpause
\begin{solution}
原式\unskip$\begin{aligned}[t]
&=\limit_{x\to1}\left(\dfrac{1+x}{1-x^2}-\dfrac{2}{1-x^2}\right)\\
&=\limit_{x\to1}\dfrac{1+x-2}{1-x^2}=\limit_{x\to1}\dfrac{x-1}{1-x^2}\\
&=\limit_{x\to1}\dfrac{x-1}{(1-x)(1+x)}=-\limit_{x\to1}\dfrac{1}{1+x}\\
&=-\dfrac{1}{1+1}=-\frac12
\end{aligned}$
\end{solution}
\end{frame}

\subsection{函数极限的性质}

\setcounter{property}{0}

\begin{iframe}
\frametitle{函数极限的性质}
\begin{property}[唯一性]
若$\limit_{x\to x_0}f(x)$存在,则极限唯一。
\end{property}
\end{iframe}

\begin{iframe}[shrink=4]
\frametitle{函数极限的性质}
\begin{property}[局部有界性]
如果$\limit_{x\to x_0}f(x)=A$,则存在$\delta>0$和$M>0$,使得当$0<|x-x_0|<\delta$时有$|f(x)|\le M$。
\end{property}
\pause
\begin{proof}
取$\epsilon=1$,则存在$\delta>0$,使得当$0<|x-x_0|<\delta$时有$|f(x)-A|<\epsilon=1$。此时
\begin{align*}
|f(x)|&=|(f(x)-A)+A|\\
      &\le|f(x)-A|+|A|<1+|A|
\end{align*}
取$M=1+|A|$,就得到函数极限的局部有界性。
\end{proof}
\pause
\begin{example*}
设$f(x)=1/x$,则$\limit_{x\to1}f(x)=1$,\pause 此时当$0<|x-1|<1/2$时有$|f(x)|\le2$。
\end{example*}
\end{iframe}

\begin{iframe}
\frametitle{函数极限的性质}
\begin{property}[局部保号性]
如果$\limit_{x\to x_0}f(x)=A$,且$A>0$(\bold{或$A<0$}),则存在$\delta>0$,
使得当$0<|x-x_0|<\delta$时有$f(x)>\frac A2>0$(\bold{或$f(x)<\frac A2 < 0$})。
\end{property}
\vpause
\begin{proof}
取$\epsilon=A/2$,则存在$\delta>0$,使得当$0<|x-x_0|<\delta$时有$|f(x)-A|<\epsilon=A/2$。
此时$f(x)>A/2>0$。
\end{proof}
\vpause
\begin{example*}
设$f(x)=2x-1$,则$\limit_{x\to1}f(x)=1>0$,\pause 此时当$0<|x-1|<1/4$时,有$f(x)>1/2>0$。
\end{example*}
\end{iframe}

\begin{frame}
\frametitle{函数极限的性质}
\begin{theorem*}[保号性]
设$f(x)\ge0$(\bold{或$f(x)\le0$}),且$\limit_{x\to x_0}f(x)=A$,则$A\ge0$(\bold{或$A\le0$})。
\end{theorem*}
\vpause
\begin{corollary*}
如果函数$g(x)  \ge h(x)$,而且$\limit_{x\to x_0}g(x)=A$,$\limit_{x\to x_0}h(x)=B$,则有$A\ge B$。
\end{corollary*}
\end{frame}

\subsection{左极限与右极限}

\begin{oframe}
\frametitle{左极限和右极限}
\begin{definition*}
设函数$f(x)$在$x_0$的\bold{左侧}有定义.如果$x$从$x_0$ \bold{左侧}趋于$x_0$时,$f(x)$无限接近一个确定的常数$A$,
则称当$x\to x_0$ 时$f(x)$以$A$为\bold{左极限},记为
\[ \lim_{x\to \bold{x_0^-}}f(x)=A \text{\quad 或\quad} f(\bold{x_0^-})=A. \]
\end{definition*}
\pause
\vfill\hrule\vfill
\begin{definition*}
设函数$f(x)$在$x_0$的\bold{右侧}有定义.如果$x$从$x_0$ \bold{右侧}趋于$x_0$时,$f(x)$无限接近一个确定的常数$A$,则称当$x\to x_0$ 时$f(x)$ 的以$A$为\bold{右极限},记为
\[ \lim_{x\to \bold{x_0^+}}f(x)=A \text{\quad 或\quad} f(\bold{x_0^+})=A. \]
\end{definition*}
\end{oframe}

\begin{iframe}%[shrink=1]
\frametitle{左极限和右极限}
\vskip-0.5em
\begin{definition*}
设$f(x)$在点$x_0$ \bold{左邻域}有定义,如果对任何$\epsilon>0$,总存在$\delta>0$,使得当$\pmb x_0-\delta<x<x_0$ 时有
\[ |f(x)-A|<\epsilon, \]
则称当$x\to x_0$时$f(x)$以$A$为\bold{左极限},记为
\[ \lim_{x\to \bold{x_0^-}}f(x)=A \text{\quad 或\quad} f(\bold{x_0^-})=A. \]
\end{definition*}
\pause
\vfill\hrule\vfill
\begin{definition*}
设$f(x)$在点$x_0$ \bold{右邻域}有定义,如果对任何$\epsilon>0$,总存在$\delta>0$,使得当$\pmb x_0<x<x_0+\delta$ 时有
\[ |f(x)-A|<\epsilon, \]
则称当$x\to x_0$时$f(x)$以$A$为\bold{右极限},记为
\[ \lim_{x\to \bold{x_0^+}}f(x)=A \text{\quad 或\quad} f(\bold{x_0^+})=A. \]
\end{definition*}
\end{iframe}

\begin{frame}
\begin{theorem}
极限存在等价于左右极限都存在且相等,即%有
\[\lim_{x\to x_0}f(x)=A \Longleftrightarrow \lim_{x\to x_0^-}f(x)=A=\lim_{x\to x_0^+}f(x)\]
\end{theorem}
\vpause
\begin{example}
设$f(x)=|x|$,研究函数极限$\limit_{x\to0}f(x)$。
\end{example}
\pause
\begin{example}
设$f(x)=\left\{\begin{array}{cc}1,&x<0\\x,&x\ge0\end{array}\right.$,则$\limit_{x\to0}f(x)$不存在。
\end{example}
\pause
\begin{example}
设$f(x)=\left\{\begin{array}{ll}x+1,&x<1\\x^2-x+2,&x>1\end{array}\right.$,求$\limit_{x\to1}f(x)$。
\end{example}
\vpause
\begin{remark*}
研究当$x\to x_0$时函数$f(x)$的左右极限,不必要求$f(x)$在$x_0$处有定义。
\end{remark*}
\end{frame}

\begin{frame}
\frametitle{左极限和右极限}
\begin{exercise}
已知函数$f(x)=\begin{cases}
x^2+x+1, & x<0 \\
e^x, & 0<x<1 \\
x^2-1, & x>1
\end{cases}$;
判断极限$\limit_{x\to0}f(x)$和$\limit_{x\to1}f(x)$是否存在,若存在求出该极限。
\end{exercise}
\end{frame}

\begin{frame}
\frametitle{前面三节复习题}
\begin{review}
求下列极限:
\begin{enumlite}
  \item $\limit_{n\to\infty}\dfrac{n^2+(-1)^n}{2n^2+n+(-1)^n}$;
  \item $\limit_{x\to{\bm\infty}}\dfrac{2x^2+3x-5}{x^2+2x-3}$;
  \item $\limit_{x\to{\bm1}}\dfrac{2x^2+3x-5}{x^2+2x-3}$。
\end{enumlite}
\end{review}
\end{frame}

\section{无穷小量与无穷大量}

\subsection{无穷小量}

\begin{frame}
\frametitle{无穷小量}
\begin{definition*}
如果$\limit_{x\to x_0}f(x)=0$,就称$f(x)$为$x\to x_0$时的\bold{无穷小量}.
\end{definition*}
\vpause
\begin{definition*}
如果$\limit_{x\to\infty}f(x)=0$,就称$f(x)$为$x\to\infty$时的\bold{无穷小量}.
\end{definition*}
\vpause
\begin{remark*}
类似地,可以定义 $x\to-\infty$、$x\to+\infty$
%、$x\to x_0^-$ 和 $x\to x_0^+$
时的无穷小量。
\end{remark*}
\end{frame}

\begin{frame}
\frametitle{无穷小量}
\begin{example}
$0$、$x$、$x^2$、$\sin x$、$1-\cos x$、$\sqrt{1+x}-1$ 和 $e^x-1$ 都是 $x\to0$时的无穷小量。
\end{example}
\vpause
\begin{example}
函数$\dfrac1x$、$\dfrac2{1+x}$ 和 $\dfrac{x}{x^2+1}$ 都是 $x\to\infty$时的无穷小量。
\end{example}
\vpause
\begin{example}
函数$y=\dfrac{x-1}{x^2-1}$ 何时是无穷小量?
\end{example}
\end{frame}

\begin{frame}
\frametitle{无穷小量的运算}
\begin{enumerate}
  \item 两个无穷小量的和差还是无穷小量.
  \item 两个无穷小量的乘积还是无穷小量.
  \item 无穷小量和有界函数的乘积还是无穷小量.
\end{enumerate}
\pause
\begin{example}
求函数极限$\limit_{x\to\infty}\dfrac{\sin x}{x}$。
\end{example}
%\pause
%\begin{example}
%求函数极限$\limit_{x\to\infty}\dfrac{\sin x+\cos x}{x^2}$.
%\end{example}
\pause
\begin{example}
求函数极限$\limit_{x\to\infty}\dfrac{3x+1}{x+\sin x}$。
\end{example}
\pause
\begin{example}
求函数极限$\limit_{x\to0}x\sin\dfrac{1}{x}$。
\end{example}
\end{frame}

\begin{frame}
\frametitle{无穷小量与有界函数的乘积}
\begin{center}
\begin{tikzpicture}[thick]
\path[use as bounding box] (-0.2,-3.5) -- (11,3.5);
\draw[very thin,color=gray] (0,-3) grid (10,3);
\draw[thin,->] (-0.2,0) -- (10.5,0) node[right]{$x$};
\draw[thin,->] (0,-3.5) -- (0,3.5) node[above]{$y$};
\draw[color=accent2] plot[domain=0.3:10,samples=50] (\x,{1/\x});
\draw[color=accent2] plot[domain=0.3:10,samples=50] (\x,{-1/\x});
\onslide+<2->{\draw[color=accent3] plot[domain=0.3:10,samples=50] (\x,{sin(\x r)/\x});}
\node[text width=10em] at (7.6,3) {\bold{绿线}:$y=\pm\dfrac1x\to0$};
\onslide+<2->{\node[text width=10em] at (7.6,1.5){\warn{红线}:$y=\dfrac{\sin x}{x}\to0$};} 
\end{tikzpicture}
\end{center}
\end{frame}

\begin{frame}
\begin{exercise}
求下列函数极限:
\begin{enumlite}
  \item $\limit_{x\to\infty}\dfrac{\cos x}{x^2}$;
  \item $\limit_{x\to\infty}\dfrac{x^2+\cos x}{\cos x +3x^2}$。
\end{enumlite}
\end{exercise}
\end{frame}

\subsection{无穷大量}

\begin{oframe}
\frametitle{无穷大量($x\to\infty$)}
如果对任何$M>0$,只要$|x|$足够大,总有$|f(x)|>M$,则称$f(x)$为$x\to\infty$时的\bold{无穷大量}.记为
$$\lim_{x\to\infty}f(x)=\infty \eqno{\digitcircled{1}}$$
\pause
类似地可以定义$x\to-\infty$的无穷大量,记为
$$\limit_{x\to-\infty}f(x)=\infty \eqno{\digitcircled{2}}$$
\pause
以及和$x\to+\infty$时的无穷大量,记为
$$\limit_{x\to+\infty}f(x)=\infty \eqno{\digitcircled{3}}$$
\pause
\bold{结论:\digitcircled{1} 成立等价于 \digitcircled{2} 和 \digitcircled{3} 同时成立。}
\end{oframe}

\begin{iframe}
\frametitle{无穷大量($x\to\infty$)}
\begin{definition*}
设$f(x)$在$|x|$大于某个正数时有定义。如果对任何$M>0$,总存在$N>0$,使得只要$|x|>N$,
就有$|f(x)|>M$,则称$f(x)$为$x\to\infty$时的\bold{无穷大量},记为
$$\lim_{x\to\infty}f(x)=\infty \eqno{\digitcircled{1}}$$
\end{definition*}
\pause
类似地可以定义$x\to-\infty$的无穷大量,记为
$$\limit_{x\to-\infty}f(x)=\infty \eqno{\digitcircled{2}}$$
\pause
以及和$x\to+\infty$时的无穷大量,记为
$$\limit_{x\to+\infty}f(x)=\infty \eqno{\digitcircled{3}}$$
\pause
\bold{结论:\digitcircled{1} 成立等价于 \digitcircled{2} 和 \digitcircled{3} 同时成立。}
\end{iframe}

\begin{frame}
\frametitle{无穷大量($x\to\infty$)}
\begin{example}
$x$,$x^2$,$x+1$ 都是$x\to\infty$时的无穷大量,即
\begin{align*}
  &\limit_{x\to\infty} x = \infty, \\
  &\limit_{x\to\infty} x^2 = \infty, \\
  &\limit_{x\to\infty}(x+1) = \infty.
\end{align*}
\end{example}
\vpause
\begin{example}
$e^x$ 是$x\to+\infty$时的无穷大量,即
\[ \limit_{x\to+\infty} e^x = \infty. \]
\end{example}
\end{frame}

\begin{oframe}
\frametitle{无穷大量($x\to x_0$)}
如果对任何$M>0$,只要$x$足够接近$x_0$,总有$|f(x)|>M$,就称$f(x)$为$x\to x_0$时的\bold{无穷大量}.记为
$$\lim_{x\to x_0}f(x)=\infty$$
\end{oframe}

\begin{iframe}
\frametitle{无穷大量($x\to x_0$)}
\begin{definition*}
设函数$f(x)$在$x_0$的某个去心邻域有定义。如果对任何给定的$M>0$,总存在$\delta>0$,
使得只要$0<|x-x_0|<\delta$,就有$|f(x)|>M$,则称$f(x)$为$x\to x_0$ 时的\bold{无穷大量},记为$\limit_{x\to x_0}f(x)=\infty$。
\end{definition*}
\end{iframe}

\begin{frame}
\frametitle{无穷大量($x\to x_0$)}
\begin{example}
$\dfrac1{x}$ 和 $\dfrac{x+1}{x^2}$ 是$x\to 0$时的无穷大量。
\end{example}
\vpause
\begin{example}
$\dfrac1{x-1}$ 和 $\dfrac{x+2}{x^2-1}$ 是$x\to 1$时的无穷大量。
\end{example}
\vpause
\begin{example}
函数$y=\dfrac{x-1}{x^2-1}$ 何时是无穷大量?
\end{example}
\end{frame}

\begin{frame}
\frametitle{无穷大量与无穷小量的关系}
\begin{theorem}
无穷大量$y$的倒数$\dfrac1y$为无穷小量,而非零无穷小量$y$的倒数$\dfrac1y$为无穷大量.
\end{theorem}
\pause
\begin{example}
$\limit_{x\to\infty}\dfrac{3x+1}{2x^2+1}=0$\quad$\Longrightarrow$\quad$\limit_{x\to\infty}\dfrac{2x^2+1}{3x+1}=\infty$。
\end{example}
\end{frame}

\begin{frame}
\frametitle{有理分式的极限}
\noindent\begin{align*}
&\lim_{x\to\infty}\frac{2x+1}{3x+1}=\frac23
&&\lim_{x\to\infty}\frac{2x^{\lead2}+1}{3x^{\lead2}+1}=\frac23\\
&\lim_{x\to\infty}\frac{2x+1}{3x^{\lead2}+1}=0
&&\lim_{x\to\infty}\frac{2x^{\lead2}+1}{3x+1}=\infty
\end{align*}
\pause\hrule
\[\bold{\lim_{x\to\infty}\frac{a_0x^n+a_1x^{n-1}+\cdots+a_n}{b_0x^m+b_1x^{m-1}+\cdots+b_m}
=\begin{cases}
  \dfrac{a_0}{b_0}, & n=m;\\
  0, & n<m;\\
  \infty, & n>m. 
\end{cases}}\]
\end{frame}

\begin{frame}
\frametitle{前面四节复习题}
\begin{review}
求下列函数极限:
\begin{enumlite}
  \item $\limit_{x\to1}\dfrac{x-1}{\sqrt{x}-1}$;
  \item $\limit_{x\to-1}\left(\dfrac1{x+1}+\dfrac2{x^2-1}\right)$;
  \item $\limit_{x\to\infty}\dfrac{\cos x+2x}{x+\cos x}$。
\end{enumlite}
\end{review}
\end{frame}

%\begin{frame}
%\frametitle{数学笑话}
%\lead{题目}\quad 已知$\dfrac{1}{\infty}=0$,证明$\dfrac{1}{0}=\infty$。\par
%\lead{证明}\quad
%因为$\dfrac{1}{\infty}=0$,两边同时旋转得
%\[ -18=0 \]
%两边同时$+8$得
%\[ -10=8 \]
%两边同时旋转得
%\[\dfrac{1}{0}=\infty\]
%\end{frame}

\section{两个重要极限}

\begin{frame}
\frametitle{本节基本内容}
\begin{tabular}{c<{\qquad}c}
  极限存在准则I & 极限存在准则II \\
  $\Downarrow$  & $\Downarrow$ \\
  重要极限I     & 重要极限II \\
  $\pmb\limit_{x\to0}\dfrac{\sin x}x=1$ &
  $\pmb\limit_{x\to\infty}\left(1+\dfrac1x\right)^x=\e$
\end{tabular}
\end{frame}

\subsection{重要极限I}

\begin{frame}
\frametitle{极限存在准则I}
\begin{theorem*}[极限存在准则I]%[三明治定理]
如果数列 $x_n \le y_n \le z_n$,而且 $\limit_{n\to\infty}x_n=A$,$\limit_{n\to\infty}z_n=A$,则有 $\limit_{n\to\infty}y_n=A$.
\end{theorem*}
\vpause
\begin{example}
求$\limit_{n\to\infty}\left(\dfrac{n}{n^2+1}+\dfrac{n}{n^2+2}+\cdots+\dfrac{n}{n^2+n}\right)$
\end{example}
\vpause
\begin{remark*}
在上述定理中,如果不等式$x_n \le y_n \le z_n$仅在$n>N$时成立,结论不变。
\end{remark*}
\end{frame}

\begin{sframe}
\frametitle{极限存在准则I}
\begin{remark*}
用极限存在准则I很难求出下面的数列极限:
\begin{align*}
&\limit_{n\to\infty}\left(\dfrac1{n+1}+\dfrac1{n+2}+\cdots+\dfrac1{n+n}\right)\\
=&\int_0^1\frac1{1+x}\d{x}=\ln2
\end{align*}
\end{remark*}
\end{sframe}

\begin{frame}
\frametitle{极限存在准则I}
\begin{theorem*}[极限存在准则I]%[三明治定理]
如果 $f(x) \le g(x) \le h(x)$,且 $\limit_{x\to x_0}f(x)=A$,$\limit_{x\to x_0}h(x)=A$,则有$\limit_{x\to x_0}g(x)=A$.
\end{theorem*}
\pause
\begin{remark*}
若将$x\to x_0$全部改为$x\to\infty$,定理仍成立。
\end{remark*}
\vpause
\begin{remark*}
在上述定理中,如果不等式$f(x) \le g(x) \le h(x)$仅在$x_0$的某个去心邻域上成立,结论不变。
\end{remark*}
\end{frame}

\begin{frame}
\frametitle{重要极限I}
\noindent\fbox{\parbox{0.956\textwidth}{%
\[ \bold{\limit_{x\to0}\dfrac{\sin x}{x}=1} \]
}}
\vpause
一般地,如果当$x\to0$时,$\phi(x)\to0$,则有\par
\noindent\fbox{\parbox{0.956\textwidth}{%
\[ \bold{\limit_{x\to0}\dfrac{\sin\phi(x)}{\phi(x)}=1} \]
}}
\end{frame}

\begin{frame}
\frametitle{重要极限I}
\begin{example}
求极限$\limit_{x\to0}\dfrac{\tan x}{x}$.
\end{example}
\pause
\begin{solution}
原式\noindent$\begin{aligned}[t]
&=\lim_{x\to0}\dfrac{\sin x}{x\cos x}=\lim_{x\to0}\left(\dfrac{\sin x}{x}\cdot\dfrac1{\cos x}\right)\\
&=\lim_{x\to0}\dfrac{\sin x}{x}\cdot\lim_{x\to0}\dfrac1{\cos x}=1\times\dfrac11=1
\end{aligned}$
\end{solution}
\end{frame}

\begin{frame}
\frametitle{重要极限I}
\begin{example}
求极限$\limit_{x\to0}\dfrac{\sin 3x}{x}$.
\end{example}
\pause
\begin{solution}
原式\noindent$\begin{aligned}[t]
&=\lim_{x\to0}\left(3\cdot\dfrac{\sin 3x}{3x}\right)\\
&=3\lim_{x\to0}\dfrac{\sin 3x}{3x}=3\times1=3
\end{aligned}$
\end{solution}
\end{frame}

\begin{frame}
\frametitle{重要极限I}
\begin{example}
求极限$\limit_{x\to0}\dfrac{\sin x}{\sin 4x}$.
\end{example}
\pause
\begin{solution}
原式\noindent$\begin{aligned}[t]
&=\lim_{x\to0}\left(\dfrac14\cdot\dfrac{\sin x}{x}\cdot\dfrac{4x}{\sin 4x}\right)\\
&=\dfrac14\cdot\lim_{x\to0}\dfrac{\sin x}{x}\cdot\lim_{x\to0}\dfrac{4x}{\sin 4x}\\
&=\dfrac14\times1\times1=\dfrac14
\end{aligned}$
\end{solution}
\end{frame}

\begin{oframe}
\frametitle{重要极限I}
\begin{exercise*}
求下列函数极限:
\begin{multicols}{2}
\begin{enumlite}
  \item $\limit_{x\to0}\dfrac{\sin 5x}{4x}$
  \item $\limit_{x\to0}\dfrac{\sin 2x}{\sin 3x}$
\end{enumlite}
\end{multicols}
\end{exercise*}
\end{oframe}

\begin{iframe}
\frametitle{重要极限I}
\begin{exercise}
求下列函数极限:
\begin{enumlite}
  \item $\limit_{x\to0}\dfrac{\sin(2x^2)}{3x^2}$
  \item $\limit_{x\to0}\dfrac{\sin 3x}{\tan 5x}$
  \item $\limit_{x\to0^+}\dfrac{\sin\sqrt{x}\;\tan\sqrt{x}}{x}$
\end{enumlite}
\end{exercise}
\end{iframe}

\begin{frame}
\frametitle{重要极限I}
\begin{example}
求极限$\limit_{x\to0}\dfrac{\arcsin x}{x}$.
\end{example}
\begin{example}
求极限$\limit_{x\to0}\dfrac{\arctan x}{x}$.
\end{example}\pause
\begin{example}
求极限$\limit_{x\to0}\dfrac{1-\cos x}{\frac12x^2}$.
\end{example}
\end{frame}

\subsection{重要极限II}

\begin{frame}
\frametitle{极限存在准则II}
\begin{theorem*}[极限存在准则II]
单调且有界的数列必定收敛.\pause\unskip
\begin{enumerate}
  \item 单调\bold{增加}且有\bold{上界}的数列必定收敛。
  \item 单调\bold{减少}且有\bold{下界}的数列必定收敛。
\end{enumerate}
\end{theorem*}
\vpause
\begin{remark*}
若数列是某一项开始单调变化,结论仍然成立。
\end{remark*}
\end{frame}

\begin{frame}
\frametitle{重要极限II}
\renewcommand*{\arraystretch}{0.95}%
\begin{columns}[onlytextwidth]
\column{0.48\textwidth}
\noindent$$\begin{array}{|r<{\quad}|>{\quad}l|}
  \hline
  \pmb n & \pmb\left(1+\frac1n\right)^n \\[2pt]
  \hline
  1 & 2 \\
  2 & 2.250 \\
  3 & 2.370 \\
  4 & 2.441 \\
  5 & 2.488 \\
  10 & 2.594 \\
  100 & 2.705 \\
  1000 & 2.717 \\
  10000 & 2.718 \\
  \hline
\end{array}$$
\column{0.08\textwidth}
\onslide<2->{$\Longrightarrow$}
\column{0.42\textwidth}
\onslide<2->{\noindent\fbox{\parbox{0.9\textwidth}{%
\[ \bold{\lim_{n\to\infty}\left(1+\frac1n\right)^n=\e}\]}}}
\end{columns}
\end{frame}

\begin{frame}
\frametitle{重要极限II}
\noindent\fbox{\parbox{0.956\textwidth}{%
\[ \bold{\lim_{x\to\infty}\left(1+\frac1x\right)^x=\e} \pause\quad\xLongrightarrow{u=1/x}
   \quad \bold{\lim_{u\to0}\big(1+u\big)^{\frac1u}=\e} \]}}
\vpause
%一般地,如果当$x\to\infty$时,$\phi(x)\to\infty$;如果当$u\to0$时,$\psi(u)\to0$,则有\par
%\noindent\fbox{\parbox{0.956\textwidth}{%
%\[ \bold{\lim_{x\to\infty}\left(1+\frac1{\phi(x)}\right)^{\phi(x)}\kern-0.6em=\e}, \pause
%   \quad\bold{\lim_{u\to0}\big(1+\psi(u)\big)^{\frac1{\psi(u)}}=\e} \]}}
一般地,如果\warn{当$x\to a$时,$\psi(x)\to0$},则有\par
\noindent\fbox{\parbox{0.956\textwidth}{\Large%
\[ \bold{\lim_{x\to a}\Big(1+\psi(x)\Big)^{\frac1{\psi(x)}}=\e} \]}}
%\par 其中 $a$ 可以是 $x_0$ 或 $\infty$。
\end{frame}

\begin{frame}
\frametitle{简单情形}
\begin{example}
求极限$\limit_{x\to\infty}\left(1+\dfrac2x\right)^x$
\end{example}
\pause
\begin{solution}
原式\unskip$\begin{aligned}[t]
&=\limit_{x\to\infty}\left(1+\dfrac2x\right)^{\frac{x}{2}\,\cdot\,2}\\
&=\limit_{x\to\infty}\left[\left(1+\dfrac2x\right)^{\frac{x}{2}}\right]^2=\e^2
\end{aligned}$
\end{solution}
\end{frame}

\begin{frame}
\frametitle{简单情形}
\begin{example}
求极限$\limit_{x\to\infty}\left(1-\dfrac1{3x}\right)^x$
\end{example}
\pause
\begin{solution}
原式\unskip$\begin{aligned}[t]
&=\limit_{x\to\infty}\left(1+\dfrac{-1}{3x}\right)^{\frac{3x}{-1}\,\cdot\,\left(-\frac13\right)}\\
&=\limit_{x\to\infty}\left[\left(1+\dfrac{-1}{3x}\right)^{\frac{3x}{-1}}\right]^{-\frac13}=\e^{-1/3}
\end{aligned}$
\end{solution}
\end{frame}

\begin{frame}
\frametitle{简单情形}
\begin{example}
求极限$\limit_{x\to0}\left(1-\dfrac{x}2\right)^{\frac1x}$
\end{example}
\pause
\begin{solution}
原式\unskip$\begin{aligned}[t]
&=\limit_{x\to0}\left(1+\dfrac{-x}{2}\right)^{\frac{2}{-x}\,\cdot\,\left(-\frac12\right)}\\
&=\limit_{x\to0}\left[\left(1+\dfrac{-x}{2}\right)^{\frac{2}{-x}}\right]^{-\frac12}=\e^{-1/2}
\end{aligned}$
\end{solution}
\end{frame}

\begin{frame}
\begin{exercise}求函数极限:
\begin{enumlite}
  %\item $\limit_{x\to\infty}\left(1+\dfrac5x\right)^x$
  \item $\limit_{x\to\infty}\left(1-\dfrac4{x}\right)^x$
  \item $\limit_{x\to0}\Big(1-2x\Big)^{\frac1{3x}}$
\end{enumlite}
\end{exercise}
\end{frame}

\begin{frame}
\frametitle{幂指情形}
\begin{example}
求极限$\limit_{x\to\infty}\left(\dfrac{x+1}{x-1}\right)^x$
\end{example}
\pause
\begin{solution}
原式$=\limit_{x\to\infty}\left(1+\dfrac2{x-1}\right)^x$\\
$\begin{aligned}[t]
&=\lim_{x\to\infty}\left(1+\dfrac2{x-1}\right)^{\!\!\frac{x-1}2\cdot\frac{2x}{x-1}}\\
&=\lim_{x\to\infty}\left[\left(1+\dfrac2{x-1}\right)^{\frac{x-1}2}\right]^{\frac{2x}{x-1}}
 =\lim_{x\to\infty}\e^{\frac{2x}{x-1}}=\e^2.
\end{aligned}$
\end{solution}
\end{frame}

\begin{frame}
\frametitle{幂指情形}
\begin{example}
求极限$\limit_{x\to0}\Big(1+3x\Big)^{\frac1{\sin x}}$
\end{example}
\pause
\begin{solution}
原式\noindent$\begin{aligned}[t]
&=\lim_{x\to0}\Big(1+3x\Big)^{\frac1{\sin x}}\\
&=\lim_{x\to0}\Big(1+3x\Big)^{\frac1{3x}\cdot\frac{3x}{\sin x}}\\
&=\lim_{x\to0}\left[\big(1+3x\big)^{\frac1{3x}}\right]^{\frac{3x}{\sin x}}\\
&=\lim_{x\to0}\e^{\frac{3x}{\sin x}}=\e^3.
\end{aligned}$
\end{solution}
\end{frame}

\begin{frame}
\begin{exercise}求函数极限:
\begin{enumlite}
  \item $\limit_{x\to0}\left(1+\sin x\right)^{\frac1{4x}}$
  \item $\limit_{x\to\infty}\left(\dfrac{x-1}x\right)^{x+1}$
  \pause
  \item $\limit_{x\to\infty}\left(\dfrac{x+1}{x-1}\right)^{2x-3}$
\end{enumlite}
\end{exercise}
\end{frame}

\begin{jframe}
\frametitle{幂指函数的极限}
\begin{theorem*}
若$\limit_{x\to\Box}u(x)=A>0$,$\limit_{x\to\Box}v(x)=B$,则有
\[ \lim_{x\to\Box}u(x)^{v(x)}=A^B.\]
\end{theorem*}
\end{jframe}

\begin{rframe}
\frametitle{常数$\mathrm{e}$的意义}
\begin{example*}
树木增长问题:假设树木的高度与细胞的个数成比例,而每个细胞平均每年分裂一次(即增加$1$倍),
则树木的高度一年后变为原来的多少倍?\pause\dotfill[$\e$倍]
\end{example*}\pause
\begin{fact*}
常数 $\mathrm{e}$ 反映了连续生长的规律.
\end{fact*}
\end{rframe}

\begin{frame}
\frametitle{常数$\mathrm{e}$的意义}
\begin{example*}
复利问题:假设银行活期存款的年利率为$0.5\%$,存入$M$元一年后最多可以得到多少钱?
\end{example*}\pause
\begin{itemize}
  \item 粗略:$M(1+0.5\%)=M\times 1.005$\pause
  \item 正常:$M\left(1+\dfrac{0.5\%}4\right)^4=M \times 1.00500938 $\pause
  \item 极端:$M\left(1+\dfrac{0.5\%}{360}\right)^{360}= M \times 1.00501249$
\end{itemize}\pause
\begin{fact*}
常数 $\mathrm{e}$ 反映了连续增长的规律.
\end{fact*}
\end{frame}

\begin{frame}
\frametitle{两个重要极限}
\begin{review}求函数极限:
\begin{enumlite}
  \item $\limit_{x\to0}\left(1-2\sin x\right)^{\frac1{3x}}$
  \item $\limit_{x\to\infty}\left(\dfrac{x^2-1}{x^2}\right)^{x\sin x}$
  \pause
  \item $\limit_{x\to\infty}\left(\dfrac{2x-1}{2x+1}\right)^{3x+1}$
\end{enumlite}
\end{review}
\end{frame}

\begin{jframe}
\frametitle{复习与提高}
\begin{example}
用极限存在准则I证明数列极限$\limit_{n\to\infty}\sqrt[n]{n}=1$。
\end{example}
\pause
\begin{proof}
令 $x_n=\sqrt[n]{n}-1$, 则$x_n>0$。当$n>1$时又有
\[ n=(1+x_n)^n\ge C_n^2x_n^2 = \frac{n(n-1)}2\,x_n^2. \]
即$x_n\le\sqrt{\frac{2}{n-1}}$.即当$n>1$时我们有
\[ 0 < x_n \le \sqrt{\frac{2}{n-1}}. \]
由极限存在准则I得$\limit_{n\to\infty}x_n=0$,即$\limit_{n\to\infty}\sqrt[n]{n}=1$。
\end{proof}
\end{jframe}

\begin{iframe}
\frametitle{复习与提高}
\begin{example}
求函数极限$\limit_{x\to0}\left(1+\cos x\right)^{\frac1{\cos x}}$。
\end{example}
\pause
\begin{solution}
因为当$x\to0$时,$\cos x$不是无穷小量,所以不能用重要极限II公式来计算。\ppause
实际上,这个函数是初等函数,且在$0$的邻域有定义,所以其极限为$(1+1)^1=2$。
\end{solution}
\end{iframe}

\begin{iframe}
\frametitle{复习与提高}
\begin{example}
设数列$x_n=\dfrac{n!}{n^n}$,研究数列的极限。
\end{example}
\vpause
\begin{example}
设数列$\{x_n\}$满足$x_1=\dfrac12$,且当$n\ge1$时有$x_{n+1}=\dfrac{1+x_n^2}{2}$。研究数列的极限。
\end{example}
\vpause
\begin{solution}
先说明数列收敛,再根据数列的递归关系求出其极限。
\end{solution}
\end{iframe}

\section{无穷小量的比较}

\subsection{无穷小量的阶}

\begin{frame}
\frametitle{无穷小量的比较}
\begin{example*}
比较$x\to0$时的三个无穷小量$x$,$2x$,$x^2$。
\end{example*}
\pause
\[\begin{array}{c|rrrrrrr}
\hline
x         & 1 & 0.1 & 0.01 & 0.001 & \cdots & \rightarrow & 0\\ \hline
2x      & 2 & 0.2 & 0.02 & 0.002  & \cdots & \rightarrow & 0\\ \hline
x^2   & 1 & 0.01 & 0.0001 & 0.000001 & \cdots & \rightarrow & 0 \\
\hline
\end{array}\]
\end{frame}

\begin{frame}
\frametitle{无穷小量的阶}
\begin{definition*}
设$\alpha$、$\beta$是同一变化过程中的两个无穷小量.
\begin{enumerate}[<+->]
  \item 称$\beta$比$\alpha$ \bold{高阶},若$\lim\frac{\beta}{\alpha}=0$,记为$\bold{\beta=o(\alpha)}$.
  \item 称$\beta$比$\alpha$ \bold{低阶},若$\lim\frac{\beta}{\alpha}=\infty$.
  \item 称$\beta$和$\alpha$ \bold{同阶},若$\lim\frac{\beta}{\alpha}=c\neq0$.
  \item[$\bigstar$] 称$\beta$和$\alpha$ \bold{等价},若$\lim\frac{\beta}{\alpha}=1$,记为$\bold{\beta\sim\alpha}$.
\end{enumerate}
\end{definition*}
\end{frame}

\begin{frame}
\frametitle{无穷小量的阶}
\begin{example}
在$x\to0$时,无穷小量$x^2$比$x$高阶。
\end{example}
\vpause
\begin{example}
在$x\to0$时,无穷小量$x^2$比$x^3$低阶。
\end{example}
\vpause
\begin{example}
在$x\to0$时,无穷小量$x^2$和$5x^2$同阶。
\end{example}
\vpause
\begin{example}
在$x\to0$时,无穷小量$x^2$和$x^2+2x^3$等价。
\end{example}
\end{frame}

\begin{iframe}
\frametitle{无穷小量的阶}
\begin{exercise}
易知 $f(x)=ax^3+bx^2+cx$ 和 $g(x)=x^2$ 均为 $x\to0$ 时的无穷小量。
\begin{enumlite}
  \item 何时 $f(x)$ 比 $g(x)$ 高阶?
  \item 何时 $f(x)$ 比 $g(x)$ 低阶?
  \item 何时 $f(x)$ 与 $g(x)$ 同阶?
  \item 何时 $f(x)$ 与 $g(x)$ 等价?
\end{enumlite}
\end{exercise}
\end{iframe}

\begin{frame}
\frametitle{常用的等价无穷小量}
当$x\to0$时,有如下这些常用的等价无穷小量:\\[0.6em]
\noindent\fbox{\parbox{0.956\textwidth}{%
\renewcommand{\arraystretch}{1.8}
\begin{tabularx}{\textwidth}{XX}
 (1)\quad $\sin x\sim x$    & (5)\quad $\ln(1+x)\sim x$ \\
 (2)\quad $\tan x\sim x$    & (6)\quad $\mathrm{e}^x-1\sim x$ \\
 (3)\quad $\arcsin x\sim x$ & (7)\quad $1-\cos x\sim \dfrac12x^2$ \\
 (4)\quad $\arctan x\sim x$ & (8)\quad $\sqrt[n]{1+x}-1\sim \dfrac{x}n$
\end{tabularx}
}}
%\begin{multicols}{2}
%\begin{itemize}
%  \item $\sin x\sim x$
%  \item $\tan x\sim x$
%  \item $\arcsin x\sim x$
%  \item $\arctan x\sim x$
%  %\vfill
%  \columnbreak
%  \item $\mathrm{e}^x-1\sim x$
%  \item $\ln(1+x)\sim x$
%  \item $1-\cos x\sim \dfrac12x^2$
%  \item $\sqrt[n]{1+x}-1\sim \dfrac{x}n$
%\end{itemize}
%\end{multicols}
\end{frame}

\begin{jframe}
\frametitle{等价无穷小量}
\begin{solution}
(8)\ $\limit_{x\to0}\dfrac{\sqrt[n]{1+x}-1}x=\limit_{t\to0}\dfrac{t}{(1+t)^n-1}=n$。\par
(5)\ $\limit_{x\to0}\dfrac{\ln(1+x)}{x}=\limit_{x\to0}\ln\big(1+x\big)^{\frac1x}\overset{\warn?}{=}\ln\e=1$。\par
(6)\ $\limit_{x\to0}\dfrac{\e^x-1}x=\limit_{t\to0}\dfrac{t}{\ln(1+t)}=1$。
\end{solution}
\end{jframe}

\begin{jframe}
\frametitle{等价无穷小量}
更一般地,我们有\par\vspace{0.5em}
(5)\ $\begin{aligned}[t]
\limit_{x\to0}\dfrac{\log_a(1+x)}{x}&=\limit_{x\to0}\log_a\big(1+x\big)^{\frac1x}\\
   &\overset{\warn?}{=}\log_a\e=\frac1{\ln a}.
\end{aligned}$\ppause
(6)\ $\begin{aligned}[t]
\limit_{x\to0}\dfrac{a^x-1}x=\limit_{t\to0}\dfrac{t}{\log_a(1+t)}=\ln a.
\end{aligned}$
\end{jframe}

\begin{jframe}
\frametitle{等价无穷小量}
(8)$\begin{aligned}[t]
 &\limit_{x\to0}\dfrac{(1+x)^a-1}x\\
=&\limit_{x\to0}\left[\dfrac{(1+x)^a-1}{\ln(1+x)^a}\cdot\dfrac{\ln(1+x)^a}{x}\right]\\
=&\limit_{x\to0}\dfrac{(1+x)^a-1}{\ln(1+x)^a}\cdot\limit_{x\to0}\dfrac{\ln(1+x)^a}{x}\\
=&\limit_{t\to0}\dfrac{t}{\ln(1+t)}\cdot\limit_{x\to0}\dfrac{a\ln(1+x)}{x}\\
=&1 \cdot a = a.
\end{aligned}$
\end{jframe}

\subsection{等价无穷小量代换}

\begin{frame}
\frametitle{等价无穷小量代换}
\begin{theorem}
设当 $x\to x_0$ 时 %$\alpha$、$\alpha'$、$\beta$、$\beta'$ 都为无穷小量,而且
$\alpha\sim\alpha'$、$\beta\sim\beta'$,则有\par
(1) $\bold{\displaystyle\lim_{x\to x_0}\alpha\gamma=\lim_{x\to x_0}\alpha'\gamma}$\qquad
(2) $\bold{\displaystyle\lim_{x\to x_0}\frac{\alpha}{\beta}=\lim_{x\to x_0}\frac{\alpha'}{\beta'}}$
\end{theorem}
\end{frame}

\begin{frame}
\frametitle{等价无穷小量代换}
\begin{example}
求函数极限$\limit_{x\to0}\dfrac{\tan 2x}{\sin 3x}$。
\end{example}
\vpause
\begin{example}
求函数极限$\limit_{x\to0}\dfrac{1-\cos 5x}{\sin(x^2)}$。
\end{example}
\vpause
\begin{example}
求函数极限$\limit_{x\to0}\dfrac{\ln(1+x)}{\sqrt[3]{1+2x}-1}$。
\end{example}
\vpause
\begin{example}
求函数极限$\limit_{x\to0}\dfrac{\e^{x^2}-1}{(\sqrt{1+x}-1)\arcsin x}$。
\end{example}
\end{frame}

\begin{frame}
\frametitle{等价无穷小量代换}
\begin{exercise}
求下列函数极限:
\begin{enumlite}
  \item $\limit_{x\to0}\dfrac{\tan 3x}{\arcsin 2x}$;
  \item $\limit_{x\to0}\dfrac{1-\cos 3x}{(\e^{2x}-1)\ln(1-2x)}$。
\end{enumlite}
\end{exercise}
\end{frame}

\begin{frame}
\frametitle{等价无穷小量代换}
\begin{example}
求函数极限$\limit_{x\to0}\dfrac{\sin x\cdot(\e^{x+1}-1)}{\sqrt{1+x+x^2}-1}$。
\end{example}
\vpause
\begin{remark*}
\emph{只能代换无穷小量,不能代换非无穷小量。}
\end{remark*}
\vpause
\begin{example}
求函数极限$\limit_{x\to0}\dfrac{\tan x-\sin x}{\sin^3x}$。
\end{example}
\vpause
\begin{remark*}
\emph{只能分别代换乘除项,不能分别代换加减项。}
\end{remark*}
\end{frame}

\begin{rframe}
\frametitle{等价无穷小量代换}
\begin{remark*}
当$\alpha_1\sim\beta_1$、$\alpha_2\sim\beta_2$时,下列等式总是成立:
$$ \alpha_1\cdot\alpha_2 \overset{\bold{\checkmark}}{\sim} \beta_1\cdot\beta_2 $$
但下列等式未必成立:
$$ \alpha_1\pm\alpha_2 \overset{\warn{\times}}{\sim} \beta_1\pm\beta_2 $$
\end{remark*}
\pause\vfill\hrule\vfill
\begin{example*}
当$x\to0$时,有
$$\boxed{\begin{aligned}
  x+x^2 &\sim x+x^3 \\
  x &\sim x
\end{aligned}}
\xrightarrow{\text{两边同时相减}}
\boxed{x^2\overset{\warn{\times}}{\sim}x^3}$$
\end{example*}
\end{rframe}

\begin{frame}
\frametitle{等价无穷小量代换}
\begin{review}
求下列函数极限:
\begin{enumlite}
  \item $\limit_{x\to0^+}\dfrac{(\sqrt{1+\sqrt{x}}-1)(\e^{\sqrt{x}}-1)}{\sin x}$;
  \item $\limit_{x\to0}\dfrac{\ln(1-2x+x^2)\cdot\tan3x}{1-\cos3x}$。
\end{enumlite}
\end{review}
\end{frame}

\begin{frame}
\frametitle{等价无穷小量代换}
\begin{review}
求下列函数极限:
\begin{enumlite}
  \item $\limit_{x\to0}\dfrac{(\sqrt[3]{1-2x^2}-1)(\e^{2x+1}-1)}{\sin(x^2)}$;
  \item $\limit_{x\to0}\dfrac{\sin x-\tan x}{\sqrt{1-x^3}-1}$。
\end{enumlite}
\end{review}
\end{frame}

\begin{frame}
\frametitle{复习与提高}
\begin{choice}
当$x\to0^+$时,下列各无穷小量中与$\sqrt{x}$等价的是\dotfill(\select{B})
\begin{choicehalf}
  \item $1-\e^{\sqrt{x}}$ ~
  \item $\ln\frac{1+x}{1-\sqrt{x}}$ ~
  \item $\sqrt{1+\sqrt{x}}-1$ ~
  \item $1-\cos\sqrt{x}$ ~ 
\end{choicehalf}
\end{choice}
\end{frame}

\section{函数的连续性}

\subsection{函数的连续性}

\begin{frame}
\frametitle{连续的概念}
\begin{example*}
自然界中有很多现象都是\bold{连续}地变化着的。
\begin{enumerate}
  \item 当时间变化很微小时,气温的变化也很微小。
  \item 当边长变化很微小时,正方形的面积变化很微小。
\end{enumerate}
\end{example*}
\vpause
对于$y=f(x)$定义域中的一点$x_0$,如果$x$从$x_0$作微小改变$\Delta x$后,
$y$的相应改变量$\Delta y$也很微小,则称$f(x)$在点$x_0$ \bold{连续}。
\end{frame}

\begin{frame}
\frametitle{连续的概念}
\begin{definition*}
设$y=f(x)$在$x_0$的某个邻域内有定义,如果
\[ \lim_{\Delta x\to0}\Delta y = \lim_{\Delta x\to0}\big(f(x_0+\Delta x)-f(x_0)\big)=0,\]
则称$f(x)$在点$x_0$ \bold{连续}。
\end{definition*}
\pause\noindent
\[ \warn{\Updownarrow} \]
\begin{definition*}
设$y=f(x)$在$x_0$的某个邻域内有定义,如果
\[ \limit_{x\to x_0}f(x)=f(x_0),\]
则称$f(x)$在点$x_0$ \bold{连续}。
\end{definition*}
\end{frame}

\begin{frame}
\frametitle{连续函数}
\begin{definition*}
如果$f(x)$在区间$I$的每一点都连续,则称$f(x)$在区间$I$上\bold{连续},
或称$f(x)$是区间$I$上的\bold{连续函数}。\pause\unskip
\begin{itemize}
  \item $f(x)$在区间左端点$a$处连续是指$f(a^+)=f(a)$
  \item $f(x)$在区间右端点$b$处连续是指$f(b^-)=f(b)$  
\end{itemize}
\end{definition*}
\vpause
\begin{remark*}
连续函数的图形是一条连续而不间断的曲线。
\end{remark*}
\end{frame}

\subsection{连续函数的运算}

\begin{iframe}
\frametitle{极限与连续}
$\limit_{x\to x_0}f(x)=A$(极限存在):
\[\forall\epsilon>0,\; \exists\delta>0,\; x\in\bold{\mathring{U}(x_0,\delta)} \rightwhitearrow f(x)\in \bold{U(A,\epsilon)}\]
\vskip0pt plus 0.5fill{\clead\cdotfill}\vskip0pt plus 0.5fill\pause
$\limit_{x\to x_0}f(x)=f(x_0)$(连续):
\[\forall\epsilon>0,\; \exists\delta>0,\; x\in\bold{U(x_0,\delta)} \rightwhitearrow f(x)\in \bold{U(f(x_0),\epsilon)}\]
\end{iframe}

\begin{iframe}
\frametitle{复合函数的极限}
\vskip1em
$\limit_{x\to x_0}g(x)=u_0,\; \limit_{u\to u_0}f(u)=A$
$\warn{\centernot\Longrightarrow}$ $\limit_{x\to x_0}f[g(x)]=A$
\vskip0pt plus 0.2fill\lead{\hrule}\vskip0pt plus 0.2fill\pause
\begin{tikzpicture}[thick]
  \node (g1) at (0,4) {$\mathring{U}(x_0,\delta)$}; \node (g2) at (0,2) {$U(u_0,\eta)$};
  \draw[->] (g1) -- node[right]{$g$} (g2);
  \node (f1) at (4,2) {$\mathring{U}(u_0,\eta)$}; \node (f2) at (4,0) {$U(A,\epsilon)$};
  \draw[->] (f1) -- node[right]{$f$} (f2);
  \node (fg1) at (8,4) {$\mathring{U}(x_0,\delta)$}; \node (fg2) at (8,0) {$U(A,\epsilon)$};
  \draw[->] (fg1) -- node[right=0.4em]{$f\text{\small$\circ$}g$} node{\Large\only<4->{\cwarn$\times$}} (fg2);
  \only<3->{\node at (2,2) {\Large\cwarn$\neq$};}
\end{tikzpicture}
\end{iframe}

\begin{iframe}
\frametitle{复合函数的极限}
\vskip1em
$\limit_{x\to x_0}g(x)=u_0,\; \limit_{u\to u_0}f(u)=A$
$\tikzmark{aa}\bold{\Longrightarrow}\tikzmark{bb}$ $\limit_{x\to x_0}f[g(x)]=A$
\vskip0pt plus 0.2fill\lead{\hrule}\vskip0pt plus 0.2fill\pause
\begin{tikzpicture}[thick]
  \node (g1) at (0,4) {$\mathring{U}(x_0,\delta)$}; \node (g2) at (0,2) {$\clead\mathring{U}(u_0,\eta)$};
  \draw[->] (g1) -- node[right]{$g$} (g2);
  \node (f1) at (4,2) {$\mathring{U}(u_0,\eta)$}; \node (f2) at (4,0) {$U(A,\epsilon)$};
  \draw[->] (f1) -- node[right]{$f$} (f2);
  \node (fg1) at (8,4) {$\mathring{U}(x_0,\delta)$}; \node (fg2) at (8,0) {$U(A,\epsilon)$};
  \draw[->] (fg1) -- node[right=0.4em]{$f\text{\small$\circ$}g$} node[xshift=-0.2em]{\Large\only<4->{\cbold$\checkmark$}} (fg2);
  \only<3->{\node at (2,2) {\Large\cbold$\phantom{o}=\phantom{o}$};}
\end{tikzpicture}%
\begin{tikzpicture}[remember picture,overlay]
  \path (aa) -- node[above=1em,shape=rectangle,draw,text width=5.5em,align=center]
    {\small\clead$x\in\mathring{U}(x_0,\delta_0)$\newline$g(x)\neq u_0$} (bb);
\end{tikzpicture}
\end{iframe}

\begin{iframe}
\frametitle{复合函数的极限}
\vskip1em
$\limit_{x\to x_0}g(x)=u_0,\; \limit_{u\to u_0}f(u)=A$
$\tikzmark{cc}\bold{\Longrightarrow}\tikzmark{dd}$ $\limit_{x\to x_0}f[g(x)]=A$
\vskip0pt plus 0.2fill\lead{\hrule}\vskip0pt plus 0.2fill\pause
\begin{tikzpicture}[thick]
  \node (g1) at (0,4) {$\mathring{U}(x_0,\delta)$}; \node (g2) at (0,2) {$U(u_0,\eta)$};
  \draw[->] (g1) -- node[right]{$g$} (g2);
  \node (f1) at (4,2) {$\clead U(u_0,\eta)$}; \node (f2) at (4,0) {$U(A,\epsilon)$};
  \draw[->] (f1) -- node[right]{$f$} (f2);
  \node (fg1) at (8,4) {$\mathring{U}(x_0,\delta)$}; \node (fg2) at (8,0) {$U(A,\epsilon)$};
  \draw[->] (fg1) -- node[right=0.4em]{$f\text{\small$\circ$}g$} node[xshift=-0.2em]{\Large\only<4->{\cbold$\checkmark$}} (fg2);
  \only<3->{\node at (2,2) {\Large\cbold$\phantom{o}=\phantom{o}$};}
\end{tikzpicture}%
\begin{tikzpicture}[remember picture,overlay]
  \path (cc) -- node[above=1em,shape=rectangle,draw,text width=5em,align=left] % align=center无效?
    {\small\clead\quad$y=f(u)$\newline 在$u_0$点连续} (dd);
\end{tikzpicture}
\end{iframe}

\begin{iframe}
\frametitle{复合函数的极限}
\begin{theorem}
如果$\limit_{x\to x_0}g(x)=u_0$,$\limit_{u\to u_0}f(u)=A$,
并且存在$\delta_0>0$使得 \bold{$x\in\mathring{U}(x_0,\delta_0)$时$g(x)\neq u_0$},
则有$$\limit_{x\to x_0}f[g(x)]=A$$
\end{theorem}
\vpause
\begin{theorem}
若$\limit_{x\to x_0}g(x)=u_0$且$f(u)$在$u_0$点连续,则
\[ \lim_{x\to x_0}f[g(x)]=f\Big[\lim_{x\to x_0}g(x)\Big]=f(u_0). \]
\end{theorem}
\vpause
\begin{example}
$\limit_{x\to2}\sqrt{\dfrac{x-2}{x^2-4}}=\sqrt{\limit_{x\to2}\dfrac{x-2}{x^2-4}}=\sqrt{\dfrac14}=\dfrac{1}{2}$.
\end{example}
\end{iframe}

\begin{frame}
\frametitle{连续函数}
\begin{theorem}
两个连续函数的\bold{四则运算}仍然是连续函数。
\end{theorem}
\pause
\begin{theorem}
两个连续函数的\bold{复合函数}仍然是连续函数。
\end{theorem}
\vpause
\begin{theorem}
基本初等函数在其定义域内都是连续函数。
\end{theorem}
\pause
\begin{theorem}
初等函数在其定义区间内都是连续函数。
\end{theorem}
\end{frame}

\begin{frame}
\frametitle{函数的连续性}
\begin{property*}
$f(x)$在$x_0$点连续 $\pmb\Leftrightarrow$ $f(x_0^-)=f(x_0^+)=f(x_0)$。
\end{property*}
\pause
\begin{example}
判断函数$f(x)$在$x=0$点的连续性:
\begin{enumlite}
  \item $f(x)=\begin{cases}x+1,&x<0\\x-1,&x\ge0\end{cases}$;
  \item $f(x)=\begin{cases}\frac{2\sin x}x,&x<0\\2,&x=0\\\frac{x}{\sqrt{1+x}-1},&x>0\end{cases}$。
\end{enumlite}
\end{example}
\end{frame}

\begin{frame}
\frametitle{函数的连续性}
\begin{exercise}
已知函数$f(x)=\begin{cases}\dfrac{\sin x}x,&x<0\\1,&x=0\\x^2+1,&x>0\end{cases}$,判断它在$x=0$点的连续性。
\end{exercise}
\end{frame}

\begin{frame}
%\frametitle{函数的连续性}
\begin{example}
已知函数$f(x)=\begin{cases}x^2+1,&x<1\\b,&x=1\\ax+1,&x>1\end{cases}$是连续的,求$a$和$b$。
\end{example}
\vpause
\begin{exercise}
已知函数$f(x)$在$x=0$点连续,求$a$和$b$:
$$f(x)=\begin{cases}\dfrac{\e^{ax}-1}x,&x<0\\2,&x=0\\\dfrac{\sqrt[3]{1-bx+x^2}-1}{x},&x>0\end{cases}.$$
\end{exercise}
\end{frame}

\begin{sframe}
\frametitle{函数的连续性}
\begin{solution}
$f(0)=2$,
\begin{align*}
f(0^-)&=\limit_{x\to0^-}\dfrac{\e^{ax}-1}{x}=\limit_{x\to0^-}\dfrac{ax}{x}=a,\\
f(0^+)&=\limit_{x\to0^+}\dfrac{\sqrt[3]{1-bx+x^2}-1}{x}\\
&=\limit_{x\to0^+}\dfrac{(-bx+x^2)/3}{x}=\limit_{x\to0^+}\dfrac{-b+x}{3}=-\frac{b}3.
\end{align*}
$f(x)$在$x=0$点连续,故$f(0)=f(0^-)=f(0^+)$,即有$a=2$,$b=-6$。
\end{solution}
\end{sframe}

\subsection{函数的间断点}

\begin{frame}
\frametitle{函数的间断点}
\begin{definition*}
设$f(x)$在点$x_0$的某个去心邻域有定义,如果$f(x)$在点$x_0$不连续,
则称它在点$x_0$ \bold{间断},或者称点$x_0$是$f(x)$的\bold{间断点}。
\end{definition*}
\end{frame}

\begin{frame}
\frametitle{间断点的分类}
\begin{itemize}[<+->]
  \item \bold{第一类间断点}:$f(x_0^-)$和$f(x_0^+)$均存在
  \begin{itemize}
    \item \bold{可去间断点}:$f(x_0^-)=f(x_0^+)$
    \item \bold{跳跃间断点}:$f(x_0^-)\neq f(x_0^+)$
  \end{itemize}
  \item \bold{第二类间断点}:$f(x_0^-)$和$f(x_0^+)$至少一个不存在
  \begin{itemize}
    \item \bold{无穷间断点}:$f(x_0^-)$和$f(x_0^+)$至少一个为无穷大
    \item \bold{振荡间断点}:$f(x_0^-)$和$f(x_0^+)$均不为无穷大
  \end{itemize}
\end{itemize}
% 问题:$x=0$ 属于 $\frac1x\sin\frac1x$ 的哪种间断点?
\end{frame}

\begin{frame}
\frametitle{函数的间断点}
\begin{example}
$f(x)=\begin{cases}x+1,&x\neq1\\1,&x=1\end{cases}$\dotfill\bold{可去间断点}
\end{example}
\pause
\begin{example}
$f(x)=\begin{cases}x-1,&x<0\\0,&x=0\\x+1,&x>0\end{cases}$\dotfill\bold{跳跃间断点}
\end{example}
\pause
\begin{example}
$f(x)=\frac1x$\dotfill\bold{无穷间断点}
\end{example}
\pause
\begin{example}
$f(x)=\sin\frac1x$\dotfill\bold{振荡间断点}
\end{example}
\end{frame}

\begin{frame}
\frametitle{函数的间断点}
\begin{remark*}
间断点常见位置:(1)\CJKunderdot{分母为零};(2)\CJKunderdot{分段点}。
\end{remark*}
\pause\vfill\hrule\vfill
\begin{example}
求函数$f(x)=\dfrac{x-1}{x^2-1}$的间断点,并判断类型。
\end{example}
\vpause
\begin{example}
求$f(x)=\left\{\begin{array}{ll}
\dfrac1{x^2}, & x\le1, x\neq0 \\[1em]
\dfrac{x^2-4}{x-2}, & x>1, x\neq2
\end{array}\right.$的间断点,并判断其类型。
\end{example}
\end{frame}

\begin{iframe}
\frametitle{复习与提高}
\begin{choice}
已知函数$f(x)=\dfrac{\e^{\frac1x}-1}{\e^{\frac1x}+1}$,则点$x=0$属于$f(x)$的\dotfill(\select{B})
\begin{choicehalf}
  \item 可去间断点 ~
  \item 跳跃间断点 ~
  \item 第二类间断点 ~
  \item 连续点 ~
\end{choicehalf}
\end{choice}
\vpause
\begin{thinking}
修改$f(x)$,使得$x\to0$时$f(x)$为无穷大量?
\end{thinking}
\vpause
\begin{remark*}
小心$\warn{\e^{\frac1x}}$和$\warn{\arctan\big(\frac1x\big)}$等复合函数的间断点!
\end{remark*}
\end{iframe}

\section{闭区间上连续函数}

\subsection{最值定理}

\begin{frame}
\frametitle{最值定理}
\begin{theorem}
设$f(x)$在闭区间$[a,b]$上连续,则$f(x)$在该区间上有界而且一定能取到最大值$M$和最小值$m$.
\end{theorem}
\vpause
\begin{remark*}
函数$y=\tan x$在开区间$\left(-\frac\pi2,\frac\pi2\right)$上是连续的,
但在这个开区间上它是无界的,而且也没有最大值和最小值。
\end{remark*}
\end{frame}

\subsection{零值定理}

\begin{frame}
\frametitle{零值定理}
\begin{theorem}
设$f(x)$在闭区间$[a,b]$上连续,且$f(a)$和$f(b)$异号,则在开区间$(a,b)$内至少存在一点$\xi$,使得$f(\xi)=0$.
\end{theorem}
\vpause
\begin{example}
证明方程$x^3-3x^2+1=0$在区间$(-1,0)$,$(0,1)$,$(1,3)$内各有一个实根。
\end{example}
\vpause
\begin{example}
证明方程$2\sin x=x+1$有实数解。
\end{example}
\end{frame}

\subsection{介值定理}

\begin{frame}
\frametitle{介值定理}
\begin{theorem}
设$f(x)$在闭区间$[a,b]$上连续,且$f(a)=A$和$f(b)=B$不相等,则对于$A$与$B$之间的任何数$C$,
在开区间$(a,b)$内至少存在一点$\xi$,使得$f(\xi)=C$.
\end{theorem}
\pause
\begin{proof}
令$g(x)=f(x)-C$.则由零值定理可得结论.
\end{proof}
\end{frame}

\end{document}
