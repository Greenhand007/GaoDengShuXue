% -*- coding: utf-8 -*-
% !TEX program = xelatex

%\documentclass[14pt]{article}
%\usepackage[notheorems]{beamerarticle}

\documentclass[14pt,notheorems,leqno,xcolor={rgb}]{beamer} % ignorenonframetext

% -*- coding: utf-8 -*-
% ----------------------------------------------------------------------------
% Author:  Jianrui Lyu <tolvjr@163.com>
% Website: https://github.com/lvjr/theme
% License: Creative Commons Attribution-ShareAlike 4.0 International License
% ----------------------------------------------------------------------------

\ProvidesPackage{beamerthemeriemann}[2018/06/05 v0.6 Beamer Theme Riemann]

\makeatletter

% compatible with old versions of beamer
\providecommand{\beamer@endinputifotherversion}[1]{}

\RequirePackage{tikz,etoolbox,adjustbox}
\usetikzlibrary{shapes.multipart}

\mode<presentation>

\setbeamersize{text margin left=8mm,text margin right=8mm}

%% ----------------- background canvas and background ----------------

\newif\ifbackgroundmarkleft
\newif\ifbackgroundmarkright

\newcommand{\insertbackgroundmark}{
  \ifbackgroundmarkleft
    \foreach \x in {1,2,3,4,5} \draw[very thick,markcolor] (0,\x*\paperheight/6) -- +(1.2mm,0);
  \fi
  \ifbackgroundmarkright
    \foreach \x in {1,2,3,4,5} \draw[very thick,markcolor] (\paperwidth,\x*\paperheight/6) -- +(-1.2mm,0);
  \fi
}

\defbeamertemplate{background}{line}{%
  \begin{tikzpicture}
    \useasboundingbox (0,0) rectangle (\paperwidth,\paperheight);
    \draw[xstep=\paperwidth,ystep=1mm,color=tcolor] (0,0) grid (\paperwidth,\paperheight);
    \insertbackgroundmark
  \end{tikzpicture}%
}

\defbeamertemplate{background}{linear}{%
  \begin{tikzpicture}
    \useasboundingbox (0,0) rectangle (\paperwidth,\paperheight);
    \draw[pattern=horizontal lines, pattern color=tcolor]
      (0,0) rectangle (\paperwidth,\paperheight);
    \insertbackgroundmark
  \end{tikzpicture}%
}

\defbeamertemplate{background}{lattice}[1][1mm]{%
  \begin{tikzpicture}
    \useasboundingbox (0,0) rectangle (\paperwidth,\paperheight);
    \draw[step=#1,color=tcolor,semithick] (0,0) grid (\paperwidth,\paperheight);
    \insertbackgroundmark
  \end{tikzpicture}%
}

\defbeamertemplate{background}{empty}{
  \begin{tikzpicture}
    \useasboundingbox (0,0) rectangle (\paperwidth,\paperheight);
    \insertbackgroundmark
  \end{tikzpicture}%
}

%% -------------------------- title page -----------------------------

% add \occasion command
\newcommand{\occasion}[1]{\def\insertoccasion{#1}}
\occasion{}

\defbeamertemplate{title page}{banner}{%
  \nointerlineskip
  \begin{adjustbox}{width=\paperwidth,center}%
    \usebeamertemplate{title page content}%
  \end{adjustbox}%
}

% need "text badly ragged" option for correct space skips
% see http://tex.stackexchange.com/a/132748/8956
\defbeamertemplate{title page content}{hexagon}{%
  \begin{tikzpicture}
  \useasboundingbox (0,0) rectangle (\paperwidth,\paperheight);
  \path[draw=dcolor,fill=fcolor,opacity=0.8]
      (0,0) rectangle (\paperwidth,\paperheight);
  \node[text width=0.86\paperwidth,text badly ragged,inner ysep=1.5cm] (main) at (0.5\paperwidth,0.55\paperheight) {%
    \begin{minipage}[c]{0.86\paperwidth}
      \centering
      \usebeamerfont{title}\usebeamercolor[fg]{title}\inserttitle
      \ifx\insertsubtitle\@empty\else
        \\[5pt]\usebeamerfont{subtitle}\usebeamercolor[fg]{subtitle}
        \insertsubtitle
      \fi
    \end{minipage}
  };
  \node[rectangle,inner sep=0pt,minimum size=3mm,fill=dcolor,right] (a) at (0,0.55\paperheight) {};
  \node[rectangle,inner sep=0pt,minimum size=3mm,fill=dcolor,left] (b) at (\paperwidth,0.55\paperheight) {};
  \ifx\insertoccasion\@empty
      \draw[thick,dcolor] (a.north east) -- (main.north west)
                   -- (main.north east) -- (b.north west);
  \else
      \node[text badly ragged] (occasion) at (main.north west -| 0.5\paperwidth,\paperheight) {
          \usebeamerfont{occasion}\usebeamercolor[fg]{occasion}\insertoccasion
      };
      \draw[thick,dcolor] (a.north east) -- (main.north west) -- (occasion.west)
                          (b.north west) -- (main.north east) -- (occasion.east);
  \fi
  \node[text badly ragged] (date) at (main.south west -| 0.5\paperwidth,0) {
      \usebeamerfont{date}\usebeamercolor[fg]{date}\insertdate
  };
  \draw[thick,dcolor] (a.south east) -- (main.south west) -- (date.west)
                      (b.south west) -- (main.south east) -- (date.east);
  \node[below=4mm,text width=0.9\paperwidth,inner xsep=0.05\paperwidth,
        text badly ragged,fill=white] at (date.south) {%
      \begin{minipage}[c]{0.9\paperwidth}
          \centering
          \textcolor{brown75}{$\blacksquare$}\hspace{0.2em}%
          \usebeamerfont{institute}\usebeamercolor[fg]{institute}\insertinstitute
          \hspace{0.4em}\textcolor{brown75}{$\blacksquare$}\hspace{0.2em}%
          \usebeamerfont{author}\usebeamercolor[fg]{author}\insertauthor
      \end{minipage}
  };
  \end{tikzpicture}
}

\defbeamertemplate{title page content}{rectangle}{%
  \begin{tikzpicture}
  \useasboundingbox (0,0) rectangle (\paperwidth,\paperheight);
  \path[draw=dcolor,fill=fcolor,opacity=0.8]
      (0,0.25\paperheight) rectangle (\paperwidth,0.85\paperheight);
  \path[draw=dcolor,very thick]
    %%(0.0075\paperwidth,0.26\paperheight) rectangle (0.9925\paperwidth,0.84\paperheight);
      (0.0375\paperwidth,0.26\paperheight) -- (0.9625\paperwidth,0.26\paperheight)
         -- ++(0,0.02\paperheight) -- ++(0.03\paperwidth,0)
         -- ++(0,-0.02\paperheight) -- ++(-0.015\paperwidth,0)
         -- ++(0,0.04\paperheight) -- ++(0.015\paperwidth,0)
      -- (0.9925\paperwidth,0.8\paperheight)
         -- ++(-0.015\paperwidth,0) -- ++(0,0.04\paperheight)
         -- ++(0.015\paperwidth,0) -- ++(0,-0.02\paperheight)
         -- ++(-0.03\paperwidth,0) -- ++(0,0.02\paperheight)
      -- (0.0375\paperwidth,0.84\paperheight)
         -- ++(0,-0.02\paperheight) -- ++(-0.03\paperwidth,0)
         -- ++(0,0.02\paperheight) -- ++(0.015\paperwidth,0)
         -- ++(0,-0.04\paperheight) -- ++(-0.015\paperwidth,0)
      -- (0.0075\paperwidth,0.3\paperheight)
         -- ++(0.015\paperwidth,0) -- ++(0,-0.04\paperheight)
         -- ++(-0.015\paperwidth,0) -- ++(0,0.02\paperheight)
         -- ++(0.03\paperwidth,0) -- ++(0,-0.02\paperheight)
      -- cycle;
  \node[text width=0.9\paperwidth,text badly ragged] at (0.5\paperwidth,0.55\paperheight) {%
    \begin{minipage}[c][0.58\paperheight]{0.9\paperwidth}
      \centering
      \usebeamerfont{title}\usebeamercolor[fg]{title}\inserttitle
      \ifx\insertsubtitle\@empty\else
        \\[5pt]\usebeamerfont{subtitle}\usebeamercolor[fg]{subtitle}
        \insertsubtitle
      \fi
    \end{minipage}
  };
  \ifx\insertoccasion\@empty\else
    \node[text badly ragged,below,draw=dcolor,fill=white] at (0.5\paperwidth,0.84\paperheight) {%
      \usebeamerfont{occasion}\usebeamercolor[fg]{occasion}\insertoccasion
    };
  \fi
  \node[text width=0.9\paperwidth,text badly ragged,below] at (0.5\paperwidth,0.25\paperheight) {%
    \begin{minipage}[t][0.25\paperheight]{0.9\paperwidth}
      \centering
      {\color{brown75}$\blacksquare$}
      \usebeamerfont{institute}\usebeamercolor[fg]{institute}\insertinstitute
      \hfill
      {\color{brown75}$\blacksquare$}
      \usebeamerfont{author}\usebeamercolor[fg]{author}\insertauthor
      \hfill
      {\color{brown75}$\blacksquare$}
      \usebeamerfont{date}\usebeamercolor[fg]{date}%
      \the\year-\ifnum\month<10 0\fi\the\month-\ifnum\day<10 0\fi\the\day
    \end{minipage}
  };
  \end{tikzpicture}
}

%% ----------------------- section and subsection --------------------

\newcounter{my@pgf@picture@count}

\def\sectionintocskip{0.5pt plus 0.1fill}
\patchcmd{\beamer@sectionintoc}{\vskip1.5em}{\vskip\sectionintocskip}{}{}

\AtBeginSection[]{%
  \begin{frame}%[plain]
    \sectionpage
  \end{frame}%
}

\defbeamertemplate{section name}{simple}{\insertsectionnumber.}

\defbeamertemplate{section name}{chinese}[1][节]{第\CJKnumber{\insertsectionnumber}#1}

\defbeamertemplate{section page}{single}{%
  \centerline{%
    \usebeamerfont{section name}%
    \usebeamercolor[fg]{section name}%
    \usebeamertemplate{section name}%
    \hspace{0.8em}%
    \usebeamerfont{section title}%
    \usebeamercolor[fg]{section title}%
    \insertsection
  }%
}

\defbeamertemplate{section name in toc}{simple}{%
  Section \ifnum\the\beamer@tempcount<10 0\fi\inserttocsectionnumber
}

\defbeamertemplate{section name in toc}{chinese}[1][节]{%
  第\CJKnumber{\inserttocsectionnumber}#1%
}

\newcounter{my@section@from}
\newcounter{my@section@to}

\defbeamertemplate{show sections in toc}{total}{%
  \setcounter{my@section@from}{1}%
  \setcounter{my@section@to}{50}%
}

% show at most five sections
\defbeamertemplate{show sections in toc}{partial}{%
  \setcounter{my@section@from}{\value{section}}%
  \addtocounter{my@section@from}{-2}%
  \setcounter{my@section@to}{\value{section}}%
  \addtocounter{my@section@to}{2}%
  \ifnum\my@totalsectionnumber>0%
    \ifnum\value{my@section@to}>\my@totalsectionnumber
      \setcounter{my@section@to}{\my@totalsectionnumber}%
      \setcounter{my@section@from}{\value{my@section@to}}%
      \addtocounter{my@section@from}{-4}%
    \fi
  \fi
  \ifnum\value{my@section@from}<1\setcounter{my@section@from}{1}%
    \setcounter{my@section@to}{\value{my@section@from}}%
    \addtocounter{my@section@to}{4}%
  \fi
}

% reset pgfid to get correct result with \tikzmark in second run
\defbeamertemplate{section page}{fill}{%
  \usebeamertemplate{show sections in toc}%
  \setcounter{my@pgf@picture@count}{\the\pgf@picture@serial@count}%
  \tableofcontents[sectionstyle=show/shaded,subsectionstyle=hide,
                   sections={\arabic{my@section@from}-\arabic{my@section@to}}]%
  \global\pgf@picture@serial@count=\value{my@pgf@picture@count}%
  \unskip
}

\defbeamertemplate{section in toc}{fill}{%
  \noindent
  \begin{tikzpicture}
  \node[rectangle split, rectangle split horizontal, rectangle split parts=2,
        rectangle split part fill={sectcolor,bg}, draw=darkgray,
        inner xsep=0pt, inner ysep=5.5pt]
       {
         \nodepart[text width=0.255\textwidth,align=center]{text}
         \usebeamertemplate{section name in toc}
         \nodepart[text width=0.74\textwidth]{second}%
         \hspace{7pt}\inserttocsection
       };
  \end{tikzpicture}%
  \par
}

\AtBeginSubsection{%
  \begin{frame}%[plain]
    \setlength{\parskip}{0pt}%
    \offinterlineskip
    \subsectionpage
  \end{frame}%
}

\defbeamertemplate{subsection name}{simple}{%
  \insertsectionnumber.\insertsubsectionnumber
}

\defbeamertemplate{subsection page}{single}{%
  \centerline{%
    \usebeamerfont{subsection name}%
    \usebeamercolor[fg]{subsection name}%
    \usebeamertemplate{subsection name}%
    \hspace{0.8em}%
    \usebeamerfont{subsection title}%
    \usebeamercolor[fg]{subsection title}%
    \insertsubsection
  }%
}

% reset pgfid to get correct result with \tikzmark in second run
\defbeamertemplate{subsection page}{fill}{%
  \setcounter{my@pgf@picture@count}{\the\pgf@picture@serial@count}%
  \tableofcontents[sectionstyle=show/hide,subsectionstyle=show/shaded/hide]%
  \global\pgf@picture@serial@count=\value{my@pgf@picture@count}%
  \unskip
}

\defbeamertemplate{subsection in toc}{fill}{%
  \noindent
  \begin{tikzpicture}
    \node[rectangle split, rectangle split horizontal, rectangle split parts=2,
          rectangle split part fill={white,bg}, draw=darkgray,
          inner xsep=0pt, inner ysep=5.5pt]
         {
           \nodepart[text width=0.255\textwidth,align=right]{text}
           \inserttocsectionnumber.\inserttocsubsectionnumber\kern7pt%
           \nodepart[text width=0.74\textwidth]{second}%
           \hspace{7pt}\inserttocsubsection
         };
  \end{tikzpicture}%
  \par
}

% chinese sections and subsections
\defbeamertemplate{section and subsection}{chinese}[1][节]{%
  \setbeamertemplate{section name in toc}[chinese][#1]%
  \setbeamertemplate{section name}[chinese][#1]%
}

%% ---------------------- headline and footline ----------------------

\defbeamertemplate{footline left}{author}{%
  \insertshortauthor
}

\defbeamertemplate{footline center}{title}{%
  \insertshorttitle
}

\defbeamertemplate{footline right}{number}{%
  \Acrobatmenu{GoToPage}{\insertframenumber{}/\inserttotalframenumber}%
}
\defbeamertemplate{footline right}{normal}{%
  \hyperlinkframeendprev{$\vartriangle$}
  \Acrobatmenu{GoToPage}{\insertframenumber{}/\inserttotalframenumber}
  \hyperlinkframestartnext{$\triangledown$}%
}

\defbeamertemplate{footline}{simple}{%
  \hbox{%
  \begin{beamercolorbox}[wd=.2\paperwidth,ht=2.25ex,dp=1ex,left]{footline}%
    \usebeamerfont{footline}\kern\beamer@leftmargin
    \usebeamertemplate{footline left}%
  \end{beamercolorbox}%
  \begin{beamercolorbox}[wd=.6\paperwidth,ht=2.25ex,dp=1ex,center]{footline}%
    \usebeamerfont{footline}\usebeamertemplate{footline center}%
  \end{beamercolorbox}%
  \begin{beamercolorbox}[wd=.2\paperwidth,ht=2.25ex,dp=1ex,right]{footline}%
    \usebeamerfont{footline}\usebeamertemplate{footline right}%
    \kern\beamer@rightmargin
  \end{beamercolorbox}%
  }%
}

\defbeamertemplate{footline}{sectioning}{%
  % default height is 0.4pt, which is ignored by adobe reader, so we increase it by 0.2pt
  {\usebeamercolor[fg]{separator line}\hrule height 0.6pt}%
  \hbox{%
  \begin{beamercolorbox}[wd=.8\paperwidth,ht=2.25ex,dp=1ex,left]{footline}%
    \usebeamerfont{footline}\kern\beamer@leftmargin\insertshorttitle
    \ifx\insertsection\@empty\else\qquad$\vartriangleright$\qquad\insertsection\fi
    \ifx\insertsubsection\@empty\else\qquad$\vartriangleright$\qquad\insertsubsection\fi
  \end{beamercolorbox}%
  \begin{beamercolorbox}[wd=.2\paperwidth,ht=2.25ex,dp=1ex,right]{footline}%
     \usebeamerfont{footline}\usebeamertemplate{footline right}%
     \kern\beamer@rightmargin
  \end{beamercolorbox}%
  }%
}

% customize mini frames template to get a section navigation bar

\defbeamertemplate{navigation box}{current}{%
  \colorbox{accent2}{%
    \rule[-1ex]{0pt}{3.25ex}\color{white}\kern1.4pt\my@navibox\kern1.4pt%
  }%
}

\defbeamertemplate{navigation box}{other}{%
  %\colorbox{white}{%
    \rule[-1ex]{0pt}{3.25ex}\color{black}\kern1.4pt\my@navibox\kern1.4pt%
  %}%
}

\newcommand{\my@navibox@subsection}{$\blacksquare$}
\newcommand{\my@navibox@frame}{$\square$}
\let\my@navibox=\my@navibox@frame

% optional navigation box for some special frame
\newcommand{\my@navibox@frame@opt}{$\boxplus$}
\newcommand{\my@change@navibox}{\let\my@navibox=\my@navibox@frame@opt}
\newcommand{\changenavibox}{%
  \addtocontents{nav}{\protect\headcommand{\protect\my@change@navibox}}%
}

\newcommand{\my@sectionentry@show}[5]{%
  \ifnum\c@section=#1%
    \setbeamertemplate{navigation box}[current]%
  \else
    \setbeamertemplate{navigation box}[other]%
  \fi
  \begingroup
    \def\my@navibox{#1}%
    \hyperlink{Navigation#3}{\usebeamertemplate{navigation box}}%
  \endgroup
}

\newif\ifmy@hidesection

\newcommand{\my@sectionentry@hide}[5]{\my@hidesectiontrue}

\pretocmd{\beamer@setuplinks}{\renewcommand{\beamer@subsectionentry}[5]{}}{}{}
\apptocmd{\beamer@setuplinks}{\global\let\beamer@subsectionentry\mybeamer@subsectionentry}{}{}

\newcommand{\mybeamer@subsectionentry}[5]{\global\let\my@navibox=\my@navibox@subsection}

\newcommand{\my@slideentry@empty}[6]{}

\newcommand{\my@slideentry@section}[6]{%
  \ifmy@hidesection
    \my@hidesectionfalse
  \else
    \ifnum\c@section=#1%
      \setbeamertemplate{navigation box}[other]%
      \ifnum\c@subsection=#2\ifnum\c@subsectionslide=#3%
         \setbeamertemplate{navigation box}[current]%
      \fi\fi
      \beamer@link(#4){\usebeamertemplate{navigation box}}%
    \fi
  \fi
  \global\let\my@navibox=\my@navibox@frame
}

\AtEndDocument{%
   \immediate\write\@auxout{%
     \noexpand\gdef\noexpand\my@totalsectionnumber{\the\c@section}%
   }%
}

\def\my@totalsectionnumber{0}

\defbeamertemplate{footline}{navigation}{%
  % default height is 0.4pt, which is ignored by adobe reader, so we increase it by 0.2pt
  {\usebeamercolor[fg]{separator line}\hrule height 0.6pt}%
  \begin{beamercolorbox}[wd=\paperwidth,ht=2.25ex,dp=1ex]{footline}%
    \usebeamerfont{footline}%
    \kern\beamer@leftmargin
    \setlength{\fboxsep}{0pt}%
    \ifnum\my@totalsectionnumber=0%
      \insertshorttitle
    \else
      \let\sectionentry=\my@sectionentry@show
      \let\slideentry=\my@slideentry@empty
      \dohead
    \fi
    \hfill
    \let\sectionentry=\my@sectionentry@hide
    \let\slideentry=\my@slideentry@section
    \dohead
    \kern\beamer@rightmargin
  \end{beamercolorbox}%
}

%% ------------------------- frame title -----------------------------

\defbeamertemplate{frametitle}{simple}[1][]
{%
  \nointerlineskip
  \begin{beamercolorbox}[wd=\paperwidth,sep=0pt,leftskip=\beamer@leftmargin,%
                         rightskip=\beamer@rightmargin,#1]{frametitle}
    \usebeamerfont{frametitle}%
    \rule[-3.6mm]{0pt}{12mm}\insertframetitle\rule[-3.6mm]{0pt}{12mm}\par
  \end{beamercolorbox}
}

%% ------------------- block and theorem -----------------------------

\defbeamertemplate{theorem begin}{simple}
{%
  \upshape%\bfseries\inserttheoremheadfont
  {\usebeamercolor[fg]{theoremname}%
  \inserttheoremname\inserttheoremnumber
  \ifx\inserttheoremaddition\@empty\else
    \ \usebeamercolor[fg]{local structure}(\inserttheoremaddition)%
  \fi%
  %\inserttheorempunctuation
  }%
  \quad\normalfont
}
\defbeamertemplate{theorem end}{simple}{\par}

\defbeamertemplate{proof begin}{simple}
{%
  %\bfseries
  \let\@addpunct=\@gobble
  {\usebeamercolor[fg]{proofname}\insertproofname}%
  \quad\normalfont
}
\defbeamertemplate{proof end}{simple}{\par}

%% ---------------------- enumerate and itemize ----------------------

\expandafter\patchcmd\csname beamer@@tmpop@enumerate item@square\endcsname
         {height1.85ex depth.4ex}{height1.85ex depth.3ex}{}{}
\expandafter\patchcmd\csname beamer@@tmpop@enumerate subitem@square\endcsname
         {height1.85ex depth.4ex}{height1.85ex depth.3ex}{}{}
\expandafter\patchcmd\csname beamer@@tmpop@enumerate subsubitem@square\endcsname
         {height1.85ex depth.4ex}{height1.85ex depth.3ex}{}{}

%% ------------------------ select templates -------------------------

\setbeamertemplate{background canvas}[default]
\setbeamertemplate{background}[line]
\setbeamertemplate{footline}[navigation]
\setbeamertemplate{footline left}[author]
\setbeamertemplate{footline center}[title]
\setbeamertemplate{footline right}[number]
\setbeamertemplate{title page}[banner]
\setbeamertemplate{title page content}[hexagon]
\setbeamertemplate{section page}[fill]
\setbeamertemplate{show sections in toc}[partial]
\setbeamertemplate{section name}[simple]
\setbeamertemplate{section name in toc}[simple]
\setbeamertemplate{section in toc}[fill]
\setbeamertemplate{section in toc shaded}[default][50]
\setbeamertemplate{subsection page}[fill]
\setbeamertemplate{subsection name}[simple]
\setbeamertemplate{subsection in toc}[fill]
\setbeamertemplate{subsection in toc shaded}[default][50]
\setbeamertemplate{theorem begin}[default]
\setbeamertemplate{theorem end}[default]
\setbeamertemplate{proof begin}[default]
\setbeamertemplate{proof end}[default]
\setbeamertemplate{frametitle}[simple]
\setbeamertemplate{navigation symbols}{}
\setbeamertemplate{itemize items}[square]
\setbeamertemplate{enumerate items}[square]

%% --------------------------- font theme ----------------------------

\setbeamerfont{title}{size=\LARGE}
\setbeamerfont{subtitle}{size=\large}
\setbeamerfont{author}{size=\normalsize}
\setbeamerfont{institute}{size=\normalsize}
\setbeamerfont{date}{size=\normalsize}
\setbeamerfont{occasion}{size=\normalsize}
\setbeamerfont{section in toc}{size=\large}
\setbeamerfont{subsection in toc}{size=\large}
\setbeamerfont{frametitle}{size=\large}
\setbeamerfont{block title}{size=\normalsize}
\setbeamerfont{item projected}{size=\footnotesize}
\setbeamerfont{subitem projected}{size=\scriptsize}
\setbeamerfont{subsubitem projected}{size=\tiny}

\usefonttheme{professionalfonts}
%\usepackage{arev}

%% ---------------------------- color theme --------------------------

% always use rgb colors in pdf files
\substitutecolormodel{hsb}{rgb}

\definecolor{red99}{Hsb}{0,0.9,0.9}
\definecolor{brown74}{Hsb}{30,0.7,0.4}
\definecolor{brown75}{Hsb}{30,0.7,0.5}
\definecolor{yellow86}{Hsb}{60,0.8,0.6}
\definecolor{yellow99}{Hsb}{60,0.9,0.9}
\definecolor{cyan95}{Hsb}{180,0.9,0.5}
\definecolor{blue67}{Hsb}{240,0.6,0.7}
\definecolor{blue74}{Hsb}{240,0.7,0.4}
\definecolor{blue77}{Hsb}{240,0.7,0.7}
\definecolor{blue99}{Hsb}{240,0.9,0.9}
\definecolor{magenta88}{Hsb}{300,0.8,0.8}

\colorlet{text1}{black}
\colorlet{back1}{white}
\colorlet{accent1}{blue99}
\colorlet{accent2}{cyan95}
\colorlet{accent3}{red99}
\colorlet{accent4}{yellow86}
\colorlet{accent5}{magenta88}
\colorlet{filler1}{accent1!40!back1}
\colorlet{filler2}{accent2!40!back1}
\colorlet{filler3}{accent3!40!back1}
\colorlet{filler4}{accent4!40!back1}
\colorlet{filler5}{accent5!40!back1}
\colorlet{gray1}{black!20}
\colorlet{gray2}{black!35}
\colorlet{gray3}{black!50}
\colorlet{gray4}{black!65}
\colorlet{gray5}{black!80}
\colorlet{tcolor}{text1!10!back1}
\colorlet{dcolor}{white}
\colorlet{fcolor}{blue77}
\colorlet{markcolor}{gray}
\colorlet{sectcolor}{brown74}

\setbeamercolor{normal text}{bg=white,fg=black}
\setbeamercolor{structure}{fg=blue99}
\setbeamercolor{local structure}{fg=cyan95}
\setbeamercolor{footline}{bg=,fg=black}
\setbeamercolor{title}{fg=yellow99}
\setbeamercolor{subtitle}{fg=white}
\setbeamercolor{author}{fg=black}
\setbeamercolor{institute}{fg=black}
\setbeamercolor{date}{fg=white}
\setbeamercolor{occasion}{fg=white}
\setbeamercolor{section name}{fg=brown75}
\setbeamercolor{section in toc}{fg=yellow99,bg=blue67}
\setbeamercolor{section in toc shaded}{fg=white,bg=blue74}
\setbeamercolor{subsection name}{parent=section name}
\setbeamercolor{subsection in toc}{use={structure,normal text},fg=structure.fg!90!normal text.bg}
\setbeamercolor{subsection in toc shaded}{parent=normal text}
\setbeamercolor{frametitle}{parent=structure}
\setbeamercolor{separator line}{fg=accent2}
\setbeamercolor{theoremname}{parent=subsection in toc}
\setbeamercolor{proofname}{parent=subsection in toc}
\setbeamercolor{block title}{fg=accent1,bg=gray}
\setbeamercolor{block body}{bg=lightgray}
\setbeamercolor{block title example}{fg=accent2,bg=gray}
\setbeamercolor{block body example}{bg=lightgray}
\setbeamercolor{block title alerted}{fg=accent3,bg=gray}
\setbeamercolor{block body alerted}{bg=lightgray}

%% ----------------------- handout mode ------------------------------

\mode<handout>{
  \setbeamertemplate{background canvas}{}
  \setbeamertemplate{background}[empty]
  \setbeamertemplate{footline}[sectioning]
  \setbeamertemplate{section page}[single]
  \setbeamertemplate{subsection page}[single]
  \setbeamerfont{subsection in toc}{size=\large}
  \colorlet{dcolor}{darkgray}
  \colorlet{fcolor}{white}
  \colorlet{sectcolor}{white}
  \setbeamercolor{normal text}{fg=black, bg=white}
  \setbeamercolor{title}{fg=blue}
  \setbeamercolor{subtitle}{fg=gray}
  \setbeamercolor{occasion}{fg=black}
  \setbeamercolor{date}{fg=black}
  \setbeamercolor{section in toc}{fg=blue!90!gray,bg=}
  \setbeamercolor{section in toc shaded}{fg=lightgray,bg=}
  \setbeamercolor{subsection in toc}{fg=blue!80!gray}
  \setbeamercolor{subsection in toc shaded}{fg=lightgray}
  \setbeamercolor{frametitle}{fg=blue!70!gray,bg=}
  \setbeamercolor{theoremname}{fg=blue!60!gray}
  \setbeamercolor{proofname}{fg=blue!60!gray}
  \setbeamercolor{footline}{bg=white,fg=black}
}

\mode
<all>

\makeatother

% -*- coding: utf-8 -*-

% ----------------------------------------------
% 中文显示相关代码
% ----------------------------------------------

% 以前要放在 usetheme 后面,否则报错;但是现在没问题了
\PassOptionsToPackage{CJKnumber}{xeCJK}
\usepackage[UTF8,noindent]{ctex}
%\usepackage[UTF8,indent]{ctexcap}

% 开明式标点:句末点号用全角,其他用半角。
%\punctstyle{kaiming}

% 在旧版本 xecjk 中用 CJKnumber 选项会自动载入 CJKnumb 包
% 但在新版本 xecjk 中 CJKnumber 选项已经被废弃,需要在后面自行载入它
\usepackage{CJKnumb}

%\CTEXoptions[today=big] % 数字年份前会有多余空白,中文年份前是正常的

\makeatletter
\ifxetex
  \setCJKsansfont{SimHei} % fix for ctex 2.0
  \renewcommand\CJKfamilydefault{\CJKsfdefault}%
\else
  \@ifpackagelater{ctex}{2014/03/01}{}{\AtBeginDocument{\heiti}} %无效?
\fi
\makeatother

%% 在旧版本 ctex 中,\today 命令生成的中文日期前面有多余空格
\makeatletter
\@ifpackagelater{ctex}{2014/03/01}{}{%
  \renewcommand{\today}{\number\year 年 \number\month 月 \number\day 日}
}
\makeatother

%% 在 xeCJK 中,默认将一些字符排除在 CJK 类别之外,需要时可以加入进来
%% 可以在 “附件->系统工具->字符映射表”中查看某字体包含哪些字符
% https://en.wikipedia.org/wiki/Number_Forms
% Ⅰ、Ⅱ、Ⅲ、Ⅳ、Ⅴ、Ⅵ、Ⅶ、Ⅷ、Ⅸ、Ⅹ、Ⅺ、Ⅻ
\xeCJKsetcharclass{"2150}{"218F}{1} % 斜线分数,全角罗马数字等
% https://en.wikipedia.org/wiki/Enclosed_Alphanumerics
\xeCJKsetcharclass{"2460}{"24FF}{1} % 带圈数字字母,括号数字字母,带点数字等

\ifxetex
% 在标点后,xeCJK 会自动添加空格,但不会去掉换行空格
%\catcode`,=\active  \def,{\textup{,} \ignorespaces}
%\catcode`;=\active  \def;{\textup{;} \ignorespaces}
%\catcode`:=\active  \def:{\textup{:} \ignorespaces}
%\catcode`。=\active  \def。{\textup{.} \ignorespaces}
%\catcode`.=\active  \def.{\textup{.} \ignorespaces}
\catcode`。=\active   \def。{.}
% 在公式中使用中文逗号和分号
%\let\douhao, \def\zhdouhao{\text{,\hskip-0.5em}}
%\let\fenhao; \def\zhfenhao{\text{;\hskip-0.5em}}
%\begingroup
%\catcode`\,=\active \protected\gdef,{\text{,\hskip-0.5em}}
%\catcode`\;=\active \protected\gdef;{\text{;\hskip-0.5em}}
% 似乎 beamer 的 \onslide<1,3> 不受影响
% 但是如果 tikz 图形中包含逗号,可能无法编译
%\catcode`\,=\active
%\protected\gdef,{\ifmmode\expandafter\zhdouhao\else\expandafter\douhao\fi}
%\catcode`\;=\active
%\protected\gdef;{\ifmmode\expandafter\zhfenhao\else\expandafter\fenhao\fi}
%\endgroup
%\AtBeginDocument{\catcode`\,=\active \catcode`\;=\active}
% 这样写反而无效
%\def\zhpunct{\catcode`\,=\active \catcode`\;=\active}
%\AtBeginDocument{%
%  \everymath\expandafter{\the\everymath\zhpunct}%
%  \everydisplay\expandafter{\the\everydisplay\zhpunct}%
%}
% 改为使用 kerkis 字体的逗号
\DeclareSymbolFont{myletters}{OML}{mak}{m}{it}
\SetSymbolFont{myletters}{bold}{OML}{mak}{b}{it}
\AtBeginDocument{%
  \DeclareMathSymbol{,}{\mathpunct}{myletters}{"3B}%
}
\fi

% 汉字下面加点表示强调
\usepackage{CJKfntef}

% ----------------------------------------------
% 字体选用相关代码
% ----------------------------------------------

% 虽然 arevtext 字体的宽度较大,但考虑到文档的美观还是同时使用 arevtext 和 arevmath
% 若在 ctex 包之前载入它,其设定的 arevtext 字体会在载入 ctex 时被重置为 lm 字体
% 因此我们在 ctex 宏包之后才载入它,但此时字体编码被改为 T1,需要修正 \nobreakspace
\usepackage{arev}
\DeclareTextCommandDefault{\nobreakspace}{\leavevmode\nobreak\ }

% 即使只需要 arevmath,也不能直接用 \usepackage{arevmath},
% 因为旧版本 fontspec 有问题,这样会导致它错误地修改数学字体

% 旧版本的 XeTeX 无法使用 arev sans 等 T1 编码字体的单独重音字符
% 因此我们恢复使用组合重音字符,见 t1enc.def, fntguide.pdf 和 encguide.pdf
\ifxetex\ifdim\the\XeTeXversion\XeTeXrevision pt<0.9999pt
  \DeclareTextAccent{\'}{T1}{1}
\fi\fi
% 在 T1enc.def 文件中定义了 T1 编码中的重音字符
% 先用 \DeclareTextAccent{\'}{T1}{1} 表示在 T1 编码中 \'e 等于 \accent"01 e
% 再用 \DeclareTextComposite{\'}{T1}{e}{233} 表示在 T1 编码中 \'e 等于 \char"E9

% ----------------------------------------------
% 版式定制相关代码
% ----------------------------------------------

\usepackage{hyperref}
\hypersetup{
  %pdfpagemode={FullScreen},
  bookmarksnumbered=true,
  unicode=true
}

%% 保证在新旧 ctex 宏包下编译得到相同的结果
\renewcommand{\baselinestretch}{1.3} % ctex 2.4.1 开始为 1,之前为 1.3

%% LaTeX 中 默认 \parskip=0pt plus 1pt,而 Beamer 中默认 \parskip=0pt

%% \parskip 用 plus 1fil 没有效果,用 plus 1fill 则节标题错位
\setlength{\parskip}{5pt plus 1pt minus 1pt}       % 段间距为 5pt + 1pt - 1pt
%\setlength{\baselineskip}{19pt plus 1pt minus 1pt} % 行间距为 5pt + 1pt - 1pt
\setlength{\lineskiplimit}{4pt}                    % 行间距小于 4pt 时重新设置
\setlength{\lineskip}{4pt}                         % 行间距太小时自动改为 4pt

\AtBeginDocument{
  \setlength{\baselineskip}{19pt plus 1pt minus 1pt} % 似乎不能放在导言区中
  \setlength{\abovedisplayskip}{4pt minus 2pt}
  \setlength{\belowdisplayskip}{4pt minus 2pt}
  \setlength{\abovedisplayshortskip}{2pt}
  \setlength{\belowdisplayshortskip}{2pt}
}

% 默认是 \raggedright,改为两边对齐
\usepackage{ragged2e}
\justifying
\let\oldraggedright\raggedright
\let\raggedright\justifying

% ----------------------------------------------
% 文本环境相关代码
% ----------------------------------------------

\setlength{\fboxsep}{0.02\textwidth}\setlength{\fboxrule}{0.002\textwidth}

\usepackage{adjustbox}
\newcommand{\mylinebox}[1]{\adjustbox{width=\linewidth}{#1}}

\usepackage{comment}
\usepackage{multicol}

% 带圈的数字
%\newcommand{\digitcircled}[1]{\textcircled{\raisebox{.8pt}{\small #1}}}
\newcommand{\digitcircled}[1]{%
  \tikz[baseline=(char.base)]{%
     \node[shape=circle,draw,inner sep=0.01em,line width=0.07em] (char) {\small #1};
  }%
}

\usepackage{pifont}
% 因为 xypic 将 \1 定义为 frm[o]{--},这里改为在文档内部定义
%\def\1{\ding{51}} % 勾
%\def\0{\ding{55}} % 叉

% 若在 enumerate 中使用自定义模板,则各项前的间距由第七项决定
% 对于我们使用的 arev 数学字体来说各个数字是等宽的,所以没问题
% 参考 https://tex.stackexchange.com/q/377959/8956
% 以及 https://chat.stackexchange.com/transcript/message/38541073#38541073
\newenvironment{enumskip}[1][]{%
  \setbeamertemplate{enumerate mini template}[default]%
  \if\relax\detokenize{#1}\relax % empty
    \begin{enumerate}[\quad(1)]
  \else
    \begin{enumerate}[#1][\quad(1)]
  \fi
}{\end{enumerate}}
\newenvironment{enumzero}[1][]{%
  \setbeamertemplate{enumerate mini template}[default]%
  \if\relax\detokenize{#1}\relax % empty
    \begin{enumerate}[(1)\,]
  \else
    \begin{enumerate}[#1][(1)\,]
  \fi
}{\end{enumerate}}
%
\newenvironment{enumlite}[1][]{%
  \setbeamertemplate{enumerate mini template}[default]%
  \setbeamercolor{enumerate item}{fg=,bg=}%
  \if\relax\detokenize{#1}\relax % empty
    \begin{enumerate}[(1)\,]%
  \else
    \begin{enumerate}[#1][(1)\,]%
  \fi
}{\end{enumerate}}
%
\newcounter{mylistcnt}
%
\newenvironment{enumhalf}{%
  \par\setcounter{mylistcnt}{0}%
  \def\item##1~{%
    \leavevmode\ifhmode\unskip\fi\linebreak[2]%
    \makebox[.5002\textwidth][l]{\stepcounter{mylistcnt}(\arabic{mylistcnt}) \,##1\ignorespaces}%
  }%
  \ignorespaces%
}{\par}
%
\newenvironment{choiceline}[1][]{%
  \par\vskip-0.5em\relax
  \setbeamertemplate{enumerate mini template}[default]%
  \setbeamercolor{enumerate item}{fg=,bg=}%
  \if\relax\detokenize{#1}\relax % empty
    \begin{enumerate}[(A)\,]
  \else
    \begin{enumerate}[#1][(A)\,]
  \fi
}{\end{enumerate}}
%
\newenvironment{choicehalf}{%
  \par\setcounter{mylistcnt}{0}%
  \def\item##1~{%
    \leavevmode\ifhmode\unskip\fi\linebreak[2]%
    \makebox[.5001\textwidth][l]{\stepcounter{mylistcnt}(\Alph{mylistcnt}) \,##1\ignorespaces}%
  }%
  \ignorespaces%
}{\par}
\newenvironment{choicequar}{%
  \par\setcounter{mylistcnt}{0}%
  \def\item##1~{%
    \leavevmode\ifhmode\unskip\fi\linebreak[0]%
    \makebox[.2501\textwidth][l]{\stepcounter{mylistcnt}(\Alph{mylistcnt}) \,##1\ignorespaces}%
  }%
  \ignorespaces%
}{\par}

% ----------------------------------------------
% 定理环境相关代码
% ----------------------------------------------

\makeatletter
\patchcmd{\@thm}{ \csname}{\kern0.18em\relax\csname}{}{}
\makeatother

\newcommand{\mynewtheorem}[2]{%
  \newtheorem{#1}{#2}[section]%
  \expandafter\renewcommand\csname the#1\endcsname{\arabic{#1}}%
}

\mynewtheorem{theorem}{定理}
\newtheorem*{theorem*}{定理}

\mynewtheorem{algorithm}{算法}
\newtheorem*{algorithm*}{算法}

\mynewtheorem{conjecture}{猜想}
\newtheorem*{conjecture*}{猜想}

\mynewtheorem{corollary}{推论}
\newtheorem*{corollary*}{推论}

\mynewtheorem{definition}{定义}
\newtheorem*{definition*}{定义}

\mynewtheorem{example}{例}
\newtheorem*{example*}{例子}

\mynewtheorem{exercise}{练习}
\newtheorem*{exercise*}{练习}

\mynewtheorem{fact}{事实}
\newtheorem*{fact*}{事实}

\mynewtheorem{guess}{猜测}
\newtheorem*{guess*}{猜测}

\mynewtheorem{lemma}{引理}
\newtheorem*{lemma*}{引理}

\mynewtheorem{method}{解法}
\newtheorem*{method*}{解法}

\mynewtheorem{origin}{引例}
\newtheorem*{origin*}{引例}

\mynewtheorem{problem}{问题}
\newtheorem*{problem*}{问题}

\mynewtheorem{property}{性质}
\newtheorem*{property*}{性质}

\mynewtheorem{proposition}{命题}
\newtheorem*{proposition*}{命题}

\mynewtheorem{puzzle}{题}
\newtheorem*{puzzle*}{题目}

\mynewtheorem{remark}{注记}
\newtheorem*{remark*}{注记}

\mynewtheorem{review}{复习}
\newtheorem*{review*}{复习}

\mynewtheorem{result}{结论}
\newtheorem*{result*}{结论}

\newtheorem*{analysis}{分析}
\newtheorem*{answer}{答案}
\newtheorem*{choice}{选择}
\newtheorem*{hint}{提示}
\newtheorem*{solution}{解答}
\newtheorem*{thinking}{思考}

\newcommand{\mynewtheoremx}[2]{%
  \newtheorem{#1}{#2}%
  \expandafter\renewcommand\csname the#1\endcsname{\arabic{#1}}%
}
\mynewtheoremx{bonus}{选做}
\newtheorem*{bonus*}{选做}

\renewcommand{\proofname}{证明}
\renewcommand{\qedsymbol}{}
\renewcommand{\tablename}{表格}

% ----------------------------------------------
% 数学环境相关代码
% ----------------------------------------------

% 选学内容的角标星号
\newcommand{\optstar}{\texorpdfstring{\kern0pt$^\ast$}{}}

\usepackage{mathtools} % \mathclap 命令
\usepackage{extarrows}

% 切换 amsmath 的公式编号位置
% 不使用 leqno 选项而在这里才修改,会导致编号与公式重叠
% 因此在 \documentclass 里都加上了 leqno 选项
\makeatletter
\newcommand{\leqnomath}{\tagsleft@true}
\newcommand{\reqnomath}{\tagsleft@false}
\makeatother
%\leqnomath

% 定义带圈数字的 tag 格式,需要 mathtools 包
\newtagform{circ}[\color{accent2}\digitcircled]{}{}
\newtagform{skip}[\quad\color{accent2}\digitcircled]{}{}

\newcounter{savedequation}

\newenvironment{aligncirc}{%
  \setcounter{savedequation}{\value{equation}}%
  \setcounter{equation}{0}%
  \usetagform{circ}%
  \align
}{
  \endalign
  \setcounter{equation}{\value{savedequation}}%
}
\newenvironment{alignskip}{%
  \setcounter{savedequation}{\value{equation}}%
  \setcounter{equation}{0}%
  \usetagform{skip}%
  \align
}{
  \endalign
  \setcounter{equation}{\value{savedequation}}%
}
\newenvironment{alignlite}{%
  \setcounter{savedequation}{\value{equation}}%
  \setcounter{equation}{0}%
  \align
}{
  \endalign
  \setcounter{equation}{\value{savedequation}}%
}

% cases 环境开始时 \def\arraystretch{1.2}
% 在中文文档中还有 \linespread{1.3},这样公式的左花括号就太大了
% 这里利用 etoolbox 包将 \linespread 临时改回 1
\AtBeginEnvironment{cases}{\linespread{1}\selectfont}
% 奇怪的是在最新的 miktex 中无此问题,
% 而且即使这样修改,在新旧 miktex 中用 arev 字体时花括号大小还是有差别
% 而用默认的 lm 字体时花括号却没有差别

% 用于给带括号的方程组编号
\usepackage{cases}

\newcommand{\R}{\mathbb{R}}
\newcommand{\Rn}{\mathbb{R}^n}

% 大型的积分号
\usepackage{relsize}
\newcommand{\Int}{\mathop{\mathlarger{\int}}}

% \oiint 命令
% \usepackage[integrals]{wasysym}

% http://tex.stackexchange.com/q/84302
\DeclareMathOperator{\arccot}{arccot}

% https://tex.stackexchange.com/a/178948/8956
% 保证 \d x 和 \d(2x) 和 \d^2 x 的间距都合适
\let\oldd=\d
\renewcommand{\d}{\mathop{}\!\mathrm{d}}
\newcommand{\dx}{\d x}
\newcommand{\dy}{\d y}
\def\dz{\d z} % 不确定命令是否已经定义
\newcommand{\du}{\d u}
\newcommand{\dv}{\d v}
\newcommand{\dr}{\d r}
\newcommand{\ds}{\d s}
\newcommand{\dt}{\d t}
\newcommand{\dS}{\d S}

\newcommand{\e}{\mathrm{e}}
\newcommand{\limit}{\lim\limits}

% 分数线长一点的分数,\wfrac[2pt]{x}{y} 表示左右加 2pt
% 参考 http://tex.stackexchange.com/a/21580/8956
\DeclareRobustCommand{\wfrac}[3][2pt]{%
  {\begingroup\hspace{#1}#2\hspace{#1}\endgroup\over\hspace{#1}#3\hspace{#1}}}

% 划去部分公式
% 横着划线,参考 http://tex.stackexchange.com/a/20613/8956
\newcommand{\hcancel}[2][accent3]{%
  \setbox0=\hbox{$#2$}%
  \rlap{\raisebox{.3\ht0}{\textcolor{#1}{\rule{\wd0}{1pt}}}}#2%
}
% 斜着划线,参考 https://tex.stackexchange.com/a/15958
\newcommand{\dcancel}[2][accent3]{%
    \tikz[baseline=(tocancel.base),ultra thick]{
        \node[inner sep=0pt,outer sep=0pt] (tocancel) {$#2$};
        \draw[#1] (tocancel.south west) -- (tocancel.north east);
    }%
}%

% 竖直居中的 \dotfill
\newcommand\cdotfill{\leavevmode\xleaders\hbox to 0.5em{\hss\footnotesize$\cdot$\hss}\hfill\kern0pt\relax}

% 保持居中的 \not 命令
\usepackage{centernot}

% 使用 stix font 中的 white arrows
\ifxetex
    \IfFileExists{STIX-Regular.otf}{%
        \newfontfamily{\mystix}{STIX} % stix v1.1
    }{%
        \newfontfamily{\mystix}{STIXGeneral} % stix v1.0
    }
    \DeclareRobustCommand\leftwhitearrow{%
      \mathrel{\text{\normalfont\mystix\symbol{"21E6}}}%
    }
    \DeclareRobustCommand\upwhitearrow{%
      \mathrel{\text{\normalfont\mystix\symbol{"21E7}}}%
    }
    \DeclareRobustCommand\rightwhitearrow{%
      \mathrel{\text{\normalfont\mystix\symbol{"21E8}}}%
    }
    \DeclareRobustCommand\downwhitearrow{%
      \mathrel{\text{\normalfont\mystix\symbol{"21E9}}}%
    }
\else
    \let \leftwhitearrow = \Leftarrow
    \let \rightwhitearrow = \Rightarrow
    \let \upwhitearrow = \Uparrow
    \let \downwhitearrow = \Downarrow
\fi

% ----------------------------------------------
% 绘图动画相关代码
% ----------------------------------------------

% pgf/tikz 的所有选项都称为 key,它们按照 unix 路径的方式组织,
% 例如:/tikz/external/force remake={boolean}
% 这些 key 可以用 \pgfkeys 定义,用 \tikzset 设置
% \tikzset 实际上等同于 \pgfkeys{/tikz/.cd,#1}.
% Using Graphic Options: P120 in manual 2.10 (\tikzset)
% Key Management: P481 in manual 2.10 (\pgfkeys)

\usepackage{tikz}
\usepackage{pgfplots}
%\usepackage{calc}

% 文档标注,通常需要编译两次就可以得到正确结果
% 但如果主题的 section page 用 tikz 绘图,将需要编译三次
% 这是因为 tikzmark 依赖 aux 文件的 pgfid 编号
% 第一次编译时缺少 toc 文件,将缺少若干个 tikz 图片
% 第二次编译时图片个数正确了,但是 aux 文件的 pgfid 仍然是错误的
% 这个问题在主题文件中已经修正了
\newcommand\tikzmark[1]{%
  \tikz[overlay,remember picture] \node[coordinate] (#1) {};%
}

% pgf 包含的  xxcolor 包存在问题,导致与 xeCJKfntef 包冲突
% 见 https://github.com/CTeX-org/ctex-kit/issues/323
% 注意此冲突在 ctex 2.9 中不存在,仅在最新的 miktex 2.9 中出现
\makeatletter
\g@addto@macro\XC@mcolor{\edef\current@color{\current@color}}
\makeatother

\usetikzlibrary{matrix} % 用于在 node 四周加括号
\usetikzlibrary{decorations}
\usetikzlibrary{decorations.markings} % 用于在箭头上作标记
\usetikzlibrary{intersections} % 用于计算路径的交点
\usetikzlibrary{positioning} % 可以更方便地定位
\usetikzlibrary{shapes.geometric} % 钻石形状节点

\usetikzlibrary{calc}
\usetikzlibrary{snakes}

% Externalizing Graphics: P343 and P651 in manual 2.10
\usetikzlibrary{external}
% 编译时需加上 --shell-escape(texlive)或 -enable-write18(miktex)选项
%\tikzexternalize[prefix=figure/] %\tikzexternalize[shell escape=-enable-write18]

% 默认 tikz 图片会用 pdflatex 编译,可以自己改为 xelatex
%\tikzset{external/system call={%
%  xelatex \tikzexternalcheckshellescape -halt-on-error -interaction=batchmode -jobname "\image" "\texsource"}}

% 强制重新生成图片, pgf 3.0 中会自动比较文件的 md5
%\tikzset{external/force remake} %\tikzset{external/remake next}

%\tikzset{draw=black,color=black}
%\mode<beamer>{\tikzset{every path/.style={color=white!90!black}}}

\usetikzlibrary{patterns}

% hack pgf prior to version 3.0 for pgf patterns in xetex
% code taken from pgfsys-dvipdfmx.def and pgfsys-xetex.def in pgf 3.0
\makeatletter
\def\myhackpgf{
  % fix typo in pgfsys-common-pdf-via-dvi.def in pgf 2.10
  \pgfutil@insertatbegineverypage{%
     \ifpgf@sys@pdf@any@resources%
        \special{pdf:put @resources
           << \ifpgf@sys@pdf@patterns@exists /Pattern @pgfpatterns \fi >>}%
     \fi%
  }
  % required to give colors on pattern objects.
  \pgfutil@addpdfresource@colorspaces{ /pgfprgb [/Pattern /DeviceRGB] }
  % hook for xdvipdfmx
  \def\pgfsys@dvipdfmx@patternobj##1{%
	 \pgfutil@insertatbegincurrentpagefrombox{##1}%
  }%
  % dvipdfmx provides a new special `pdf:stream' for a stream object
  \def\pgfsys@dvipdfmx@stream##1##2##3{%
     \special{pdf:stream ##1 (##2) << ##3 >>}%
  }%
  % declare patterns and set patterns
  \def\pgfsys@declarepattern##1##2##3##4##5##6##7##8##9{%
     \pgf@xa=##2\relax \pgf@ya=##3\relax%
     \pgf@xb=##4\relax \pgf@yb=##5\relax%
     \pgf@xc=##6\relax \pgf@yc=##7\relax%
     \pgf@sys@bp@correct\pgf@xa \pgf@sys@bp@correct\pgf@ya%
     \pgf@sys@bp@correct\pgf@xb \pgf@sys@bp@correct\pgf@yb%
     \pgf@sys@bp@correct\pgf@xc \pgf@sys@bp@correct\pgf@yc%
     \pgfsys@dvipdfmx@patternobj{%
        \pgfsys@dvipdfmx@stream{@pgfpatternobject##1}{##8}{%
           /Type /Pattern
           /PatternType 1
           /PaintType \ifnum##9=0 2 \else 1 \fi
           /TilingType 1
           /BBox [\pgf@sys@tonumber\pgf@xa\space\pgf@sys@tonumber\pgf@ya\space
                  \pgf@sys@tonumber\pgf@xb\space\pgf@sys@tonumber\pgf@yb]
           /XStep \pgf@sys@tonumber\pgf@xc\space
           /YStep \pgf@sys@tonumber\pgf@yc\space
           /Resources << >> %<<
        }%
     }%
     \pgfutil@addpdfresource@patterns{/pgfpat##1\space @pgfpatternobject##1}%
  }
  \def\pgfsys@setpatternuncolored##1##2##3##4{%
     \pgfsysprotocol@literal{/pgfprgb cs ##2 ##3 ##4 /pgfpat##1\space scn}%
  }
  \def\pgfsys@setpatterncolored##1{%
     \pgfsysprotocol@literal{/Pattern cs /pgfpat##1\space scn}%
  }
}
\@ifpackagelater{pgf}{2013/12/18}{}{\ifxetex\expandafter\myhackpgf\fi}%
\makeatother

% ----------------------------------------------
% 表格制作相关代码
% ----------------------------------------------

\newcommand{\narrowsep}[1][2pt]{\setlength{\arraycolsep}{#1}}
\newcommand{\narrowtab}[1][3pt]{\setlength{\tabcolsep}{#1}}

% diagbox 依赖 pict2e,但 miktex 中旧版本 pict2e 打包错误,使得引擎判别错误
% 从而导致在编译时出现大量警告,以及导致底栏右下角按钮链接错位
\ifxetex\PassOptionsToPackage{xetex}{pict2e}\fi
\usepackage{diagbox}

\usepackage{multirow} % 跨行表格

\usepackage{array} % 可以用 \extrarowheight
% 双倍宽度的横线和竖线,\arrayrulewidth 默认为 0.4pt
\setlength{\doublerulesep}{0pt}
\newcommand{\dhline}{\noalign{\global\arrayrulewidth0.8pt}\hline\noalign{\global\arrayrulewidth0.4pt}}
\newcolumntype{?}{!{\vrule width 0.8pt}} % 即使 \doublerulesep 为 0pt,|| 也不能得到双倍宽度
% 最好还是用 tabu,更简单

\usepackage{tabularx}

%\usepackage{arydshln} % 在分块矩阵中加虚线
%\setlength{\dashlinedash}{2pt} % 默认4pt
%\setlength{\dashlinegap}{2pt} % 默认4pt

% tabu 与 arydshln 会冲突,可以不使用 arydshln,
% 而用 tabu 定义虚线 \newcolumntype{:}{|[on 2pt off 2pt]}
% 参考 http://bbs.ctex.org/forum.php?mod=viewthread&tid=63944#pid405057
\usepackage{tabu}
\newcolumntype{:}{|[on 2pt off 2pt]}
\newcommand{\hdashline}{\tabucline[0.4pt on 2pt off 2pt]{-}} % 兼容 arydshln 的命令
\setlength{\tabulinesep}{4pt} % 拉开大型公式与表格横线的距离

%\usepackage{colortbl} % 否则 \taburowcolors 命令无效

% ----------------------------------------------
% 绝对定位相关代码
% ----------------------------------------------

\usepackage[absolute,overlay]{textpos}

% 将整个页面分为 32 乘 24 个边长为 4mm 的小正方形
\setlength{\TPHorizModule}{4mm}
\setlength{\TPVertModule}{4mm}

\setlength{\TPboxrulesize}{0.6pt}
\newlength{\tpmargin}
\setlength{\tpmargin}{2mm}

\newenvironment{bblock}[1][black]{%
  \begingroup
  \TPshowboxestrue\TPMargin{\tpmargin}%
  \textblockrulecolor{#1}\textblockcolour{}%
  \begin{textblock}%
}{%
  \end{textblock}%
  \endgroup
}
\newenvironment{cblock}[2][black]{%
  \begingroup
  \TPshowboxestrue\TPMargin{\tpmargin}%
  \textblockrulecolor{#1}\textblockcolour{#2}%
  \begin{textblock}%
}{%
  \end{textblock}%
  \endgroup
}
\newenvironment{cblocka}{\begin{cblock}{filler1}}{\end{cblock}}
\newenvironment{cblockb}{\begin{cblock}{filler2}}{\end{cblock}}
\newenvironment{cblockc}{\begin{cblock}{filler3}}{\end{cblock}}
\newenvironment{cblockd}{\begin{cblock}{filler4}}{\end{cblock}}
\newenvironment{cblocke}{\begin{cblock}{filler5}}{\end{cblock}}

% ----------------------------------------------
% 模版定制相关代码
% ----------------------------------------------

\usepackage{bookmark}

\newcommand{\mybookmark}[1]{%
  \bookmark[page=\thepage,level=3]{#1}%
  \changenavibox
}

%\setbeamercovered{transparent=5}

\setbeamersize{text margin left=4mm,text margin right=4mm}
\mode<beamer>{\setbeamertemplate{background}[linear]}
\setbeamertemplate{footline}[sectioning]
\setbeamertemplate{footline right}[normal]
\setbeamertemplate{theorem begin}[simple]
\setbeamertemplate{theorem end}[simple]
\setbeamertemplate{proof begin}[simple]
\setbeamertemplate{proof end}[simple]

% 段间距在 block 中也许无效 http://tex.stackexchange.com/q/6111/8956
%\addtobeamertemplate{block begin}{}{\setlength{\parskip}{6pt plus 2pt minus 2pt}}

%\mode<beamer>{\tikzset{every path/.style={color=black}}}

% 在 amsfonts.sty 中已经废弃 \bold 命令,改用 \mathbf 命令
\def\lead#1{\textcolor{accent1}{#1}}
\def\bold#1{\textcolor{accent2}{#1}}
\def\warn#1{\textcolor{accent3}{#1}}
\def\clead{\color{accent1}}
\def\cbold{\color{accent2}}
\def\cwarn{\color{accent3}}

\mode<handout>{
  \colorlet{filler1}{filler1!40!white}
  \colorlet{filler2}{filler2!40!white}
  \colorlet{filler3}{filler3!40!white}
  \colorlet{filler4}{filler4!40!white}
  \colorlet{filler5}{filler5!40!white}
  \colorlet{gray1}{gray1!60!white}
  \colorlet{gray2}{gray2!60!white}
  \colorlet{gray3}{gray3!60!white}
  \colorlet{gray4}{gray4!60!white}
  \colorlet{gray5}{gray5!60!white}
}

% 兼容性命令,在 beamer 中应该避免使用它们,而改用上面几个命令
\let\textbf=\bold \def\pmb{\usebeamercolor[fg]{local structure}}
\let\emph=\warn   \def\bm{\usebeamercolor[fg]{alerted text}}

\newcommand{\vpause}{\pause\vskip 0pt plus 0.5fill\relax}
\newcommand{\ppause}{\par\pause}

\newcommand{\mybackground}{\setbeamertemplate{background}[lattice][4mm]}
% 几个 \varxxx 命令是 arevmath 包提供的
% $\heartsuit\varheart\diamondsuit\vardiamond$
% $\varspade\spadesuit\varclub\clubsuit$
% rframe 为例题解答,sframe 为练习解答,可以选择不包含它们
\newenvironment{rframe}{\mybackground\begin{frame}}{\end{frame}}
\newenvironment{sframe}{%
  \mybackground
  \colorlet{markcolor}{accent4}%
  \backgroundmarklefttrue\backgroundmarkrighttrue
  \begin{frame}
}{\end{frame}}
\ifdefined\slide
  \setbeamertemplate{footline}[navigation]
  \renewenvironment{rframe}{\begin{frame}<beamer:0>}{\end{frame}}%
  \renewenvironment{sframe}{\begin{frame}<beamer:0>}{\end{frame}}%
\fi
\ifdefined\print
  \renewenvironment{sframe}{\begin{frame}<handout:0>}{\end{frame}}%
\fi
% 用于标示只针对内招或外招的内容:iframe 为内招,oframe 为外招
\newenvironment{iframe}{\backgroundmarklefttrue\begin{frame}}{\end{frame}}
\newenvironment{oframe}{\backgroundmarkrighttrue\begin{frame}}{\end{frame}}
\newenvironment{jframe}{\mybackground\backgroundmarklefttrue\begin{frame}}{\end{frame}}
\newenvironment{pframe}{\mybackground\backgroundmarkrighttrue\begin{frame}}{\end{frame}}
\def\myimode{i}
\def\myomode{o}
\ifx\slide\myimode
  \renewenvironment{oframe}{\begin{frame}<presentation:0>}{\end{frame}}%
  \renewenvironment{pframe}{\begin{frame}<presentation:0>}{\end{frame}}%
  \renewenvironment{jframe}{\begin{frame}<presentation:0>}{\end{frame}}%
\fi
\ifx\slide\myomode
  \renewenvironment{iframe}{\begin{frame}<presentation:0>}{\end{frame}}%
  \renewenvironment{jframe}{\begin{frame}<presentation:0>}{\end{frame}}%
  \renewenvironment{pframe}{\begin{frame}<presentation:0>}{\end{frame}}%
\fi
\ifx\print\myimode
  \renewenvironment{oframe}{\begin{frame}<presentation:0>}{\end{frame}}%
  \renewenvironment{pframe}{\begin{frame}<presentation:0>}{\end{frame}}%
\fi
\ifx\print\myomode
  \renewenvironment{iframe}{\begin{frame}<presentation:0>}{\end{frame}}%
  \renewenvironment{jframe}{\begin{frame}<presentation:0>}{\end{frame}}%
\fi

% 利用 tikzmark 作边注
\newcommand{\imark}[1][gray]{%
  \begin{tikzpicture}[overlay,remember picture]
    \node[coordinate] (A) {};
    \fill[color=#1] (current page.west |- A) rectangle +(1.2mm,0.6em);
  \end{tikzpicture}%
}
\newcommand{\omark}[1][gray]{%
  \begin{tikzpicture}[overlay,remember picture]
    \node[coordinate] (A) {};
    \fill[color=#1] (A -| current page.east) rectangle +(-1.2mm,0.6em);
  \end{tikzpicture}%
}
\newcommand{\smark}{%
  \imark[accent4]\omark[accent4]%
}
\newcommand{\itext}[1]{%
  \ifx\slide\myomode\else
    \ifx\print\myomode\else
      #1%
    \fi
  \fi
}
\newcommand{\otext}[1]{%
  \ifx\slide\myimode\else
    \ifx\print\myimode\else
      #1%
    \fi
  \fi
}
\newcommand{\stext}[1]{%
  \ifdefined\slide\else
    \ifdefined\print\else
      #1%
    \fi
  \fi
}

% 选择题的答案
\newcommand{\select}[1]{\qquad\stext{\llap{\makebox[2em]{\color{accent4}#1}}}}

%% 内外招同编号的定理,例子或练习等,需要将编号减一
\newcommand{\minusone}[1]{%
  \ifdefined\slide\else
    \ifdefined\print\else
      \addtocounter{#1}{-1}%
    \fi
  \fi
}

%\mode<beamer>{
%\def\mytoctemplate{
%  \setbeamerfont{section in toc}{size=\normalsize}
%  \setbeamerfont{subsection in toc}{size=\small}
%  \setbeamertemplate{section in toc shaded}[default][100]
%  \setbeamertemplate{subsection in toc}[subsections numbered]
%  \setbeamertemplate{subsection in toc shaded}[default][100]
%  \setbeamercolor{section in toc}{fg=structure.fg}
%  \setbeamercolor{section in toc shaded}{fg=structure.fg!50!black}
%  \setbeamercolor{subsection in toc}{fg=structure.fg}
%  \setbeamercolor{subsection in toc shaded}{fg=normal text.fg}
%  \begin{multicols}{2}
%  \tableofcontents[sectionstyle=show/shaded,subsectionstyle=show/shaded]
%  \end{multicols}
%}
%\AtBeginSection[]{\begin{frame}\frametitle{目录结构}\mytoctemplate\end{frame}}
%\AtBeginSubsection[]{\begin{frame}\frametitle{目录结构}\mytoctemplate\end{frame}}
%}

\mode<presentation>

\setbeamertemplate{section and subsection}[chinese]
\usebeamertemplate{section and subsection}

\mode
<all>

% -*- coding: utf-8 -*-

% ----------------------------------------------
% 高等数学中的定义和改动
% ----------------------------------------------

\newif\ifligong % 理工类或经济类
\ligongtrue

% Repeating Things: P504 in manual 2.10
\newcommand{\drawline}[4][]{%
  \foreach \v [remember=\v as \u,count=\i] in {#4} {
    \ifnum \i > 1
      \ifodd \i \draw[#1,#3] \u -- \v; \else \draw[#1,#2] \u -- \v; \fi
    \fi
  }
}
\newcommand{\drawplot}[5][]{%
  \foreach \v [remember=\v as \u,count=\i] in {#4} {
    \ifnum \i > 1
      \ifodd \i \draw[#1,#3] plot[domain=\u:\v] #5; \else \draw[#1,#2] plot[domain=\u:\v] #5; \fi
    \fi
  }
}

% http://tex.stackexchange.com/q/84302
\DeclareMathOperator{\Prj}{Prj}
\DeclareMathOperator{\grad}{grad}

\newcommand{\va}{\vec{a\vphantom{b}}}
\newcommand{\vb}{\vec{b}}
\newcommand{\vc}{\vec{c\vphantom{b}}}
\newcommand{\vd}{\vec{d}}
\newcommand{\ve}{\vec{e}}
\newcommand{\vi}{\vec{i}}
\newcommand{\vj}{\vec{j}}
\newcommand{\vk}{\vec{k}}
\newcommand{\vn}{\vec{n}}
\newcommand{\vs}{\vec{s}}
\newcommand{\vv}{\vec{v}}

\let\ov=\overrightarrow

% xcolor 支持 hsb 色彩模型,但 pgf 不支持,因此需要指定输出的色彩模型为 rgb
% 在 article 中可以用 \usepackage[rgb]{xcolor} \usepackage{tikz} 解决此问题
% 在 beamer 中可以用 \documentclass[xcolor={rgb}]{beamer} 解决此问题

%\definecolor{bcolor0}{Hsb}{0,0.6,0.9}   % red 红色
%\definecolor{bcolor1}{Hsb}{60,0.6,0.9}  % yellow 黄色
%\definecolor{bcolor2}{Hsb}{120,0.6,0.9} % green 绿色
%\definecolor{bcolor3}{Hsb}{180,0.6,0.9} % cyan 青色
%\definecolor{bcolor4}{Hsb}{240,0.6,0.9} % blue 蓝色
%\definecolor{bcolor5}{Hsb}{300,0.6,0.9} % magenta 洋红色

\colorlet{bcolor0}{accent3}
\colorlet{bcolor1}{accent1}
\colorlet{bcolor2}{accent2}
\colorlet{bcolor3}{accent4}
\colorlet{bcolor5}{accent5}


\begin{document}

\occasion{高等数学课程}
\title{第三章·导数的应用}
\author{\href{https://lvjr.bitbucket.io}{吕荐瑞}}
\institute{暨南大学数学系}

\begin{frame}[plain]
\titlepage
\end{frame}

\section{微分中值定理}

\subsection{罗尔中值定理}

\begin{iframe}
\frametitle{费马引理}
\lead{费马引理}\quad
设$f(x)$在点$x_0$的某邻域$U(x_0)$内有定义,
且$\forall x\in U(x_0)$有$f(x)\le f(x_0)$(或$f(x)\ge f(x_0)$).
如果$f(x)$在$x_0$处可导.则有$f'(x_0)=0$.
\end{iframe}

\begin{frame}
\frametitle{罗尔定理}
\begin{theorem}
如果函数$f(x)$满足条件:
\begin{enumskip}
\item 在闭区间$[a,b]$上连续,
\item 在开区间$(a,b)$上可导,
\item 在端点处$f(a)=f(b)$,
\end{enumskip}
则至少存在一点$\xi\in(a,b)$,使得$f'(\xi)=0$.
\end{theorem}
\vpause
\begin{fact}
如果定理的三个条件有一个不满足,则结论可能不成立.
\end{fact}
\end{frame}

\begin{frame}
\begin{example}
下列函数只满足罗尔定理的条件(2)和(3),不满足条件(1),因此没有导数为零的点.
\[f(x)=\begin{cases}x,&-1\leq x<1 \\
-1, &x=1\end{cases}\]
\end{example}
\end{frame}

\begin{frame}
\begin{example}
下列函数只满足罗尔定理的条件(1)和(3),不满足条件(2),因此没有导数为零的点.
\[f(x)=|x|, -1\leq x\leq 1\]
\end{example}
\vpause
\begin{example}
下列函数只满足罗尔定理的条件(1)和(2),不满足条件(3),因此没有导数为零的点.
\[f(x)=x, -1\leq x\leq 1\]
\end{example}
\end{frame}

\begin{frame}
\begin{example}
对$f(x)=x^2-2x-3$在区间$[-1,3]$上验证罗尔定理.
\end{example}%\pause
%\begin{example}
%对$f(x)=e^{x^2}-1$在区间$[-1,1]$上验证罗尔定理.
%\end{example}
\vpause
\begin{exercise}
%对$f(x)=2x^2-x-3$在区间$[-1,\frac32]$上验证罗尔定理.\\
对$f(x)=\dfrac1{1+x^2}$在区间$[-2,2]$上验证罗尔定理.
\end{exercise}
\end{frame}

\begin{frame}
\frametitle{罗尔定理}
\begin{example}
设$f(x)$在$[0,1]$上连续,在$(0,1)$上可导,而且$f(0)=0$,$f(1)=1$.证明:存在$\xi\in(0,1)$,使得$f'(\xi)=2\xi$.
\end{example}
\end{frame}

\subsection{拉格朗日中值定理}

\begin{frame}
\frametitle{拉格朗日定理}
\begin{theorem*}
如果函数$f(x)$满足下列条件:
\begin{enumskip}
\item 在闭区间$[a,b]$上连续,
\item 在开区间$(a,b)$内可导,
\end{enumskip}
则至少存在一点$\xi\in(a,b)$使得$f'(\xi)=\frac{f(b)-f(a)}{b-a}$.
\end{theorem*}
\end{frame}

\begin{frame}
\begin{example}
对函数$f(x)=x^3$在区间$[0,1]$上验证拉格朗日定理.
\end{example}%\pause
%\begin{example}
%对$f(x)=\mathrm{e}^x$在区间$[0,1]$上验证拉格朗日中值定理.
%\end{example}
\vpause
\begin{exercise}
对$f(x)=x^3+x$在区间$[-1,1]$上验证拉格朗日定理.\\
%对$f(x)=\ln x$在区间$[1,2]$上验证拉格朗日中值定理.
\end{exercise}
\end{frame}

\begin{frame}
\begin{corollary}
如果函数$f(x)$在区间$I$上的导数恒为$0$,那么$f(x)$在区间$I$上是一个常数.
\end{corollary}
\vpause
\begin{example}
证明当$-1\le x \le 1$时,有
$$\arcsin x + \arccos x = \dfrac{\pi}2.$$
\end{example}
\end{frame}

\begin{frame}
\begin{example}
证明当$x_2>x_1$时不等式成立:
$$\arctan x_2-\arctan x_1 \le x_2-x_1.$$
\end{example}
\vpause
\begin{exercise}
证明:当$x_2>x_1$时有
$$\sin x_2-\sin x_1 \le x_2-x_1.$$
\end{exercise}
\end{frame}

\begin{frame}
\frametitle{拉格朗日定理}
\begin{example}
证明当$x>0$时不等式成立:
$$\dfrac{x}{1+x}<\ln(1+x)<x.$$
\end{example}
\pause
\begin{remark*}
当$-1<x<0$时,不等式同样成立.
\end{remark*}
\end{frame}

\begin{sframe}
\frametitle{对数函数不等式}
\begin{remark*}
这个例子包含的两个不等式是等价的:先作换元$t=1+x$,则不等式变成
\[ \frac{t-1}{t} < \ln t < t - 1 . \]
再利用换元$u=1/t$就可看出这两个不等式是等价的.
\end{remark*}
\end{sframe}

\subsection{柯西中值定理}

\begin{frame}
\frametitle{柯西定理}
\begin{theorem*}
如果函数$f(x)$和$g(x)$满足下列条件:
\begin{enumskip}
\item 在闭区间$[a,b]$上都连续,
\item 在开区间$(a,b)$内都可导,
\item 在开区间$(a,b)$内$g'(x)\neq0$,
\end{enumskip}
则至少存在一点$\xi\in(a,b)$使得$\frac{f'(\xi)}{g'(\xi)}=\frac{f(b)-f(a)}{g(b)-g(a)}$.
\end{theorem*}
\end{frame}

\begin{frame}
\begin{example}
对函数$f(x)=x^3$和$g(x)=x^2+1$在区间$[1,2]$上验证柯西定理.
\end{example}
\end{frame}

\begin{frame}
\begin{review}
证明:当$x>1$时,$\e^x-\e>\e(x-1)$.
\end{review}
\end{frame}

\begin{sframe}
\frametitle{复习与提高:存在性问题}
\begin{review}
设$f(x)$在$[a,b]$上连续,在$(a,b)$内有二阶导数,且$f(a)=f(b)=0$。
证明:$\forall c\in(a,b)$,$\exists\xi\in(a,b)$,使得
\[ f(c)=\tfrac12f''(\xi)(c-a)(c-b). \]
\end{review}
\pause
\begin{solution}
若能找到$L(x)$,使得$L(a)=L(c)=L(b)$,且
\[ L''(x)=f''(x)(c-a)(c-b)-2f(c). \]
则由罗尔定理,存在$\xi_1\in(a,c)$,$\xi_2\in(c,b)$,使得
$L'(\xi_1)=L'(\xi_2)=0$.从而$\exists\xi\in(\xi_1,\xi_2)$,
使得$L''(\xi)=0$,从而结论成立.
\end{solution}
\end{sframe}

\begin{sframe}
\frametitle{复习与提高:存在性问题}
对$L''(x)$积分,逐步寻找$L(x)$:
\noindent\begin{align*}
&& L''(x)&=f''(x)(c-a)(c-b)-2f(c) \\
\rightwhitearrow&& L'(x)&=f'(x)(c-a)(c-b)-2f(c)x+k \\
\rightwhitearrow&& L(x)&=f(x)(c-a)(c-b)-f(c)x^2+kx
\end{align*}
利用条件$L(a)=L(b)$可得$k=f(c)(a+b)$.因此
$$\bold{L(x)=f(x)(c-a)(c-b)-f(c)x^2+f(c)(a+b)x}$$
为所求.或者也可给$L(x)$减去$L(c)=f(c)ab$变成
$$\bold{\tilde{L}(x)=f(x)(c-a)(c-b)-f(c)(x-a)(x-b)}.$$
\end{sframe}

\section{洛必达法则}

\begin{frame}
\frametitle{洛必达法则}
在一定条件下,我们有下面的洛必达法则:
\[ \lim\frac{f(x)}{g(x)}=\lim\frac{f'(x)}{g'(x)}\]
\end{frame}

\begin{frame}
\frametitle{一、$\frac00$型的洛必达法则}
\begin{theorem}
如果$\lim\limits_{x\to a}f(x)=0$,$\lim\limits_{x\to a}g(x)=0$,
而且$\lim\limits_{x\to a}\frac{f'(x)}{g'(x)}$ 的极限存在(或为$\infty$),则有
\[\lim_{x\to a}\frac{f(x)}{g(x)}=\lim\limits_{x\to a}\frac{f'(x)}{g'(x)}\]
\end{theorem}\pause
\begin{example}
求极限$\lim\limits_{x\to2}\dfrac{x^2+x-6}{x^2-4}$.
\end{example}
\end{frame}

\begin{frame}
\begin{example}
求极限$\lim\limits_{x\to0}\dfrac{(1+x)^a-1}{x}$.
\end{example}\pause
\begin{example}
求极限$\lim\limits_{x\to0}\dfrac{e^x-1}{x^2-x}$.
\end{example}\pause
\begin{example}
求极限$\lim\limits_{x\to0}\dfrac{x-\sin x}{x^3}$.
\end{example}\pause
\begin{example}
求极限$\lim\limits_{x\to0}\dfrac{\ln(1+x)}{x^2}$.
\end{example}
\end{frame}

\begin{frame}
\begin{exercise}
用洛必达法则求函数极限.
\begin{enumlite}[<+->]
  \item $\lim\limits_{x\to1}\dfrac{x^3-3x+2}{x^3-x^2-x+1}$
  \item $\lim\limits_{x\to4}\dfrac{\sqrt{x}-2}{x-4}$
  \item $\lim\limits_{x\to0}\dfrac{\sin 3x}{\sin 5x}$
\end{enumlite}
\end{exercise}
\end{frame}

\begin{frame}
\frametitle{两种方法比较}
\begin{remark}
对于$x\to0$时的$\frac00$型极限,现在我们有两种方法可以使用:
\begin{enumskip}
  \item 等价无穷小量代换
  \item 洛必达法则
\end{enumskip}
一般地,方法(1)应该优先使用,因为方法(2)可能变得复杂.
\end{remark}
\end{frame}

\begin{frame}
\frametitle{两种方法比较}
\begin{example}
求函数极限.
\begin{enumlite}
  \item $\lim\limits_{x\to0}\dfrac{\sin 3x}{\tan 6x}$\pause
  \item $\lim\limits_{x\to0}\dfrac{\mathrm{e}^{x-\sin x}-1}{\arcsin(x^3)}$
\end{enumlite}
\end{example}
\end{frame}

\begin{frame}
\frametitle{两种方法比较}
\begin{exercise}
求函数极限.
\begin{enumlite}
  \item $\lim\limits_{x\to0}\dfrac{\sin x-x\cos x}{\sin^3x}$
  \item $\lim\limits_{x\to0^+}\dfrac{\sqrt{1+x^3}-1}{1-\cos\sqrt{x-\sin x}}$
\end{enumlite}
\end{exercise}
\end{frame}

\begin{frame}
\frametitle{二、$\frac{\infty}{\infty}$型的洛必达法则}
\begin{theorem}
如果$\lim\limits f(x)=\infty$,$\lim\limits g(x)=\infty$,
而且$\lim\limits \frac{f'(x)}{g'(x)}$的极限存在(或为$\infty$),则有
\[\lim \frac{f(x)}{g(x)}=\lim\limits \frac{f'(x)}{g'(x)}\]
\end{theorem}\pause
\begin{example}
求极限$\lim\limits_{x\to\infty}\dfrac{2x^2+x+1}{3x^2-x+4}$.
\end{example}
\end{frame}

\begin{frame}
\begin{example}
求函数极限.
\begin{enumlite}
  \item 求极限$\lim\limits_{x\to+\infty}\dfrac{\ln x}{x^n}$($n>0$)\pause
  \item 求极限$\lim\limits_{x\to+\infty}\dfrac{x^3}{\mathrm{e}^x}$\pause
\end{enumlite}
\end{example}\pause
\begin{exercise}
求函数极限:
$\lim\limits_{x\to0^+}\dfrac{\ln\sin x}{\ln x}$
\end{exercise}\pause
\begin{thinking}
求极限$\lim\limits_{x\to\frac{\pi}2}\dfrac{\tan x}{\tan 3x}$
\end{thinking}
\end{frame}

\begin{frame}
\begin{remark}洛必达法则未必总是有效.例如:
\begin{enumhalf}
  \item $\lim\limits_{x\to\infty}\dfrac{x+\sin x}{x}$ ~\pause
  \item $\lim\limits_{x\to+\infty}\dfrac{\sqrt{1+x^2}}{x}$ ~
\end{enumhalf}
\end{remark}
\end{frame}

\begin{frame}
\frametitle{三、$0\cdot\infty$型和$\infty-\infty$型的未定式}
对于$0\cdot\infty$型和$\infty-\infty$型的未定式,
我们可以将它们变换为$\frac00$ 型或$\frac{\infty}{\infty}$型的未定式,
然后使用洛必达法则.\pause
\begin{example}
求函数极限:
\begin{enumlite}
  \item $\lim\limits_{x\to+\infty}x\left(\dfrac\pi2-\arctan x\right)$\pause
  \item $\lim\limits_{x\to1}\left(\dfrac{1}{x-1}-\dfrac1{\ln x}\right)$
\end{enumlite}
\end{example}
\end{frame}

\begin{frame}
\begin{exercise}
求函数极限:
\begin{enumhalf}
  \item $\lim\limits_{x\to0^+}x^2\ln x$ ~
  \item $\lim\limits_{x\to0}\left(\dfrac{1}{x}-\dfrac1{e^x-1}\right)$ ~
\end{enumhalf}
\end{exercise}
\end{frame}

\begin{frame}
\frametitle{四、$1^\infty$型,$0^0$型和$\infty^0$型的未定式}
对于$1^\infty$型,$0^0$型和$\infty^0$型的未定式,我们可以将它们变换为$0\cdot\infty$型未定式,
进而化为$\frac00$ 型或$\frac{\infty}{\infty}$型,然后使用洛必达法则.\pause
\[\lim u(x)^{v(x)}=\lim \mathrm{e}^{v(x)\ln u(x)} = \mathrm{e}^{\lim v(x)\ln u(x)}\]
\end{frame}

\begin{frame}
\begin{example}
求函数极限:
\begin{enumlite}
  \item \ $\lim\limits_{x\to1}x^{\frac{1}{x-1}}$\pause
  \item \ $\lim\limits_{x\to0^+}x^x$\pause
  \item $\lim\limits_{x\to+\infty}\left(1+\mathrm{e}^x\right)^{\frac1x}$
\end{enumlite}
\end{example}
\end{frame}

\begin{frame}
\begin{exercise}
求函数极限:
\begin{enumlite}
  \item $\lim\limits_{x\to0}\left(1+\sin x\right)^{\frac1x}$
  \item $\lim\limits_{x\to+\infty}x^{\frac1x}$
  \pause
  \item $\lim\limits_{x\to0^+}x^{\sin x}$
\end{enumlite}
\end{exercise}
\end{frame}

\begin{frame}
\frametitle{复习与提高}
\begin{review}
求函数极限.
\begin{enumlite}
  \item $\lim\limits_{x\to0}\dfrac{e^x-e^{-x}}{x}$
  \item $\lim\limits_{x\to1}\dfrac{\ln x}{x-1}$
  \item $\lim\limits_{x\to1}\dfrac{x^3-x^2-x+1}{x^3-3x+2}$
\end{enumlite}
\end{review}
\end{frame}

\begin{frame}
\frametitle{复习与提高}
\begin{review}
求函数极限:
\begin{enumhalf}
  \item $\lim\limits_{x\to0^+}\sqrt{x}\ln x$ ~
  \item $\lim\limits_{x\to0}\left(\dfrac1{\ln(1+x)}-\dfrac1x\right)$ ~
\end{enumhalf}
\end{review}
\end{frame}

\begin{iframe}
\frametitle{复习与提高:倒代换}
\begin{review}
求函数极限:
\begin{enumlite}
  \item $\lim\limits_{x\to+\infty}\left(x^2-x\ln\left(1+\frac1x\right)\right)$
  \item $\lim\limits_{x\to+\infty}\left(\sqrt{x^2+x}-\sqrt[3]{x^3+x^2}\right)$
\end{enumlite}
\end{review}
\end{iframe}

\begin{frame}
\frametitle{复习与提高}
\begin{review}
求函数极限 $\lim\limits_{x\to0^+}\left(\dfrac1x\right)^{\tan x}$。
\end{review}
\end{frame}

\section{泰勒公式}

\begin{frame}
\frametitle{近似估计}
假设$f'(x_0)$存在。已经知道当$x\to x_0$时有
\[ f(x) = f(x_0) + f'(x_0)(x-x_0) + \mathrm{o}\big(x-x_0\big)\]
\vpause
是否存在\bold{二次}多项式$g(x)$使得当$x\to x_0$时有
\[ f(x) \overset{\warn{?}}{=} g(x) + \bold{\mathrm{o}\left((x-x_0)^2\right)}\]
\vpause
令$g(x)=A+B(x-x_0)+C(x-x_0)^2$,则有
\[ A=f(x_0),\qquad B=f'(x_0),\qquad C=\wfrac12f''(x_0).\]
\end{frame}

\begin{frame}
%\frametitle{近似估计}
\noindent
\[ \bold{f(x) = A+B(x-x_0)+C(x-x_0)^2 + \mathrm{o}\left((x-x_0)^2\right)} \]
{\clead\cdotfill}\vpause
令$x\to x_0$,得到$A=f(x_0)$,\pause 从而
\[ \dfrac{f(x)-f(x_0)}{x-x_0} = B + C(x-x_0) + \mathrm{o}(x-x_0). \]\pause
再令$x\to x_0$,得到$B=f'(x_0)$。\pause 因此\vspace{2pt}
\begin{align*}
  C &=\lim_{x\to x_0}\frac{f(x)-f(x_0)-f'(x_0)(x-x_0)}{(x-x_0)^2} \\
    &\onslide<6->{=\lim_{x\to x_0}\frac{f'(x)-f'(x_0)}{2(x-x_0)}\qquad\fbox{洛必达法则}}\\[-2pt]
    &\onslide<7->{=\wfrac12f''(x_0)\qquad{\fbox{导数的定义}}}
\end{align*}
\end{frame}

\begin{frame}
%\frametitle{泰勒公式}
\begin{theorem}[带佩亚诺余项的泰勒公式]\par
设$f(x)$在$x_0$点存在$\warn{n}$阶导数,则有
\begin{align*}
f(x)&=f(x_0)+f'(x_0)(x-x_0)+\frac{f''(x_0)}{2!}(x-x_0)^2\\
    &\phantom{=}+\cdots+\frac{f^{(n)}(x_0)}{n!}(x-x_0)^n+\bold{\mathrm{o}\big((x-x_0)^n\big)}
\end{align*}
\end{theorem}
\pause
\begin{solution}
连续用$n-1$次洛必达法则,再用导数的定义.
\end{solution}
\end{frame}

\begin{frame}
%\frametitle{泰勒公式}
\begin{theorem}[带拉格朗日余项的泰勒公式]\par
设$f(x)$在$x_0$的某邻域$U(x_0)$内存在$\warn{n+1}$阶导数,则$\forall x\in U(x_0)$有
\begin{align*}
  f(x)=f(x_0)&+f'(x_0)(x-x_0)+\frac{f''(x_0)}{2!}(x-x_0)^2\\
             &+\cdots+\frac{f^{(n)}(x_0)}{n!}(x-x_0)^n+R_n(x),
\end{align*}
其中余项$R_n(x)=\bold{\dfrac{f^{(n+1)}(\xi)}{(n+1)!}(x-x_0)^{n+1}}$,
$\xi$介于$x_0$和$x$之间。
\end{theorem}
\vpause
\begin{solution}
连续用$n+1$次柯西中值定理.
\end{solution}
\end{frame}

\begin{frame}
%\frametitle{麦克劳林公式}
当$x_0=0$时,泰勒公式称为\bold{麦克劳林公式}
\begin{align*}
  f(x)=f(0)&+f'(0)x+\frac{f''(0)}{2!}x^2\\
             &+\cdots+\frac{f^{(n)}(0)}{n!}x^n+R_n(x),
\end{align*}
\pause
其中$R_n(x)=\mathrm{o}\left(x^n\right)$\cdotfill \bold{佩亚诺余项}\par
或者$R_n(x)=\dfrac{f^{(n+1)}(\xi)}{(n+1)!}x^{n+1}$\cdotfill\bold{拉格朗日余项}\newline
  $\xi$介于$0$和$x$之间。
\vfill\pause{\clead\hrule}\vfill
令$\xi=\theta x$,则$R_n(x)=\dfrac{f^{(n+1)}(\theta x)}{(n+1)!}x^{n+1}$, $0<\theta<1$。
\end{frame}

\begin{frame}[label=maclaurin]%若这里用 \label 命令,后面将无法跳回各个 slide
\begin{example}
求$f(x)$的带拉格朗日余项的麦克劳林公式.\onslide<5>{\hfill\hyperlink{maclaurin:list}{\beamerbutton{小结}}\par}
\begin{enumlite}[<+->]
  \item $f(x)=\e^x$ \cdotfill\hyperlink{maclaurin:exp}{\beamerbutton{应用}}
  \item $f(x)=\sin x$ \cdotfill\hyperlink{maclaurin:sin}{\beamerbutton{图形}}
  \item $f(x)=\cos x$
  \item $f(x)=\ln(1+x)$ \cdotfill\hyperlink{maclaurin:log}{\beamerbutton{应用}}
  \item $f(x)=(1+x)^{\alpha}$ \cdotfill\hyperlink{maclaurin:sqrt}{\beamerbutton{应用}}
\end{enumlite}
\end{example}
\end{frame}

\begin{frame}\label{maclaurin:exp}
\begin{example}
证明常数$\e$是无理数。
\end{example}
\pause
\begin{proof}
假设$\e=\dfrac{m}n$为有理数,其中$n\ge2$。\pause 在$\e^x$的麦克劳林公式中令$x=1$,得到($0<\theta<1$)
\[ \frac{m}n = \e = 1 + 1 + \frac1{2!} + \frac1{3!} + \cdots + \frac1{n!} + \frac{\e^\theta}{(n+1)!} \]\pause
两边同时乘以$n!$,得到
\[ \frac{m\cdot n!}n = n! + n! + \frac{n!}{2!} + \frac{n!}{3!} + \cdots
 + \frac{n!}{n!} + \frac{\e^\theta}{n+1} \]\pause
由于$0<\e^\theta<3$,\pause 所以最后一项为分数,但是其他各项都为整数。\pause 矛盾。
%\hfill % FIXME
\hyperlink{maclaurin}{\beamerbutton{返回}}
\end{proof}
\end{frame}

% 数字太大,tikz 无法处理,加上 fpu 库就可以了
\usepgflibrary{fpu}

% 分为三个 frame 保证打印时不会混成一团

\begin{frame}\label{maclaurin:sin}
\frametitle{正弦函数的近似}
\noindent % 去掉空行
\[ \sin x = x - \frac1{3!}x^3 + \frac1{5!}x^5 - \frac1{7!}x^7 + \frac1{9!}x^9 + \cdots \]
\begin{center}
\begin{tikzpicture}[thick,scale=.8,font=\small,/pgf/fpu,/pgf/fpu/output format=fixed]
\path[use as bounding box] (-6.5,-3.5) -- (6.6,3.5);
\draw[very thin,color=gray] (-6,-3) grid (6,3);
\draw[thin,->] (-6.5,0) -- (6.6,0) node[right] {$x$};
\draw[thin,->] (0,-3.2) -- (0,3.2) node[above] {$y$};
\onslide+<2->\draw plot[domain=-6.28:6.28,samples=50] (\x,{sin(\x r)}); %node[above=7mm] {$y=\sin x$};
\onslide+<3->\draw[color=accent2] plot[domain=-3:3]
  (\x,\x) node[above,inner sep=1pt] {$y=x$};
\onslide+<4>\draw[color=accent3] plot[domain=-3.35:3.35,samples=40]
  (\x,{\x-(1/3!)*(\x)^3}) node[below,inner sep=1pt]{$y=x-\frac{1}{3!}x^3$};
\onslide+<1->%否则底栏在最后才显示
\end{tikzpicture}
\end{center}
\end{frame}

\begin{frame}
\frametitle{正弦函数的近似}
\noindent % 去掉空行
\[ \sin x = x - \frac1{3!}x^3 + \frac1{5!}x^5 - \frac1{7!}x^7 + \frac1{9!}x^9 + \cdots \]
\begin{center}
\begin{tikzpicture}[thick,scale=.8,font=\small,/pgf/fpu,/pgf/fpu/output format=fixed]
\path[use as bounding box] (-6.5,-3.5) -- (6.6,3.5);
\draw[very thin,color=gray] (-6,-3) grid (6,3);
\draw[thin,->] (-6.5,0) -- (6.6,0) node[right] {$x$};
\draw[thin,->] (0,-3.2) -- (0,3.2) node[above] {$y$};
\draw plot[domain=-6.28:6.28,samples=50] (\x,{sin(\x r)}); %node[above=7mm] {$y=\sin x$};
\onslide+<1|handout:0>\draw[color=accent3] plot[domain=-3.35:3.35,samples=40]
  (\x,{\x-(1/3!)*(\x)^3}) node[below,inner sep=1pt]{$y=x-\frac{1}{3!}x^3$};
\onslide+<1->\draw[color=accent2] plot[domain=-4.2:4.2,samples=50]
  (\x,{\x-(1/3!)*(\x)^3+(1/5!)*(\x)^5}) node[above,inner sep=1pt]{$y=x-\frac1{3!}x^3+\frac1{5!}x^5$};
\onslide+<2->\draw[color=accent3] plot[domain=-4.3:4.3,samples=60]
  (\x,{\x-(1/3!)*(\x)^3+(1/5!)*(\x)^5-(1/7!)*(\x)^7})
      node[below,inner sep=1pt]{$y=x - \frac1{3!}x^3 + \frac1{5!}x^5 - \frac1{7!}x^7$};
\onslide+<1->%否则底栏在最后才显示
\end{tikzpicture}
\end{center}
\end{frame}

\begin{frame}
\frametitle{正弦函数的近似}
\noindent % 去掉空行
\[ \sin x = x - \frac1{3!}x^3 + \frac1{5!}x^5 - \frac1{7!}x^7 + \frac1{9!}x^9 + \cdots \]
\begin{center}
\begin{tikzpicture}[thick,scale=.8,font=\small,/pgf/fpu,/pgf/fpu/output format=fixed]
\path[use as bounding box] (-6.5,-3.5) -- (6.6,3.5);
\draw[very thin,color=gray] (-6,-3) grid (6,3);
\draw[thin,->] (-6.5,0) -- (6.6,0) node[right] {$x$};
\draw[thin,->] (0,-3.2) -- (0,3.2) node[above] {$y$};
\draw plot[domain=-6.28:6.28,samples=50] (\x,{sin(\x r)}); %node[above=7mm] {$y=\sin x$};
\onslide+<1|handout:0>\draw[color=accent3] plot[domain=-4.3:4.3,samples=60]
  (\x,{\x-(1/3!)*(\x)^3+(1/5!)*(\x)^5-(1/7!)*(\x)^7})
      node[below,inner sep=1pt]{$y=x - \frac1{3!}x^3 + \frac1{5!}x^5 - \frac1{7!}x^7$};
\onslide+<1->\draw[color=accent2] plot[domain=-5.55:5.55,samples=70]
  (\x,{\x-(1/3!)*(\x)^3+(1/5!)*(\x)^5-(1/7!)*(\x)^7+(1/9!)*(\x)^9})
      node[left]{$y=x - \frac1{3!}x^3 + \frac1{5!}x^5 - \frac1{7!}x^7 + \frac1{9!}x^9$};
\onslide+<2->\draw[color=accent3] plot[domain=-6.2:6.2,samples=80]
  (\x,{\x-(1/3!)*(\x)^3+(1/5!)*(\x)^5-(1/7!)*(\x)^7+(1/9!)*(\x)^9-(1/11!)*(\x)^11})
      node[left]{$y=x - \frac1{3!}x^3 + \frac1{5!}x^5 - \frac1{7!}x^7 + \frac1{9!}x^9 - \frac1{11!}x^{11}$};
\onslide+<1->%否则底栏在最后才显示
\end{tikzpicture}
\end{center}
\begin{bblock}[white]{4}(28,1)\hyperlink{maclaurin<2>}{\beamerbutton{返回}}\end{bblock}
\end{frame}

\begin{frame}\label{maclaurin:log}
\frametitle{利用泰勒公式证明不等式}
\begin{example}
证明:当$x>0$时,有$\ln(1+x)>x-\dfrac{x^2}{2}$。
\end{example}
\pause
\begin{solution}
利用$\ln(1+x)$的$1$阶麦克劳林公式.\hfill
\hyperlink{maclaurin<4>}{\beamerbutton{返回}}
\end{solution}
\end{frame}

%\begin{sframe}
%\frametitle{利用泰勒公式求极限}
%\begin{example}
%求极限$\limit_{x\to0}\dfrac{\sin x-x\cos x}{\sin^3 x}$。
%\end{example}
%\end{sframe}

\begin{frame}\label{maclaurin:sqrt}
\frametitle{利用泰勒公式求极限}
\begin{example}
求极限$\limit_{x\to0}\dfrac{\sqrt{4+3x}+\sqrt{4-3x}-4}{x^2}$。
\end{example}
\pause
\begin{solution}
利用$\sqrt{1+x}$的$2$阶麦克劳林公式,求得极限等于$-\dfrac{9}{32}$.\hfill
\hyperlink{maclaurin<5>}{\beamerbutton{返回}}
\end{solution}
\end{frame}

\begin{frame}[shrink=10]\label{maclaurin:list}
\frametitle{初等函数的麦克劳林公式}
\noindent\begin{align*}
\e^x &= 1 + x + \frac{x^2}{2!} + \frac{x^3}{3!} + \frac{x^4}{4!} + \cdots + \frac{x^n}{n!} + R_n(x) \\
\sin x &= x-\frac{x^3}{3!}+\frac{x^5}{5!}-\cdots + (-1)^{n-1}\frac{x^{2n-1}}{(2n-1)!} + R_{2n}(x) \\
\cos x &= 1-\frac{x^2}{2!}+\frac{x^4}{4!}-\cdots+(-1)^n\frac{x^{2n}}{(2n)!} + R_{2n+1}(x)
\end{align*}
\vfill\bold{\cdotfill}\vfill
\noindent\begin{align*}
\ln(1+x) &= x - \frac{x^2}2 + \frac{x^3}{3} - \frac{x^4}{4} +\cdots + (-1)^{n-1}\frac{x^n}{n} + R_n(x) \\
(1+x)^{\alpha} &= 1 + C_\alpha^1x + C_\alpha^2x^2 + C_\alpha^3x^3 +\cdots + C_\alpha^n{x^n} + R_n(x)
%\arctan x  &=x-\frac{x^3}{3}+\frac{x^5}{5}-\frac{x^7}{7}+\cdots + (-1)^n\frac{x^{2n+1}}{(2n+1)} + \cdots \\
\end{align*}
\end{frame}

\section{单调性与凹凸性}

\subsection{函数的单调性}

\begin{frame}
\begin{theorem}
设$f(x)$在闭区间$[a,b]$上连续,在开区间$(a,b)$上可导,那么
\begin{enumzero}
  \item 如果在$(a,b)$上恒有$f'(x)>0$,则$f(x)$在$[a,b]$上单调增加.
  \item 如果在$(a,b)$上恒有$f'(x)<0$,则$f(x)$在$[a,b]$上单调减少.
\end{enumzero}
\end{theorem}
%\vpause
%\begin{remark*}
%对开区间或者无限区间,也有类似的结论。
%\end{remark*}
\vpause
\begin{remark*}
若在区间上$f'(x)=0$的点仅有有限个,仍有
\begin{enumzero}
  \item 在$(a,b)$上$f'(x)\ge0$\hspace{0.5em}$\lead{\rightwhitearrow}$\hspace{0.4em}在$[a,b]$上单调增加.
  \item 在$(a,b)$上$f'(x)\le0$\hspace{0.5em}$\lead{\rightwhitearrow}$\hspace{0.4em}在$[a,b]$上单调减少.
\end{enumzero}
\end{remark*}
\end{frame}

\begin{frame}
\frametitle{函数的单调性}
\begin{example}
确定下列函数的单调增减区间.
\begin{enumhalf}
  \item $f(x)=x^3-3x$ ~
  \item $f(x)=x^3$ ~\pause
  \item $f(x)=x-\sin x$ ~
  \item $f(x)=\sqrt[3]{x^2}$ ~  
\end{enumhalf}
\end{example}
\vpause
\begin{remark*}
通常可用这两类点来划分单调区间:
\begin{enumhalf}
  \item \CJKunderdot{导数为零的点}; ~
  \item \CJKunderdot{导数不存在的点}。 ~
\end{enumhalf}
\end{remark*}
\vpause
\begin{definition*}
导数为零的点称为函数的\bold{驻点}。
\end{definition*}
\end{frame}

\begin{oframe}
\frametitle{函数的单调性}
\begin{exercise}
确定下列函数的单调增减区间.
\begin{enumhalf}
  \item $y=3x^2+6x+5$ ~
  \item $y=x-\mathrm{e}^x$ ~
\end{enumhalf}
\end{exercise}
\end{oframe}

%\begin{frame}
%\begin{example}
%证明函数$y=x-\ln(1+x^2)$单调增加.
%\end{example}
%\pause
%\begin{exercise}
%证明函数$y=\sin x-x$单调减少.
%\end{exercise}
%\end{frame}

\begin{frame}
\frametitle{不等式问题}
\begin{example}
证明当$x>0$时有不等式$\mathrm{e}^x>1+x$.
\end{example}
\pause
\begin{example}
证明当$x>0$时有$x-\dfrac{x^3}{6}<\sin x$.
\end{example}
\end{frame}

%\begin{frame}
%\frametitle{不等式问题}
%\begin{exercise}
%证明当$x>1$时有$3-\dfrac1x<2\sqrt{x}$.
%\end{exercise}
%\end{frame}

\subsection{曲线的凹凸性}

%\begin{frame}
%%\frametitle{凹凸性}
%\begin{definition}
%设函数$f(x)$在区间$I$上有定义.
%\begin{enumskip}
%  \item 如果对任何$x\in I$,函数$f(x)$的曲线总位于该点切线的上方,
%        则称曲线$f(x)$在区间$I$上是\bold{凹}(上凹)的。
%  \pause
%  \item 如果对任何$x\in I$,函数$f(x)$的曲线总位于该点切线的下方,
%        则称曲线$f(x)$在区间$I$上是\bold{凸}(下凹)的。
%\end{enumskip}
%\end{definition}
%\end{frame}

\begin{frame}
%\frametitle{凹凸性}
\begin{definition}
设函数$f(x)$在区间$I$上连续.
\begin{enumzero}
  \item 如果对任何$I$上任何两点$x_1$和$x_2$,恒有
        \[ \frac{f(x_1)+f(x_2)}{2} \,\mathbin{\cbold>}\, f\left(\frac{x_1+x_2}{2}\right)\]
        则称曲线$f(x)$在区间$I$上是\bold{凹}(上凹)的。
  \pause
  \item 如果对任何$I$上任何两点$x_1$和$x_2$,恒有
        \[ \frac{f(x_1)+f(x_2)}{2} \,\mathbin{\cbold<}\, f\left(\frac{x_1+x_2}{2}\right)\]
        则称曲线$f(x)$在区间$I$上是\bold{凸}(下凹)的。
\end{enumzero}
\end{definition}
\end{frame}

\begin{frame}
\frametitle{凹凸性的判别法}
\begin{theorem}
设函数$f(x)$在$[a,b]$上连续,在$(a,b)$内有二阶导数,那么
\begin{enumzero}
  \item 如果$x\in(a,b)$时,恒有$f''(x)>0$,则函数的曲线在$(a,b)$上是凹的。\pause
  \item 如果$x\in(a,b)$时,恒有$f''(x)<0$,则函数的曲线在$(a,b)$上是凸的。
\end{enumzero}
\end{theorem}
\end{frame}

\begin{frame}
\begin{definition*}
曲线凹和凸的分界点$(x_0,y_0)$称为\bold{拐点}。
\end{definition*}
\vpause
\begin{property*}
在拐点$(x_0,y_0)$处,要么$f''(x_0)=0$,要么$f''(x_0)$不存在。
\end{property*}
\pause\vskip0pt plus0.2fill\cdotfill\vskip0pt plus0.2fill
\begin{example}
求曲线$y=\sqrt[3]{x}$的凹凸区间和拐点。
\end{example}
\begin{remark}
$(x_0,y_0)$为拐点 $\cwarn\centernot\Longrightarrow$ $f''(x_0)=0$。
\end{remark}
\vpause
\begin{example}
求曲线$y=x^4$的凹凸区间和拐点。
\end{example}
\pause
\begin{remark}
$f''(x_0)=0$ $\cwarn\centernot\Longrightarrow$ $(x_0,y_0)$为拐点。\pause
\end{remark}
\end{frame}

\begin{frame}
\frametitle{曲线的凹凸性}
\begin{example}
求曲线$y=x^4-2x^3+1$的凹凸区间和拐点。
\end{example}
\vpause
\begin{exercise}
求下列曲线的凹凸区间和拐点。
\begin{enumhalf}
  \item $y=x^2-x^3$ ~\pause
  \item $y=\mathrm{e}^{-x}$ ~
\end{enumhalf}
\end{exercise}
\end{frame}

\begin{oframe}
\frametitle{复习与提高}
\begin{review}
确定函数$y=2x^2-\ln x$的单调增减区间.
\end{review}
\end{oframe}

\begin{iframe}
\frametitle{复习与提高}
\begin{review}
设$x>0$,$y>0$,$x\neq y$,$n>1$,证明
$$\frac12\left(x^n+y^n\right)>\left(\frac{x+y}{2}\right)^n.$$
\end{review}
\end{iframe}

\begin{frame}
\frametitle{复习与提高}
\begin{choice}
设在$[0,1]$上$f''(x)>0$, 则$f'(0)$, $f'(1)$, $f(1)-f(0)$
或$f(0)-f(1)$的大小顺序是\dotfill(\qquad)
\begin{choiceline}
  \item $f'(1)>f'(0)>f(1)-f(0)$
  \item $f'(1)>f(1)-f(0)>f'(0)$
  \item $f(1)-f (0)>f'(1)>f'(0)$
  \item $f'(1)>f(0)-f(1)>f'(0)$
\end{choiceline} % 答案 B
%提示:利用$f''(x)>0$得到$f'(x)$单调增加.
%再用中值定理得到$f(1)-f(0)=f'(\xi)$,$0<\xi<1$.
\end{choice}
\end{frame}

\section{极值与最值}

\subsection{函数的极值}

\begin{frame}
\begin{definition}
设$f(x)$在点$x_0$的某邻域$U(x_0)$内有定义.
\begin{enumzero}
  \item 若$\forall x\in\mathring{U}(x_0)$,总有$\bold{f(x)<f(x_0)}$,则称
    \begin{itemize}
      \item $x_0$为$f(x)$ 的一个\lead{极大值点},
      \item $f(x_0)$为$f(x)$ 的一个\warn{极大值}。
    \end{itemize}
  \pause
  \item 若$\forall x\in\mathring{U}(x_0)$,总有$\bold{f(x)>f(x_0)}$,则称
    \begin{itemize}
      \item $x_0$为$f(x)$ 的一个\lead{极小值点},
      \item $f(x_0)$为$f(x)$ 的一个\warn{极小值}。
    \end{itemize}
\end{enumzero}
\end{definition}
\vpause
\begin{remark*}
极大值点和极小值点统称为\lead{极值点},极大值和极小值统称为\warn{极值}.  
\end{remark*}
\end{frame}

\begin{frame}
\frametitle{极值的必要条件}
\begin{theorem}
设$f(x)$在$x_0$点可导,而且$x_0$为极值点,则$f'(x_0)=0$.
\end{theorem}
\vpause
\begin{remark*}
我们称导数为零的点为\bold{驻点}.\pause
\begin{itemize}
  \item 驻点\warn{未必}都是极值点:\pause 比如$y=x^3$.
  \pause
  \item 极值点\warn{未必}都是驻点:\pause 比如$y=|x|$.
\end{itemize}
\end{remark*}
\end{frame}

\begin{frame}
\frametitle{极值的第一判别法}
\begin{theorem}
设$f(x)$在$x_0$点连续,而且在它的某个去心邻域内可导.\pause
\begin{enumskip}
  \item 若在$x_0$的左邻域内$f'(x)>0$,在右邻域内$f'(x)<0$,则$x_0$为极大值点.\pause
  \item 若在$x_0$的左邻域内$f'(x)<0$,在右邻域内$f'(x)>0$,则$x_0$为极小值点.\pause
  \item 若在$x_0$的左邻域内和右邻域内$f'(x)$的符号不变,则$x_0$不为极值点.
\end{enumskip}
\end{theorem}
\end{frame}

\begin{frame}
\begin{example}
求函数的单调增减区间和极值.
\begin{enumlite}
  \item $f(x)=(x-1)^2(x+1)^3$\pause
  \item $f(x)=x-\frac32x^{\frac23}$
\end{enumlite}
\end{example}
\end{frame}

\begin{oframe}
\frametitle{函数的极值}
\begin{exercise}
求函数的单调增减区间和极值.
\begin{enumlite}
  \item $f(x)=x^3-3x^2+7$\pause
  \item $f(x)=\dfrac{2x}{1+x^2}$
\end{enumlite}
\end{exercise}
\end{oframe}

\begin{frame}
\frametitle{极值的第二判别法}
\begin{theorem}
设$f'(x_0)=0$而且$f''(x_0)$存在。
\begin{enumskip}
  \item 若$f''(x_0)>0$,则$x_0$为$f(x)$的极小值点。
  \item 若$f''(x_0)<0$,则$x_0$为$f(x)$的极大值点。
\end{enumskip}
\end{theorem}\pause
\begin{remark}
当$f''(x_0)=0$时,上面的定理无法判定。例如$f(x)=x^3$和$f(x)=x^4$。
\end{remark}
\end{frame}

\begin{frame}
\begin{example}
用极值的第二判别法求$f(x)=x^3-3x$的极值。
\end{example}\pause
\begin{exercise}
用极值的第二判别法求函数的极值:
$$y=x^3-3x^2-9x-5$$
\end{exercise}
\end{frame}

\subsection{函数的最值}

\begin{frame}
\frametitle{函数的最值}
\begin{definition}
设$f(x)$在区间$I$上有定义。如果有$x_0\in I$,
使得对所有$x\in I$都有
$$f(x)\le f(x_0) \quad\text{(或$f(x)\ge f(x_0)$)},$$
则称$f(x_0)$是$f(x)$ 在区间$I$上的\bold{最大值}(或\bold{最小值})。
\end{definition}
\end{frame}

\begin{frame}
\frametitle{函数最值的求法:闭区间情形}
设函数$f(x)$在闭区间$[a,b]$上连续,而且在除有限个点外都可导.则可按照下面步骤求出函数的最值:\pause
\begin{enumerate}
  \item 求出函数所有的驻点,不可导点,和区间端点一起列出来作为最值可疑点.\pause
  \item 求出函数在这些点的取值并比较,最大(小)者就为函数的最大(小)值.
\end{enumerate}
\end{frame}

\begin{frame}
\begin{example}
求以下函数在指定区间上的最值。
\begin{enumlite}
  \item $f(x)=x^3-3x^2+7$在区间$[-2,3]$上. \pause
  \item $f(x)=x-\frac32x^{\frac23}$在区间$[-1,8]$上.
\end{enumlite}
\end{example}
\vpause
\begin{exercise}
求以下函数在指定区间上的最值。
\begin{enumlite}
  \item $f(x)=x^4-2x^2+5$在区间$[-2,3]$上.
  %% 不好,去掉!(去掉 x=-2 和 x=-1)
  %\item $f(x)=\dfrac{x^2}{1+x}$在区间$[-\frac12,2]$上.
\end{enumlite}
\end{exercise}
\end{frame}

\begin{frame}
\frametitle{函数最值的求法:唯一驻点情形}
如果函数$f(x)$在区间(开或闭,有限或无限)上可导,而且只有一个驻点$x_0$。
则$f(x_0)$为极大值时就是最大值,为极小值时就是最小值。
\end{frame}

\begin{frame}
\begin{example}
将边长为$a$的一块正方形铁皮,四角各截去一个大小相同的小正方形,然后将四边折起做成一个无盖的方盒。
问截掉的小正方形的长为多少时,所得方盒的容积最大?
\end{example}
\end{frame}

%\begin{frame}
%\begin{exercise}
%一房地产公司有$50$套公寓要出租.当月租定为$2000$元时,公寓会全部租出去.
%当月租每增加$100$ 元,就会多剩一套公寓租不出去.
%而租出去的每套公寓每月需要花费$200$ 元的维修费用.问房租定为多少时可获得最大收入?
%\end{exercise}
%\end{frame}

\begin{oframe}
\frametitle{复习与提高}
\begin{review}
求函数$y=(x-3)^2(x-2)$的极值。
\end{review}
\end{oframe}

\begin{oframe}
\frametitle{复习与提高}
\begin{review}
求以下函数在指定区间上的最值。
\begin{enumlite}
  \item $f(x)=x^3-3x$在区间$[-2,3]$上.
  \item $f(x)=x^2\mathrm{e}^{-x}$在区间$[-2,3]$上.
\end{enumlite}
\end{review}
\end{oframe}

\begin{iframe}
\frametitle{极值的判别法}
\begin{example*}
设$f(x)$在$x_0$处有$n$阶导数,且$f'(x_0)=f''(x_0)=\cdots=f^{(n-1)}(x_0)=0$, $f^{(n)}(x_0)\neq0$.证明:\unskip
\begin{enumzero}
  \item 当$n$为奇数时,$f(x)$在$x_0$点不取得极值.
  \item 当$n$为偶数时,$f(x)$在$x_0$点取得极值,且
  \begin{itemize}
    \item 当$f^{(n)}(x_0)<0$时,$f(x_0)$为极大值;
    \item 当$f^{(n)}(x_0)>0$时,$f(x_0)$为极小值.
  \end{itemize}
\end{enumzero}
\end{example*}
\end{iframe}

\begin{iframe}
\begin{solution}
我们只证明$f^{(n)}(x_0)>0$的情形.\pause 由带佩亚诺余项的$n$阶泰勒公式,有
\begin{align*}
f(x)&=f(x_0)+\frac{f^{(n)}(x_0)}{n!}(x-x_0)^n+o\big((x-x_0)^n\big)\\
&=f(x_0)+(x-x_0)^n\left[\frac{f^{(n)}(x_0)}{n!}+o(1)\right]
\end{align*}\pause
其中的$o(1)$是$x\to x_0$时的无穷小量.\pause 由极限的局部保号性,存在$x_0$的去心邻域,%$\mathring{U}(x_0)$
使得在此邻域中$$\frac{f^{(n)}(x_0)}{n!}+o(1)>0.$$\pause
所以当$n$为奇数时$f(x)$在$x_0$点不取得极值,当$n$为偶数时$f(x)$在$x_0$点取得极小值.
\end{solution}
\end{iframe}

\begin{frame}
\frametitle{复习与提高}
\begin{choice}
设函数$f(x)$满足$\limit_{x\to a}\dfrac{f(x)-f(a)}{(x-a)^2}=-1$,则在点$a$处\dotfill(\qquad)
\begin{choiceline}
  \item $f(x)$的导数存在,且$f'(a)\neq0$
  \item $f(x)$的导数不存在
  \item $f(x)$取得极大值
  \item $f(x)$取得极小值
\end{choiceline}
\end{choice}% 答案 C
\end{frame}

\section{函数图形的描绘}

\begin{frame}
\begin{definition}
给定曲线$y=f(x)$。
\begin{enumskip}
  \item 若$\lim\limits_{x\to\infty}f(x)=b$,称$y=b$为其\bold{水平渐近线}。
  \item 若$\lim\limits_{x\to a}f(x)=\infty$,称$x=a$为其\bold{铅直渐近线}。
\end{enumskip}
\end{definition}
\vpause
\begin{remark*}
$(1)$ $x\to\infty$可以改为$x\to+\infty$或$x\to-\infty$。\pause
$(2)$ $x\to a$可以改为$x\to a^+$或$x\to a^-$。
\end{remark*}
\end{frame}

\begin{frame}
\begin{example}
求曲线$y=\dfrac1{x-1}$的水平和铅直渐近线.
\end{example}\pause
\begin{example}
求曲线$y=\dfrac{-x^2+x+1}{x^2}$的水平和铅直渐近线.
\end{example}
\end{frame}

\begin{frame}
\frametitle{曲线的渐近线}
\begin{exercise}
求曲线的水平和铅直渐近线.
\begin{enumhalf}
  \item $y=\dfrac{2x+1}{x+2}$ ~
  \item $y=\dfrac{2x}{x^2-1}$ ~
\end{enumhalf}
\end{exercise}
\end{frame}

\begin{frame}
\begin{definition}
若直线$y=kx+b$($k\neq0$)满足
$$\lim\limits_{x\to\infty}[f(x)-(kx+b)]=0,$$
则称它是曲线$y=f(x)$的一条\bold{斜渐近线}。
\end{definition}
\vpause
\begin{theorem}
直线$y=kx+b$是曲线$y=f(x)$的斜渐近线,当且仅当
$$\lim_{x\to\infty}\frac{f(x)}{x}=k\text{~~~而且~~}\lim_{x\to\infty}[f(x)-kx]=b.$$
\end{theorem}
\vpause
\begin{remark*}
$x\to\infty$可以改为$x\to+\infty$或$x\to-\infty$。
\end{remark*}
\end{frame}

\begin{frame}
\frametitle{曲线的渐近线}
\begin{example}
求曲线$y=\dfrac{x^2}{x+1}$的斜渐近线.
\end{example}\pause
\begin{exercise}
求曲线$y=\dfrac{x^3}{(x-1)^2}$的斜渐近线.
\end{exercise}
\end{frame}

\begin{frame}
\frametitle{函数图形的描绘}
\begin{example}
描绘函数$y=\dfrac{x^2}{x+1}$的曲线。
\end{example}
\end{frame}

\section{曲线的曲率}

\begin{frame}
\frametitle{曲率与曲率半径}
\begin{itemize}
  \item 弧微分: $\ds=\sqrt{1+y'^2}{\dx}$
  \item 角微分: $\d\alpha=\dfrac{y''}{1+y'^2}\dx$
  \item 曲率: $K=\left|\dfrac{\d\alpha}{\ds}\right|=\dfrac{|y''|}{(1+y'^2)^{3/2}}$
  \item 曲率半径: $\rho=\dfrac1K$
\end{itemize}
\end{frame}

\end{document}
