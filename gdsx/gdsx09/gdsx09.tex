% -*- coding: utf-8 -*-
% !TEX program = xelatex

%\documentclass[14pt]{article}
%\usepackage[notheorems]{beamerarticle}

\documentclass[14pt,notheorems,leqno,xcolor={rgb}]{beamer} % ignorenonframetext

% -*- coding: utf-8 -*-
% ----------------------------------------------------------------------------
% Author:  Jianrui Lyu <tolvjr@163.com>
% Website: https://github.com/lvjr/theme
% License: Creative Commons Attribution-ShareAlike 4.0 International License
% ----------------------------------------------------------------------------

\ProvidesPackage{beamerthemeriemann}[2018/06/05 v0.6 Beamer Theme Riemann]

\makeatletter

% compatible with old versions of beamer
\providecommand{\beamer@endinputifotherversion}[1]{}

\RequirePackage{tikz,etoolbox,adjustbox}
\usetikzlibrary{shapes.multipart}

\mode<presentation>

\setbeamersize{text margin left=8mm,text margin right=8mm}

%% ----------------- background canvas and background ----------------

\newif\ifbackgroundmarkleft
\newif\ifbackgroundmarkright

\newcommand{\insertbackgroundmark}{
  \ifbackgroundmarkleft
    \foreach \x in {1,2,3,4,5} \draw[very thick,markcolor] (0,\x*\paperheight/6) -- +(1.2mm,0);
  \fi
  \ifbackgroundmarkright
    \foreach \x in {1,2,3,4,5} \draw[very thick,markcolor] (\paperwidth,\x*\paperheight/6) -- +(-1.2mm,0);
  \fi
}

\defbeamertemplate{background}{line}{%
  \begin{tikzpicture}
    \useasboundingbox (0,0) rectangle (\paperwidth,\paperheight);
    \draw[xstep=\paperwidth,ystep=1mm,color=tcolor] (0,0) grid (\paperwidth,\paperheight);
    \insertbackgroundmark
  \end{tikzpicture}%
}

\defbeamertemplate{background}{linear}{%
  \begin{tikzpicture}
    \useasboundingbox (0,0) rectangle (\paperwidth,\paperheight);
    \draw[pattern=horizontal lines, pattern color=tcolor]
      (0,0) rectangle (\paperwidth,\paperheight);
    \insertbackgroundmark
  \end{tikzpicture}%
}

\defbeamertemplate{background}{lattice}[1][1mm]{%
  \begin{tikzpicture}
    \useasboundingbox (0,0) rectangle (\paperwidth,\paperheight);
    \draw[step=#1,color=tcolor,semithick] (0,0) grid (\paperwidth,\paperheight);
    \insertbackgroundmark
  \end{tikzpicture}%
}

\defbeamertemplate{background}{empty}{
  \begin{tikzpicture}
    \useasboundingbox (0,0) rectangle (\paperwidth,\paperheight);
    \insertbackgroundmark
  \end{tikzpicture}%
}

%% -------------------------- title page -----------------------------

% add \occasion command
\newcommand{\occasion}[1]{\def\insertoccasion{#1}}
\occasion{}

\defbeamertemplate{title page}{banner}{%
  \nointerlineskip
  \begin{adjustbox}{width=\paperwidth,center}%
    \usebeamertemplate{title page content}%
  \end{adjustbox}%
}

% need "text badly ragged" option for correct space skips
% see http://tex.stackexchange.com/a/132748/8956
\defbeamertemplate{title page content}{hexagon}{%
  \begin{tikzpicture}
  \useasboundingbox (0,0) rectangle (\paperwidth,\paperheight);
  \path[draw=dcolor,fill=fcolor,opacity=0.8]
      (0,0) rectangle (\paperwidth,\paperheight);
  \node[text width=0.86\paperwidth,text badly ragged,inner ysep=1.5cm] (main) at (0.5\paperwidth,0.55\paperheight) {%
    \begin{minipage}[c]{0.86\paperwidth}
      \centering
      \usebeamerfont{title}\usebeamercolor[fg]{title}\inserttitle
      \ifx\insertsubtitle\@empty\else
        \\[5pt]\usebeamerfont{subtitle}\usebeamercolor[fg]{subtitle}
        \insertsubtitle
      \fi
    \end{minipage}
  };
  \node[rectangle,inner sep=0pt,minimum size=3mm,fill=dcolor,right] (a) at (0,0.55\paperheight) {};
  \node[rectangle,inner sep=0pt,minimum size=3mm,fill=dcolor,left] (b) at (\paperwidth,0.55\paperheight) {};
  \ifx\insertoccasion\@empty
      \draw[thick,dcolor] (a.north east) -- (main.north west)
                   -- (main.north east) -- (b.north west);
  \else
      \node[text badly ragged] (occasion) at (main.north west -| 0.5\paperwidth,\paperheight) {
          \usebeamerfont{occasion}\usebeamercolor[fg]{occasion}\insertoccasion
      };
      \draw[thick,dcolor] (a.north east) -- (main.north west) -- (occasion.west)
                          (b.north west) -- (main.north east) -- (occasion.east);
  \fi
  \node[text badly ragged] (date) at (main.south west -| 0.5\paperwidth,0) {
      \usebeamerfont{date}\usebeamercolor[fg]{date}\insertdate
  };
  \draw[thick,dcolor] (a.south east) -- (main.south west) -- (date.west)
                      (b.south west) -- (main.south east) -- (date.east);
  \node[below=4mm,text width=0.9\paperwidth,inner xsep=0.05\paperwidth,
        text badly ragged,fill=white] at (date.south) {%
      \begin{minipage}[c]{0.9\paperwidth}
          \centering
          \textcolor{brown75}{$\blacksquare$}\hspace{0.2em}%
          \usebeamerfont{institute}\usebeamercolor[fg]{institute}\insertinstitute
          \hspace{0.4em}\textcolor{brown75}{$\blacksquare$}\hspace{0.2em}%
          \usebeamerfont{author}\usebeamercolor[fg]{author}\insertauthor
      \end{minipage}
  };
  \end{tikzpicture}
}

\defbeamertemplate{title page content}{rectangle}{%
  \begin{tikzpicture}
  \useasboundingbox (0,0) rectangle (\paperwidth,\paperheight);
  \path[draw=dcolor,fill=fcolor,opacity=0.8]
      (0,0.25\paperheight) rectangle (\paperwidth,0.85\paperheight);
  \path[draw=dcolor,very thick]
    %%(0.0075\paperwidth,0.26\paperheight) rectangle (0.9925\paperwidth,0.84\paperheight);
      (0.0375\paperwidth,0.26\paperheight) -- (0.9625\paperwidth,0.26\paperheight)
         -- ++(0,0.02\paperheight) -- ++(0.03\paperwidth,0)
         -- ++(0,-0.02\paperheight) -- ++(-0.015\paperwidth,0)
         -- ++(0,0.04\paperheight) -- ++(0.015\paperwidth,0)
      -- (0.9925\paperwidth,0.8\paperheight)
         -- ++(-0.015\paperwidth,0) -- ++(0,0.04\paperheight)
         -- ++(0.015\paperwidth,0) -- ++(0,-0.02\paperheight)
         -- ++(-0.03\paperwidth,0) -- ++(0,0.02\paperheight)
      -- (0.0375\paperwidth,0.84\paperheight)
         -- ++(0,-0.02\paperheight) -- ++(-0.03\paperwidth,0)
         -- ++(0,0.02\paperheight) -- ++(0.015\paperwidth,0)
         -- ++(0,-0.04\paperheight) -- ++(-0.015\paperwidth,0)
      -- (0.0075\paperwidth,0.3\paperheight)
         -- ++(0.015\paperwidth,0) -- ++(0,-0.04\paperheight)
         -- ++(-0.015\paperwidth,0) -- ++(0,0.02\paperheight)
         -- ++(0.03\paperwidth,0) -- ++(0,-0.02\paperheight)
      -- cycle;
  \node[text width=0.9\paperwidth,text badly ragged] at (0.5\paperwidth,0.55\paperheight) {%
    \begin{minipage}[c][0.58\paperheight]{0.9\paperwidth}
      \centering
      \usebeamerfont{title}\usebeamercolor[fg]{title}\inserttitle
      \ifx\insertsubtitle\@empty\else
        \\[5pt]\usebeamerfont{subtitle}\usebeamercolor[fg]{subtitle}
        \insertsubtitle
      \fi
    \end{minipage}
  };
  \ifx\insertoccasion\@empty\else
    \node[text badly ragged,below,draw=dcolor,fill=white] at (0.5\paperwidth,0.84\paperheight) {%
      \usebeamerfont{occasion}\usebeamercolor[fg]{occasion}\insertoccasion
    };
  \fi
  \node[text width=0.9\paperwidth,text badly ragged,below] at (0.5\paperwidth,0.25\paperheight) {%
    \begin{minipage}[t][0.25\paperheight]{0.9\paperwidth}
      \centering
      {\color{brown75}$\blacksquare$}
      \usebeamerfont{institute}\usebeamercolor[fg]{institute}\insertinstitute
      \hfill
      {\color{brown75}$\blacksquare$}
      \usebeamerfont{author}\usebeamercolor[fg]{author}\insertauthor
      \hfill
      {\color{brown75}$\blacksquare$}
      \usebeamerfont{date}\usebeamercolor[fg]{date}%
      \the\year-\ifnum\month<10 0\fi\the\month-\ifnum\day<10 0\fi\the\day
    \end{minipage}
  };
  \end{tikzpicture}
}

%% ----------------------- section and subsection --------------------

\newcounter{my@pgf@picture@count}

\def\sectionintocskip{0.5pt plus 0.1fill}
\patchcmd{\beamer@sectionintoc}{\vskip1.5em}{\vskip\sectionintocskip}{}{}

\AtBeginSection[]{%
  \begin{frame}%[plain]
    \sectionpage
  \end{frame}%
}

\defbeamertemplate{section name}{simple}{\insertsectionnumber.}

\defbeamertemplate{section name}{chinese}[1][节]{第\CJKnumber{\insertsectionnumber}#1}

\defbeamertemplate{section page}{single}{%
  \centerline{%
    \usebeamerfont{section name}%
    \usebeamercolor[fg]{section name}%
    \usebeamertemplate{section name}%
    \hspace{0.8em}%
    \usebeamerfont{section title}%
    \usebeamercolor[fg]{section title}%
    \insertsection
  }%
}

\defbeamertemplate{section name in toc}{simple}{%
  Section \ifnum\the\beamer@tempcount<10 0\fi\inserttocsectionnumber
}

\defbeamertemplate{section name in toc}{chinese}[1][节]{%
  第\CJKnumber{\inserttocsectionnumber}#1%
}

\newcounter{my@section@from}
\newcounter{my@section@to}

\defbeamertemplate{show sections in toc}{total}{%
  \setcounter{my@section@from}{1}%
  \setcounter{my@section@to}{50}%
}

% show at most five sections
\defbeamertemplate{show sections in toc}{partial}{%
  \setcounter{my@section@from}{\value{section}}%
  \addtocounter{my@section@from}{-2}%
  \setcounter{my@section@to}{\value{section}}%
  \addtocounter{my@section@to}{2}%
  \ifnum\my@totalsectionnumber>0%
    \ifnum\value{my@section@to}>\my@totalsectionnumber
      \setcounter{my@section@to}{\my@totalsectionnumber}%
      \setcounter{my@section@from}{\value{my@section@to}}%
      \addtocounter{my@section@from}{-4}%
    \fi
  \fi
  \ifnum\value{my@section@from}<1\setcounter{my@section@from}{1}%
    \setcounter{my@section@to}{\value{my@section@from}}%
    \addtocounter{my@section@to}{4}%
  \fi
}

% reset pgfid to get correct result with \tikzmark in second run
\defbeamertemplate{section page}{fill}{%
  \usebeamertemplate{show sections in toc}%
  \setcounter{my@pgf@picture@count}{\the\pgf@picture@serial@count}%
  \tableofcontents[sectionstyle=show/shaded,subsectionstyle=hide,
                   sections={\arabic{my@section@from}-\arabic{my@section@to}}]%
  \global\pgf@picture@serial@count=\value{my@pgf@picture@count}%
  \unskip
}

\defbeamertemplate{section in toc}{fill}{%
  \noindent
  \begin{tikzpicture}
  \node[rectangle split, rectangle split horizontal, rectangle split parts=2,
        rectangle split part fill={sectcolor,bg}, draw=darkgray,
        inner xsep=0pt, inner ysep=5.5pt]
       {
         \nodepart[text width=0.255\textwidth,align=center]{text}
         \usebeamertemplate{section name in toc}
         \nodepart[text width=0.74\textwidth]{second}%
         \hspace{7pt}\inserttocsection
       };
  \end{tikzpicture}%
  \par
}

\AtBeginSubsection{%
  \begin{frame}%[plain]
    \setlength{\parskip}{0pt}%
    \offinterlineskip
    \subsectionpage
  \end{frame}%
}

\defbeamertemplate{subsection name}{simple}{%
  \insertsectionnumber.\insertsubsectionnumber
}

\defbeamertemplate{subsection page}{single}{%
  \centerline{%
    \usebeamerfont{subsection name}%
    \usebeamercolor[fg]{subsection name}%
    \usebeamertemplate{subsection name}%
    \hspace{0.8em}%
    \usebeamerfont{subsection title}%
    \usebeamercolor[fg]{subsection title}%
    \insertsubsection
  }%
}

% reset pgfid to get correct result with \tikzmark in second run
\defbeamertemplate{subsection page}{fill}{%
  \setcounter{my@pgf@picture@count}{\the\pgf@picture@serial@count}%
  \tableofcontents[sectionstyle=show/hide,subsectionstyle=show/shaded/hide]%
  \global\pgf@picture@serial@count=\value{my@pgf@picture@count}%
  \unskip
}

\defbeamertemplate{subsection in toc}{fill}{%
  \noindent
  \begin{tikzpicture}
    \node[rectangle split, rectangle split horizontal, rectangle split parts=2,
          rectangle split part fill={white,bg}, draw=darkgray,
          inner xsep=0pt, inner ysep=5.5pt]
         {
           \nodepart[text width=0.255\textwidth,align=right]{text}
           \inserttocsectionnumber.\inserttocsubsectionnumber\kern7pt%
           \nodepart[text width=0.74\textwidth]{second}%
           \hspace{7pt}\inserttocsubsection
         };
  \end{tikzpicture}%
  \par
}

% chinese sections and subsections
\defbeamertemplate{section and subsection}{chinese}[1][节]{%
  \setbeamertemplate{section name in toc}[chinese][#1]%
  \setbeamertemplate{section name}[chinese][#1]%
}

%% ---------------------- headline and footline ----------------------

\defbeamertemplate{footline left}{author}{%
  \insertshortauthor
}

\defbeamertemplate{footline center}{title}{%
  \insertshorttitle
}

\defbeamertemplate{footline right}{number}{%
  \Acrobatmenu{GoToPage}{\insertframenumber{}/\inserttotalframenumber}%
}
\defbeamertemplate{footline right}{normal}{%
  \hyperlinkframeendprev{$\vartriangle$}
  \Acrobatmenu{GoToPage}{\insertframenumber{}/\inserttotalframenumber}
  \hyperlinkframestartnext{$\triangledown$}%
}

\defbeamertemplate{footline}{simple}{%
  \hbox{%
  \begin{beamercolorbox}[wd=.2\paperwidth,ht=2.25ex,dp=1ex,left]{footline}%
    \usebeamerfont{footline}\kern\beamer@leftmargin
    \usebeamertemplate{footline left}%
  \end{beamercolorbox}%
  \begin{beamercolorbox}[wd=.6\paperwidth,ht=2.25ex,dp=1ex,center]{footline}%
    \usebeamerfont{footline}\usebeamertemplate{footline center}%
  \end{beamercolorbox}%
  \begin{beamercolorbox}[wd=.2\paperwidth,ht=2.25ex,dp=1ex,right]{footline}%
    \usebeamerfont{footline}\usebeamertemplate{footline right}%
    \kern\beamer@rightmargin
  \end{beamercolorbox}%
  }%
}

\defbeamertemplate{footline}{sectioning}{%
  % default height is 0.4pt, which is ignored by adobe reader, so we increase it by 0.2pt
  {\usebeamercolor[fg]{separator line}\hrule height 0.6pt}%
  \hbox{%
  \begin{beamercolorbox}[wd=.8\paperwidth,ht=2.25ex,dp=1ex,left]{footline}%
    \usebeamerfont{footline}\kern\beamer@leftmargin\insertshorttitle
    \ifx\insertsection\@empty\else\qquad$\vartriangleright$\qquad\insertsection\fi
    \ifx\insertsubsection\@empty\else\qquad$\vartriangleright$\qquad\insertsubsection\fi
  \end{beamercolorbox}%
  \begin{beamercolorbox}[wd=.2\paperwidth,ht=2.25ex,dp=1ex,right]{footline}%
     \usebeamerfont{footline}\usebeamertemplate{footline right}%
     \kern\beamer@rightmargin
  \end{beamercolorbox}%
  }%
}

% customize mini frames template to get a section navigation bar

\defbeamertemplate{navigation box}{current}{%
  \colorbox{accent2}{%
    \rule[-1ex]{0pt}{3.25ex}\color{white}\kern1.4pt\my@navibox\kern1.4pt%
  }%
}

\defbeamertemplate{navigation box}{other}{%
  %\colorbox{white}{%
    \rule[-1ex]{0pt}{3.25ex}\color{black}\kern1.4pt\my@navibox\kern1.4pt%
  %}%
}

\newcommand{\my@navibox@subsection}{$\blacksquare$}
\newcommand{\my@navibox@frame}{$\square$}
\let\my@navibox=\my@navibox@frame

% optional navigation box for some special frame
\newcommand{\my@navibox@frame@opt}{$\boxplus$}
\newcommand{\my@change@navibox}{\let\my@navibox=\my@navibox@frame@opt}
\newcommand{\changenavibox}{%
  \addtocontents{nav}{\protect\headcommand{\protect\my@change@navibox}}%
}

\newcommand{\my@sectionentry@show}[5]{%
  \ifnum\c@section=#1%
    \setbeamertemplate{navigation box}[current]%
  \else
    \setbeamertemplate{navigation box}[other]%
  \fi
  \begingroup
    \def\my@navibox{#1}%
    \hyperlink{Navigation#3}{\usebeamertemplate{navigation box}}%
  \endgroup
}

\newif\ifmy@hidesection

\newcommand{\my@sectionentry@hide}[5]{\my@hidesectiontrue}

\pretocmd{\beamer@setuplinks}{\renewcommand{\beamer@subsectionentry}[5]{}}{}{}
\apptocmd{\beamer@setuplinks}{\global\let\beamer@subsectionentry\mybeamer@subsectionentry}{}{}

\newcommand{\mybeamer@subsectionentry}[5]{\global\let\my@navibox=\my@navibox@subsection}

\newcommand{\my@slideentry@empty}[6]{}

\newcommand{\my@slideentry@section}[6]{%
  \ifmy@hidesection
    \my@hidesectionfalse
  \else
    \ifnum\c@section=#1%
      \setbeamertemplate{navigation box}[other]%
      \ifnum\c@subsection=#2\ifnum\c@subsectionslide=#3%
         \setbeamertemplate{navigation box}[current]%
      \fi\fi
      \beamer@link(#4){\usebeamertemplate{navigation box}}%
    \fi
  \fi
  \global\let\my@navibox=\my@navibox@frame
}

\AtEndDocument{%
   \immediate\write\@auxout{%
     \noexpand\gdef\noexpand\my@totalsectionnumber{\the\c@section}%
   }%
}

\def\my@totalsectionnumber{0}

\defbeamertemplate{footline}{navigation}{%
  % default height is 0.4pt, which is ignored by adobe reader, so we increase it by 0.2pt
  {\usebeamercolor[fg]{separator line}\hrule height 0.6pt}%
  \begin{beamercolorbox}[wd=\paperwidth,ht=2.25ex,dp=1ex]{footline}%
    \usebeamerfont{footline}%
    \kern\beamer@leftmargin
    \setlength{\fboxsep}{0pt}%
    \ifnum\my@totalsectionnumber=0%
      \insertshorttitle
    \else
      \let\sectionentry=\my@sectionentry@show
      \let\slideentry=\my@slideentry@empty
      \dohead
    \fi
    \hfill
    \let\sectionentry=\my@sectionentry@hide
    \let\slideentry=\my@slideentry@section
    \dohead
    \kern\beamer@rightmargin
  \end{beamercolorbox}%
}

%% ------------------------- frame title -----------------------------

\defbeamertemplate{frametitle}{simple}[1][]
{%
  \nointerlineskip
  \begin{beamercolorbox}[wd=\paperwidth,sep=0pt,leftskip=\beamer@leftmargin,%
                         rightskip=\beamer@rightmargin,#1]{frametitle}
    \usebeamerfont{frametitle}%
    \rule[-3.6mm]{0pt}{12mm}\insertframetitle\rule[-3.6mm]{0pt}{12mm}\par
  \end{beamercolorbox}
}

%% ------------------- block and theorem -----------------------------

\defbeamertemplate{theorem begin}{simple}
{%
  \upshape%\bfseries\inserttheoremheadfont
  {\usebeamercolor[fg]{theoremname}%
  \inserttheoremname\inserttheoremnumber
  \ifx\inserttheoremaddition\@empty\else
    \ \usebeamercolor[fg]{local structure}(\inserttheoremaddition)%
  \fi%
  %\inserttheorempunctuation
  }%
  \quad\normalfont
}
\defbeamertemplate{theorem end}{simple}{\par}

\defbeamertemplate{proof begin}{simple}
{%
  %\bfseries
  \let\@addpunct=\@gobble
  {\usebeamercolor[fg]{proofname}\insertproofname}%
  \quad\normalfont
}
\defbeamertemplate{proof end}{simple}{\par}

%% ---------------------- enumerate and itemize ----------------------

\expandafter\patchcmd\csname beamer@@tmpop@enumerate item@square\endcsname
         {height1.85ex depth.4ex}{height1.85ex depth.3ex}{}{}
\expandafter\patchcmd\csname beamer@@tmpop@enumerate subitem@square\endcsname
         {height1.85ex depth.4ex}{height1.85ex depth.3ex}{}{}
\expandafter\patchcmd\csname beamer@@tmpop@enumerate subsubitem@square\endcsname
         {height1.85ex depth.4ex}{height1.85ex depth.3ex}{}{}

%% ------------------------ select templates -------------------------

\setbeamertemplate{background canvas}[default]
\setbeamertemplate{background}[line]
\setbeamertemplate{footline}[navigation]
\setbeamertemplate{footline left}[author]
\setbeamertemplate{footline center}[title]
\setbeamertemplate{footline right}[number]
\setbeamertemplate{title page}[banner]
\setbeamertemplate{title page content}[hexagon]
\setbeamertemplate{section page}[fill]
\setbeamertemplate{show sections in toc}[partial]
\setbeamertemplate{section name}[simple]
\setbeamertemplate{section name in toc}[simple]
\setbeamertemplate{section in toc}[fill]
\setbeamertemplate{section in toc shaded}[default][50]
\setbeamertemplate{subsection page}[fill]
\setbeamertemplate{subsection name}[simple]
\setbeamertemplate{subsection in toc}[fill]
\setbeamertemplate{subsection in toc shaded}[default][50]
\setbeamertemplate{theorem begin}[default]
\setbeamertemplate{theorem end}[default]
\setbeamertemplate{proof begin}[default]
\setbeamertemplate{proof end}[default]
\setbeamertemplate{frametitle}[simple]
\setbeamertemplate{navigation symbols}{}
\setbeamertemplate{itemize items}[square]
\setbeamertemplate{enumerate items}[square]

%% --------------------------- font theme ----------------------------

\setbeamerfont{title}{size=\LARGE}
\setbeamerfont{subtitle}{size=\large}
\setbeamerfont{author}{size=\normalsize}
\setbeamerfont{institute}{size=\normalsize}
\setbeamerfont{date}{size=\normalsize}
\setbeamerfont{occasion}{size=\normalsize}
\setbeamerfont{section in toc}{size=\large}
\setbeamerfont{subsection in toc}{size=\large}
\setbeamerfont{frametitle}{size=\large}
\setbeamerfont{block title}{size=\normalsize}
\setbeamerfont{item projected}{size=\footnotesize}
\setbeamerfont{subitem projected}{size=\scriptsize}
\setbeamerfont{subsubitem projected}{size=\tiny}

\usefonttheme{professionalfonts}
%\usepackage{arev}

%% ---------------------------- color theme --------------------------

% always use rgb colors in pdf files
\substitutecolormodel{hsb}{rgb}

\definecolor{red99}{Hsb}{0,0.9,0.9}
\definecolor{brown74}{Hsb}{30,0.7,0.4}
\definecolor{brown75}{Hsb}{30,0.7,0.5}
\definecolor{yellow86}{Hsb}{60,0.8,0.6}
\definecolor{yellow99}{Hsb}{60,0.9,0.9}
\definecolor{cyan95}{Hsb}{180,0.9,0.5}
\definecolor{blue67}{Hsb}{240,0.6,0.7}
\definecolor{blue74}{Hsb}{240,0.7,0.4}
\definecolor{blue77}{Hsb}{240,0.7,0.7}
\definecolor{blue99}{Hsb}{240,0.9,0.9}
\definecolor{magenta88}{Hsb}{300,0.8,0.8}

\colorlet{text1}{black}
\colorlet{back1}{white}
\colorlet{accent1}{blue99}
\colorlet{accent2}{cyan95}
\colorlet{accent3}{red99}
\colorlet{accent4}{yellow86}
\colorlet{accent5}{magenta88}
\colorlet{filler1}{accent1!40!back1}
\colorlet{filler2}{accent2!40!back1}
\colorlet{filler3}{accent3!40!back1}
\colorlet{filler4}{accent4!40!back1}
\colorlet{filler5}{accent5!40!back1}
\colorlet{gray1}{black!20}
\colorlet{gray2}{black!35}
\colorlet{gray3}{black!50}
\colorlet{gray4}{black!65}
\colorlet{gray5}{black!80}
\colorlet{tcolor}{text1!10!back1}
\colorlet{dcolor}{white}
\colorlet{fcolor}{blue77}
\colorlet{markcolor}{gray}
\colorlet{sectcolor}{brown74}

\setbeamercolor{normal text}{bg=white,fg=black}
\setbeamercolor{structure}{fg=blue99}
\setbeamercolor{local structure}{fg=cyan95}
\setbeamercolor{footline}{bg=,fg=black}
\setbeamercolor{title}{fg=yellow99}
\setbeamercolor{subtitle}{fg=white}
\setbeamercolor{author}{fg=black}
\setbeamercolor{institute}{fg=black}
\setbeamercolor{date}{fg=white}
\setbeamercolor{occasion}{fg=white}
\setbeamercolor{section name}{fg=brown75}
\setbeamercolor{section in toc}{fg=yellow99,bg=blue67}
\setbeamercolor{section in toc shaded}{fg=white,bg=blue74}
\setbeamercolor{subsection name}{parent=section name}
\setbeamercolor{subsection in toc}{use={structure,normal text},fg=structure.fg!90!normal text.bg}
\setbeamercolor{subsection in toc shaded}{parent=normal text}
\setbeamercolor{frametitle}{parent=structure}
\setbeamercolor{separator line}{fg=accent2}
\setbeamercolor{theoremname}{parent=subsection in toc}
\setbeamercolor{proofname}{parent=subsection in toc}
\setbeamercolor{block title}{fg=accent1,bg=gray}
\setbeamercolor{block body}{bg=lightgray}
\setbeamercolor{block title example}{fg=accent2,bg=gray}
\setbeamercolor{block body example}{bg=lightgray}
\setbeamercolor{block title alerted}{fg=accent3,bg=gray}
\setbeamercolor{block body alerted}{bg=lightgray}

%% ----------------------- handout mode ------------------------------

\mode<handout>{
  \setbeamertemplate{background canvas}{}
  \setbeamertemplate{background}[empty]
  \setbeamertemplate{footline}[sectioning]
  \setbeamertemplate{section page}[single]
  \setbeamertemplate{subsection page}[single]
  \setbeamerfont{subsection in toc}{size=\large}
  \colorlet{dcolor}{darkgray}
  \colorlet{fcolor}{white}
  \colorlet{sectcolor}{white}
  \setbeamercolor{normal text}{fg=black, bg=white}
  \setbeamercolor{title}{fg=blue}
  \setbeamercolor{subtitle}{fg=gray}
  \setbeamercolor{occasion}{fg=black}
  \setbeamercolor{date}{fg=black}
  \setbeamercolor{section in toc}{fg=blue!90!gray,bg=}
  \setbeamercolor{section in toc shaded}{fg=lightgray,bg=}
  \setbeamercolor{subsection in toc}{fg=blue!80!gray}
  \setbeamercolor{subsection in toc shaded}{fg=lightgray}
  \setbeamercolor{frametitle}{fg=blue!70!gray,bg=}
  \setbeamercolor{theoremname}{fg=blue!60!gray}
  \setbeamercolor{proofname}{fg=blue!60!gray}
  \setbeamercolor{footline}{bg=white,fg=black}
}

\mode
<all>

\makeatother

% -*- coding: utf-8 -*-

% ----------------------------------------------
% 中文显示相关代码
% ----------------------------------------------

% 以前要放在 usetheme 后面,否则报错;但是现在没问题了
\PassOptionsToPackage{CJKnumber}{xeCJK}
\usepackage[UTF8,noindent]{ctex}
%\usepackage[UTF8,indent]{ctexcap}

% 开明式标点:句末点号用全角,其他用半角。
%\punctstyle{kaiming}

% 在旧版本 xecjk 中用 CJKnumber 选项会自动载入 CJKnumb 包
% 但在新版本 xecjk 中 CJKnumber 选项已经被废弃,需要在后面自行载入它
\usepackage{CJKnumb}

%\CTEXoptions[today=big] % 数字年份前会有多余空白,中文年份前是正常的

\makeatletter
\ifxetex
  \setCJKsansfont{SimHei} % fix for ctex 2.0
  \renewcommand\CJKfamilydefault{\CJKsfdefault}%
\else
  \@ifpackagelater{ctex}{2014/03/01}{}{\AtBeginDocument{\heiti}} %无效?
\fi
\makeatother

%% 在旧版本 ctex 中,\today 命令生成的中文日期前面有多余空格
\makeatletter
\@ifpackagelater{ctex}{2014/03/01}{}{%
  \renewcommand{\today}{\number\year 年 \number\month 月 \number\day 日}
}
\makeatother

%% 在 xeCJK 中,默认将一些字符排除在 CJK 类别之外,需要时可以加入进来
%% 可以在 “附件->系统工具->字符映射表”中查看某字体包含哪些字符
% https://en.wikipedia.org/wiki/Number_Forms
% Ⅰ、Ⅱ、Ⅲ、Ⅳ、Ⅴ、Ⅵ、Ⅶ、Ⅷ、Ⅸ、Ⅹ、Ⅺ、Ⅻ
\xeCJKsetcharclass{"2150}{"218F}{1} % 斜线分数,全角罗马数字等
% https://en.wikipedia.org/wiki/Enclosed_Alphanumerics
\xeCJKsetcharclass{"2460}{"24FF}{1} % 带圈数字字母,括号数字字母,带点数字等

\ifxetex
% 在标点后,xeCJK 会自动添加空格,但不会去掉换行空格
%\catcode`,=\active  \def,{\textup{,} \ignorespaces}
%\catcode`;=\active  \def;{\textup{;} \ignorespaces}
%\catcode`:=\active  \def:{\textup{:} \ignorespaces}
%\catcode`。=\active  \def。{\textup{.} \ignorespaces}
%\catcode`.=\active  \def.{\textup{.} \ignorespaces}
\catcode`。=\active   \def。{.}
% 在公式中使用中文逗号和分号
%\let\douhao, \def\zhdouhao{\text{,\hskip-0.5em}}
%\let\fenhao; \def\zhfenhao{\text{;\hskip-0.5em}}
%\begingroup
%\catcode`\,=\active \protected\gdef,{\text{,\hskip-0.5em}}
%\catcode`\;=\active \protected\gdef;{\text{;\hskip-0.5em}}
% 似乎 beamer 的 \onslide<1,3> 不受影响
% 但是如果 tikz 图形中包含逗号,可能无法编译
%\catcode`\,=\active
%\protected\gdef,{\ifmmode\expandafter\zhdouhao\else\expandafter\douhao\fi}
%\catcode`\;=\active
%\protected\gdef;{\ifmmode\expandafter\zhfenhao\else\expandafter\fenhao\fi}
%\endgroup
%\AtBeginDocument{\catcode`\,=\active \catcode`\;=\active}
% 这样写反而无效
%\def\zhpunct{\catcode`\,=\active \catcode`\;=\active}
%\AtBeginDocument{%
%  \everymath\expandafter{\the\everymath\zhpunct}%
%  \everydisplay\expandafter{\the\everydisplay\zhpunct}%
%}
% 改为使用 kerkis 字体的逗号
\DeclareSymbolFont{myletters}{OML}{mak}{m}{it}
\SetSymbolFont{myletters}{bold}{OML}{mak}{b}{it}
\AtBeginDocument{%
  \DeclareMathSymbol{,}{\mathpunct}{myletters}{"3B}%
}
\fi

% 汉字下面加点表示强调
\usepackage{CJKfntef}

% ----------------------------------------------
% 字体选用相关代码
% ----------------------------------------------

% 虽然 arevtext 字体的宽度较大,但考虑到文档的美观还是同时使用 arevtext 和 arevmath
% 若在 ctex 包之前载入它,其设定的 arevtext 字体会在载入 ctex 时被重置为 lm 字体
% 因此我们在 ctex 宏包之后才载入它,但此时字体编码被改为 T1,需要修正 \nobreakspace
\usepackage{arev}
\DeclareTextCommandDefault{\nobreakspace}{\leavevmode\nobreak\ }

% 即使只需要 arevmath,也不能直接用 \usepackage{arevmath},
% 因为旧版本 fontspec 有问题,这样会导致它错误地修改数学字体

% 旧版本的 XeTeX 无法使用 arev sans 等 T1 编码字体的单独重音字符
% 因此我们恢复使用组合重音字符,见 t1enc.def, fntguide.pdf 和 encguide.pdf
\ifxetex\ifdim\the\XeTeXversion\XeTeXrevision pt<0.9999pt
  \DeclareTextAccent{\'}{T1}{1}
\fi\fi
% 在 T1enc.def 文件中定义了 T1 编码中的重音字符
% 先用 \DeclareTextAccent{\'}{T1}{1} 表示在 T1 编码中 \'e 等于 \accent"01 e
% 再用 \DeclareTextComposite{\'}{T1}{e}{233} 表示在 T1 编码中 \'e 等于 \char"E9

% ----------------------------------------------
% 版式定制相关代码
% ----------------------------------------------

\usepackage{hyperref}
\hypersetup{
  %pdfpagemode={FullScreen},
  bookmarksnumbered=true,
  unicode=true
}

%% 保证在新旧 ctex 宏包下编译得到相同的结果
\renewcommand{\baselinestretch}{1.3} % ctex 2.4.1 开始为 1,之前为 1.3

%% LaTeX 中 默认 \parskip=0pt plus 1pt,而 Beamer 中默认 \parskip=0pt

%% \parskip 用 plus 1fil 没有效果,用 plus 1fill 则节标题错位
\setlength{\parskip}{5pt plus 1pt minus 1pt}       % 段间距为 5pt + 1pt - 1pt
%\setlength{\baselineskip}{19pt plus 1pt minus 1pt} % 行间距为 5pt + 1pt - 1pt
\setlength{\lineskiplimit}{4pt}                    % 行间距小于 4pt 时重新设置
\setlength{\lineskip}{4pt}                         % 行间距太小时自动改为 4pt

\AtBeginDocument{
  \setlength{\baselineskip}{19pt plus 1pt minus 1pt} % 似乎不能放在导言区中
  \setlength{\abovedisplayskip}{4pt minus 2pt}
  \setlength{\belowdisplayskip}{4pt minus 2pt}
  \setlength{\abovedisplayshortskip}{2pt}
  \setlength{\belowdisplayshortskip}{2pt}
}

% 默认是 \raggedright,改为两边对齐
\usepackage{ragged2e}
\justifying
\let\oldraggedright\raggedright
\let\raggedright\justifying

% ----------------------------------------------
% 文本环境相关代码
% ----------------------------------------------

\setlength{\fboxsep}{0.02\textwidth}\setlength{\fboxrule}{0.002\textwidth}

\usepackage{adjustbox}
\newcommand{\mylinebox}[1]{\adjustbox{width=\linewidth}{#1}}

\usepackage{comment}
\usepackage{multicol}

% 带圈的数字
%\newcommand{\digitcircled}[1]{\textcircled{\raisebox{.8pt}{\small #1}}}
\newcommand{\digitcircled}[1]{%
  \tikz[baseline=(char.base)]{%
     \node[shape=circle,draw,inner sep=0.01em,line width=0.07em] (char) {\small #1};
  }%
}

\usepackage{pifont}
% 因为 xypic 将 \1 定义为 frm[o]{--},这里改为在文档内部定义
%\def\1{\ding{51}} % 勾
%\def\0{\ding{55}} % 叉

% 若在 enumerate 中使用自定义模板,则各项前的间距由第七项决定
% 对于我们使用的 arev 数学字体来说各个数字是等宽的,所以没问题
% 参考 https://tex.stackexchange.com/q/377959/8956
% 以及 https://chat.stackexchange.com/transcript/message/38541073#38541073
\newenvironment{enumskip}[1][]{%
  \setbeamertemplate{enumerate mini template}[default]%
  \if\relax\detokenize{#1}\relax % empty
    \begin{enumerate}[\quad(1)]
  \else
    \begin{enumerate}[#1][\quad(1)]
  \fi
}{\end{enumerate}}
\newenvironment{enumzero}[1][]{%
  \setbeamertemplate{enumerate mini template}[default]%
  \if\relax\detokenize{#1}\relax % empty
    \begin{enumerate}[(1)\,]
  \else
    \begin{enumerate}[#1][(1)\,]
  \fi
}{\end{enumerate}}
%
\newenvironment{enumlite}[1][]{%
  \setbeamertemplate{enumerate mini template}[default]%
  \setbeamercolor{enumerate item}{fg=,bg=}%
  \if\relax\detokenize{#1}\relax % empty
    \begin{enumerate}[(1)\,]%
  \else
    \begin{enumerate}[#1][(1)\,]%
  \fi
}{\end{enumerate}}
%
\newcounter{mylistcnt}
%
\newenvironment{enumhalf}{%
  \par\setcounter{mylistcnt}{0}%
  \def\item##1~{%
    \leavevmode\ifhmode\unskip\fi\linebreak[2]%
    \makebox[.5002\textwidth][l]{\stepcounter{mylistcnt}(\arabic{mylistcnt}) \,##1\ignorespaces}%
  }%
  \ignorespaces%
}{\par}
%
\newenvironment{choiceline}[1][]{%
  \par\vskip-0.5em\relax
  \setbeamertemplate{enumerate mini template}[default]%
  \setbeamercolor{enumerate item}{fg=,bg=}%
  \if\relax\detokenize{#1}\relax % empty
    \begin{enumerate}[(A)\,]
  \else
    \begin{enumerate}[#1][(A)\,]
  \fi
}{\end{enumerate}}
%
\newenvironment{choicehalf}{%
  \par\setcounter{mylistcnt}{0}%
  \def\item##1~{%
    \leavevmode\ifhmode\unskip\fi\linebreak[2]%
    \makebox[.5001\textwidth][l]{\stepcounter{mylistcnt}(\Alph{mylistcnt}) \,##1\ignorespaces}%
  }%
  \ignorespaces%
}{\par}
\newenvironment{choicequar}{%
  \par\setcounter{mylistcnt}{0}%
  \def\item##1~{%
    \leavevmode\ifhmode\unskip\fi\linebreak[0]%
    \makebox[.2501\textwidth][l]{\stepcounter{mylistcnt}(\Alph{mylistcnt}) \,##1\ignorespaces}%
  }%
  \ignorespaces%
}{\par}

% ----------------------------------------------
% 定理环境相关代码
% ----------------------------------------------

\makeatletter
\patchcmd{\@thm}{ \csname}{\kern0.18em\relax\csname}{}{}
\makeatother

\newcommand{\mynewtheorem}[2]{%
  \newtheorem{#1}{#2}[section]%
  \expandafter\renewcommand\csname the#1\endcsname{\arabic{#1}}%
}

\mynewtheorem{theorem}{定理}
\newtheorem*{theorem*}{定理}

\mynewtheorem{algorithm}{算法}
\newtheorem*{algorithm*}{算法}

\mynewtheorem{conjecture}{猜想}
\newtheorem*{conjecture*}{猜想}

\mynewtheorem{corollary}{推论}
\newtheorem*{corollary*}{推论}

\mynewtheorem{definition}{定义}
\newtheorem*{definition*}{定义}

\mynewtheorem{example}{例}
\newtheorem*{example*}{例子}

\mynewtheorem{exercise}{练习}
\newtheorem*{exercise*}{练习}

\mynewtheorem{fact}{事实}
\newtheorem*{fact*}{事实}

\mynewtheorem{guess}{猜测}
\newtheorem*{guess*}{猜测}

\mynewtheorem{lemma}{引理}
\newtheorem*{lemma*}{引理}

\mynewtheorem{method}{解法}
\newtheorem*{method*}{解法}

\mynewtheorem{origin}{引例}
\newtheorem*{origin*}{引例}

\mynewtheorem{problem}{问题}
\newtheorem*{problem*}{问题}

\mynewtheorem{property}{性质}
\newtheorem*{property*}{性质}

\mynewtheorem{proposition}{命题}
\newtheorem*{proposition*}{命题}

\mynewtheorem{puzzle}{题}
\newtheorem*{puzzle*}{题目}

\mynewtheorem{remark}{注记}
\newtheorem*{remark*}{注记}

\mynewtheorem{review}{复习}
\newtheorem*{review*}{复习}

\mynewtheorem{result}{结论}
\newtheorem*{result*}{结论}

\newtheorem*{analysis}{分析}
\newtheorem*{answer}{答案}
\newtheorem*{choice}{选择}
\newtheorem*{hint}{提示}
\newtheorem*{solution}{解答}
\newtheorem*{thinking}{思考}

\newcommand{\mynewtheoremx}[2]{%
  \newtheorem{#1}{#2}%
  \expandafter\renewcommand\csname the#1\endcsname{\arabic{#1}}%
}
\mynewtheoremx{bonus}{选做}
\newtheorem*{bonus*}{选做}

\renewcommand{\proofname}{证明}
\renewcommand{\qedsymbol}{}
\renewcommand{\tablename}{表格}

% ----------------------------------------------
% 数学环境相关代码
% ----------------------------------------------

% 选学内容的角标星号
\newcommand{\optstar}{\texorpdfstring{\kern0pt$^\ast$}{}}

\usepackage{mathtools} % \mathclap 命令
\usepackage{extarrows}

% 切换 amsmath 的公式编号位置
% 不使用 leqno 选项而在这里才修改,会导致编号与公式重叠
% 因此在 \documentclass 里都加上了 leqno 选项
\makeatletter
\newcommand{\leqnomath}{\tagsleft@true}
\newcommand{\reqnomath}{\tagsleft@false}
\makeatother
%\leqnomath

% 定义带圈数字的 tag 格式,需要 mathtools 包
\newtagform{circ}[\color{accent2}\digitcircled]{}{}
\newtagform{skip}[\quad\color{accent2}\digitcircled]{}{}

\newcounter{savedequation}

\newenvironment{aligncirc}{%
  \setcounter{savedequation}{\value{equation}}%
  \setcounter{equation}{0}%
  \usetagform{circ}%
  \align
}{
  \endalign
  \setcounter{equation}{\value{savedequation}}%
}
\newenvironment{alignskip}{%
  \setcounter{savedequation}{\value{equation}}%
  \setcounter{equation}{0}%
  \usetagform{skip}%
  \align
}{
  \endalign
  \setcounter{equation}{\value{savedequation}}%
}
\newenvironment{alignlite}{%
  \setcounter{savedequation}{\value{equation}}%
  \setcounter{equation}{0}%
  \align
}{
  \endalign
  \setcounter{equation}{\value{savedequation}}%
}

% cases 环境开始时 \def\arraystretch{1.2}
% 在中文文档中还有 \linespread{1.3},这样公式的左花括号就太大了
% 这里利用 etoolbox 包将 \linespread 临时改回 1
\AtBeginEnvironment{cases}{\linespread{1}\selectfont}
% 奇怪的是在最新的 miktex 中无此问题,
% 而且即使这样修改,在新旧 miktex 中用 arev 字体时花括号大小还是有差别
% 而用默认的 lm 字体时花括号却没有差别

% 用于给带括号的方程组编号
\usepackage{cases}

\newcommand{\R}{\mathbb{R}}
\newcommand{\Rn}{\mathbb{R}^n}

% 大型的积分号
\usepackage{relsize}
\newcommand{\Int}{\mathop{\mathlarger{\int}}}

% \oiint 命令
% \usepackage[integrals]{wasysym}

% http://tex.stackexchange.com/q/84302
\DeclareMathOperator{\arccot}{arccot}

% https://tex.stackexchange.com/a/178948/8956
% 保证 \d x 和 \d(2x) 和 \d^2 x 的间距都合适
\let\oldd=\d
\renewcommand{\d}{\mathop{}\!\mathrm{d}}
\newcommand{\dx}{\d x}
\newcommand{\dy}{\d y}
\def\dz{\d z} % 不确定命令是否已经定义
\newcommand{\du}{\d u}
\newcommand{\dv}{\d v}
\newcommand{\dr}{\d r}
\newcommand{\ds}{\d s}
\newcommand{\dt}{\d t}
\newcommand{\dS}{\d S}

\newcommand{\e}{\mathrm{e}}
\newcommand{\limit}{\lim\limits}

% 分数线长一点的分数,\wfrac[2pt]{x}{y} 表示左右加 2pt
% 参考 http://tex.stackexchange.com/a/21580/8956
\DeclareRobustCommand{\wfrac}[3][2pt]{%
  {\begingroup\hspace{#1}#2\hspace{#1}\endgroup\over\hspace{#1}#3\hspace{#1}}}

% 划去部分公式
% 横着划线,参考 http://tex.stackexchange.com/a/20613/8956
\newcommand{\hcancel}[2][accent3]{%
  \setbox0=\hbox{$#2$}%
  \rlap{\raisebox{.3\ht0}{\textcolor{#1}{\rule{\wd0}{1pt}}}}#2%
}
% 斜着划线,参考 https://tex.stackexchange.com/a/15958
\newcommand{\dcancel}[2][accent3]{%
    \tikz[baseline=(tocancel.base),ultra thick]{
        \node[inner sep=0pt,outer sep=0pt] (tocancel) {$#2$};
        \draw[#1] (tocancel.south west) -- (tocancel.north east);
    }%
}%

% 竖直居中的 \dotfill
\newcommand\cdotfill{\leavevmode\xleaders\hbox to 0.5em{\hss\footnotesize$\cdot$\hss}\hfill\kern0pt\relax}

% 保持居中的 \not 命令
\usepackage{centernot}

% 使用 stix font 中的 white arrows
\ifxetex
    \IfFileExists{STIX-Regular.otf}{%
        \newfontfamily{\mystix}{STIX} % stix v1.1
    }{%
        \newfontfamily{\mystix}{STIXGeneral} % stix v1.0
    }
    \DeclareRobustCommand\leftwhitearrow{%
      \mathrel{\text{\normalfont\mystix\symbol{"21E6}}}%
    }
    \DeclareRobustCommand\upwhitearrow{%
      \mathrel{\text{\normalfont\mystix\symbol{"21E7}}}%
    }
    \DeclareRobustCommand\rightwhitearrow{%
      \mathrel{\text{\normalfont\mystix\symbol{"21E8}}}%
    }
    \DeclareRobustCommand\downwhitearrow{%
      \mathrel{\text{\normalfont\mystix\symbol{"21E9}}}%
    }
\else
    \let \leftwhitearrow = \Leftarrow
    \let \rightwhitearrow = \Rightarrow
    \let \upwhitearrow = \Uparrow
    \let \downwhitearrow = \Downarrow
\fi

% ----------------------------------------------
% 绘图动画相关代码
% ----------------------------------------------

% pgf/tikz 的所有选项都称为 key,它们按照 unix 路径的方式组织,
% 例如:/tikz/external/force remake={boolean}
% 这些 key 可以用 \pgfkeys 定义,用 \tikzset 设置
% \tikzset 实际上等同于 \pgfkeys{/tikz/.cd,#1}.
% Using Graphic Options: P120 in manual 2.10 (\tikzset)
% Key Management: P481 in manual 2.10 (\pgfkeys)

\usepackage{tikz}
\usepackage{pgfplots}
%\usepackage{calc}

% 文档标注,通常需要编译两次就可以得到正确结果
% 但如果主题的 section page 用 tikz 绘图,将需要编译三次
% 这是因为 tikzmark 依赖 aux 文件的 pgfid 编号
% 第一次编译时缺少 toc 文件,将缺少若干个 tikz 图片
% 第二次编译时图片个数正确了,但是 aux 文件的 pgfid 仍然是错误的
% 这个问题在主题文件中已经修正了
\newcommand\tikzmark[1]{%
  \tikz[overlay,remember picture] \node[coordinate] (#1) {};%
}

% pgf 包含的  xxcolor 包存在问题,导致与 xeCJKfntef 包冲突
% 见 https://github.com/CTeX-org/ctex-kit/issues/323
% 注意此冲突在 ctex 2.9 中不存在,仅在最新的 miktex 2.9 中出现
\makeatletter
\g@addto@macro\XC@mcolor{\edef\current@color{\current@color}}
\makeatother

\usetikzlibrary{matrix} % 用于在 node 四周加括号
\usetikzlibrary{decorations}
\usetikzlibrary{decorations.markings} % 用于在箭头上作标记
\usetikzlibrary{intersections} % 用于计算路径的交点
\usetikzlibrary{positioning} % 可以更方便地定位
\usetikzlibrary{shapes.geometric} % 钻石形状节点

\usetikzlibrary{calc}
\usetikzlibrary{snakes}

% Externalizing Graphics: P343 and P651 in manual 2.10
\usetikzlibrary{external}
% 编译时需加上 --shell-escape(texlive)或 -enable-write18(miktex)选项
%\tikzexternalize[prefix=figure/] %\tikzexternalize[shell escape=-enable-write18]

% 默认 tikz 图片会用 pdflatex 编译,可以自己改为 xelatex
%\tikzset{external/system call={%
%  xelatex \tikzexternalcheckshellescape -halt-on-error -interaction=batchmode -jobname "\image" "\texsource"}}

% 强制重新生成图片, pgf 3.0 中会自动比较文件的 md5
%\tikzset{external/force remake} %\tikzset{external/remake next}

%\tikzset{draw=black,color=black}
%\mode<beamer>{\tikzset{every path/.style={color=white!90!black}}}

\usetikzlibrary{patterns}

% hack pgf prior to version 3.0 for pgf patterns in xetex
% code taken from pgfsys-dvipdfmx.def and pgfsys-xetex.def in pgf 3.0
\makeatletter
\def\myhackpgf{
  % fix typo in pgfsys-common-pdf-via-dvi.def in pgf 2.10
  \pgfutil@insertatbegineverypage{%
     \ifpgf@sys@pdf@any@resources%
        \special{pdf:put @resources
           << \ifpgf@sys@pdf@patterns@exists /Pattern @pgfpatterns \fi >>}%
     \fi%
  }
  % required to give colors on pattern objects.
  \pgfutil@addpdfresource@colorspaces{ /pgfprgb [/Pattern /DeviceRGB] }
  % hook for xdvipdfmx
  \def\pgfsys@dvipdfmx@patternobj##1{%
	 \pgfutil@insertatbegincurrentpagefrombox{##1}%
  }%
  % dvipdfmx provides a new special `pdf:stream' for a stream object
  \def\pgfsys@dvipdfmx@stream##1##2##3{%
     \special{pdf:stream ##1 (##2) << ##3 >>}%
  }%
  % declare patterns and set patterns
  \def\pgfsys@declarepattern##1##2##3##4##5##6##7##8##9{%
     \pgf@xa=##2\relax \pgf@ya=##3\relax%
     \pgf@xb=##4\relax \pgf@yb=##5\relax%
     \pgf@xc=##6\relax \pgf@yc=##7\relax%
     \pgf@sys@bp@correct\pgf@xa \pgf@sys@bp@correct\pgf@ya%
     \pgf@sys@bp@correct\pgf@xb \pgf@sys@bp@correct\pgf@yb%
     \pgf@sys@bp@correct\pgf@xc \pgf@sys@bp@correct\pgf@yc%
     \pgfsys@dvipdfmx@patternobj{%
        \pgfsys@dvipdfmx@stream{@pgfpatternobject##1}{##8}{%
           /Type /Pattern
           /PatternType 1
           /PaintType \ifnum##9=0 2 \else 1 \fi
           /TilingType 1
           /BBox [\pgf@sys@tonumber\pgf@xa\space\pgf@sys@tonumber\pgf@ya\space
                  \pgf@sys@tonumber\pgf@xb\space\pgf@sys@tonumber\pgf@yb]
           /XStep \pgf@sys@tonumber\pgf@xc\space
           /YStep \pgf@sys@tonumber\pgf@yc\space
           /Resources << >> %<<
        }%
     }%
     \pgfutil@addpdfresource@patterns{/pgfpat##1\space @pgfpatternobject##1}%
  }
  \def\pgfsys@setpatternuncolored##1##2##3##4{%
     \pgfsysprotocol@literal{/pgfprgb cs ##2 ##3 ##4 /pgfpat##1\space scn}%
  }
  \def\pgfsys@setpatterncolored##1{%
     \pgfsysprotocol@literal{/Pattern cs /pgfpat##1\space scn}%
  }
}
\@ifpackagelater{pgf}{2013/12/18}{}{\ifxetex\expandafter\myhackpgf\fi}%
\makeatother

% ----------------------------------------------
% 表格制作相关代码
% ----------------------------------------------

\newcommand{\narrowsep}[1][2pt]{\setlength{\arraycolsep}{#1}}
\newcommand{\narrowtab}[1][3pt]{\setlength{\tabcolsep}{#1}}

% diagbox 依赖 pict2e,但 miktex 中旧版本 pict2e 打包错误,使得引擎判别错误
% 从而导致在编译时出现大量警告,以及导致底栏右下角按钮链接错位
\ifxetex\PassOptionsToPackage{xetex}{pict2e}\fi
\usepackage{diagbox}

\usepackage{multirow} % 跨行表格

\usepackage{array} % 可以用 \extrarowheight
% 双倍宽度的横线和竖线,\arrayrulewidth 默认为 0.4pt
\setlength{\doublerulesep}{0pt}
\newcommand{\dhline}{\noalign{\global\arrayrulewidth0.8pt}\hline\noalign{\global\arrayrulewidth0.4pt}}
\newcolumntype{?}{!{\vrule width 0.8pt}} % 即使 \doublerulesep 为 0pt,|| 也不能得到双倍宽度
% 最好还是用 tabu,更简单

\usepackage{tabularx}

%\usepackage{arydshln} % 在分块矩阵中加虚线
%\setlength{\dashlinedash}{2pt} % 默认4pt
%\setlength{\dashlinegap}{2pt} % 默认4pt

% tabu 与 arydshln 会冲突,可以不使用 arydshln,
% 而用 tabu 定义虚线 \newcolumntype{:}{|[on 2pt off 2pt]}
% 参考 http://bbs.ctex.org/forum.php?mod=viewthread&tid=63944#pid405057
\usepackage{tabu}
\newcolumntype{:}{|[on 2pt off 2pt]}
\newcommand{\hdashline}{\tabucline[0.4pt on 2pt off 2pt]{-}} % 兼容 arydshln 的命令
\setlength{\tabulinesep}{4pt} % 拉开大型公式与表格横线的距离

%\usepackage{colortbl} % 否则 \taburowcolors 命令无效

% ----------------------------------------------
% 绝对定位相关代码
% ----------------------------------------------

\usepackage[absolute,overlay]{textpos}

% 将整个页面分为 32 乘 24 个边长为 4mm 的小正方形
\setlength{\TPHorizModule}{4mm}
\setlength{\TPVertModule}{4mm}

\setlength{\TPboxrulesize}{0.6pt}
\newlength{\tpmargin}
\setlength{\tpmargin}{2mm}

\newenvironment{bblock}[1][black]{%
  \begingroup
  \TPshowboxestrue\TPMargin{\tpmargin}%
  \textblockrulecolor{#1}\textblockcolour{}%
  \begin{textblock}%
}{%
  \end{textblock}%
  \endgroup
}
\newenvironment{cblock}[2][black]{%
  \begingroup
  \TPshowboxestrue\TPMargin{\tpmargin}%
  \textblockrulecolor{#1}\textblockcolour{#2}%
  \begin{textblock}%
}{%
  \end{textblock}%
  \endgroup
}
\newenvironment{cblocka}{\begin{cblock}{filler1}}{\end{cblock}}
\newenvironment{cblockb}{\begin{cblock}{filler2}}{\end{cblock}}
\newenvironment{cblockc}{\begin{cblock}{filler3}}{\end{cblock}}
\newenvironment{cblockd}{\begin{cblock}{filler4}}{\end{cblock}}
\newenvironment{cblocke}{\begin{cblock}{filler5}}{\end{cblock}}

% ----------------------------------------------
% 模版定制相关代码
% ----------------------------------------------

\usepackage{bookmark}

\newcommand{\mybookmark}[1]{%
  \bookmark[page=\thepage,level=3]{#1}%
  \changenavibox
}

%\setbeamercovered{transparent=5}

\setbeamersize{text margin left=4mm,text margin right=4mm}
\mode<beamer>{\setbeamertemplate{background}[linear]}
\setbeamertemplate{footline}[sectioning]
\setbeamertemplate{footline right}[normal]
\setbeamertemplate{theorem begin}[simple]
\setbeamertemplate{theorem end}[simple]
\setbeamertemplate{proof begin}[simple]
\setbeamertemplate{proof end}[simple]

% 段间距在 block 中也许无效 http://tex.stackexchange.com/q/6111/8956
%\addtobeamertemplate{block begin}{}{\setlength{\parskip}{6pt plus 2pt minus 2pt}}

%\mode<beamer>{\tikzset{every path/.style={color=black}}}

% 在 amsfonts.sty 中已经废弃 \bold 命令,改用 \mathbf 命令
\def\lead#1{\textcolor{accent1}{#1}}
\def\bold#1{\textcolor{accent2}{#1}}
\def\warn#1{\textcolor{accent3}{#1}}
\def\clead{\color{accent1}}
\def\cbold{\color{accent2}}
\def\cwarn{\color{accent3}}

\mode<handout>{
  \colorlet{filler1}{filler1!40!white}
  \colorlet{filler2}{filler2!40!white}
  \colorlet{filler3}{filler3!40!white}
  \colorlet{filler4}{filler4!40!white}
  \colorlet{filler5}{filler5!40!white}
  \colorlet{gray1}{gray1!60!white}
  \colorlet{gray2}{gray2!60!white}
  \colorlet{gray3}{gray3!60!white}
  \colorlet{gray4}{gray4!60!white}
  \colorlet{gray5}{gray5!60!white}
}

% 兼容性命令,在 beamer 中应该避免使用它们,而改用上面几个命令
\let\textbf=\bold \def\pmb{\usebeamercolor[fg]{local structure}}
\let\emph=\warn   \def\bm{\usebeamercolor[fg]{alerted text}}

\newcommand{\vpause}{\pause\vskip 0pt plus 0.5fill\relax}
\newcommand{\ppause}{\par\pause}

\newcommand{\mybackground}{\setbeamertemplate{background}[lattice][4mm]}
% 几个 \varxxx 命令是 arevmath 包提供的
% $\heartsuit\varheart\diamondsuit\vardiamond$
% $\varspade\spadesuit\varclub\clubsuit$
% rframe 为例题解答,sframe 为练习解答,可以选择不包含它们
\newenvironment{rframe}{\mybackground\begin{frame}}{\end{frame}}
\newenvironment{sframe}{%
  \mybackground
  \colorlet{markcolor}{accent4}%
  \backgroundmarklefttrue\backgroundmarkrighttrue
  \begin{frame}
}{\end{frame}}
\ifdefined\slide
  \setbeamertemplate{footline}[navigation]
  \renewenvironment{rframe}{\begin{frame}<beamer:0>}{\end{frame}}%
  \renewenvironment{sframe}{\begin{frame}<beamer:0>}{\end{frame}}%
\fi
\ifdefined\print
  \renewenvironment{sframe}{\begin{frame}<handout:0>}{\end{frame}}%
\fi
% 用于标示只针对内招或外招的内容:iframe 为内招,oframe 为外招
\newenvironment{iframe}{\backgroundmarklefttrue\begin{frame}}{\end{frame}}
\newenvironment{oframe}{\backgroundmarkrighttrue\begin{frame}}{\end{frame}}
\newenvironment{jframe}{\mybackground\backgroundmarklefttrue\begin{frame}}{\end{frame}}
\newenvironment{pframe}{\mybackground\backgroundmarkrighttrue\begin{frame}}{\end{frame}}
\def\myimode{i}
\def\myomode{o}
\ifx\slide\myimode
  \renewenvironment{oframe}{\begin{frame}<presentation:0>}{\end{frame}}%
  \renewenvironment{pframe}{\begin{frame}<presentation:0>}{\end{frame}}%
  \renewenvironment{jframe}{\begin{frame}<presentation:0>}{\end{frame}}%
\fi
\ifx\slide\myomode
  \renewenvironment{iframe}{\begin{frame}<presentation:0>}{\end{frame}}%
  \renewenvironment{jframe}{\begin{frame}<presentation:0>}{\end{frame}}%
  \renewenvironment{pframe}{\begin{frame}<presentation:0>}{\end{frame}}%
\fi
\ifx\print\myimode
  \renewenvironment{oframe}{\begin{frame}<presentation:0>}{\end{frame}}%
  \renewenvironment{pframe}{\begin{frame}<presentation:0>}{\end{frame}}%
\fi
\ifx\print\myomode
  \renewenvironment{iframe}{\begin{frame}<presentation:0>}{\end{frame}}%
  \renewenvironment{jframe}{\begin{frame}<presentation:0>}{\end{frame}}%
\fi

% 利用 tikzmark 作边注
\newcommand{\imark}[1][gray]{%
  \begin{tikzpicture}[overlay,remember picture]
    \node[coordinate] (A) {};
    \fill[color=#1] (current page.west |- A) rectangle +(1.2mm,0.6em);
  \end{tikzpicture}%
}
\newcommand{\omark}[1][gray]{%
  \begin{tikzpicture}[overlay,remember picture]
    \node[coordinate] (A) {};
    \fill[color=#1] (A -| current page.east) rectangle +(-1.2mm,0.6em);
  \end{tikzpicture}%
}
\newcommand{\smark}{%
  \imark[accent4]\omark[accent4]%
}
\newcommand{\itext}[1]{%
  \ifx\slide\myomode\else
    \ifx\print\myomode\else
      #1%
    \fi
  \fi
}
\newcommand{\otext}[1]{%
  \ifx\slide\myimode\else
    \ifx\print\myimode\else
      #1%
    \fi
  \fi
}
\newcommand{\stext}[1]{%
  \ifdefined\slide\else
    \ifdefined\print\else
      #1%
    \fi
  \fi
}

% 选择题的答案
\newcommand{\select}[1]{\qquad\stext{\llap{\makebox[2em]{\color{accent4}#1}}}}

%% 内外招同编号的定理,例子或练习等,需要将编号减一
\newcommand{\minusone}[1]{%
  \ifdefined\slide\else
    \ifdefined\print\else
      \addtocounter{#1}{-1}%
    \fi
  \fi
}

%\mode<beamer>{
%\def\mytoctemplate{
%  \setbeamerfont{section in toc}{size=\normalsize}
%  \setbeamerfont{subsection in toc}{size=\small}
%  \setbeamertemplate{section in toc shaded}[default][100]
%  \setbeamertemplate{subsection in toc}[subsections numbered]
%  \setbeamertemplate{subsection in toc shaded}[default][100]
%  \setbeamercolor{section in toc}{fg=structure.fg}
%  \setbeamercolor{section in toc shaded}{fg=structure.fg!50!black}
%  \setbeamercolor{subsection in toc}{fg=structure.fg}
%  \setbeamercolor{subsection in toc shaded}{fg=normal text.fg}
%  \begin{multicols}{2}
%  \tableofcontents[sectionstyle=show/shaded,subsectionstyle=show/shaded]
%  \end{multicols}
%}
%\AtBeginSection[]{\begin{frame}\frametitle{目录结构}\mytoctemplate\end{frame}}
%\AtBeginSubsection[]{\begin{frame}\frametitle{目录结构}\mytoctemplate\end{frame}}
%}

\mode<presentation>

\setbeamertemplate{section and subsection}[chinese]
\usebeamertemplate{section and subsection}

\mode
<all>

% -*- coding: utf-8 -*-

% ----------------------------------------------
% 高等数学中的定义和改动
% ----------------------------------------------

\newif\ifligong % 理工类或经济类
\ligongtrue

% Repeating Things: P504 in manual 2.10
\newcommand{\drawline}[4][]{%
  \foreach \v [remember=\v as \u,count=\i] in {#4} {
    \ifnum \i > 1
      \ifodd \i \draw[#1,#3] \u -- \v; \else \draw[#1,#2] \u -- \v; \fi
    \fi
  }
}
\newcommand{\drawplot}[5][]{%
  \foreach \v [remember=\v as \u,count=\i] in {#4} {
    \ifnum \i > 1
      \ifodd \i \draw[#1,#3] plot[domain=\u:\v] #5; \else \draw[#1,#2] plot[domain=\u:\v] #5; \fi
    \fi
  }
}

% http://tex.stackexchange.com/q/84302
\DeclareMathOperator{\Prj}{Prj}
\DeclareMathOperator{\grad}{grad}

\newcommand{\va}{\vec{a\vphantom{b}}}
\newcommand{\vb}{\vec{b}}
\newcommand{\vc}{\vec{c\vphantom{b}}}
\newcommand{\vd}{\vec{d}}
\newcommand{\ve}{\vec{e}}
\newcommand{\vi}{\vec{i}}
\newcommand{\vj}{\vec{j}}
\newcommand{\vk}{\vec{k}}
\newcommand{\vn}{\vec{n}}
\newcommand{\vs}{\vec{s}}
\newcommand{\vv}{\vec{v}}

\let\ov=\overrightarrow

% xcolor 支持 hsb 色彩模型,但 pgf 不支持,因此需要指定输出的色彩模型为 rgb
% 在 article 中可以用 \usepackage[rgb]{xcolor} \usepackage{tikz} 解决此问题
% 在 beamer 中可以用 \documentclass[xcolor={rgb}]{beamer} 解决此问题

%\definecolor{bcolor0}{Hsb}{0,0.6,0.9}   % red 红色
%\definecolor{bcolor1}{Hsb}{60,0.6,0.9}  % yellow 黄色
%\definecolor{bcolor2}{Hsb}{120,0.6,0.9} % green 绿色
%\definecolor{bcolor3}{Hsb}{180,0.6,0.9} % cyan 青色
%\definecolor{bcolor4}{Hsb}{240,0.6,0.9} % blue 蓝色
%\definecolor{bcolor5}{Hsb}{300,0.6,0.9} % magenta 洋红色

\colorlet{bcolor0}{accent3}
\colorlet{bcolor1}{accent1}
\colorlet{bcolor2}{accent2}
\colorlet{bcolor3}{accent4}
\colorlet{bcolor5}{accent5}


\begin{document}
%\ifxetex\else\heiti\fi

\occasion{高等数学课程}
\title{第九章·多元函数微分法}
\author{\href{https://lvjr.bitbucket.io}{吕荐瑞}}
\institute{暨南大学数学系}

\begin{frame}[plain]
\titlepage
\end{frame}

\section{多元函数的基本概念}

\subsection{多元函数的概念}

\begin{frame}
\frametitle{二元函数}
\begin{definition}
从平面$\mathbb{R}^2$的非空子集$D$到$\mathbb{R}$的对应关系
$$f:D\longrightarrow\mathbb{R}$$
称为\bold{二元函数},其中对应$f$ 将点$(x,y)$对应到$f(x,y)$。记为$z=f(x,y)$。
\pause $x$和$y$称为\bold{自变量},$z$称为\bold{因变量}.
%\vpause
%类似地可以定义三元函数。
\end{definition}
\end{frame}

\begin{frame}
\frametitle{二元函数的定义域}
二元函数的\bold{自然定义域}:由所有使得$f(x,y)$有意义的点$(x,y)$组成的集合.
\ppause\vspace{1em}\hrule\vspace{1em}
\begin{example}
$z=\ln(x+y)$的定义域为\onslide<4->{\cdotfill\bold{无界\onslide<5->{开区域}}}
$$D=\{(x,y) \mid x+y>0\}.$$
\end{example}
\pause
\begin{example}
$z=\sqrt{1-x^2-y^2}$的定义域为\onslide<4->{\cdotfill\bold{有界\onslide<5->{闭区域}}}
$$D=\{(x,y) \mid x^2+y^2\leq1\}.$$
\end{example}
\end{frame}

\begin{frame}
\frametitle{自然定义域}
\begin{exercise}
求二元函数的定义域并画出该区域。
\begin{enumlite}
  \item $f(x,y)=\ln(x^2+y^2-1)$
  \item $f(x,y)=\dfrac1{\sqrt{x+y-2}}$
  \pause
  \item $f(x,y)=\sqrt{1-|x|-|y|}$
\end{enumlite}
\end{exercise}
\end{frame}

\begin{frame}
\frametitle{平面点集的分类}
\begin{description}[一二三]
  \item[有界集] 限制在有限范围的点集
  \item[无界集] 延伸到无穷远的点集
\end{description}
\vpause
\begin{description}[一二三]
  \item[开区域] 不包含边界的区域
  \item[闭区域] 包含边界的区域
\end{description}
\vpause\cdotfill
\begin{problem*}
如何准确描述上述几种平面点集?
\end{problem*}
\end{frame}

\begin{frame}
\frametitle{直线点集与平面点集}
\begin{tabularx}{\textwidth}{XX}
  \hline
  \bold{直线$\mathbb{R}$} & \bold{平面$\mathbb{R}^2$} \\
  \hline
  邻域             & 邻域 \\
  有限集           & 有界集 \\
  无限集           & 无界集 \\
  开区间           & 开区域 \\
  闭区间           & 闭区域 \\
  端点             & 边界 \\
  \hline
\end{tabularx}
\end{frame}

\begin{frame}
%\frametitle{平面中的邻域}
\begin{definition}
平面上的点集
\[ \left\{ (x,y) \;\middle|\; \sqrt{(x-x_0)^2+(y-y_0)^2} < \delta \right\}\]
称为点$P_0(x_0,y_0)$的 \bold{$\delta$邻域},记为$\bold{U(P_0,\delta)}$。
\end{definition}
\vpause
\begin{definition}
平面上的点集
\[ \left\{ (x,y) \;\middle|\; 0 < \sqrt{(x-x_0)^2+(y-y_0)^2} < \delta \right\}\]
称为点$P_0(x_0,y_0)$的 \bold{去心$\delta$邻域},记为$\bold{\mathring{U}(P_0,\delta)}$。
\end{definition}
\pause\cdotfill
\begin{description}[一二三]
  \item[有界集] 存在某个$r>0$,使得$E\subset U(O,r)$.\pause
  \item[无界集] 对于任何$r>0$,总有$E\not\subset U(O,r)$.
\end{description}
\end{frame}

\begin{frame}
%\frametitle{点与点集的关系}
\begin{description}[一二]
  \item[内点] 若存在$P$的某个邻域$U(P)$使得$U(P)\subset E$,\newline 则称$P$为$E$的内点.\pause
  \item[外点] 若存在$P$的某个邻域$U(P)$使得$U(P)\cap E=\emptyset$,\newline 则称$P$为$E$的外点.
\end{description}
\pause
\begin{description}[一二三]
  \item[边界点] 若$P$的任何邻域,既含有属于$E$的点,又含有不属于$E$的点,则称$P$为$E$的边界点.
\end{description}
\pause
\begin{remark*}
(1) 内点一定属于$E$;(2) 外点一定不属于$E$;(3) 边界点可能属于$E$也可能不属于$E$.
\end{remark*}
\pause\cdotfill
\begin{description}[一二]
  \item[边界] $E$的边界点的全体,称为$E$的边界,记为$\bold{\partial E}$.
\end{description}
\end{frame}

\begin{frame}
%\frametitle{点集的特征}
\begin{description}[一二]
  \item[开集] 若$E$的边界点都不属于$E$,即$\partial E\cap E=\emptyset$,\newline 则称$E$为开集.\pause
  \item[闭集] 若$E$的边界点都属于$E$,即$\partial E\subset E$,\newline 则称$E$为闭集.
\end{description}
\pause
\begin{description}[一二三]
  \item[连通集] 若$E$中任何两点都可用$E$中的折线联结起来.则称$E$为连通集.
\end{description}
\pause
\begin{description}[一二三]
  \item[开区域] \CJKunderdot{非空}的连通开集称为开区域.\pause %也可简称为\bold{区域}.\pause
  \item[闭区域] 开区域及其边界一起构成的点集称为闭区域.
\end{description}
\end{frame}

\begin{sframe}
\frametitle{点集的特征}
\begin{remark*}
我们希望这里定义的开区域(闭区域)概念是上册的开区间(闭区间)概念在
$\mathbb{R}^n$中的推广,因此
\begin{enumerate}
  \item 需要将开区域定义为\CJKunderdot{非空}的连通开集
  \item 忽略可以将开区域简称为区域的说法
\end{enumerate}
\end{remark*}
\vpause
\begin{remark*}
即使开区域可以是空集,连通的闭集也未必是闭区域,比如平面上的直线是连通闭集但不是闭区域.\par
事实上,在$\mathbb{R}^2$中直线的内点集为空集,而空集的边界也是空集,因此直线不是闭区域.
\end{remark*}
\end{sframe}

\subsection{多元函数的极限}

\begin{iframe}
\frametitle{二元函数的极限:定义}
\begin{definition}
如果任意给定$\epsilon>0$,总存在一个$\delta>0$,使得当点$(x,y)\in \mathring{U}(P_0,\delta)$时,
$$|f(x,y)-A|<\epsilon$$
总成立,则称当$(x,y)$趋于点$P_0(x_0,y_0)$时,函数$f(x,y)$以$A$为\bold{极限},记为
$$\limit_{(x,y)\to(x_0,y_0)}f(x,y)=A \text{\quad 或\quad} \limit_{P\to P_0}f(x,y)=A.$$
\end{definition}
\vpause
\begin{example}
证明$\limit_{(x,y)\to(0,0)}f(x,y)=0$,其中\vspace{-0.5em}
$$f(x,y)=(x^2+y^2)\sin\dfrac1{x^2+y^2}.$$
\end{example}
\end{iframe}

\begin{oframe}
\frametitle{二元函数的极限:解释}
若$(x,y)$以任意方式趋于$(x_0,y_0)$时,$f(x,y)$总趋于$A$,则称$f(x,y)$以$A$为\bold{极限},记为
$$\limit_{(x,y)\to(x_0,y_0)}f(x,y)=A \text{\quad 或\quad} \limit_{P\to P_0}f(x,y)=A.$$
\end{oframe}

\begin{iframe}
\frametitle{二元函数的极限:解释}
\begin{remark*}
函数极限$\limit_{(x,y)\to(x_0,y_0)}f(x,y)=A$成立等价于当$(x,y)$ \warn{以任意方式}趋于$(x_0,y_0)$ 时,
$f(x,y)$ 总趋于$A$。
\end{remark*}
\vpause
\begin{example}
证明$\limit_{(x,y)\to(0,0)}f(x,y)$不存在,其中
$$f(x,y)=\left\{\begin{matrix}
  \dfrac{xy}{x^2+y^2}, & (x,y)\neq(0,0); \\ 0, & (x,y)=(0,0).
\end{matrix}\right.$$
\end{example}
\end{iframe}

\begin{frame}
\frametitle{二重极限的运算法则}
\begin{remark*}
二元函数的极限运算法则与一元函数的类似.
\end{remark*}
\vpause
\begin{example}
求极限$\limit_{(x,y)\to(0,2)}\dfrac{\sin{xy}}{x}$.
\end{example}
\vpause\cdotfill
\begin{remark*}
在二元函数极限的定义中,不要求函数在$P_0$的某个去心邻域有定义,
只要求$P_0$是定义域$D$的\bold{聚点}.
\end{remark*}
\vpause
\begin{definition*}
若$\forall\delta>0$,$\mathring{U}(P,\delta)$内总有$E$中的点,则称$P$为$E$的\bold{聚点}.
\end{definition*}
\end{frame}

\subsection{多元函数的连续性}

\begin{frame}
\frametitle{连续函数}
\begin{definition}
设$f(x,y)$的定义域$D$有聚点$(x_0,y_0)$.如果
$$\limit_{(x,y)\to(x_0,y_0)}f(x,y)=f(x_0,y_0),$$
则称$f(x,y)$在$(x_0,y_0)$处\bold{连续},
或者称$(x_0,y_0)$是$f(x,y)$的一个\bold{连续点}.
\end{definition}
\vpause
\begin{remark*}
函数$f(x,y)$的不连续点$(x_0,y_0)$称为\bold{间断点}.
\end{remark*}
\vpause
\begin{property}
二元初等函数在\CJKunderdot{定义区域}上总是连续的.
\end{property}
\end{frame}

\begin{frame}
\frametitle{二元函数的极限}
\begin{example}
求二元函数极限$\limit_{(x,y)\to(1,2)}\dfrac{x+y}{xy}$.
\end{example}
\vpause
\begin{example}
求极限$\limit_{(x,y)\to(0,0)}\dfrac{\sqrt{1+xy}-1}{xy}$.
\end{example}
\end{frame}

\begin{frame}
\frametitle{二元连续函数的性质}
\begin{property}
若$f(x,y)$在有界闭区域$D$上连续,则有
\begin{enumerate}
  \item 它在$D$上有界,且能取得最大值和最小值.
  \item 它能取到介于最大值和最小值之间的任何值.
\end{enumerate}
\end{property}
\end{frame}

\begin{frame}
\frametitle{二元函数的间断点}
\begin{example}
找出下列二元函数的所有间断点:
\begin{enumlite}
  \item $f(x,y)=\sin\dfrac1{x^2+y^2-1}$
  \item $g(x,y)=\begin{cases}
       \dfrac{xy}{x^2+y^2}, & (x,y)\neq(0,0); \\ 0, & (x,y)=(0,0).
       \end{cases}$
\end{enumlite}
\end{example}
\end{frame}

\mybookmark{复习与提高}

\begin{frame}
\frametitle{复习与提高}
\begin{review}
求二元函数的定义域并画出该区域。
\begin{enumlite}
  \item $f(x,y)=\sqrt{x^2+y^2-1}+\sqrt{4-x^2-y^2}$
  \item $f(x,y)=\ln(x-y+2)+\ln(2x+y-2)$
\end{enumlite}
\end{review}
\vpause
\begin{remark*}
\parbox[t]{0.8\textwidth}{%
$y>f(x)$表示$y=f(x)$的上方区域。\newline
$y<f(x)$表示$y=f(x)$的下方区域。}
\end{remark*}
\end{frame}

\begin{frame}
\frametitle{复习与提高}
\begin{puzzle}
设$f(x+y,x-y)=xy$,求$f(x,y)$.
\end{puzzle}
\vpause
\begin{puzzle}
求极限$\limit_{(x,y)\to(0,0)}\dfrac{x^2y^2}{x^2+y^2}$.
\end{puzzle}
\end{frame}

\begin{frame}
\frametitle{复习与提高}
\begin{puzzle}
说明极限$\limit_{(x,y)\to(0,0)}\dfrac{xy}{x+y}$不存在.
\end{puzzle}
\end{frame}

\begin{frame}
\frametitle{复习与提高}
\begin{puzzle}
找出下面函数的所有间断点:
$$f(x,y)=\begin{cases}
x\sin\dfrac{1}{y}, & y \neq 0; \\ 0, & y=0.
\end{cases}$$
\end{puzzle}
\end{frame}

\begin{frame}
\frametitle{复习与提高}
\begin{choice}
有且仅有一个间断点的函数为\dotfill(\select{C})
\begin{choicehalf}
  \item $\dfrac{x}{y}$ ~
  \item $\dfrac{x}{x+y}$ ~
  \item $\e^{-x}\ln(x^2+y^2)$ ~
  \item $\arctan(|xy|+1)$ ~
\end{choicehalf} 
\end{choice}
\end{frame}

\section{偏导数}

\subsection{一阶偏导数}

\begin{frame}
\frametitle{偏导数}
\begin{definition}
设函数$f(x,y)$在$(x_0,y_0)$某邻域内有定义,如果极限
\[ \onslide*<2->{f'_x(x_0,y_0)=}\lim_{\Delta x\to0} \frac{f(x_0+\Delta x,y_0)-f(x_0,y_0)}{\Delta x}\]
存在,则称该极限为函数在点$(x_0,y_0)$处\bold{对$x$的偏导数},记为$f'_x(x_0,y_0)$。
\ppause
\onslide<3->类似地定义函数在点$(x_0,y_0)$处\bold{对$y$的偏导数}为
\[ f'_y(x_0,y_0) = \lim_{\Delta y\to0} \frac{f(x_0,y_0+\Delta y)-f(x_0,y_0)}{\Delta y}\]
\end{definition}
\vpause\onslide<4->
\begin{example}
设$f(x,y)=xy^2$,求$f'_x(2,1)$和$f'_y(2,1)$.
\end{example}
\end{frame}

\begin{frame}
\frametitle{偏导函数}
\begin{definition}
设$f(x,y)$在区域$D$的每一点对$x$的偏导数都存在,则有\bold{对$x$的偏导函数}
\[ f'_x(x,y)=\lim_{\Delta x\to0} \frac{f(x+\Delta x,y)-f(x,y)}{\Delta x}\]
\ppause
类似地,设$f(x,y)$在区域$D$的每一点对$y$的偏导数都存在,则有\bold{对$y$的偏导函数}
\[ f'_y(x,y) = \lim_{\Delta y\to0} \frac{f(x,y+\Delta y)-f(x,y)}{\Delta y}\]
\end{definition}
\vpause
\begin{remark*}
在不至于混淆时,可把\bold{偏导函数}简称为\bold{偏导数}.
\end{remark*}
\end{frame}

\begin{frame}
\frametitle{偏导函数}
对于$z=f(x,y)$,将$y$看为常数,对$x$求导,得到$z$对$x$的\bold{偏导数},
记为$\dfrac{\partial z}{\partial x}$,\pause 或$\dfrac{\partial f}{\partial x}$,或$z'_x$,或$f'_x$。
\vpause
对于$z=f(x,y)$,将$x$看为常数,对$y$求导,得到$z$对$y$的\bold{偏导数},
记为$\dfrac{\partial z}{\partial y}$,\pause 或$\dfrac{\partial f}{\partial y}$,或$z'_y$,或$f'_y$。
\end{frame}

\begin{frame}
\begin{example}
求$z=x^2+xy+y^2$的偏导数。
\end{example}
\pause
\begin{example}
求$z=\dfrac{xy}{2x-y}$的偏导数。
\end{example}
\pause
\begin{example}
求$f(x,y)=\mathrm{e}^{x^2y}$在点$(1,2)$处的偏导数。
\end{example}
\vpause
\begin{exercise}
求下列函数的偏导数。
\begin{enumlite}
  \item $z=2x^3-5xy^2+x^2y$
  \item $z=\arctan\dfrac{y}x$
\end{enumlite}
\end{exercise}
\end{frame}

\begin{frame}
\frametitle{偏导数的几何意义}
\begin{columns}[onlytextwidth]
\column{0.6\textwidth}
\begin{tikzpicture}[thick,inner sep=2pt,
                    x={(-0.5cm,-0.6cm)},y={(1.2cm,0)},z={(0,1cm)}]
  \draw[-stealth,thin,gray] (0,0,0) -- (3.6,0,0);
  \draw[-stealth,thin,gray] (0,0,0) -- (0,4.4,0);
  \draw[-stealth,thin,gray] (0,0,0) -- (0,0,5.2) node[left,color=text1]{$z$};
  \coordinate (x0) at (1,0,0);
  \coordinate (y0) at (0,1,0);
  \coordinate (x1) at (3,0,0);
  \coordinate (y1) at (0,4,0);
  \coordinate (P0) at (1,1,0);
  \coordinate (P1) at (3,4,0);
  \coordinate (Py) at (1,4,0);
  \coordinate (Px) at (3,1,0);
  \coordinate (Qy) at (1,4,3);
  \coordinate (Qx) at (3,1,3);
  \coordinate (Q1) at (3,4,3);
  \coordinate (M0) at (1,1,3);
  \coordinate (My) at (1,4,4.7);
  \coordinate (Mx) at (3,1,5.5);
  \coordinate (M1) at (3,4,7); 
  \coordinate (Ty) at (1,4,3.8);
  \coordinate (Tx) at (3,1,4.8);
  \coordinate (T1) at (3,4,5.6);
  \draw[densely dashed] (x0) node[left]{$x_0$} -- node[pos=0.6,fill=back1,inner sep=0pt]{$\Delta y$} (Py);
  \draw[densely dashed] (y0) node[above]{$y_0$} -- node[pos=0.65,fill=back1,inner sep=0pt]{$\Delta x$} (Px);
  \draw (x1) node[left]{$x$} -- (P1) node[right]{$P$} -- (y1) node[above]{$y$};
  \draw[densely dashed] (Qx) -- (M0) node[below right]{$M_0$} -- (Qy);
  \draw (Qx) node[left]{$Q_1$} -- (Q1) node[above left]{$Q$} -- (Qy) node[right]{$Q_2$};
  \draw[densely dashed] (P0) node[below right]{$P_0$} -- (M0);
  \draw (P1) -- (M1) node[left=2pt]{$M$} (Py) -- (My) (Px) -- (Mx);
  \draw[densely dashed,color=accent1] (Mx) to[bend right=25] (M0) to[bend right=15] (My);
  \draw[color=accent1] (Mx) node[left,color=text1] {$M_1$}
       to[bend right=30] (M1) to[bend right=30] (My) node[right,color=text1] {$M_2$};
  \draw[densely dashed,color=accent3] (Tx) -- (M0) -- (Ty);
  \draw (Tx) node[left] {$T_1$} -- (T1) node[above left] {$T$} -- (Ty) node[right] {$T_2$};
\end{tikzpicture}
\column{0.39\textwidth}
\setlength{\fboxsep}{4pt}%
\adjustbox{cfbox=accent2}{\parbox{0.85\linewidth}{
\begin{align*}
  M_0M_1 \text{切线} &= M_0T_1 \\
  \angle Q_1M_0T_1 &= \alpha \\
  f'_x(x_0,y_0) &= \tan\alpha
\end{align*}
\vspace{0.5em}%
\begin{align*}
  M_0M_2 \text{切线} &= M_0T_2 \\
  \angle Q_2M_0T_2 &= \beta \\
  f'_y(x_0,y_0) &= \tan\beta
\end{align*}}}
\end{columns}
\end{frame}

\begin{frame}
\frametitle{偏导数与连续性}
\begin{example}
说明$f(x,y)$在点$(0,0)$的两个偏导数存在,但在点$(0,0)$不连续,其中
$$f(x,y)=\left\{\begin{matrix}
  \dfrac{xy}{x^2+y^2}, & (x,y)\neq(0,0); \\ 0, & (x,y)=(0,0).
\end{matrix}\right.$$
\end{example}
\end{frame}

\begin{frame}
\frametitle{三元函数的偏导数}
类似地,对于三元函数$u=f(x,y,z)$,可以定义三个偏导数$\dfrac{\partial u}{\partial x}$,
$\dfrac{\partial u}{\partial y}$ 和 $\dfrac{\partial u}{\partial z}$。
\pause
\begin{example}
求三元函数$u=xy^2z^3$的偏导数。
\end{example}
\pause
\begin{example}
求三元函数$r=\sqrt{x^2+y^2+z^2}$的偏导数.
\end{example}
\end{frame}

\subsection{高阶偏导数}

\begin{frame}
\frametitle{二阶偏导数}
对$z=f(x,y)$的偏导数$z'_x$和$z'_y$再求偏导数,就得到四个\bold{二阶偏导数}:\pause
\begin{itemize}[<+->]
  \item $(z'_x)'_x=z^{''}_{xx}$ 或 $(f'_x)'_x=f^{''}_{xx}$
  \item $(z'_x)'_y=z^{''}_{xy}$ 或 $(f'_x)'_y=f^{''}_{xy}$
  \item $(z'_y)'_x=z^{''}_{yx}$ 或 $(f'_y)'_x=f^{''}_{yx}$
  \item $(z'_y)'_y=z^{''}_{yy}$ 或 $(f'_y)'_y=f^{''}_{yy}$
\end{itemize}
\end{frame}

\begin{frame}
\begin{example}
求$z=x^3+y^3-3xy^2$的各二阶偏导数。
\end{example}
\pause
\begin{example}
求$z=x^2ye^y$的各二阶偏导数。
\end{example}
\vpause
\begin{theorem*}
当二阶偏导数$f''_{xy}(x,y)$和$f''_{yx}(x,y)$都连续时,两者必定相等。
\end{theorem*}
\vpause
\begin{exercise}
求下列函数的二阶偏导数。
\begin{enumlite}
  \item $z=x^2y^3+\e^x\sin y$
  \item $z=\ln(x-y)$
\end{enumlite}
\end{exercise}
\end{frame}

\begin{frame}
\frametitle{二阶偏导数}
$z=f(x,y)$的二阶偏导数也可以这样表示:\pause
\begin{itemize}[<+->]
  \item $\dfrac{\partial}{\partial x}\left(\dfrac{\partial z}{\partial x}\right)=\dfrac{\partial^2z}{\partial x^2}=z^{''}_{xx}$
  \item $\dfrac{\partial}{\partial y}\left(\dfrac{\partial z}{\partial x}\right)=\dfrac{\partial^2z}{\partial x\partial y}=z^{''}_{xy}$
  \item $\dfrac{\partial}{\partial x}\left(\dfrac{\partial z}{\partial y}\right)=\dfrac{\partial^2z}{\partial y\partial x}=z^{''}_{yx}$
  \item $\dfrac{\partial}{\partial y}\left(\dfrac{\partial z}{\partial y}\right)=\dfrac{\partial^2z}{\partial y^2}=z^{''}_{yy}$
\end{itemize}
\end{frame}

\mybookmark{复习与提高}

\begin{frame}
\frametitle{复习与提高}
\begin{review}
求下列函数的偏导数。
\begin{enumlite}
  \item $z=\dfrac{x}{x^2-y^2}$\pause
  \item $z=\arctan(x-y)$
\end{enumlite}
\end{review}
\end{frame}

\begin{frame}
\frametitle{复习与提高}
\begin{review}
求下列函数的二阶偏导数。
\begin{enumlite}
  \item $z=y^2\e^{x-y}$\pause
  \item $z=xy\cos y$
\end{enumlite}
\end{review}
\end{frame}

\begin{frame}
\frametitle{复习与提高}
\begin{puzzle}
设$z=\int_x^y f(t)\dt$,求$\dfrac{\partial z}{\partial x}$和$\dfrac{\partial z}{\partial y}$.
\end{puzzle}
\vpause
\begin{puzzle}
设$f(x,y)=x^2y+(y-1)\arctan\left(x/y\right)$,求$f'_x(1,1)$.
\end{puzzle}
\end{frame}

\begin{frame}
\frametitle{复习与提高}
\begin{puzzle}
判断$f(x,y)=\sqrt{x^2+y^2}$在$(0,0)$是否连续,以及偏导数是否存在.
\end{puzzle}
\vpause
\begin{puzzle}
判断$f(x,y)=\begin{cases}
  1, & xy=0 \\
  0, & xy\neq0
\end{cases}$在$(0,0)$是否连续,以及偏导数是否存在.
\end{puzzle}
\end{frame}

\begin{frame}
\frametitle{复习与提高}
\begin{puzzle}
设二元函数
$$f(x,y)=\begin{cases}
  xy\,\dfrac{x^2-y^2}{x^2+y^2}, & x^2+y^2\neq0; \\
  0, & x^2+y^2=0.
\end{cases}$$
说明混合偏导数$f''_{xy}(0,0)$和$f''_{yx}(0,0)$不相等.
\end{puzzle}
\end{frame}

\begin{frame}
\frametitle{复习与提高}
\begin{puzzle}
求满足条件$\dfrac{\partial f}{\partial y}=x^2+2y$,$f(x,x^2)=1$的二元函数$f(x,y)$.
\end{puzzle}
\end{frame}

\section{全微分}

\subsection{全微分的定义}

\begin{frame}
\frametitle{全微分}
\begin{example*}
用$S$表示边长分别为$x$与$y$的矩形的面积,则$S=xy$.\pause
如果边长$x$与$y$分别取得改变量$\Delta x$与$\Delta y$,\pause
则面积$S$相应地有一个改变量
\[ \Delta S = y \Delta x + x \Delta y + \Delta x\Delta y. \]
\end{example*}
\stext{\vpause
\begin{remark*}\smark
当$\Delta P\to0$时,$\Delta x\Delta y=o(\Delta P)$.事实上,
\[\limit_{\Delta P\to0}\frac{\Delta x\Delta y}{\Delta P}
 =\limit_{(\Delta x,\Delta y)\to(0,0)}\frac{\Delta x\Delta y}{\sqrt{(\Delta x)^2+(\Delta y)^2}}=0.\]
其中从极限定义知$\Delta P\to0$等价于$(\Delta x,\Delta y)\to(0,0)$.
\end{remark*}}
\end{frame}

\begin{frame}
\begin{definition*}
如果$z=f(x,y)$在点$(x,y)$的某邻域内满足
\[ \Delta z = A\Delta x + B\Delta y + o(\Delta P),\quad(\Delta P\to0) \]
其中$A$,$B$与$\Delta x$,$\Delta y$无关,$\Delta P=\sqrt{(\Delta x)^2+(\Delta y)^2}$,
则称函数在点$(x,y)$ \bold{可微分},并称它的\bold{全微分}为
\[ \d{z} = A\Delta{x} + B\Delta{y} \]
\end{definition*}
\vpause
\begin{theorem*}
函数在$(x,y)$点可微 \bold{\large$\rightwhitearrow$} 函数在$(x,y)$点连续.
\end{theorem*}
\vspace{0.3em}\vpause
\begin{theorem*}%[可微的必要条件]
如果函数$z=f(x,y)$可微,则$A=f'_x(x,y)$,$B=f'_y(x,y)$.即有\vspace{-0.5em}
\[\bold{\d z=f'_x(x,y)\d x+f'_y(x,y)\d y},\]
\vskip-0.8em 其中$\d{x}=\Delta{x}$,$\d{y}=\Delta{y}$.
\end{theorem*}
\end{frame}

\begin{frame}
\frametitle{微分的几何意义:以直代曲}
\begin{tikzpicture}[thick,inner sep=2pt,declare function={
    f(\x)=0.3*(\x-2)^2+2;
    g(\x)=0.6*(\x-2); % 导函数
}] % 定义函数后的分号不能省略
%\useasboundingbox (-1,-1) rectangle (9,6.5);
\draw[thin,-stealth] (-0.3,0) -- (6.9,0);
\draw[thin,-stealth] (0,-0.4) -- (0,6.5) node[below left]{$y$};
\draw[domain=0.7:5.7,samples=30,color=accent1] plot (\x,{f(\x)});
\draw[domain=-0.3:7,color=accent3] plot (\x,{g(3)*(\x-3)+f(3)}); % 切线
\fill (3,{f(3)}) node[above left]{$M_0$} circle (1.2pt);
\fill (5,{f(5)}) node[above left]{$M$} circle (1.2pt);
\fill (3,0) node[above left]{$P_0$} circle (1.2pt);
\fill (5,0) node[above left]{$P$} circle (1.2pt);
\draw[densely dashed] (3,0) node[below]{$x_0$} -- (3,{f(3)});
\draw[densely dashed] (5,0) node[below]{$x$} -- (5,{f(5)});
\draw[densely dashed,stealth-stealth,] (3.05,{f(3)}) -- node[below]{$\Delta x$} (4.99,{f(3)}) node[above left] {$Q$};
\draw[stealth-stealth,color=accent3] (5,{f(3.01)}) -- node[pos=0.5,right]{$\dy$} (5,{g(3)*(5-3)+f(3)});
\draw (4.9,{g(3)*(5-3)+f(3)}) -- node[left]{$T$} (5.1,{g(3)*(5-3)+f(3)});
\draw (5,{f(3)}) -- (6.3,{f(3)}) (5,{f(5)}) -- (6.3,{f(5)});
\draw[stealth-stealth,color=accent1] (6.15,{f(3)}) -- node[pos=0.45,right]{$\Delta y$} (6.15,{f(5)});
\node[above right,text width=4cm,align=left,draw=accent2,inner sep=0.4em] at (7.1,-0.4)
{当$P_0P\to0$时,\\[-0.3em] $TM=o(P_0P)$ \\[0.5em] 当$\Delta x\to0$时,\\[-0.3em] $\Delta y-\dy=o(\Delta x)$};
\end{tikzpicture}
\end{frame}

\begin{frame}
\frametitle{全微分的几何意义:以平代曲}
\begin{tikzpicture}[thick,inner sep=2pt,
                    x={(-0.5cm,-0.6cm)},y={(1.2cm,0)},z={(0,1cm)}]
\draw[-stealth,thin,gray] (0,0,0) -- (3.6,0,0);
\draw[-stealth,thin,gray] (0,0,0) -- (0,4.4,0);
\draw[-stealth,thin,gray] (0,0,0) -- (0,0,5.2) node[left,color=text1]{$z$};
\coordinate (x0) at (1,0,0);
\coordinate (y0) at (0,1,0);
\coordinate (x1) at (3,0,0);
\coordinate (y1) at (0,4,0);
\coordinate (P0) at (1,1,0);
\coordinate (P1) at (3,4,0);
\coordinate (Py) at (1,4,0);
\coordinate (Px) at (3,1,0);
\coordinate (Qy) at (1,4,3);
\coordinate (Qx) at (3,1,3);
\coordinate (Q1) at (3,4,3);
\coordinate (Q2) at (3,5.8,3);
\coordinate (Q3) at (3,6.8,3);
\coordinate (M0) at (1,1,3);
\coordinate (My) at (1,4,4.7);
\coordinate (Mx) at (3,1,5.5);
\coordinate (M1) at (3,4,7);
\coordinate (M3) at (3,6.8,7);
\coordinate (Ty) at (1,4,3.8);
\coordinate (Tx) at (3,1,4.8);
\coordinate (T1) at (3,4,5.6);
\coordinate (T2) at (3,5.8,5.6);
\draw[densely dashed] (x0) node[left]{$x_0$} -- node[pos=0.6,fill=back1,inner sep=0pt]{$\Delta y$} (Py);
\draw[densely dashed] (y0) node[above]{$y_0$} -- node[pos=0.65,fill=back1,inner sep=0pt]{$\Delta x$} (Px);
\draw[densely dashed] (P0) -- node[pos=0.5,fill=back1,inner sep=-1pt]{$\Delta P$} (P1);
\draw (x1) node[left]{$x$} -- (P1) node[right]{$P$} -- (y1) node[above]{$y$};
\draw[densely dashed] (Qx) -- (M0) node[below right]{$M_0$} -- (Qy);
\draw (Qx) node[left]{$Q_1$} -- (Q1) node[above left]{$Q$} -- (Qy) node[right]{$Q_2$};
\draw[densely dashed] (P0) node[above left]{$P_0$} -- (M0);
\draw (P1) -- (M1) node[left=2pt]{$M$} (Py) -- (My) (Px) -- (Mx);
\draw[densely dashed,color=accent1] (Mx) to[bend right=25] (M0) to[bend right=15] (My);
\draw[color=accent1] (Mx) node[left,color=text1] {$M_1$}
   to[bend right=30] (M1) to[bend right=30] (My) node[right,color=text1] {$M_2$};
\draw[densely dashed,color=accent3] (Tx) -- (M0) -- (Ty);
\draw[color=accent3] (Tx) node[left,color=text1] {$T_1$}
                -- (T1) node[above left,color=text1] {$T$}
                -- (Ty) node[right,color=text1] {$T_2$};
\draw[dashdotted,thin] (Q1) -- (Q2);
\draw[dashdotted,thin] (T1) -- (T2) -- +(0,0.05,0);
\draw[stealth-stealth,accent3] (Q2) -- node[fill=back1]{$\dz$} (T2);
\draw[dashdotted,thin] (Q2) -- (Q3) -- +(0,0.05,0);
\draw[dashdotted,thin] (M1) -- (M3) -- +(0,0.05,0);
\draw[stealth-stealth,accent1] (Q3) -- node[fill=back1]{$\Delta z$} (M3);
\node[right,text width=3.8cm,align=left,draw=accent2,inner sep=0.4em] at (0.9,5.2,0)
{当$P_0P\to0$时,\\[-0.3em] $TM=o(P_0P)$ \\[0.5em] 当$\Delta P\to0$时,\\[-0.3em] $\Delta z-\dz=o(\Delta P)$};
\end{tikzpicture}
\end{frame}

\begin{rframe}
%\frametitle{全微分的几何意义}
\begin{remark*}
在前面的图形中,我们断言$QT=\dz$.
\end{remark*}
%\pause
\begin{solution}
由偏导数几何意义,可得
\[Q_1T_1=f'_x\cdot\Delta x,\quad Q_2T_2=f'_y\cdot\Delta y.\]
因此我们只需说明$QT = Q_1T_1 + Q_2T_2$.事实上,
\begin{align*}
\ov{QT} &= \ov{M_0T} - \ov{M_0Q} \\
&= (\ov{M_0T_1}+\ov{M_0T_2}) - (\ov{M_0Q_1}+\ov{M_0Q_2}) \\
&= (\ov{M_0T_1}-\ov{M_0Q_1}) + (\ov{M_0T_2}-\ov{M_0Q_2}) \\
&= \ov{Q_1T_1} + \ov{Q_2T_2}
\end{align*}
再由$\ov{QT}$,$\ov{Q_1T_1}$,$\ov{Q_2T_2}$平行,得到上述结论.
\end{solution}
\end{rframe}

\begin{frame}
\frametitle{全微分与偏导数}
\begin{example}
证明函数$f(x,y)$在点$(0,0)$的偏导数存在,但不可微分.其中
$$f(x,y)=\begin{cases}
  \dfrac{xy}{\sqrt{x^2+y^2}}, & (x,y)\neq(0,0); \\[1em]
  0, & (x,y) = (0,0).
\end{cases}$$
\end{example}
\vpause
\begin{theorem*}%[可微的充分条件]
若多元函数各个偏导数都连续,则全微分存在.
\end{theorem*}
\end{frame}

\begin{frame}
\frametitle{全微分}
设二元函数$z=f(x,y)$可微,则全微分为
$$\cbold\mathrm{d}z=f'_x(x,y)\mathrm{d}x+f'_y(x,y)\mathrm{d}y$$
\vskip-1.3em\pause 设三元函数$u=f(x,y,z)$可微,则全微分为
$$\cbold\mathrm{d}u=f'_x(x,y,z)\mathrm{d}x+f'_y(x,y,z)\mathrm{d}y+f'_z(x,y,z)\mathrm{d}z$$
\vspace{-0.5em}\pause\hrule\par\vspace{0.4em}
\begin{example}
求$z=x^2y^3$在$x=1$,$y=2$,$\Delta x=0.2$,$\Delta y=0.1$时的全微分。
\end{example}
\vpause
\begin{example}
求$z=\e^{xy}$的全微分。
\end{example}
\vpause
\begin{example}
求$u=xy+yz+zx$的全微分。
\end{example}
\end{frame}

\subsection{用全微分求近似值\optstar}

\begin{frame}
\frametitle{近似计算}
利用全微分公式,我们有下列近似计算公式:
\[ \cbold f(x+\Delta x,y+\Delta y) \approx f(x,y) + f'_x(x,y)\Delta x + f'_y(x,y)\Delta y \]
\vpause
\begin{example}
求$1.01^{2.99}$的近似值。
\end{example}
\end{frame}

\mybookmark{复习与提高}

\begin{frame}
\frametitle{复习与提高}
\begin{puzzle}
求$z=\arctan\left(\dfrac{x-y}{x+y}\right)$的全微分$\dz$.
\end{puzzle}
\vpause
\begin{puzzle}
求$f(x,y,z)=\sqrt[z]{x/y}$在点$(1,1,1)$的全微分.
\end{puzzle}
\end{frame}

\begin{frame}
\frametitle{复习与提高}
\begin{puzzle}
证明$f(x,y)=\sqrt{|xy|}$在点$(0,0)$偏导数存在,但不可微分.
\end{puzzle}
\end{frame}

\begin{frame}
\frametitle{复习与提高}
\begin{choice}%[2002]
考虑二元函数$f(x,y)$的下面$4$条性质:\par
\quad\digitcircled1 $f(x,y)$在点$(x_0,y_0)$处连续\par
\quad\digitcircled2 $f(x,y)$在点$(x_0,y_0)$处的两个偏导数连续\par
\quad\digitcircled3 $f(x,y)$在点$(x_0,y_0)$处可微\par
\quad\digitcircled4 $f(x,y)$在点$(x_0,y_0)$处的两个偏导数存在\par
若用$P\Rightarrow Q$表示可由性质$P$推出$Q$,
则有\dotfill(\select{A})
\begin{choicehalf}
  \item \digitcircled2 $\Rightarrow$ \digitcircled3 $\Rightarrow$ \digitcircled1 ~
  \item \digitcircled3 $\Rightarrow$ \digitcircled2 $\Rightarrow$ \digitcircled1 ~
  \item \digitcircled3 $\Rightarrow$ \digitcircled4 $\Rightarrow$ \digitcircled1 ~
  \item \digitcircled3 $\Rightarrow$ \digitcircled1 $\Rightarrow$ \digitcircled4 ~
\end{choicehalf}
\end{choice}
\end{frame}

\section{多元复合函数求导}

\subsection{复合函数求导法则}

\begin{frame}
\frametitle{复合函数求导:情形1}
设$z=f(x,y)$, $x=\phi(t)$, $y=\psi(t)$,\pause 则我们得到复合函数$z=f(\phi(t),\psi(t))$。\pause
此时我们有\bold{全导数}
\[ \bold{\frac{\dz}{\dt}=\frac{\partial z}{\partial x}\frac{\dx}{\dt}+\frac{\partial z}{\partial y}\frac{\dy}{\dt}} \]
这里要求$f(x,y)$的偏导数连续.
\pause\vspace{0.5em}\hrule % TODO: 若显示公式后没有正文,则这里的 \pause 将导致多余的竖直空隙
\begin{example}
设$z=xy$,$x=\e^t$, $y=\sin t$,求全导数$\dfrac{\dz}{\dt}$。
\end{example}
\end{frame}

\begin{frame}
\frametitle{复合函数求导:情形2}
设$z=f(u,v)$, $u=\phi(x,y)$, $v=\psi(x,y)$,\pause 则有复合函数$z=f(\phi(x,y),\psi(x,y))$。\pause
此时我们有偏导数\vspace{.5em}
{\cbold\begin{align*}
\frac{\partial z}{\partial x}
&=\frac{\partial z}{\partial u}\frac{\partial u}{\partial x}+\frac{\partial z}{\partial v}\frac{\partial v}{\partial x},\\
\frac{\partial z}{\partial y}
&=\frac{\partial z}{\partial u}\frac{\partial u}{\partial y}+\frac{\partial z}{\partial v}\frac{\partial v}{\partial y}.
\end{align*}}
这里要求$f(u,v)$的偏导数连续.
\pause\vspace{0.5em}\hrule%\vspace{-1em}
\begin{example}
设$z=uv$,$u=3x^2+y^2$, $v=2x+y$,求偏导数$\dfrac{\partial z}{\partial x}$和$\dfrac{\partial z}{\partial y}$。
\end{example}
\end{frame}

\begin{frame}
\frametitle{复合函数求导}
\begin{exercise}
\begin{enumlite}
  \item 设$z=\e^{x-2y}$,$x=\sin t$, $y=\cos t$,求全导数$\dfrac{\mathrm{d}z}{\mathrm{d}t}$。\pause
  \item 设$z=\e^u\sin v$,$u=xy$, $v=x-y$,求偏导数$\dfrac{\partial z}{\partial x}$和$\dfrac{\partial z}{\partial y}$。
\end{enumlite}
\end{exercise}
\end{frame}

\begin{frame}
\frametitle{复合函数求导:情形3}
\begin{example}
设$z=f(u,v,t)=uv+\sin t$,而$u=\e^t$,$v=\cos t$.求全导数$\dfrac{\dz}{\dt}$.
\end{example}
\vpause
\begin{example}
设$w=f(x+y+z,xyz)$,且$f$具有二阶连续偏导数,求$\dfrac{\partial w}{\partial x}$
及$\dfrac{\partial^2 w}{\partial x\partial z}$.
\end{example}
\end{frame}

\begin{iframe}
\frametitle{复合函数求导}
\begin{example}
设$u=f(x,y)$的所有二阶偏导数连续,把下列表达式转换为极坐标系中的形式:
\begin{enumhalf}
  \item $\left(\dfrac{\partial u}{\partial x}\right)^2+\left(\dfrac{\partial u}{\partial y}\right)^2$ ~
  \item $\dfrac{\partial^2 u}{\partial x^2}+\dfrac{\partial^2 u}{\partial y^2}$ ~
\end{enumhalf}
\end{example}
\end{iframe}

\subsection{全微分形式不变性}

\begin{frame}
\frametitle{全微分的形式不变性}
设有$z=f(u,v)$,$u$, $v$为自变量,则全微分为
$$\d{z}=\frac{\partial z}{\partial u}\d{u}+\frac{\partial z}{\partial v}\d{v}$$
\pause
若又有$u=\phi(x,y)$, $v=\psi(x,y)$,$u$, $v$ 为中间变量,则全微分仍为
$$\d{z}=\frac{\partial z}{\partial u}\d{u}+\frac{\partial z}{\partial v}\d{v}$$
这里假定所有函数的偏导数都连续.
\end{frame}

\begin{frame}
\begin{example}
利用全微分的形式不变性,求二元函数$$z=(x^2-y^2)\e^{xy}$$的偏导数
$\dfrac{\partial z}{\partial x}$和$\dfrac{\partial z}{\partial y}$.
\end{example}
\end{frame}

\mybookmark{复习与提高}

\begin{frame}
\frametitle{复习与提高}
\begin{review}
\begin{enumlite}
  \item 设$z=\dfrac{x}y$,$x=\e^t$, $y=\sqrt{t}$,求全导数$\dfrac{\dz}{\dt}$。
  \item 设$z=uv$,$u=x\sin y$, $v=y\cos x$,求偏导数$\dfrac{\partial z}{\partial x}$和$\dfrac{\partial z}{\partial y}$。
\end{enumlite}
\end{review}
\end{frame}

\begin{frame}
\frametitle{复习与提高}
\begin{puzzle}
设$f(u,v)$有二阶连续偏导数,求$z=f(x,x/y)$的一阶和二阶偏导数.
\end{puzzle}
\vpause
\begin{puzzle}
设$f(u,v)$有连续偏导数,求$w=f(x/y,y/z)$的全微分.
\end{puzzle}
\end{frame}

\begin{frame}
\frametitle{复习与提高}
\begin{puzzle}
已知$f(x,y)|_{y=x^2}=1$,$f'_1(x,y)|_{y=x^2}=2x$,求$f'_2(x,y)|_{y=x^2}$.
\pause\cdotfill$-1$
\end{puzzle}
\vpause
\begin{puzzle}%[2001年考研]
设$z=f(x,y)$在点$(1,1)$处可微,$f(1,1)=1$,$f'_1(1,1)=2$,求$f'_2(1,1)=3$,
$\phi(x)=f(x,f(x,x))$,求$\left.\dfrac{\d}{\dx}\phi^3(x)\right|_{x=1}$.
\pause\cdotfill$51$
\end{puzzle}
\end{frame}

\begin{frame}
\frametitle{复习与提高}
\begin{choice}
已知二元函数$f(x,y)$具有一阶连续偏导数,且$f(x,y)=f(y,x)$,则有\dotfill(\select{B})
\begin{choicehalf}
  \item $f'_1(x,y)=f'_1(y,x)$ ~
  \item $f'_1(x,y)=f'_2(y,x)$ ~
  \item $f'_1(x,y)=f'_2(x,y)$ ~
  \item $f'_2(x,y)=f'_2(y,x)$ ~
\end{choicehalf} 
\end{choice}
\end{frame}

\section{隐函数求导}

\subsection{一个方程的情形}

\begin{frame}
\frametitle{隐函数的导数1}
\begin{theorem*}
设$F(x,y)$在$(x_0,y_0)$邻域有连续偏导数
\[ \left\{\begin{array}{@{}l@{}}
   F(x,y)=0 \\ F(x_0,y_0)=0
\end{array}\right.
\quad\overset{\cwarn F'_y\neq0}{\text{\LARGE\bold{$\rightwhitearrow$}}}\quad
   \left\{\begin{array}{@{}l@{}}
   y=f(x) \\ y_0=f(x_0)
\end{array}\right. \]\pause
而且隐函数$y=f(x)$也有连续偏导数
\[ \wfrac{\dy}{\dx}=-\wfrac{F'_x}{F'_y} \]
\end{theorem*}
\pause\hrule
%% 二阶导数的计算较繁琐,换成下面的例子
%\begin{example}
%设方程$y-x\e^y+x=0$确定了隐函数$y=f(x)$,求$\dfrac{\dy}{\dx}$和$\dfrac{\d^2y}{\dx^2}$.
%\end{example}
\begin{example}
设方程$x^4+y^4=1$确定了隐函数$y=f(x)$,求$\dfrac{\dy}{\dx}$和$\dfrac{\d^2y}{\dx^2}$.
\end{example}
\end{frame}

\begin{sframe}
\frametitle{隐函数的导数1}
\begin{solution}
隐函数的一阶导数$\smash[t]{\dfrac{\dy}{\dx}=-\dfrac{x^3}{y^3}}$.因此二阶导数
\begin{align*}
  \frac{\d^2y}{\dx^2} &= -\frac{\dfrac{\d}{\dx}(x^3)\cdot y^3-x^3\cdot\bold{\dfrac{\d}{\dx}(y^3)}}{y^6} \\
  &= -\frac{3x^2\cdot y^3-x^3\cdot\bold{3y^2\dfrac{\dy}{\dx}}}{y^6} = -\frac{3x^2}{y^7}
\end{align*}
\end{solution}
\begin{remark*}
注意不要错误认为$\dfrac{\d}{\dx}(y^3)=0$.
\end{remark*}
\end{sframe}

\begin{frame}
\frametitle{隐函数的导数2}
\begin{theorem*}
设$F(x,y,z)$在$(x_0,y_0,z_0)$邻域有连续偏导数
\[ \left\{\begin{array}{@{}l@{}}
   F(x,y,z)=0 \\ F(x_0,y_0,z_0)=0
\end{array}\right.
\quad\overset{\cwarn F'_z\neq0}{\text{\LARGE\bold{$\rightwhitearrow$}}}\quad
   \left\{\begin{array}{@{}l@{}}
   z=f(x,y) \\ z_0=f(x_0,y_0)
\end{array}\right. \]\pause
而且隐函数$z=f(x,y)$也有连续偏导数
\[
\wfrac{\partial z}{\partial x}=-\wfrac{F'_x}{F'_z}\qquad\qquad
\wfrac{\partial z}{\partial y}=-\wfrac{F'_y}{F'_z}
\]
\end{theorem*}
\pause\hrule
\begin{example}
设方程$x^2+y^2+z^2-4z=0$确定了隐函数$z=f(x,y)$,求二阶偏导数$\dfrac{\partial^2 z}{\partial x^2}$.
\end{example}
\end{frame}

\begin{frame}
\begin{exercise}
\begin{enumlite}
  \item 设方程$\sin y+\e^x-xy^2=0$确定了隐函数$y=f(x)$,求导数$\dfrac{\dy}{\dx}$。
  \item 设方程$\e^z=xyz$确定了隐函数$z=f(x,y)$,求偏导数$\dfrac{\partial z}{\partial x}$ 和$\dfrac{\partial z}{\partial y}$。
\end{enumlite}
\end{exercise}
\end{frame}

\subsection{方程组的情形}

\begin{iframe}
\frametitle{方程组确定的隐函数}
隐函数存在定理可以推广到方程组情形.比如
\[ 
\left\{\begin{array}{@{}l@{}}F(x,y,u,v)=0 \\ G(x,y,u,v)=0\end{array}\right.
\quad\overset{\cwarn?}{\text{\LARGE\bold{$\rightwhitearrow$}}}\quad
\left\{\begin{array}{@{}l@{}}u=u(x,y) \\ v=v(x,y)\end{array}\right.
\]
\vspace{0.5em}\hrule\vspace{0.5em}\pause
由$F$、$G$的偏导数组成的行列式
\[ J=\frac{\partial(F,G)}{\partial(u,v)}=\begin{vmatrix}
  \frac{\partial F}{\partial u} & \frac{\partial F}{\partial v} \\[.5em]
  \frac{\partial G}{\partial u} & \frac{\partial G}{\partial v}
\end{vmatrix} \]
称为$F$、$G$的\bold{雅可比行列式}.\pause 隐函数存在要求$\warn{J\neq0}$.
\end{iframe}

\begin{iframe}
%\frametitle{方程组确定的隐函数}
\begin{theorem*}
设$F$, $G$在$(x_0,y_0,u_0,v_0)$邻域有连续偏导数
\[ \left\{\begin{array}{@{}l@{}}
   F(x,y,u,v)=0 \\ G(x,y,u,v)=0 \\ F(x_0,y_0,u_0,v_0)=0 \\ G(x_0,y_0,u_0,v_0)=0
\end{array}\right.
\quad\overset{\cwarn J\neq0}{\text{\LARGE\bold{$\rightwhitearrow$}}}\quad
   \left\{\begin{array}{@{}l@{}}
   u=u(x,y) \\ v=v(x,y) \\ u_0=u(x_0,y_0) \\ v_0=v(x_0,y_0)
\end{array}\right. \]\pause
而且隐函数也有连续偏导数\vspace{0.3em}
\begin{align*}
  \frac{\partial u}{\partial x} &= -\frac1J\,\frac{\partial(F,G)}{\partial(x,v)} &
  \frac{\partial v}{\partial x} &= -\frac1J\,\frac{\partial(F,G)}{\partial(u,x)} \\
  \frac{\partial u}{\partial y} &= -\frac1J\,\frac{\partial(F,G)}{\partial(y,v)} &
  \frac{\partial v}{\partial y} &= -\frac1J\,\frac{\partial(F,G)}{\partial(u,y)}
\end{align*}%
\end{theorem*}
\end{iframe}

\begin{frame}
\frametitle{方程组确定的隐函数}
\begin{example}
设$xu-yv=0$,$yu+xv=1$,求偏导数
$$\dfrac{\partial u}{\partial x},\quad \dfrac{\partial u}{\partial y},\quad
  \dfrac{\partial v}{\partial x},\quad \dfrac{\partial v}{\partial y}.$$
\end{example}
\end{frame}

\mybookmark{复习与提高}

\begin{frame}
\frametitle{复习与提高}
\begin{review}
\begin{enumlite}
  \item 设方程$xy+\ln y-\ln x=0$确定了隐函数$y=f(x)$,求导数$\dfrac{\dy}{\dx}$。
  \item 设方程$\dfrac{x}z=\ln z-\ln y$确定了隐函数$z=f(x,y)$,求偏导数
        $\dfrac{\partial z}{\partial x}$和$\dfrac{\partial z}{\partial y}$。
\end{enumlite}
\end{review}
\end{frame}

\begin{frame}
\frametitle{复习与提高}
\begin{puzzle}
设$F(u,v)$有连续偏导数,由方程$F\left(\frac{x}z,\frac{y}z\right)=0$求全微分$\dz$.
\end{puzzle}
\end{frame}

\begin{frame}
\frametitle{复习与提高}
\begin{choice}%[2006]
设有三元方程$xy-z\ln y+\e^{xz}=1$,根据隐函数存在定理,存在点$(0,1,1)$的一个邻域,
在此邻域内该方程所确定的具有连续偏导数的隐函数\dotfill(\select{D})
\begin{choiceline}
  \item 只有$z=z(x,y)$这一个
  \item 只有$y=y(x,z)$和$z=z(x,y)$这两个
  \item 只有$x=x(y,z)$和$z=z(x,y)$这两个
  \item 只有$x=x(y,z)$和$y=y(x,z)$这两个
\end{choiceline} 
\end{choice}
\end{frame}

\ifligong % >>>>>>>>>>>>>>>>>>>>>>>>>>>>>>>>>>>>>>>>>>>>>>>>>>>>>>>>>>>>>>>>>>>>

\section{多元函数微分学的几何应用}

\subsection{向量值函数及其导数}

\begin{frame}
\frametitle{函数与向量值函数}
\begin{tabularx}{\textwidth}{|X|X|}
  \hline
  \bold{(一元)\kern0pt 函数}\par\vspace{0.5em}
  $\begin{aligned}f:\mathbb{R}&\longrightarrow\mathbb{R}\\[-0.5em]
                    x&\longmapsto y\end{aligned}$ &
  \bold{多元函数}\par\vspace{0.5em}
  $\begin{aligned}f:\mathbb{R}^3&\longrightarrow\mathbb{R}\\[-0.5em]
                    (x,y,z)&\longmapsto u\end{aligned}$ \\
  \hline
  \bold{(一元)\kern0pt 向量值函数}\par\vspace{0.5em}
  $\begin{aligned}\vec{f}:\mathbb{R}&\longrightarrow\mathbb{R}^3\\[-0.5em]
                    t&\longmapsto (x,y,z)\end{aligned}$&
  \bold{多元向量值函数}\par\vspace{0.5em}
  $\begin{aligned}\vec{f}:\mathbb{R}^3&\longrightarrow\mathbb{R}^3\\[-0.5em]
                    (r,s,t)&\longmapsto (x,y,z)\end{aligned}$ \\
  \hline
\end{tabularx}
\end{frame}

\begin{frame}
\begin{definition}
设数集$D\subset \mathbb{R}$.则称映射$\vec{f}:D\rightarrow\mathbb{R}^n$为\bold{向量值函数},通常记为
\[ \vec{r}=\vec{f}(t),\;t\in D. \]
其中数集$D$称为函数的定义域,$t$称为自变量,$\vec{r}$称为因变量.
\end{definition}
\vpause
\begin{remark*}
普通的实值函数也称为\bold{数量函数}.
\end{remark*}
\end{frame}

\begin{frame}
\begin{example*}
在$\mathbb{R}^3$中,向量值函数$\vec{f}(t)$可表示为
\[ \bold{\vec{f}(t) = f_1(t)\,\vec{i} + f_2(t)\,\vec{j} + f_3(t)\,\vec{k},\quad t \in D} \]
或者
\[ \bold{\vec{f}(t) = \big( f_1(t), f_2(t), f_3(t) \big),\quad t \in D} \]
\end{example*}
\vpause
\begin{remark*}
$\mathbb{R}^3$中的向量值函数与空间曲线一一对应.
\end{remark*}
\end{frame}

\begin{frame}
\begin{definition}
设向量值函数$\vec{f}(t)$在点$t_0$的某一去心邻域有定义.如果存在一个常向量$\vec{r}_0$,
对于任意给定的$\epsilon>0$,总存在$\delta>0$,使得当$0<|t-t_0|<\epsilon$时,
对应的函数值$\vec{f}(t_0)$都满足不等式
\[ |\vec{f}(t)-\vec{r}_0|<\epsilon, \]
那么,常向量$\vec{r}_0$就叫做向量值函数$\vec{f}(t)$当$t\to t_0$时的\bold{极限},记为
\[ \bold{\lim_{t\to t_0}\vec{f}(t)=\vec{r}_0}, \text{\quad 或者\quad} \bold{\vec{f}(t)\to\vec{r}_0,\,t\to t_0}. \]
\end{definition}
\end{frame}

\begin{frame}
\begin{example}
设$\vec{f}(t)=(\cos t)\,\vec{i}+(\sin t)\,\vec{j}+t\,\vec{k}$,求$\limit_{t\to\pi/4}\vec{f}(t)$.
\end{example}
\end{frame}

\begin{frame}
\begin{definition}
设向量值函数$\vec{r}=\vec{f}(t)$在点$t_0$的某一邻域内有定义,如果
\[ \lim_{\Delta t\to 0}\frac{\Delta \vec{r}}{\Delta t}
  =\lim_{\Delta t\to 0}\frac{\vec{f}(t_0+\Delta t)-\vec{f}(t_0)}{\Delta t} \]
存在,那么就称这个极限向量为向量值函数$\vec{r}=\vec{f}(t)$在$t_0$处的\bold{导数}或\bold{导向量},
记为$\bold{\vec{f}'(t_0)}$或$\bold{\left.\dfrac{\d\vec{r}}{\dt}\right|_{t=t_0}}$.
\end{definition}
\vpause
\begin{remark*}
向量值函数的导向量是所对应曲线$\Gamma$的\bold{切向量}.
\end{remark*}
\end{frame}

\begin{frame}
\begin{example}
设空间曲线$\Gamma$的向量方程为
\[ \vec{r}=\vec{f}(t)=(t^2+1,4t-3,2t^2-6t),\quad t\in\mathbb{R}. \]
求曲线$\Gamma$在与$t=2$相应的点处的单位切向量.
\end{example}
\end{frame}

\begin{iframe}
\frametitle{向量值函数}
\begin{example}
一个人在悬挂式滑翔机上由于快速上升气流而沿位置向量为
\[ \vec{r}=\vec{f}(t)=(3\cos t)\,\vec{i}+(3\sin t)\,\vec{j}+t^2\,\vec{k}\]
的路径螺旋式向上.求
\begin{enumlite}
  \item 滑翔机在任意时刻$t$的速度向量和加速度向量.
  \item 滑翔机在任意时刻$t$的速率.
  \item 滑翔机的加速度与速度正交的时刻.
\end{enumlite}
\end{example}
\end{iframe}

\subsection{空间曲线的切线与法平面}

\begin{frame}
\frametitle{空间曲线的切线与法平面}
\vspace{-0.6em}%
设空间曲线$\Gamma$的参数方程为$\bold{\left\{\begin{array}{l}
  x = \phi(t), \\
  y = \psi(t), \\
  z = \omega(t),
\end{array}\right.}\ t\in[\alpha,\beta]$.\par
则曲线在点$(x_0,y_0,z_0)=(\phi(t_0),\psi(t_0),\omega(t_0))$处有
\begin{description}
  \item[切向量] $\overrightarrow{T}=(T_1,T_2,T_3)=\bold{\big(\phi'(t_0),\psi'(t_0),\omega'(t_0)\big)}$\vspace{0.8em}
  \item[切线\quad] $\dfrac{x-x_0}{T_1}=\dfrac{y-y_0}{T_2}=\dfrac{z-z_0}{T_3}$\vspace{0.8em}
  \item[法平面] $T_1(x-x_0)+T_2(y-y_0)+T_3(z-z_0)=0$
\end{description}
\end{frame}

\begin{frame}
\frametitle{空间曲线的切线方程}
\begin{columns}[onlytextwidth]
\column{0.5\textwidth}
\onslide<2->{用点向式写出\CJKunderdot{割线方程}:
\[ \frac{x-x_0}{\Delta x}=\frac{y-y_0}{\Delta y}=\frac{z-z_0}{\Delta z} \]}
\onslide<3->{\vskip0pt plus 0.5fill 分母同时除以$\Delta t$得到:
\[ \frac{x-x_0}{\Delta x/\Delta t}=\frac{y-y_0}{\Delta y/\Delta t}=\frac{z-z_0}{\Delta z/\Delta t} \]}
\onslide<4->{\vskip0pt plus 0.5fill 令$\Delta t\to0$得到\CJKunderdot{切线方程}:
\[ \bold{\frac{x-x_0}{\phi'(t_0)} = \frac{y-y_0}{\psi'(t_0)} = \frac{z-z_0}{\omega'(t_0)}} \]}
\column{0.44\textwidth}
\hfill
\begin{tikzpicture}[thick,inner sep=3pt]
\draw[thin,-stealth] (0,0) -- (4,0) node[below] {$y$};
\draw[thin,-stealth] (0,0) -- (0,4) node[left] {$z$};
\draw[thin,-stealth] (0,0) -- (-1,-1) node[right] {$x$};
\draw[color=accent1] (-0.8,2.2) .. controls (0,2.2) and (0,1) .. (1,1) to[bend right=45] (4,4);
\draw[color=accent3] (0.25,0.75) -- (1,1) -- (2.8,1.6) -- (4,2);
\fill (1,1) node[below right] {$M_0$}circle (1.2pt) (2.8,1.6) node[below right] {$M_1$} circle (1.2pt);
\end{tikzpicture}
\begin{align*}
  t_0 &\longmapsto M_0(x_0,y_0,z_0) \\[-0.3em]
  \Delta t &\longmapsto (\Delta x,\Delta y,\Delta z)
\end{align*}
\end{columns}
\end{frame}

\begin{frame}
\frametitle{空间曲线的切线与法平面}
\begin{example}
求曲线$x=t$, $y=t^2$, $z=t^3$在点$(1,1,1)$处的切线及法平面方程.
\end{example}
\end{frame}

\begin{frame}
\frametitle{空间曲线的切线与法平面}
设空间曲线$\Gamma$的显式方程为$\bold{\left\{\begin{array}{l}
  y = \psi(x), \\
  z = \omega(x),
\end{array}\right.}\ x\in[a,b]$.\par
则曲线在点$(x_0,y_0,z_0)=(x_0,\psi(x_0),\omega(x_0))$处有
\begin{description}
  \item[切向量] $\overrightarrow{T}=(T_1,T_2,T_3)=\bold{\big(1,\psi'(x_0),\omega'(x_0)\big)}$\vspace{1em}
  \item[切线\quad] $\dfrac{x-x_0}{T_1}=\dfrac{y-y_0}{T_2}=\dfrac{z-z_0}{T_3}$\vspace{1em}
  \item[法平面] $T_1(x-x_0)+T_2(y-y_0)+T_3(z-z_0)=0$
\end{description}
\end{frame}

\begin{iframe}
\frametitle{空间曲线的切线与法平面}
\vspace{-0.6em}%
设空间曲线$\Gamma$的隐式方程为$\bold{\left\{\begin{array}{l}
  F(x,y,z) = 0, \\
  G(x,y,z) = 0.
\end{array}\right.}$\par\vspace{-0.3em}
则曲线在点$(x_0,y_0,z_0)$处有
\begin{description}
  \item[切向量] $\begin{aligned}[t]&\overrightarrow{T}=(T_1,T_2,T_3)\\[-0.3em]&=\bold{\left.\left(
          \begin{vmatrix}F'_y & F'_z \\ G'_y & G'_z\end{vmatrix},
          \begin{vmatrix}F'_z & F'_x \\ G'_z & G'_x\end{vmatrix},
          \begin{vmatrix}F'_x & F'_y \\ G'_x & G'_y\end{vmatrix}
        \right)\right|_{(x_0,y_0,z_0)}}\end{aligned}$\vspace{0.2em}
  \item[切线\quad] $\dfrac{x-x_0}{T_1}=\dfrac{y-y_0}{T_2}=\dfrac{z-z_0}{T_3}$\vspace{0.3em}
  \item[法平面] $T_1(x-x_0)+T_2(y-y_0)+T_3(z-z_0)=0$
\end{description}
\end{iframe}

\begin{iframe}
\frametitle{空间曲线的切线与法平面}
\begin{example}
求曲线$x^2+y^2+z^2=6$, $x+y+z=0$在点$(1,-2,1)$处的切线及法平面方程.
\end{example}
\end{iframe}

\begin{jframe}
\frametitle{空间曲线的切线与法平面}
\begin{solution}
两曲面在点$(1,-2,1)$处的法向量分别为
\begin{align*}
  \vn_1 &= (2x,2y,2z)\big|_{(1,-1,2)}=(2,-4,2) \\
  \vn_2 &= (1,1,1)
\end{align*}
因此,曲线同时在两平面上,因此曲线有切向量为
\[ \ov{T} = \vn_1\times\vn_2 = (-6,0,6) \text{\ 或者\ } \ov{T} = (1,0,-1) \]
切线方程为
$\dfrac{x-1}{1}=\dfrac{y+2}{0}=\dfrac{z-1}{-1}$, 即$\begin{cases}
  x+z=2, \\ y=-2.
\end{cases}$\par
法平面方程为$(x-1)-(z-1)=0$,即$x-z=0$.
\end{solution}
\end{jframe}

\subsection{空间曲面的法线与切平面}

\begin{frame}
\frametitle{空间曲面的法线与切平面}
设空间曲面$\Sigma$的隐式方程为$\bold{F(x,y,z)=0}$.\par
则曲面在点$(x_0,y_0,z_0)$处有
\begin{description}
  \item[法向量] $\vec{n}=(n_1,n_2,n_3)=\bold{\left.\left(F'_x,\;F'_y,\;F'_z\right)\right|_{(x_0,y_0,z_0)}}$\vspace{1em}
  \item[法线\quad] $\dfrac{x-x_0}{n_1}=\dfrac{y-y_0}{n_2}=\dfrac{z-z_0}{n_3}$\vspace{1em}
  \item[切平面] $n_1(x-x_0)+n_2(y-y_0)+n_3(z-z_0)=0$
\end{description}
\end{frame}

\begin{frame}
\frametitle{空间曲面的法线与切平面}
在曲面$\Sigma$上任取一条过$M_0(x_0,y_0,z_0)$的曲线$\Gamma$.
\[ \Gamma:\quad (x,y,z)=\big(\phi(t),\psi(t),\omega(t)\big) \]
\begin{columns}[onlytextwidth]
\column{0.6\textwidth}
\begin{tikzpicture}[x={(170:1cm)},y={(55:.7cm)},z={(90:1cm)},thick,font=\small,inner sep=3pt]
%%coordinates
%\draw[->,thin,gray4] (0,0,0) -- (3,0,0) node[left] {$x$};
%\draw[->,thin,gray4] (0,0,0) -- (0,3,0) node[above right] {$y$};
%\draw[->,thin,gray4] (0,0,0) -- (0,0,3) node[right] {$z$};
%surface S
\draw[thick,looseness=.7] (2.5,-2.5,-1) node[above right=3pt and 3pt] {$\Sigma:\,F(x,y,z)=0$}
  to[bend left]  (2.5,2.5,-1)   to[bend left]  (-2.5,2.5,-1)
  to[bend right] (-2.5,-2.5,-1) to[bend right] (2.5,-2.5,-1) -- cycle;
%curve on S
\visible<2->{
\draw[very thick,accent3,looseness=.4] (1,-1,-0.5) coordinate (c1) to[bend left]
     (0,0,0) to[bend left] (-1.2,1.2,-0.5) ;
\node[above] at (c1) {$\Gamma$};
}
%normal line through point M
\visible<6->{\draw[densely dashed] (0,0,-2.5)--(0,0,0);}
%tangent plane
\visible<5->{
\draw[accent1,fill=filler1,opacity=0.7] (2.5,-2.5,0) -- (2.5,2.5,0) -- (-2.5,2.5,0) -- (-2.5,-2.5,0) -- cycle;
\draw[<-] (2.4,2.6,0) to[bend right=20] (3.2,3,0) node[left]{切平面};
}
%normal line through point M
\visible<6->{\draw (0,0,0)--(0,0,2.3) node[right]{法线} --(0,0,2.5);}
%tangent vector at point M
\visible<3->{\draw[-latex,accent1](0,0,0)--(-0.8,0.7,0) node[right]{$\ov{T}$切向量};}
%normal vector at point M
\visible<4->{\draw[-latex,accent1](0,0,0)--(0,0,1) node[below left] {法向量$\vn$};}
%point M on surface
\draw[fill] (0,0,0) circle (1pt) node[below right] {$M_0$};
\end{tikzpicture}
\column{0.4\textwidth}
\bold{通过$M_0$的任何曲线,它们在$M_0$点的切线都在同一个平面上.}
\end{columns}
\end{frame}

\begin{frame}
\frametitle{空间曲面的法线与切平面}
\begin{example}
求球面$x^2+y^2+z^2=14$在点$(1,2,3)$处的法线和切平面方程.
\end{example}
\end{frame}

\begin{frame}
\frametitle{空间曲面的法线与切平面}
设空间曲面$\Sigma$的显式方程为$\bold{z=f(x,y)}$.\par
则曲面在点$(x_0,y_0,z_0)=(x_0,y_0,f(x_0,y_0))$处有
\begin{description}
  \item[法向量] $\vec{n}=(n_1,n_2,n_3)=\bold{\left.\left(f'_x,\;f'_y,\;-1\right)\right|_{(x_0,y_0)}}$\vspace{1em}
  \item[法线\quad] $\dfrac{x-x_0}{n_1}=\dfrac{y-y_0}{n_2}=\dfrac{z-z_0}{n_3}$\vspace{1em}
  \item[切平面] $n_1(x-x_0)+n_2(y-y_0)+n_3(z-z_0)=0$
\end{description}
\end{frame}

\begin{frame}
\frametitle{空间曲面的法线与切平面}
\begin{example}
求旋转抛物面$z=x^2+y^2-1$在点$(2,1,4)$处的法线和切平面方程.
\end{example}
\end{frame}

\mybookmark{复习与提高}

\begin{frame}
\frametitle{复习与提高}
\begin{puzzle}
求螺旋线$x=a\cos\phi$, $y=a\sin\phi$, $z=b\phi$在
$\phi=\pi/2$对应点处的切线方程和法平面方程.
\end{puzzle}
\end{frame}

\begin{frame}
\frametitle{复习与提高}
\begin{puzzle}
已知曲面$xyz=\sigma$与球面$x^2+y^2+z^2=a^2$在点$M(x_0,y_0,z_0)$处相切,求正数$\sigma$.
\end{puzzle}
\vpause
\begin{puzzle}
已知平面$3x+\lambda y-3z+16=0$与椭球面$3x^2+y^2+z^2=16$相切,求$\lambda$.
\pause\cdotfill$\pm2$
\end{puzzle}
\end{frame}

\begin{frame}
\frametitle{复习与提高}
\begin{puzzle}
证明曲面$z=xf(y/x)$上任一点处的切平面都通过原点,其中$f(u)$可微.
\end{puzzle}
\vpause
\begin{puzzle}
证明曲面$F(x-my,z-ny)=0$的所有切平面恒与定直线平行, 其中$F(u,v)$可微.
\end{puzzle}
\end{frame}

\section{方向导数与梯度}

\subsection{方向导数}

%\begin{rframe}
%\frametitle{偏导数与方向导数}
%\begin{description}
%  \item[偏导数]   反映函数沿坐标轴的变化率%(四个方向)
%  \item[方向导数] 反映函数沿任一方向的变化率
%\end{description}
%\end{rframe}

\begin{frame}
\begin{origin*}
有一只蚂蚁在铁盘上某点$P_0(x_0,y_0)$处,铁盘上任意点的温度为$z=f(x,y)$.
问这只蚂蚁应沿什么方向爬行,才能最快到达较凉快的地点?
\end{origin*}
\pause\vspace{0.5em}\hrule
\begin{itemize}
  \item 本质在于求出温度由高到低变化最剧烈的方向.\pause
  \item 需要先研究温度函数沿任意方向的变化率问题.
\end{itemize}
\pause\hrule\vspace{0.5em}
\begin{columns}[onlytextwidth]
\column{0.58\textwidth}
温度函数$f(x,y)$在$(x_0,y_0)$点沿方向$l$的变化率为\vspace{0.5em}%
\[\left.\frac{\partial f}{\partial l}\right|_{(x_0,y_0)}=\limit_{\Delta P\to0^+}\frac{\Delta z}{\Delta P}\]%
\vskip0.5em 称为函数沿方向$l$的\bold{方向导数}.
\column{0.4\textwidth}
\hfill
\begin{tikzpicture}[thick,inner sep=0.2em]
 \draw[-stealth,thin] (-0.5,-0.5) -- (3.8,-0.5);
 \draw[-stealth,thin] (-0.5,-0.5) -- (-0.5,3.3);
 \coordinate (P0) at (0.7,0.7);
 \coordinate (P1) at (2.0,2.0);
 \coordinate (P2) at (3.0,3.0);
 \coordinate (Q0) at (2.0,0.7);
 \coordinate (Q1) at (0.7,3.0);
 \draw[-stealth,color=accent1] (P0) -- node[pos=0.7,color=text1,left]{$\Delta P$} (P1) -- (P2) node[right]{$l$};
 \draw (P0) -- node[pos=0.5,below]{$\Delta x$} (Q0) -- node[pos=0.4,right]{$\Delta y$} (P1);
 \path[fill=accent3] (P0) circle (1.5pt) node[left] {$P_0$}
                     (P1) circle (1.5pt) node[right,inner sep=0.3em] {$P(x,y)$};
\end{tikzpicture}
\end{columns}
\end{frame}

\begin{frame}
\begin{columns}[onlytextwidth]
\column{0.55\textwidth}
设方向$l$与两个坐标轴的夹角分别为$\alpha$和$\beta$,则有
\begin{align*}
  \Delta x&=\Delta P\cdot\cos\alpha \\[-0.3em]
  \Delta y&=\Delta P\cdot\cos\beta
\end{align*}%
\onslide<2->{故函数沿方向$l$的\bold{方向导数}为\par\vspace{0.5em}%
$\displaystyle\phantom{={}}\left.\frac{\partial f}{\partial l}\right|_{(x_0,y_0)}
 =\limit_{\Delta P\to0^+}\frac{\Delta z}{\Delta P}$}
\column{0.4\textwidth}
\hfill
\begin{tikzpicture}[thick,scale=1.5,inner sep=0.2em]
 \draw[-stealth,thin] (0,0) -- (3,0);
 \draw[-stealth,thin] (0,0) -- (0,3);
 \coordinate (P0) at (0.7,0.7);
 \coordinate (P1) at (2.0,2.0);
 \coordinate (P2) at (2.9,2.9);
 \coordinate (Q0) at (2.0,0.7);
 \coordinate (Q1) at (0.7,3.0);
 \draw[densely dashed] (P0) -- +(2.2,0) (P0) -- +(0,2);
 \draw[-stealth,color=accent1] (P0) -- node[pos=0.7,color=text1,left]{$\Delta P$} (P1) -- (P2) node[right]{$l$};
 \draw (P0) -- node[pos=0.5,below]{$\Delta x$} (Q0) -- node[pos=0.5,right]{$\Delta y$} (P1);
 \path[fill=accent3] (P0) circle (1.5pt) node[left] {$P_0$}
                     (P1) circle (1.5pt) node[right,inner sep=0.3em] {$P(x,y)$};
 \path (P0) +(22.5:0.4) node[color=accent2] {$\alpha$};          
 \path (P0) +(67.5:0.4) node[color=accent2] {$\beta$};
\end{tikzpicture}
\end{columns}\pause[3]
$\!\!\begin{aligned}[t]
  &=\limit_{\Delta P\to0^+}\!\frac{f(x_0+\Delta x,y_0+\Delta y)-f(x_0,y_0)}{\Delta P} \\
  &=\limit_{\Delta P\to0^+}\!\frac{f(x_0+\Delta P\cos\alpha,y_0+\Delta P\cos\beta)-f(x_0,y_0)}{\Delta P}
\end{aligned}$
\end{frame}

\begin{frame}
\begin{remark*}
\bold{单位向量} $\ve_l=(\cos\alpha,\cos\beta)$与方向$l$同方向.
即$\cos\alpha$和$\cos\beta$是方向$l$的两个\bold{方向余弦}.
\end{remark*}
\pause\cdotfill
\begin{theorem}
若函数$f(x,y)$在点$P_0(x_0,y_0)$可微分,那么它在该点沿任一方向$l$的方向导数存在,且有
\[ \left.\frac{\partial f}{\partial l}\right|_{(x_0,y_0)}
  =f'_x(x_0,y_0)\cos\alpha + f'_y(x_0,y_0)\cos\beta. \]
其中$\ve_l=(\cos\alpha,\cos\beta)$是与$l$同方向的单位向量.
\end{theorem}
\pause\cdotfill
\begin{example}
求函数$z=x\e^{2y}$在点$P(1,0)$处沿从点$P(1,0)$到点$Q(2,-1)$的方向导数.
\end{example}
\end{frame}

\begin{sframe}
\begin{proof}
因为函数在点$P_0(x_0,y_0)$可微分,所以\vspace{0.4em}
\begin{align*}
&\phantom{={}}\limit_{\Delta P\to0^+}\frac{\Delta z}{\Delta P}
 =\limit_{\Delta P\to0^+}\frac{\dz+o(\Delta P)}{\Delta P} \\[0.3em]
&=\limit_{\Delta P\to0^+}\frac{\dz}{\Delta P}+\limit_{\Delta P\to0^+}\frac{o(\Delta P)}{\Delta P} \\[0.3em]
&=\limit_{\Delta P\to0^+}\frac{f'_x(x_0,y_0)\bold{\Delta x}+f'_y(x_0,y_0)\bold{\Delta y}}{\bold{\Delta P}} \\[0.3em]
&=\limit_{\Delta P\to0^+}\frac{f'_x(x_0,y_0)\,\bold{\Delta P\cos\alpha}
                              +f'_y(x_0,y_0)\,\bold{\Delta P\cos\beta}}{\bold{\Delta P}} \\[0.3em]
&=f'_x(x_0,y_0)\cos\alpha + f'_y(x_0,y_0)\cos\beta
\end{align*}
\end{proof}
\end{sframe}

\begin{frame}
\frametitle{三元函数的方向导数}
\begin{remark*}
在二元函数方向导数的定义中,令$t=\Delta P$,则\par
\mylinebox{%
  $\displaystyle\left.\frac{\partial f}{\partial l}\right|_{(x_0,y_0)}\!\!\!
  =\limit_{t\to 0^+}\frac{f(x_0+t\cos\alpha,y_0+t\cos\beta)-f(x_0,y_0)}{t}$}
\end{remark*}
\vpause
\begin{definition*}
对三元函数$f(x,y,z)$,它在点$P_0(x_0,y_0,z_0)$沿方向$\vec{e}_l=(\cos\alpha,\cos\beta,\cos\gamma)$
的方向导数为
\[ \left.\frac{\partial f}{\partial l}\right|_{(x_0,y_0,z_0)}
  =\limit_{t\to 0^+}\frac{f(x_1,y_1,z_1)-f(x_0,y_0,z_0)}{t}, \]
$(x_1,y_1,z_1)=(x_0+t\cos\alpha,y_0+t\cos\beta,z_0+\cos\gamma)$.
\end{definition*}
\end{frame}

\begin{frame}
\frametitle{三元函数的方向导数}
\begin{theorem}
若函数$f(x,y,z)$在点$P_0(x_0,y_0,z_0)$可微分,那么它在该点沿任一方向
$\vec{e}_l=(\cos\alpha,\cos\beta,\cos\gamma)$的方向导数存在,且有
\begin{align*}
  \left.\frac{\partial f}{\partial l}\right|_{P_0}
    = f'_x(x_0,y_0,z_0)\cos\alpha 
    &+ f'_y(x_0,y_0,z_0)\cos\beta \\[-0.7em]
    &+ f'_z(x_0,y_0,z_0)\cos\gamma.
\end{align*}
\end{theorem}
\vpause\cdotfill
\begin{example}
求函数$f(x,y,z)=xy+yz+zx$在点$(1,1,2)$沿方向$l$的方向导数,
其中$l$的方向角分别为$60^\circ$, $45^\circ$, $60^\circ$.
\end{example}
\end{frame}

\subsection{梯度}

\begin{frame}
\begin{origin*}
有一只蚂蚁在铁盘上某点$P_0(x_0,y_0)$处,铁盘上任意点的温度为$z=f(x,y)$.
问这只蚂蚁应沿什么方向爬行,才能最快到达较凉快的地点?
\end{origin*}
\pause\vspace{0.5em}\hrule
\begin{itemize}
  \item 已经研究好温度函数沿任意方向的变化率问题.\pause
  \item 现在要寻找出温度由高到低变化最剧烈的方向.
\end{itemize}
\pause\hrule\vspace{0.5em}
\begin{align*}
\left.\frac{\partial f}{\partial l}\right|_{(x_0,y_0)}
  &=f'_x(x_0,y_0)\cos\alpha + f'_y(x_0,y_0)\cos\beta \\
  &=\big(f'_x(x_0,y_0),f'_y(x_0,y_0)\big)\cdot\big(\cos\alpha,\cos\beta\big) \\
  &=\big|\big(f'_x(x_0,y_0),f'_y(x_0,y_0)\big)\big|\cdot\cos\theta
\end{align*}
其中$\theta$是向量$(f'_x(x_0,y_0),f'_y(x_0,y_0)\big)$与$l$的夹角。
\end{frame}

\begin{frame}
\begin{definition*}
二元函数$f(x,y)$在点$P_0(x_0,y_0)$处的\bold{梯度}为
\begin{align*}
  \bold{\grad f(x_0,y_0)} &= \left(f'_x(x_0,y_0),\,f'_y(x_0,y_0)\right) \\
  &= f'_x(x_0,y_0)\,\vi + f'_y(x_0,y_0)\,\vj
\end{align*}
或者记为$\bold{\nabla f(x_0,y_0)}$.
\end{definition*}
\vpause
\begin{remark*}
梯度与方向导数有如下关系:
\begin{enumerate}
  \item 当方向向量与梯度方向相同时,函数增加最快
  \item 当方向向量与梯度方向相反时,函数减少最快
  \item 当方向向量与梯度方向垂直时,函数变化率为零
\end{enumerate}
\end{remark*}
\end{frame}

\begin{frame}
\frametitle{二元函数的梯度}
\begin{example*}
有一只蚂蚁在铁盘上点$(1,1)$处,\vspace{0.3em}%
铁盘上任意点的温度为$f(x,y)=\dfrac{240}{x^2+\bold{2}y^2+1}$.\vspace{0.3em}%
问这只蚂蚁应沿什么方向爬行,才能最快到达较凉快的地点?
\end{example*}
\pause
\begin{solution}
$\grad f(x,y)=\left(-\frac{480x}{(x^2+2y^2+1)^2},-\frac{960y}{(x^2+2y^2+1)^2}\right)$.\ppause
蚂蚁的逃生路线应与梯度向量相切,\pause 解微分方程%\let\displaystyle=\textstyle
\[ \frac{\dy}{\dx}=\frac{\bold{2}y}{x},\qquad y\big|_{x=1}=1, \]
\pause 得曲线方程$y=x^{\bold{2}}$.\pause 沿远离原点方向温度下降最快.
\end{solution}
\end{frame}

\begin{frame}
%\frametitle{等值线}
\begin{tikzpicture}[thick,scale=1.7,font=\small,inner sep=0.3pt]
  \draw[-stealth,thin] (0,0) node[below left]{$O$} -- (3.6,0) node[below=3pt]{$x$};
  \draw[-stealth,thin] (0,0) -- (0,2.5) node[left=2pt]{$y$};
  \draw[color=accent3] (0,0) ellipse ({sqrt(16/8)} and {sqrt(16/16)});
  \draw[color=accent3!50!accent4] (0,0) ellipse ({sqrt(17/7)} and {sqrt(17/14)});
  \draw[color=accent4] (0,0) ellipse ({sqrt(18/6)} and {sqrt(18/12)});
  \node[fill=back1] at (-10:{sqrt(18/6)} ) {$60$};
  \draw[color=accent4!50!accent2] (0,0) ellipse ({sqrt(19/5)} and {sqrt(19/10)});
  \draw[color=accent2] (0,0) ellipse ({sqrt(20/4)} and {sqrt(20/8)});
  \node[fill=back1] at (-10:{sqrt(20/4)} ) {$40$};
  \draw[color=accent2!50!accent1] (0,0) ellipse ({sqrt(21/3)} and {sqrt(21/6)});
  \draw[color=accent1] (0,0) ellipse ({sqrt(22/2)} and {sqrt(22/4)});
  \node[fill=back1] at (-10:{sqrt(22/2)} ) {$20$};
  \draw[densely dashed,domain=-1.5:1.5,smooth] plot (\x,{(\x)^2});
  \draw[densely dashed,domain=-2.5:2.5,smooth,thin] plot (\x,{-0.3*(\x)^2});
  \draw (1.4,0.8) -- (0.6,1.2); 
  \draw[-stealth,accent3] (1,1) -- +(0.25,0.5);
  \draw[-stealth,accent1] (1,1) -- +(-0.25,-0.5);
  \path[draw=text1,fill=accent3] (1,1) circle (0.8pt);
  \node[right,font=\normalsize,accent1] at (-3.3,2.5) {等值线$f(x,y)=c$};
  \node[left,font=\normalsize] at (3.6,2.5) {梯度曲线$y=x^2$};
\end{tikzpicture}
\end{frame}

\begin{rframe}
\begin{property*}
过点$(x_0,y_{0})$处的梯度$\grad f(x_0,y_0)$垂直于过该点的等值线.
即梯度向量是等值线的法向量.
\end{property*}
\begin{proof}
等值线$f(x,y)=c$的切线斜率为$\displaystyle\wfrac[0.2em]{\dy}{\dx}=-\wfrac[0.4em]{f'_x}{f'_y}$,
而梯度向量$\big(f'_x,\;f'_y\big)$的斜率为$\displaystyle\smash[t]{\wfrac[0.4em]{f'_y}{f'_x}}$.
两个斜率乘积为$-1$,从而两者相互垂直.
\end{proof}
%\begin{proof}
%设该等值线参数方程为$(x(t),y(t))$,且$t=t_0$时对应$(x_0,y_0)$点.
%由$f\big(x(t),y(t)\big)\equiv c$得
%\begin{align*}
%0&=\left.\frac{\d}{\dt}\;f\big(x(t),y(t)\big)\right|_{t=t_0} \\
% &=f'_{x}(x_0,y_0)\cdot x'(t_0)+f'_{y}(x_0,y_0)\cdot y'(t_0) \\
% &=\grad f(x_0,y_0)\cdot\big(x'(t_0),y'(t_0)\big)
%\end{align*}\vskip-\baselineskip
%\end{proof}
\end{rframe}

%\begin{sframe}
%\frametitle{二元函数的梯度}
%\begin{example}
%求梯度$\grad\dfrac1{x^2+y^2}$.
%\end{example}
%\vpause
%\begin{example}
%设$f(x,y)=\frac12(x^2+y^2)$,$P_0(1,1)$,求
%\begin{enumlite}
%  \item $f(x,y)$在$P_0$处增加最快的方向以及$f(x,y)$在这个方向的方向导数;
%  \item $f(x,y)$在$P_0$处减少最快的方向以及$f(x,y)$在这个方向的方向导数;
%  \item $f(x,y)$在$P_0$处的变化率为零的方向. 
%\end{enumlite}
%\end{example}
%\end{sframe}

\begin{frame}
\frametitle{三元函数的梯度}
\begin{definition*}
函数$f(x,y,z)$在点$P_0(x_0,y_0,z_0)$处的\bold{梯度}为
\begin{align*}
   &\bold{\grad f(x_0,y_0,z_0) = \nabla f(x_0,y_0,z_0)} \\
={}&\left(f'_x(x_0,y_0,z_0),\,f'_y(x_0,y_0,z_0),\,f'_z(x_0,y_0,z_0)\right) \\
={}& f'_x(x_0,y_0,z_0)\,\vi 
   + f'_y(x_0,y_0,z_0)\,\vj
   + f'_z(x_0,y_0,z_0)\,\vk.
\end{align*}
\end{definition*}
\vpause
\begin{example}
设$f(x,y,z)=x^3-xy^2-z$,$P_0(1,1,0)$,问$f(x,y,z)$在$P_0$处沿什么方向变化最快,
在这个方向的变化率是多少?
\end{example}
%\vpause
%\begin{example}
%求曲面$x^2+y^2+z=9$在点$P_0(1,2,4)$的切平面和法线方程.
%\end{example}
\end{frame}

\begin{frame}
%\frametitle{数量场与向量场}
%\begin{definition}
%(1) 如果对于空间区域$G$内的任一点$M$,都有一个确定的数量$f(M)$,
%那么称在这空间区域$G$内确定了一个\bold{数量场}.
%一个数量场可用一个数量函数$f(M)$来确定.\par
%(2) 如果对于空间区域$G$内的任一点$M$,都有一个确定的向量$\vec{F}(M)$,
%那么称在这空间区域$G$内确定了一个\bold{向量场}.
%一个向量场可用一个向量值函数$\vec{F}(M)$来确定,即
%\[ \vec{F}(M) = P(M)\vec{i} + Q(M)\vec{j} + R(M)\vec{k}, \]
%其中$P(M)$, $Q(M)$, $R(M)$ 是点$M$的数量函数.
%\end{definition}
\begin{definition}
(1) 空间区域上的数量函数$f(M)$称为\bold{数量场}.\ppause
(2) 空间区域上的向量值函数$\vec{F}(M)$称为\bold{向量场}.即有
\[ \vec{F}(M) = P(M)\;\vec{i} + Q(M)\;\vec{j} + R(M)\;\vec{k}, \]
其中$P(M)$, $Q(M)$, $R(M)$ 是点$M$的数量函数.
\end{definition}
\vpause
\begin{remark*}
由数量函数$f(M)$产生的梯度$\grad f(M)$是一个向量场.
\end{remark*}
\vpause
\begin{example}
求数量场$\dfrac{m}{r}$产生的梯度场,其中常数$m>0$,
$r=\sqrt{x^2+y^2+z^2}$为原点$O$与点$M(x,y,z)$的距离.
\end{example}
\end{frame}

\mybookmark{复习与提高}

\begin{frame}
\frametitle{复习与提高}
\begin{puzzle}%[1996]
求函数$u=\ln(x+\sqrt{y^2+z^2})$在点$A(1,0,1)$处沿点$A$指向$B(3,-2,2)$方向的方向导数.
\pause\cdotfill$\frac12$
\end{puzzle}
\vpause
\begin{puzzle}%[1992]
求函数$u=\ln(x^2+y^2+z^2)$在点$M(1,2,-2)$处的梯度$\grad u\big|_M$.
\pause\cdotfill$\big(\frac{2}{9},\frac{4}{9},-\frac{4}{9}\big)$
\end{puzzle}
\end{frame}

\begin{frame}
\frametitle{复习与提高}
\begin{puzzle}%[总习题16]
求$u=x^2+y^2+z^2$在椭球面$\dfrac{x^2}{a^2}+\dfrac{y^2}{b^2}+\dfrac{z^2}{c^2}=1$
上点$M_0(x_0,y_0,z_0)$处沿\CJKunderdot{外法线方向}的方向导数.
\end{puzzle}
\end{frame}

\begin{frame}
\frametitle{复习与提高}
\begin{puzzle}
判定下列函数在$(0,0)$点的偏导数和方向导数是否存在:
\begin{enumhalf}
  \item $f(x,y)=\sqrt{x^2+y^2}$ ~
  \item $g(x,y)=\sqrt[3]{xy}$ ~
\end{enumhalf}
\end{puzzle}
\end{frame}

\fi % <<<<<<<<<<<<<<<<<<<<<<<<<<<<<<<<<<<<<<<<<<<<<<<<<<<<<<<<<<<<<<<<<<<<<<<<<<

\makeatletter
\beamer@tocsectionnumber=7\relax % 没有 \beamer@tocsubsectionnumber,修改 subsection 计数器时不用改
\setcounter{section}{7}
\makeatother

\section{多元函数的极值}

\subsection{多元函数的极值与最值}

\begin{frame}%[shrink=7]
\frametitle{极值和极值点}
\begin{definition*}
设点$(x_0,y_0)$在$z=f(x,y)$定义域内部。\pause
\begin{enumerate}
  \item 若对$(x_0,y_0)$去心邻域中任何点$(x,y)$,总有$$f(x,y)<f(x_0,y_0),$$\pause
        称$(x_0,y_0)$为\bold{极大值点},$f(x_0,y_0)$为\bold{极大值}。\pause
  \item 若对$(x_0,y_0)$去心邻域中任何点$(x,y)$,总有$$f(x,y)>f(x_0,y_0),$$\pause
        称$(x_0,y_0)$为\bold{极小值点},$f(x_0,y_0)$为\bold{极小值}。
\end{enumerate}
\end{definition*}
\vpause\hrule
\begin{itemize}
  \item 极大值和极小值统称\bold{极值}。
  \item 极大值点和极小值点统称\bold{极值点}。
\end{itemize}
\end{frame}

\begin{frame}
\frametitle{极值和极值点}
\begin{example}
对下列各函数,判定点$(0,0)$是否为极值点:\pause
\begin{enumlite}[<+->]
  \item $f(x,y)=x^2+y^2$
  \item $f(x,y)=\sqrt{1-x^2-y^2}$
  \item $f(x,y)=xy$
\end{enumlite}
\end{example}
\end{frame}

\begin{rframe}
\frametitle{温习:二元函数与二次曲面}
(1) \ $f(x,y)=x^2+y^2$\cdotfill 椭圆抛物面\par
(2) \ $f(x,y)=\sqrt{1-x^2-y^2}$\cdotfill 上半球面\par
(3) \ $f(x,y)=xy$\cdotfill 双曲抛物面
\begin{remark*}
在平面解析几何中,将双曲线$\dfrac{x^2}{2}-\dfrac{y^2}{2}=1$
绕原点逆时针旋转$45^\circ$,双曲线的方程变成$xy=1$.\par
在空间解析几何中,将双曲抛物面$z=\dfrac{x^2}{2}-\dfrac{y^2}{2}$
绕$z$轴逆时针旋转$45^\circ$,双曲抛物面的方程变成$z=xy$.
\end{remark*}
\end{rframe}

\begin{frame}
\frametitle{极值和与驻点}
\begin{theorem}[极值的必要条件]
如果 $f(x,y)$ 在 $(x_0,y_0)$ 处取得极值,且偏导数存在,\pause 则有
\[ f'_x(x_0,y_0)=0,\qquad f'_y(x_0,y_0)=0. \]
\end{theorem}
\vpause
\begin{definition*}
使两个偏导数都为零的点称为\bold{驻点}。
\end{definition*}
\vpause
\begin{remark*}
(1) 驻点\warn{未必}都是极值点.\pause 例如$z=xy$.%在原点不是极值点.
\pause\newline
(2) 极值点\warn{未必}都是驻点.\pause 例如$z=\sqrt{x^2+y^2}$.%在原点的偏导数不存在.
\end{remark*}
\end{frame}

\begin{frame}
\begin{theorem}[极值的充分条件]
设函数$f(x,y)$在$(x_0,y_0)$某邻域有连续的二阶偏导数,且$(x_0,y_0)$是它的驻点。\pause
设 $A=f''_{xx}(x_0,y_0)$,$B=f''_{xy}(x_0,y_0)$,$C=f''_{yy}(x_0,y_0)$,则有\pause
\begin{enumerate}[<+->][(1)]
  \item 如果 $AC-B^2>0$,则 $f(x_0,y_0)$ 为极值。
  \begin{itemize}[<+->]
    \item 若 $A<0$,则 $f(x_0,y_0)$ 为极大值
    \item 若 $A>0$,则 $f(x_0,y_0)$ 为极小值
  \end{itemize}
  \item 如果 $AC-B^2<0$,则 $f(x_0,y_0)$ 不是极值。
  \item 如果 $AC-B^2=0$,则 $f(x_0,y_0)$ 是否为极值需另外判定。
\end{enumerate}
\end{theorem}
\end{frame}

\begin{frame}
\frametitle{极值问题}
\begin{example}
求二元函数的极值:
$$f(x,y)=y^3-x^2+6x-12y+5.$$
\end{example}
\vpause
\begin{exercise}
求二元函数的极值:
\begin{enumlite}
  \item $f(x,y)=4(x-y)-x^2-y^2;$
  \item $f(x,y)=x^3+y^3-3xy.$
\end{enumlite}
\end{exercise}
\end{frame}

\begin{frame}
\frametitle{最值问题}
设$f(x,y)$在有界闭区域$D$上连续,在$D$内可微分,且只有有限个驻点.\pause
则它在$D$上的最值只可能出现在
\begin{multicols}{2}\begin{enumerate}%[(1)]
  \item 内部的驻点 \item 边界上的最值点 
\end{enumerate}\end{multicols}
\pause\cdotfill\par
在很多实际问题中,最值一定存在,且在$D$内部取得.此时若驻点唯一,则该驻点就是最值点.
\end{frame}

\begin{frame}
\frametitle{最值问题}
\begin{example}
要造一个容积为$2$的长方体箱子。问选择长、宽、高为多少时,所用的材料最少?
\end{example}
\vpause
\begin{example}
有一宽为$24$厘米的长方形铁板,把它两边折起来做成一断面为等腰梯形的水槽。
问怎样折法才能使断面的面积最大?
\end{example}
\end{frame}

\begin{sframe}
\frametitle{最值问题}
\begin{example}
某工厂生产两种产品I和II,出售单价分别为10元与9元。生产$x$单位的产品I与生产$y$单位的产品II的总费用为
$$C(x,y)=400+2x+3y+0.01(3x^2+xy+3y^2).$$
问两种产品各生产多少时,总利润最大?
\end{example}
\end{sframe}

\subsection{条件极值与拉格朗日乘数法}

\begin{frame}
\frametitle{极值问题}
极值问题$\left\{\begin{array}{@{}l@{}}
\text{\bold{无条件极值}\quad 对自变量只有定义域限制} \\
\text{\bold{\makebox[5em][s]{条\ 件\ 极\ 值}}\quad
    \smash[b]{\parbox[t]{12em}{要求自变量满足某些方程}}}
\end{array}
\right.$
\ppause\cdotfill
\begin{example*}
求$z=xy$在条件$x^2+y^2=1$下的极值.
\end{example*}
\cdotfill\ppause
条件极值的求法$\left\{\begin{array}{@{}l@{}}
\text{\bold{代入法}\quad 消去变量化为无条件极值} \\
\text{\bold{拉格朗日乘数法}\quad 直接求解方程组}
\end{array}
\right.$
\end{frame}

\begin{frame}
\frametitle{拉格朗日乘数法1}
\begin{problem*}
求函数$u=f(x,y)$在约束条件$g(x,y)=0$下的极值。
\end{problem*}
\pause
\begin{method*}
令$L(x,y)=f(x,y)+\lambda g(x,y)$,\pause 由
$$\left\{\begin{aligned}
L'_x(x,y)&=0\\L'_y(x,y)&=0\\g(x,y)&=0
\end{aligned}\right.$$\pause
消去$\lambda$,解得的$(x,y)$即为极值可疑点。
\end{method*}
\end{frame}

\begin{sframe}
\frametitle{条件最值}
\begin{example}
求$f(x,y)=xy^2$在条件$x^2+y^2=1$,$x \ge 0$,$y \ge 0$下的最大值。
\end{example}
\vpause
\begin{exercise}
求$f(x,y)=x^2y$在条件$x+y=1$,$x \ge 0$,$y \ge 0$下的最大值。
\end{exercise}
\end{sframe}

\begin{frame}
\frametitle{拉格朗日乘数法1}
\begin{problem*}
求$u=f(x,y,z)$在约束条件$g(x,y,z)=0$下的极值。
\end{problem*}
\pause
\begin{method*}
令$L(x,y,z)=f(x,y,z)+\lambda g(x,y,z)$,\pause 由
$$\left\{\begin{aligned}
L'_x(x,y,z)&=0\\L'_y(x,y,z)&=0\\L'_z(x,y,z)&=0\\g(x,y,z)&=0
\end{aligned}\right.$$\pause
消去$\lambda$,解得的$(x,y,z)$即为极值可疑点。
\end{method*}
\end{frame}

\begin{frame}
\frametitle{条件最值}
\begin{example}
求表面积为$a^2$而体积为最大的长方体的体积.
\end{example}
\stext{\pause
\begin{solution}\smark
$L(x,y,z)=xyz+\lambda(2xy+2yz+2xz-a^2)$\pause
\[\begin{cases}
  yz+2\lambda(y+z)=0& \qquad\digitcircled{1}\\
  xz+2\lambda(x+z)=0& \qquad\digitcircled{2}\\
  xy+2\lambda(y+x)=0& \qquad\digitcircled{3}
\end{cases}\]\pause
$\digitcircled{1}$乘$x$减去$\digitcircled{2}$乘$y$,得到$x=y$.\pause 类似地有$y=z$.
\end{solution}}
\end{frame}

\begin{iframe}
\frametitle{拉格朗日乘数法2}
\begin{problem*}
求$u=f(x,y,z)$在约束条件$g(x,y,z)=0$和$h(x,y,z)=0$下的极值。
\end{problem*}
\pause
\begin{method*}
令$$L(x,y,z)=f(x,y,z)+\lambda g(x,y,z)+\mu h(x,y,z).$$\pause
由下面方程组$$\left\{\begin{aligned}
L'_x(x,y,z)&=0, & L'_y(x,y,z)&=0, & L'_z(x,y,z)&=0\\
g(x,y,z)&=0, & h(x,y,z)&=0
\end{aligned}\right.$$\pause
消去$\lambda$和$\mu$,解得的$(x,y,z)$即为极值可疑点。
\end{method*}
\end{iframe}

\begin{iframe}
\frametitle{条件最值}
\begin{example}%[2008数一第17题]\newline
已知曲线$C$的方程为$\left\{\begin{array}{l}x^2+y^2-2z^2=0 \\ x+y+3z=5\end{array}\right.$.
求$C$上距离$xOy$面最远的点和最近的点.
\end{example}
\end{iframe}

\begin{jframe}
%\frametitle{条件最值}
\begin{solution}
$L(x,y,z)=\bold{z^2}+\lambda\,g(x,y,z)+\mu\,h(x,y,z)$\pause
\[\begin{cases}
  2\lambda x + \mu = 0, & \qquad\digitcircled{1}\\
  2\lambda y + \mu = 0, & \qquad\digitcircled{2}\\
  2z - 4\lambda z +3\mu = 0, & \qquad\digitcircled{3}\\
  x^2+y^2-2z^2=0, & \qquad\digitcircled{4}\\
  x+y+3z=5. & \qquad\digitcircled{5}
\end{cases}\]\pause
由$\digitcircled{1}$和$\digitcircled{2}$得$x=y$,\pause
代入$\digitcircled{4}$和$\digitcircled{5}$得
\[\begin{cases}
  2x^2-2z^2=0, \\ 2x+3z=5.
\end{cases}\]\pause
解得$(x,y,z)=(-5,-5,5)$或$(x,y,z)=(1,1,1)$.
\end{solution}
\end{jframe}

\begin{iframe}
\frametitle{拉格朗日乘数法2}
\begin{problem*}
求$u=f(x,y,z,t)$在约束条件$g(x,y,z,t)=0$和$h(x,y,z,t)=0$下的极值。
\end{problem*}
\pause
\begin{method*}
令\begin{align*}L(x,y,z,t)=f(x,y,z,t)&+\lambda g(x,y,z,t)\\&+\mu h(x,y,z,t).\end{align*}\pause
由下面各式$$\begin{aligned}
L'_x&=0, & L'_y&=0, & L'_z&=0, & L'_t&=0, & g&=0, & h&=0
\end{aligned}$$\pause
消去$\lambda$和$\mu$,解得的$(x,y,z,t)$即为极值可疑点。
\end{method*}
\end{iframe}

\begin{sframe}
\frametitle{条件最值}
\begin{example}
求抛物线$y=x^2$和直线$x-y-2=0$的最短距离.
\end{example}
\end{sframe}

\mybookmark{复习与提高}

\begin{frame}
\frametitle{复习与提高}
\begin{review}
求二元函数的极值:
\begin{enumlite}
  \item $f(x,y)=xy(1-x-y)$
  \itext{\pause\item \imark$f(x,y)=(6x-x^2)(4y-y^2)$}
\end{enumlite}
\end{review}
\vpause
\begin{review}
求二元函数$f(x,y)=x+2y$在约束条件$x^2+y^2=1$下的最值。
\end{review}
\end{frame}

\begin{frame}
\frametitle{复习与提高}
\begin{puzzle}
已知平面上两定点$A(1,3)$, $B(4,2)$.试在椭圆
$$\dfrac{x^2}{9}+\dfrac{y^2}{4}=1\quad (x\ge0,y\ge0)$$
圆周上求一点$C$,使$\triangle ABC$面积$S_{\triangle}$最大或最小.
\end{puzzle}
\end{frame}

\begin{frame}
\frametitle{复习与提高}
\begin{puzzle}
求旋转抛物面$z=x^2+y^2$与平面$x+y-2z=2$之间的最短距离.
\end{puzzle}
\end{frame}

\begin{frame}
\frametitle{复习与提高}
\begin{puzzle}
求二元函数$f(x,y)=x+y-xy$在闭区域$D=\{(x,y)\mid x^2+y^2\le5\}$上的最值.
\end{puzzle}
\end{frame}

\begin{sframe}
\frametitle{复习与提高}
\vspace{-0.3em}\begin{solution}
(1) 在区域内部求无条件极值:由
\[ f'_x(x,y)=1-y=0,\quad f'_y(x,y)=1-x=0 \]
得驻点$(1,1)$,函数值$f(1,1)=1$.\ppause
(2) 在区域边界求条件极值:令
$$L(x,y)=x+y-xy+\lambda(x^2+y^2-5),$$
解得四组解\vspace{-0.3em}
\[ 
(-1,2),\quad (2,-1),\quad 
\big(\tfrac{\sqrt{10}}{2},\tfrac{\sqrt{10}}{2}\big),\quad
\big(-\tfrac{\sqrt{10}}{2},-\tfrac{\sqrt{10}}{2}\big)
\]%\vspace*{-0.1em}%
因此,函数有最小值$f\big(-\tfrac{\sqrt{10}}{2},-\tfrac{\sqrt{10}}{2}\big)=-\sqrt{10}-\frac{5}{2}$,
有最大值$f(-1,2)=f(2,-1)=3$.
\end{solution}
\end{sframe}

\begin{frame}
\frametitle{复习与提高}
\begin{choice}
已知函数$f(x,y)$在点$(0,0)$的某个邻域内连续,
且$\limit_{(x,y)\to(0,0)}\dfrac{f(x,y)-xy}{(x^2+y^2)^2}=1$,则\dotfill(\select{A})
\begin{choiceline}
  \item 点$(0,0)$不是$f(x,y)$的极值点
  \item 点$(0,0)$是$f(x,y)$的极大值点
  \item 点$(0,0)$是$f(x,y)$的极小值点
  \item 根据所给条件无法判断点$(0,0)$是否为$f(x,y)$的极值点
\end{choiceline} 
\end{choice}
\end{frame}

\begin{sframe}
\frametitle{复习与提高}
\begin{solution}
首先必有$f(0,0)=0$,且当$(x,y)\to(0,0)$时
\[ f(x,y)=xy+\big(x^2+y^2\big)^2+o\left(\big(x^2+y^2\big)^2\right) \]
当$y=x$,且$|x|$充分小时,有
\[ f(x,y)=x^2+4x^4+o(4x^4)=x^2\big[1+o(1)\big]>0 \]
当$y=-x$,且$|x|$充分小时,有
\[ f(x,y)=-x^2+4x^4+o(4x^4)=-x^2\big[1+o(1)\big]<0 \]
因此$(0,0)$不是$f(x,y)$的极值点.
\end{solution}
\end{sframe}

\end{document}
